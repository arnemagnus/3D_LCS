%
% Text format options
%

% Specify colors
\definecolor{gray80}{gray}{0.8}

%% Spacing
%\setstretch{1.1}
\setSpacing{1.1}
\setlength{\parskip}{1.4em}
\setlength{\parindent}{0em}

% Margins set in accordance to guidelines set by NTNU grafisk senter
% (i.e., minimum 2cm top, bottom, left and right in A4 format)
\setlrmarginsandblock{3cm}{2.5cm}{*}
\setulmarginsandblock{2.5cm}{2.5cm}{*}
\checkandfixthelayout

% Prettier unordered lists
\newcommand{\sbt}{\,\begin{picture}(-1,1)(-1,-3)\circle*{2.2}\end{picture}\ }
\renewcommand{\labelitemi}{\sbt}
\renewcommand{\labelitemii}{\sbt}
\renewcommand{\labelitemiii}{\sbt}


%
% Page format
%

%%% HEADER AND FOOTER

\makepagestyle{standard} %Make standard pagestyle

\makeatletter                 %Define standard pagestyle
\makeevenfoot{standard}{\thepage}{}{} %
\makeoddfoot{standard}{}{}{\thepage}  %
\makefootrule{standard}{\textwidth}{\normalrulethickness}{\footruleskip} % Remove footrule if footnotes are troublesome
\makeevenhead{standard}{\leftmark}{}{}
\makeoddhead{standard}{}{}{\rightmark}
\makeheadrule{standard}{\textwidth}{\normalrulethickness}
\makeatother                  %

\makeatletter
\makepsmarks{standard}{
\createmark{chapter}{both}{shownumber}{\@chapapp\ }{ \quad }
\createmark{section}{right}{shownumber}{}{ \quad }
\createplainmark{toc}{both}{\contentsname}
\createplainmark{lof}{both}{\listfigurename}
\createplainmark{lot}{both}{\listtablename}
\createplainmark{bib}{both}{\bibname}
\createplainmark{index}{both}{\indexname}
\createplainmark{glossary}{both}{\glossaryname}
   }
\makeatother                               %


\makepagestyle{firstchappage} %Make standard pagestyle for first page in chapter

\makeatletter                 %Define standard pagestyle
\makeevenfoot{firstchappage}{\thepage}{}{} %
\makeoddfoot{firstchappage}{}{}{\thepage}  %
\makefootrule{firstchappage}{\textwidth}{\normalrulethickness}{\footruleskip} % Remove footrule if footnotes are troublesome
\makeevenhead{firstchappage}{}{}{}
\makeoddhead{firstchappage}{}{}{}
\makeatother                  %

\makeatletter
\makepsmarks{firstchappage}{
%\createmark{chapter}{both}{shownumber}{\@chapapp\ }{ \quad }
%\createmark{section}{right}{shownumber}{}{ \quad }
%\createplainmark{toc}{both}{\contentsname}
%\createplainmark{lof}{both}{\listfigurename}
%\createplainmark{lot}{both}{\listtablename}
%\createplainmark{bib}{both}{\bibname}
%\createplainmark{index}{both}{\indexname}
%\createplainmark{glossary}{both}{\glossaryname}
   }
\makeatother                               %

\nouppercaseheads
\pagestyle{standard}               %Choosing pagestyle and chapter pagestyle
\aliaspagestyle{chapter}{firstchappage}    %Change plain to standard if you want the header on pages with chapter headings
\aliaspagestyle{chapter*}{firstchappage}


%
% Titlesec format options
%

% Default font size: 12 pt.
% Hierarchy: Usual text < Subsubsection < Subsection < Section < Chapter < Part
%            12 pt        13 pt           14 pt        15 pt     16 pt     17 pt

%\newfont{\partFont}{\fontsize{17pt}{17pt}\selectfont\bfseries}

\titleformat
{\part} % Command
[block] % Shape
{\fontsize{17pt}{17pt}\selectfont\bfseries} % Format
{\thepart} % Label
{1em} % Sep
{} % Before-code
[] % After-code

\titlespacing{\part}{0pt}{3.5ex+1ex-0.35ex}{2.3ex-0.5ex}
\titleclass{\part}{straight}

\titleformat
{\chapter}
[block]%
{\fontsize{16pt}{16pt}\selectfont\bfseries}
{\thechapter}%
{0.5em}
{}
[\normalsize\vspace*{.8\baselineskip}\titlerule]
\titlespacing*{\chapter}{0pt}{2.5ex plus 1ex minus .2ex}{-0.3ex minus -.2ex}
\titleclass{\chapter}{straight}

\titleformat
{name=\chapter,numberless}
[block]
{\fontsize{16pt}{16pt}\selectfont\bfseries\filcenter}
{}
{0.5em}
{}
[\normalsize\vspace*{.8\baselineskip}\titlerule]

\titleformat
{\section}
[block]
{\fontsize{15pt}{15pt}\selectfont}
{\thesection}
%{\makebox[0cm][r]{\thesection\hspace{1em}}}
{0.5em}
%{\fontsize{15pt}{15pt}\selectfont\scshape\lowercase}
{\fontsize{15pt}{15pt}\selectfont\scshape}
[]

\titlespacing*{\section}{0pt}{2ex plus 1ex minus .2ex}{-1.3ex minus -.2ex}
\titleclass{\section}{straight}


\titleformat
{\subsection}
[block]
{\fontsize{14pt}{14pt}\selectfont}
{\thesubsection}
{.6em}
{\itshape}
[]
\titlespacing*{\subsection}{0pt}{1.5ex plus 1ex minus .2ex}{-1.3ex minus -.2ex}
\titleclass{\subsection}{straight}

\titleformat
{\subsubsection}
[block]
{\fontsize{13pt}{13pt}\selectfont\bfseries}
{}
{}
{}
[]
\titlespacing{\subsubsection}{0pt}{1ex plus 1ex minus .2ex}{-1.3ex minus -.2ex}
\titleclass{\subsubsection}{straight}




%
% Tocloft format options
%

% Table of Contents:

\renewcommand\printtoctitle[1]{\vspace{-15.5ex}\chapter*{#1}}

\addto\captionsenglish{%
    \renewcommand{\contentsname}{Table of Contents}%
}

\newlength{\aftertocskip}
\setlength{\aftertocskip}{0ex}
\renewcommand{\aftertoctitle}{\par\nobreak\vskip\aftertocskip}

% Chapters:
\renewcommand*{\cftchapterleader}{\hfill}
%\renewcommand*{\cftchapterdotsep}{\cftdotsep}
%\renewcommand*{\cftsectionleader}{\cftdotsep}
%\renewcommand*{\cftsubsectionleader}{\cftdotfill}
\renewcommand*{\cftchapterpagefont}{\normalfont}
\renewcommand*{\cftchapterformatpnum}[1]{\bfseries{#1}}
\renewcommand*{\cftsectionformatpnum}[1]{#1}
\renewcommand*{\cftsubsectionformatpnum}[1]{#1}
%\renewcommand{\cftchapterafterpnum}{\cftparfillskip}
%\renewcommand{\cftsectionafterpnum}{\cftparfillskip}
%\renewcommand{\cftsubsectionafterpnum}{\cftparfillskip}
\setrmarg{3.55em plus 1fil}
\setsecnumdepth{subsection}
\maxsecnumdepth{subsection}
\settocdepth{subsection}
%%\renewcommand{\cftbeforechapskip}{1pt}
%\renewcommand{\cftchappresnum}{\bfseries}
%\renewcommand{\cftchapfont}{\bfseries}
%\renewcommand{\cftchappagefont}{\bfseries}
%\renewcommand{\cftchapafterpnum}{\cftparfillskip}

%% Sections:
%\renewcommand{\cftsecpresnum}{\scshape}
%\renewcommand{\cftsecfont}{\normalfont}
%\renewcommand{\cftsecpagefont}{\normalfont}
%\renewcommand{\cftsecleader}{\cftdotfill{\cftsecdotsep}}
%\renewcommand{\cftsecafterpnum}{\cftparfillskip}


%% Subsections:
%\renewcommand{\cftsubsecpresnum}{\scshape}
%\renewcommand{\cftsubsecfont}{\normalfont}
%\renewcommand{\cftsubsecpagefont}{\normalfont}
%\renewcommand{\cftsubsecleader}{\cftdotfill{\cftsubsecdotsep}}
%\renewcommand{\cftsubsecafterpnum}{\cftparfillskip}

% List of Figures:
\renewcommand\printloftitle[1]{\vspace{-15.5ex}\chapter*{#1}}

\addto\captionsenglish{%
    \renewcommand{\listfigurename}{List of Figures}%
}


\addtocontents{lof}{\vspace{-2ex}}

% List of Tables:
\renewcommand\printlottitle[1]{\vspace{-15.5ex}\chapter*{#1}}

\addto\captionsenglish{%
    \renewcommand{\listtablename}{List of Tables}%
}

\addtocontents{lot}{\vspace{-2ex}}

%
% "Function" definitions for front-, main- and backmatter
%

% Roman numbering for the frontmatter
\def\frontmatter{%
    \pagenumbering{roman}%
    \setcounter{page}{1}%
    \renewcommand{\thepart}{}%
    \renewcommand{\thechapter}{}%
    \renewcommand{\thesection}{}%
    \renewcommand{\thesubsection}{}%
    \renewcommand{\thesubsubsection}{}%
}

% Arabic numbering for the mainmatter
\def\mainmatter{%
    \pagenumbering{arabic}%
    \setcounter{page}{1}%
    \renewcommand{\thepart}{\arabic{part}}%
    \renewcommand{\thechapter}{\arabic{chapter}}%
    \renewcommand{\thesection}{\thechapter.\arabic{section}}%
    \renewcommand{\thesubsection}{\thesection.\arabic{subsection}}%
    \renewcommand{\thesubsubsection}{\arabic{subsubsection}}%
}

% Alphabetic "numbering" for the backmatter
\def\backmatter{%
    \setcounter{part}{0}%
    \setcounter{chapter}{0}%
    \setcounter{section}{0}%
    \setcounter{subsection}{0}%
    \setcounter{subsubsection}{0}%
    \renewcommand{\thepart}{\Alph{part}}%
    \renewcommand{\thechapter}{\Alph{chapter}}%
    \renewcommand{\thesection}{\thechapter.\arabic{section}}%
    \renewcommand{\thesubsection}{\thesection.\arabic{subsection}}%
    \renewcommand{\thesubsubsection}{\Alph{subsubsection}}%
    %\renewcommand{\cftchapnumwidth}{8em}
    \addtocontents{toc}{\def\protect\cftchappresnum{\MakeUppercase{\appendixname{}} }}
    \addtocontents{toc}{\def\protect\cftchapnumwidth{8em}}
}

% Add spacing between frontmatter and mainmatter in ToC
\let\oldfrontmatter\frontmatter
\renewcommand{\frontmatter}{%
    \addtocontents{toc}{\protect\vspace{10pt}}%
    \oldfrontmatter
}

% Add spacing between mainmatter (incl. references) and backmatter in ToC
\let\oldbackmatter\backmatter
\renewcommand{\backmatter}{%
    \addtocontents{toc}{\protect\vspace{10pt}}%
    \oldbackmatter%
}

%
% References
%

% When referring to a section, include the chapter no.
%\makeatletter
%\renewcommand{\p@section}{%
%    \thechapter.\arabic{section}%
%    \expandafter\@gobble%
%}
%\makeatother

%% When referring to a section, include the chapter and section no.s
%\makeatletter
%\renewcommand{\p@subsection}{%
%    \thechapter.\thesection.\arabic{subsection}%
%    \expandafter\@gobble%
%}
%\makeatother

% Ensure that the entire citation from BibLaTeX becomes part of the
% hyperref hyperlink, rather than just the year
    \DeclareCiteCommand{\cite}
    {\usebibmacro{prenote}}
    {\usebibmacro{citeindex}%
    \printtext[bibhyperref]{\usebibmacro{cite}}}
    {\multicitedelim}
    {\usebibmacro{postnote}}

    \DeclareCiteCommand*{\cite}
    {\usebibmacro{prenote}}
    {\usebibmacro{citeindex}%
    \printtext[bibhyperref]{\usebibmacro{citeyear}}}
    {\multicitedelim}
    {\usebibmacro{postnote}}

    \DeclareCiteCommand{\parencite}[\mkbibparens]
    {\usebibmacro{prenote}}
    {\usebibmacro{citeindex}%
    \printtext[bibhyperref]{\usebibmacro{cite}}}
    {\multicitedelim}
    {\usebibmacro{postnote}}

    \DeclareCiteCommand*{\parencite}[\mkbibparens]
    {\usebibmacro{prenote}}
    {\usebibmacro{citeindex}%
    \printtext[bibhyperref]{\usebibmacro{citeyear}}}
    {\multicitedelim}
    {\usebibmacro{postnote}}

    \DeclareCiteCommand{\footcite}[\mkbibfootnote]
    {\usebibmacro{prenote}}
    {\usebibmacro{citeindex}%
    \printtext[bibhyperref]{ \usebibmacro{cite}}}
    {\multicitedelim}
    {\usebibmacro{postnote}}

    \DeclareCiteCommand{\footcitetext}[\mkbibfootnotetext]
    {\usebibmacro{prenote}}
    {\usebibmacro{citeindex}%
    \printtext[bibhyperref]{\usebibmacro{cite}}}
    {\multicitedelim}
    {\usebibmacro{postnote}}

    \DeclareCiteCommand{\textcite}
    {\boolfalse{cbx:parens}}
    {\usebibmacro{citeindex}%
    \printtext[bibhyperref]{\usebibmacro{textcite}}}
    {\ifbool{cbx:parens}
        {\bibcloseparen\global\boolfalse{cbx:parens}}
        {}%
    {\multicitedelim}}
    {\usebibmacro{textcite:postnote}}

%
% Definitions for mathematical symbols
%

% Macros for greek letters in roman style, in math mode
\DeclareRobustCommand{\mathup}[1]{%
\begingroup\changegreek\mathrm{#1}\endgroup}
\DeclareRobustCommand{\mathbfup}[1]{%
\begingroup\changegreekbf\mathbf{#1}\endgroup}

\makeatletter
\def\changegreek{\@for\next:={%
        alpha,beta,gamma,delta,epsilon,zeta,eta,theta,iota,kappa,lambda,mu,nu,%
        xi,pi,rho,sigma,tau,upsilon,phi,chi,psi,omega,varepsilon,varpi,%
    varrho,varsigma,varphi}%
\do{\expandafter\let\csname\next\expandafter\endcsname\csname\next up\endcsname}}
\def\changegreekbf{\@for\next:={%
        alpha,beta,gamma,delta,epsilon,zeta,eta,theta,iota,kappa,lambda,mu,nu,%
        xi,pi,rho,sigma,tau,upsilon,phi,chi,psi,omega,varepsilon,varpi,%
    varrho,varsigma,varphi}%
    \do{\expandafter\def\csname\next\expandafter\endcsname\expandafter{%
\expandafter\bm\expandafter{\csname\next up\endcsname}}}}
\makeatother

% Define vectors in bold roman font
\newcommand{\vct}[1]{\ensuremath{\mathbfup{#1}}}

% Define unit vectors in bold roman font, with hats
\newcommand{\uvct}[1]{\ensuremath{\mathbfup{\hat{#1}}}}

% Define matrices in bold italic font
\newcommand{\mtrx}[1]{\ensuremath{\bm{#1}}}

% Define macro for the sign function
\DeclareMathOperator{\sgn}{sgn}

% Define macros for aboslute and relative tolerances
\DeclareMathOperator{\atol}{Atol}
\DeclareMathOperator{\rtol}{Rtol}

% Define macro for error of embedded methods
\DeclareMathOperator{\errem}{err}

% Define macro for sc in embedded methods
\DeclareMathOperator{\scem}{sc}

% Define macro for safety factor in embedded methods
\DeclareMathOperator{\facem}{fac}

% Define macro for RMS deviation
\DeclareMathOperator{\rmsd}{RMSD}

% Declare macro for display style (partial) derivatives (cf. physics package)
\DeclareDocumentCommand\ddv{}{\displaystyle\derivative}
\DeclareDocumentCommand\dpdv{}{\displaystyle\partialderivative}

% Declare macro for (scalable) inner product notation
\DeclarePairedDelimiterX{\inp}[2]{\langle}{\rangle}{#1, #2}

% Declare environment for aligned, gathered math environments
\newenvironment{nalign}{
    \begin{equation}
    \begin{aligned}
}{
    \end{aligned}
    \end{equation}
    \ignorespacesafterend
}
\makeatletter
%http://groups.google.com/group/comp.text.tex/msg/7e812e5d6e67fcc5
\def\convertto#1#2{\strip@pt\dimexpr #2*65536/\number\dimexpr 1#1}
\makeatother
