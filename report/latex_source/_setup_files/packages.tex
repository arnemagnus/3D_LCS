% ---------------- %
% Text-related packages
% ---------------- %


\usepackage[showframe]{geometry}

% Margins
%\usepackage[a4paper,margin=2.5cm]{geometry}

% Typeset UTF-8 symbols:
\usepackage[T1]{fontenc}
\usepackage[utf8]{inputenc}

% Typographical refinement
\usepackage[activate={true,nocompatibility},% Let microtype do its magic
%                                           % without trying to keep page breaks
%                                           % etc.
final=true,% Activate microtype even when in draft mode
kerning=true,% Fixes kerning issues
spacing=true,% Fixes spacing issues
tracking=true,% Adds letter spacing for small caps
shrink=30,% Characters are stretched or shrunk in order to improve justification
stretch=30,% up to 3 %
factor=0% Controls how much punctuation protrudes past the end of the line
]{microtype}
\microtypecontext{spacing=nonfrench} % <-- No spacing found for Linux Libertine

% Text spacing
\usepackage[]{setspace}

% Manage culturally-determined typographical, and other, rules:
\usepackage[main=english,norsk]{babel}

% Choose Libertine font:
\usepackage[]{libertine}

% Include author's affiliation
\usepackage[]{authblk}

% Typeset relevant quotations or sayings as epigraphs:
\usepackage[]{epigraph}

% Simple syntax for superscript 1st, 2nd, 3rd etc: \nth{#}
\usepackage[super]{nth}

%% Paragraphs separated by blank lines rather than indentation
%\usepackage[parfill]{parskip}

% Customize title/heading styles
\usepackage[]{titlesec}

%% Control ToC, LoF, LoT etc.
%\usepackage[titles]{tocloft}

% Verbatim
\usepackage[]{verbatim}

% Frame any object in the document
\usepackage[]{framed}

% Break lines in citations which do not fit on a single line
\usepackage[]{breakcites}

% Insert placeholder text
\usepackage[]{lipsum}

% Include the possibility of colored text
\usepackage[]{color}

% Include the possibility to strikethrough text
\usepackage[normalem]{ulem}

%
% Float-related packages
%

% Interface for floating objects
\usepackage[]{float}

% Interface for stacked floating objects:
\usepackage[]{subcaption}

% Enhanced graphics support
\usepackage[]{graphicx}

% PGF files
\usepackage[]{pgf}

% PGF plots
\usepackage[]{pgfplots}

% Wrap text around figure
\usepackage[]{wrapfig}

% Extend array and tabular environments
\usepackage[]{array}

% Define math mode version of centered type column
\newcolumntype{C}{>$c<$}

% Rotation tools for floats. Default: Clockwise rotation
\usepackage[figuresright]{rotating}

% Format float captions any which way you want.
% My preference:    No more than 90 % of the overall textwidth
%                   Slightly smaller fontsize than the text otherwise
%                   Label typeset in bold
%                   Separate label from caption with colon
\usepackage[%format={\fontsize{11}{13}\selectfont},%
width=0.9\textwidth,%
labelfont=bf,%
labelsep=colon%
]{caption}

% Include entire (external) PDF pages in document
\usepackage[]{pdfpages}

% Better formatting for tables
\usepackage[]{booktabs}


%
% Mathematical functions
%

% Necessary for most kinds of mathematical typesetting:
\usepackage[]{amsmath,amssymb}

% Package containing useful macros for typsetting vector calculus, linear
% algebra etc.
\usepackage[]{physics}

% Typeset mathematical symbols in bold, useful when working with vectors,
% tensors, etc.
\usepackage[]{bm}

% Necessary to write greek letters in upright in math mode, e.g. for vectors
\usepackage[libertine]{newtxmath}

% Clever typesetting of SI units
\usepackage[]{siunitx}



%
% Reference-related packages
%

% Use biblatex for handling refernces
\usepackage[backend=biber,%
style=authoryear,%
language=british,%
dashed=false,%
url=false,%
doi=false,%
eprint=false,%
giveninits=true,%
uniquename=init%
]{biblatex}

\defbibheading{bibliography}[\bibname]{\chapter*{#1}\markboth{#1}{#1}}
\AtBeginDocument{\renewcommand{\bibname}{References}}
% Refer to all authors by their last name first
\DeclareNameAlias{author}{last-first}

% Ensure that quoted texts are typeset according to rules of main language
\usepackage[]{csquotes}

% Access a myriad of different coloring options for e.g. hyperref links
\usepackage[]{xcolor}

% Make all referrals hyperlinks (within the document)
\usepackage[]{hyperref}
\hypersetup{%
%hidelinks=true,%
colorlinks,%
linkcolor={red!50!black},%
citecolor={blue!50!black},%
urlcolor={blue!80!black},%
pdfauthor={Arne Magnus Tveita Løken},%
pdftitle={Sensitivity to Numerical Integration Scheme in Calculation of Lagrangian Coherent Structures},%
pdfsubject={Computational nonlinear physics},%
pdfkeywords={Computational physics, Nonlinear dynamics, Lagrangian coherent structures, Numerical integration}%
}

% Hide hyperlinks upon printing, i.e., they are printed in the regular
% text color (typically key)
\usepackage[ocgcolorlinks]{ocgx2}

% Bookmark organization for the hyperref package
\usepackage[]{bookmark}

% Clever referrals to objects (figures, tables, etc.)
\usepackage[]{cleveref}

% Use endash in a cleveref-range
\newcommand{\crefrangeconjunction}{--}

% Clever typesetting of theorems, definitions etc.
\let\openbox\relax
\usepackage[]{amsthm}
\theoremstyle{plain}
\newtheorem{thm}{Theorem}
\crefname{thm}{theorem}{theorems}
\theoremstyle{definition}
\newtheorem{defn}{Definition}
\crefname{defn}{definition}{definitions}

\crefname{figure}{figure}{figures}
\crefname{equation}{equation}{equations}
