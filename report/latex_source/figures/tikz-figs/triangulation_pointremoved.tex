\documentclass[crop]{standalone}
\usepackage{tikz}
\usepackage[]{tikz-3dplot}
\usepackage{pgfplots}
\pgfplotsset{compat=1.15}
\usepackage[]{amsmath}
\usepackage[]{libertine}
\usepackage[libertine]{newtxmath}
\usepackage[]{bm}
\usepackage[]{physics}
% Macros for greek letters in roman style, in math mode
\DeclareRobustCommand{\mathup}[1]{%
\begingroup\ensuremath\changegreek\mathrm{#1}\endgroup}
\DeclareRobustCommand{\mathbfup}[1]{%
\begingroup\ensuremath\changegreek\bm{\mathrm{#1}}\endgroup}


\makeatletter
\def\changegreek{\@for\next:={%
        alpha,beta,gamma,delta,epsilon,zeta,eta,theta,iota,kappa,lambda,mu,nu,%
        xi,pi,rho,sigma,tau,upsilon,phi,chi,psi,omega,varepsilon,varpi,%
    varrho,varsigma,varphi}%
\do{\expandafter\let\csname\next\expandafter\endcsname\csname\next up\endcsname}}
\makeatother

% Define vectors in bold, roman, lowercase font
\newcommand{\vct}[1]{\ensuremath{\mathbfup{\MakeLowercase{#1}}}}

% Define unit vectors in bold, roman, lowercase font, with hats
\newcommand{\uvct}[1]{\ensuremath{\mathbfup{\hat{\MakeLowercase{#1}}}}}

% Define matrices in bold, roman, uppercase font
\newcommand{\mtrx}[1]{\ensuremath{\mathbfup{\MakeUppercase{#1}}}}
\usetikzlibrary{%
    angles,%
    arrows.meta,%
    backgrounds,%
    calc,%
    decorations,%
    fit,%
    hobby,%
    patterns,%
    positioning,%
    quotes
}


\begin{document}

\tdplotsetmaincoords{60}{25}

\begin{tikzpicture}[tdplot_main_coords]
    \pgfmathsetmacro{\innerradius}{5}
    \pgfmathsetmacro{\outerradius}{7.5}
    \pgfmathsetmacro{\size}{2.5}
    \pgfmathsetmacro{\outerarclowerangle}{35}
    \pgfmathsetmacro{\outerarcupperangle}{100}

    % Place inner points in inner level set
    \foreach [count=\i] \ang in {30,52,73,92}%
    {%
        \draw[draw=black!80,fill=gray!20] ( {\innerradius*cos(\ang)}, {\innerradius*sin(\ang)}, 0) circle (\size pt) coordinate (i\i);
    }

    % Place the set of initial points which remain after deletion
    \foreach [count=\i] \ang in {32,68,89}%
    {%
        \draw[draw=black!80,fill=gray!20] ( {\outerradius*cos(\ang)}, {\outerradius*sin(\ang)}, 0 ) circle (\size pt) coordinate (o\i);
    }
    \draw[draw=gray!80,fill=gray!10] ( {\outerradius*cos(54)}, {\outerradius*sin(54)}, 0 ) circle (\size pt) coordinate (doomed);

    % Place nodes of outer level set
    \node[above right = 0pt and -5pt of o1] {$\mathcal{M}_{i+1,j-1}$};
    \node[above right = 0pt and -5pt of o2] {$\mathcal{M}_{i+1,j+1}$};
    \node[above right = 0pt and -5pt of doomed,color=gray!80] {$\mathcal{M}_{i+1,j}$};
    \node[above right = -5pt and 2pt of o3] {$\mathcal{M}_{i+1,j+2}$};

    % Place nodes of inner level set
    \node[below left = 0pt and -10pt of i1] {$\mathcal{M}_{i,j-1}$};
    \node[below left = 0pt and -7pt of i2] {$\mathcal{M}_{i,j}$};
    \node[below left = 0pt and -10pt of i3] {$\mathcal{M}_{i,j+1}$};
    \node[below left = 0pt and -10pt of i4] {$\mathcal{M}_{i,j+2}$};


    % Draw interpolation triangles
    \begin{scope}[on background layer]


        \draw[draw=none,pattern=north east lines, pattern color = gray!20]
               (i3) -- (o2) -- (o3) -- cycle;
        \draw[draw=black!65,dashed] (i3) -- (o2) -- (o3) -- cycle;

        \draw[draw=none,pattern=north west lines, pattern color = gray!20] (i3) -- (o3) -- (i4) -- cycle;
        \draw[draw=black!65,dashed] (i3) -- (o3) -- (i4) -- cycle;

        \draw[draw=none,pattern=north west lines, pattern color = gray!20] (i1) -- (o1) -- (i2) -- cycle;
        \draw[draw=black!65,dashed] (i1) -- (o1) -- (i2) -- cycle;

        % Triangles relevant for this special case
        \draw[draw=black!60,dashed,fill=gray!15] (i2) -- (o1) -- (o2) -- cycle;
%        \draw[draw=black!65,dashed,fill=gray!30] (i2) -- (o2) -- (o3) -- cycle;
        \draw[draw=black!65,dashed,fill=gray!30] (i2) -- (o2) -- (i3) -- cycle;
    \end{scope}

\end{tikzpicture}
\end{document}
