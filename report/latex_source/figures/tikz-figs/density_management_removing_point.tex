\documentclass[crop]{standalone}
\usepackage{tikz}
\usepackage[]{tikz-3dplot}
\usepackage{pgfplots}
\pgfplotsset{compat=1.15}
\usepackage[]{amsmath}
\usepackage[]{libertine}
\usepackage[libertine]{newtxmath}
\usepackage[]{bm}
\usepackage[]{physics}
% Macros for greek letters in roman style, in math mode
\DeclareRobustCommand{\mathup}[1]{%
\begingroup\ensuremath\changegreek\mathrm{#1}\endgroup}
\DeclareRobustCommand{\mathbfup}[1]{%
\begingroup\ensuremath\changegreek\bm{\mathrm{#1}}\endgroup}


\makeatletter
\def\changegreek{\@for\next:={%
        alpha,beta,gamma,delta,epsilon,zeta,eta,theta,iota,kappa,lambda,mu,nu,%
        xi,pi,rho,sigma,tau,upsilon,phi,chi,psi,omega,varepsilon,varpi,%
    varrho,varsigma,varphi}%
\do{\expandafter\let\csname\next\expandafter\endcsname\csname\next up\endcsname}}
\makeatother

% Define vectors in bold, roman, lowercase font
\newcommand{\vct}[1]{\ensuremath{\mathbfup{\MakeLowercase{#1}}}}

% Define unit vectors in bold, roman, lowercase font, with hats
\newcommand{\uvct}[1]{\ensuremath{\mathbfup{\hat{\MakeLowercase{#1}}}}}

% Define matrices in bold, roman, uppercase font
\newcommand{\mtrx}[1]{\ensuremath{\mathbfup{\MakeUppercase{#1}}}}
\usetikzlibrary{%
    angles,%
    arrows.meta,%
    backgrounds,%
    calc,%
    decorations,%
    fit,%
    hobby,%
    patterns,%
    positioning,%
    quotes
}


\begin{document}

\tdplotsetmaincoords{60}{25}

\begin{tikzpicture}[tdplot_main_coords]
    \pgfmathsetmacro{\innerradius}{7.5}
    \pgfmathsetmacro{\size}{2.5}
    \pgfmathsetmacro{\innerarclowerangle}{35}
    \pgfmathsetmacro{\innerarcupperangle}{100}

    % Place the set of initial points which remain after deletion
    \foreach [count=\i] \ang in {35,64,90}%
    {%
        \draw[draw=black!80,fill=gray!20] ( {\innerradius*cos(\ang)}, {\innerradius*sin(\ang)}, 0 ) circle (\size pt) coordinate (i\i);
    }
    \draw[draw=gray!80,fill=gray!10] ( {\innerradius*cos(75)}, {\innerradius*sin(75)}, 0 ) circle (\size pt) coordinate (doomed);

    % Place nodes of inner level set
    \node[above right = 0pt and -5pt of i1] {\footnotesize $\mathcal{M}_{i,j-1}$};
    \node[above right = 0pt and -5pt of i2] {\footnotesize $\mathcal{M}_{i,j}$};
    \node[above right = -1pt and -3.5pt of doomed,color=gray!80] {\footnotesize $\mathcal{M}_{i,j+1}$};
    \node[above right = 0pt and -5pt of i3] {\footnotesize $\mathcal{M}_{i,j+2}$};

    % Define helper coordinates to place small bracket
    \coordinate[above right = 0pt and 0pt of doomed] (sb1);
    \coordinate[above right = 0pt and 0pt of i2] (sb2);
    % Place small bracket
    \draw[decorate,decoration={brace,amplitude=2.5pt,raise=5pt},yshift=0pt] (sb2) -- (sb1) node [midway,below right = -12pt and -35pt,rotate=-26] {\footnotesize $\norm{\vct{x}_{i+1,j+1}-\vct{x}_{i+1,j}} < \Delta_{\min}$};

    % Define helper coordinates to place first of two big brackets
    \coordinate[below left = 0pt and 0pt of i3] (bb1);
    \coordinate[below left = 0pt and 0pt of i2] (bb2);
    % Place first bracket
    \draw[decorate,decoration={brace,amplitude=2.5pt,raise=25pt},yshift=0pt] (bb2) -- (bb1) node [midway,below right = 10pt and -45pt,rotate=-20.8] {\footnotesize $\norm{\vct{x}_{i+1,j+2}-\vct{x}_{i+1,j}} < \Delta_{\max}$};

    % Define helper coordinates to place second of two big brackets
    \coordinate[above right = 0pt and 0pt of doomed] (gb1);
    \coordinate[below left = 0pt and 0pt of i1] (gb2);
    % Place second bracket
    \draw[decorate,decoration={brace,amplitude=2.5pt,raise=25pt},yshift=0pt] (gb1) -- (gb2) node [midway,above right = 50pt and -20pt,rotate=-40.5] {\footnotesize $\norm{\vct{x}_{i+1,j+1}-\vct{x}_{i+1,j-1}}<\Delta_{\max}$};

%    % Define helper coordinates to place bracket
%    \coordinate[above right = 3.5pt and 1.5pt of o3]  (b1);
%    \coordinate[above right = 3.5pt and 1.5pt of o2]  (b2);

%    % Draw bracket



\end{tikzpicture}
\end{document}
