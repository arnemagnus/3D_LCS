\documentclass[crop]{standalone}
\usepackage{tikz}
\usepackage[]{tikz-3dplot}
\usepackage{pgfplots}
\usepackage[]{amsmath}
\usepackage[]{libertine}
\usepackage[libertine]{newtxmath}
\usepackage[]{bm}
\usepackage[]{physics}
% Macros for greek letters in roman style, in math mode
\DeclareRobustCommand{\mathup}[1]{%
\begingroup\ensuremath\changegreek\mathrm{#1}\endgroup}
\DeclareRobustCommand{\mathbfup}[1]{%
\begingroup\ensuremath\changegreek\bm{\mathrm{#1}}\endgroup}

\makeatletter
\def\changegreek{\@for\next:={%
        alpha,beta,gamma,delta,epsilon,zeta,eta,theta,iota,kappa,lambda,mu,nu,%
        xi,pi,rho,sigma,tau,upsilon,phi,chi,psi,omega,varepsilon,varpi,%
    varrho,varsigma,varphi}%
\do{\expandafter\let\csname\next\expandafter\endcsname\csname\next up\endcsname}}
\makeatother

% Define vectors in bold, roman, lowercase font
\newcommand{\vct}[1]{\ensuremath{\mathbfup{\MakeLowercase{#1}}}}

% Define unit vectors in bold, roman, lowercase font, with hats
\newcommand{\uvct}[1]{\ensuremath{\mathbfup{\hat{\MakeLowercase{#1}}}}}

% Define matrices in bold, roman, uppercase font
\newcommand{\mtrx}[1]{\ensuremath{\mathbfup{\MakeUppercase{#1}}}}
\usetikzlibrary{%
    angles,%
    arrows.meta,%
    backgrounds,%
    calc,%
    decorations,%
    fit,%
    hobby,%
    patterns,%
    positioning,%
    quotes
}
\makeatletter
\tikzset{
  fitting node/.style={
    inner sep=0pt,
    fill=none,
    draw=none,
    reset transform,
    fit={(\pgf@pathminx,\pgf@pathminy) (\pgf@pathmaxx,\pgf@pathmaxy)}
  },
  reset transform/.code={\pgftransformreset}
}
\makeatother
% A simple empty decoration, that is used to ignore the last bit of the path
\pgfdeclaredecoration{ignore}{final}
{
\state{final}{}
}

% Declare the actual decoration.
\pgfdeclaremetadecoration{middle}{initial}{
    \state{initial}[
        width={0pt},
        next state=middle
    ]
    {\decoration{moveto}}

    \state{middle}[
        width={\pgfdecorationsegmentlength*\pgfmetadecoratedpathlength},
        next state=final
    ]
    {\decoration{curveto}}

    \state{final}
    {\decoration{ignore}}
}


% Create a key for easy access to the decoration
\tikzset{middle segment/.style={decoration={middle},decorate, segment length=#1}}

\def\getangle(#1)(#2)#3{%
  \begingroup%
    \pgftransformreset%
    \pgfmathanglebetweenpoints{\pgfpointanchor{#1}{center}}{\pgfpointanchor{#2}{center}}%
    \expandafter\xdef\csname angle#3\endcsname{\pgfmathresult}%
  \endgroup%
}


\begin{document}

\tdplotsetmaincoords{70}{20}

\begin{tikzpicture}[tdplot_main_coords]
	\pgfmathsetmacro{\innerxscl}{2.5}
    \pgfmathsetmacro{\inneryscl}{2.5}
	\pgfmathsetmacro{\outerxscl}{4.5}
	\pgfmathsetmacro{\outeryscl}{4.5}
    \pgfmathsetmacro{\innernum}{11} % = Degree span / separation, outer for loop
    \pgfmathsetmacro{\outernum}{21}
    % Inner geodesic level set
    \foreach [count = \i] \a in {15,48,...,345}%
    {%
        \coordinate (i\i) at ( {\innerxscl*cos(\a)} , {\inneryscl*sin(\a)} , 0  ) ;
    }%
    \draw[stroke=black!80,thin,dotted] (i1) to [ curve through ={(i2) .. (i3) .. (i4) .. (i5) .. (i6) .. (i7) .. (i8) .. (i9) .. (i10) .. (i11) }] (i1);
    \foreach \i in {1,...,\innernum}%
    {%
        \draw[stroke=black!80, fill=gray!60] (i\i) circle (3pt);
    }%

    % Outer geodesic level set
    \foreach [count = \i] \a in {10,27,...,350}%
    {%
        \coordinate (o\i) at ( {\outerxscl*cos(\a)} , {\outeryscl*sin(\a)} , 0 ) ;
    }%
    \draw[stroke=black!80,thin,dotted] (o1) to [ tension = 0.3, curve through ={(o2)  .. (o3)  .. (o4)  .. (o5)  .. (o6)  .. (o7)  ..
                                                                 (o8)  .. (o9)  .. (o10) .. (o11) .. (o12) .. (o13) ..
                                                             (o14) .. (o15) .. (o16) .. (o17) .. (o18) .. (o19) .. (o20) .. (o21)}] (o1);

    \foreach \i in {1,...,\outernum}%
    {%
        \draw[stroke=black!80, fill=gray!60] (o\i) circle (3pt);
    }%
    \begin{scope}[on background layer]
    % Draw line parallel to prev_vec
    \draw[stroke=black!80,->,densely dashed] (i1) to ($(i1)!2.25!(o1)$) coordinate (vf); % Draw 225 % of the path from (i1) to (o1)
    % Suggested new point
    \coordinate (n1) at ( {5.2*cos(10)} , {5.1*cos(10)}, 0.1 );
    \draw[stroke=black!80,fill=gray!60] (n1) circle (3pt) ;
    % Draw prev_vec for new point
    \draw[stroke=black!80,->,dashed] (o1) to ($(o1)!0.95!(n1)$);

    \pic [draw,-{>[scale=1.1]}, stroke=black!80, angle radius = 15,angle eccentricity = 1.5] {angle= vf--o1--n1};
    \node at (o1) [above right = -3pt and 15pt] {$\theta$};

    \end{scope}

    % Add nodes
    \node at (i1) [below right] {$\mathcal{P}_{i-1,j}$};
    \node at (o1) [below right] {$\mathcal{P}_{i,j}$};
    \node at (n1) [below right= 3pt and -5pt] {$\mathcal{P}_{i+1,j}$};

    % Denote inner and outer parametrizations
    \node at ($(i3)!0.6!(i4)$) (innerpt) {};
    \node [below left =6pt and 1pt of innerpt] (innermrk) {$\mathcal{C}_{i-1}$};

    \draw[stroke=black!80] ($(innerpt)!0.7!(innermrk)$) to [out=30,in=-60] ($(innermrk)!0.9!(innerpt)$);

    \node at ($(o4)!0.6!(o5)$) (outerpt) {};
    \node [below right=5pt and 1pt of outerpt] (outermrk) {$\mathcal{C}_{i}$};
    \draw[stroke=black!80] ($(outerpt)!0.6!(outermrk)$) to [out=135,in=-70] ($(outermrk)!0.9!(outerpt)$);



\end{tikzpicture}
\end{document}
