\documentclass[crop]{standalone}
\usepackage{tikz}
\usepackage[]{tikz-3dplot}
\usepackage{pgfplots}
\pgfplotsset{compat=1.15}
\usepackage[]{amsmath}
\usepackage[]{libertine}
\usepackage[libertine]{newtxmath}
\usepackage[]{bm}
\usepackage[]{physics}
% Macros for greek letters in roman style, in math mode
\DeclareRobustCommand{\mathup}[1]{%
\begingroup\ensuremath\changegreek\mathrm{#1}\endgroup}
\DeclareRobustCommand{\mathbfup}[1]{%
\begingroup\ensuremath\changegreek\bm{\mathrm{#1}}\endgroup}


\makeatletter
\def\changegreek{\@for\next:={%
        alpha,beta,gamma,delta,epsilon,zeta,eta,theta,iota,kappa,lambda,mu,nu,%
        xi,pi,rho,sigma,tau,upsilon,phi,chi,psi,omega,varepsilon,varpi,%
    varrho,varsigma,varphi}%
\do{\expandafter\let\csname\next\expandafter\endcsname\csname\next up\endcsname}}
\makeatother

% Define vectors in bold, roman, lowercase font
\newcommand{\vct}[1]{\ensuremath{\mathbfup{\MakeLowercase{#1}}}}

% Define unit vectors in bold, roman, lowercase font, with hats
\newcommand{\uvct}[1]{\ensuremath{\mathbfup{\hat{\MakeLowercase{#1}}}}}

% Define matrices in bold, roman, uppercase font
\newcommand{\mtrx}[1]{\ensuremath{\mathbfup{\MakeUppercase{#1}}}}
\usetikzlibrary{%
    angles,%
    arrows.meta,%
    backgrounds,%
    calc,%
    decorations,%
    fit,%
    hobby,%
    patterns,%
    positioning,%
    quotes
}


\begin{document}

\tdplotsetmaincoords{60}{120}

\begin{tikzpicture}[tdplot_main_coords]
    \pgfmathsetmacro{\radius}{2.5}
    \pgfmathsetmacro{\size}{2.5}
    % Shade initial tangent plane
    \draw[fill=gray!25,draw opacity=0] ($({-1.2*\radius},{-1.2*\radius},0)$) -- ($({1.2*\radius},{-1.2*\radius},0)$) -- ($({1.2*\radius},{1.2*\radius},0)$) -- ($({-1.2*\radius},{1.2*\radius},0)$) -- cycle;
    % Draw unit normal vector to surface
    \draw[->,very thick,color=black!90] (0,0,0) -- (0,0,1.7) node[anchor=south]{$\vct{\xi}_{3}(\vct{x}_{0})$};
    % Suggest initial radius
    \draw[stroke=black!90,thin,dashed] (0,0,0) to ($({\radius*cos(-110)},{\radius*sin(-110)},0)$) coordinate (p);
    % Place the set of initial points
    \foreach [count = \i] \ang in {0,40,...,320}%
    {%
        \coordinate (\i) at ( {\radius*cos(\ang-90)}, {\radius*sin(\ang-90)}, {0} );
    }
    % Draw interpolation curve
    \draw[stroke = black!80,thin,dotted] (1) to [curve through = {(2) .. (3) .. (4) .. (5) .. (6) .. (7) .. (8) .. (9)}] (1);
    % Draw point given as initial condition
    \draw[fill=gray!5,stroke=black!65,fill opacity=1] (0,0,0) circle (\size pt) coordinate (0);
    % Draw points in initial level set
    \foreach \i in {1,...,9}%
    {%
        \draw[fill=gray!10,stroke=black!65] (\i) circle (\size pt);
    }
    % Label initial condition
    \node [below right] at (0) {$\vct{x}_{0}$};
    % Label nodes
    \node [below right] at (4) {$\mathcal{M}_{1,1}$};
    \node [below right] at (5) {$\mathcal{M}_{1,2}$};
    \node [right] at (6) {$\mathcal{M}_{1,3}$};
    \node[below left = 0pt and -2pt] at (3) {$\mathcal{M}_{1,n}$};
    \node[below left] at (2) {$\mathcal{M}_{1,n-1}$};
    % Add pseudoarclength interpolation parameter labelling
    \node[above right = 9pt and -14pt] at (4) {\footnotesize$s_{1}=0$,};
    \node[above = 2pt] at (4) {\footnotesize$s=1$};
    \node[above left] at (5) {\footnotesize$s_{2}$};
    \node[above left = -2pt and 1pt] at (6) {\footnotesize$s_{3}$};
    \node[above right = 0pt and -2pt] at (3) {\footnotesize$s_{n}$};
    \node[above right = -5pt and 2pt] at (2) {\footnotesize$s_{n-1}$};
    \node[below left,rotate=-30] at ($(0)!0.3!(p)$) {$\delta_{\text{init}}$};

    % Add coordinate on interpolated circle, in order to denote it properly
    \coordinate (q) at ($({\radius*cos(-190)},{\radius*sin(-190)},0)$);
    % Add pivot coordinate on which to place interpolation circle label
    \coordinate (r) at ($({1.4*\radius*cos(-190)},{1.4*\radius*sin(-190)},0)$);
    \node at (r) {$\mathcal{C}_{1}$};
    \draw[stroke=black!80] ($(q)!0.1!(r)$) to [out = 20,in=36] ($(q)!0.65!(r)$);
\end{tikzpicture}
\end{document}
