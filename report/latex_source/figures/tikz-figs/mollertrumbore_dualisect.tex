\documentclass[crop]{standalone}
\usepackage{tikz}
\usepackage[]{tikz-3dplot}
\usepackage{pgfplots}
\pgfplotsset{compat=1.15}
\usepackage[]{amsmath}
\usepackage[]{libertine}
\usepackage[libertine]{newtxmath}
\usepackage[]{bm}
\usepackage[]{physics}
% Macros for greek letters in roman style, in math mode
\DeclareRobustCommand{\mathup}[1]{%
\begingroup\ensuremath\changegreek\mathrm{#1}\endgroup}
\DeclareRobustCommand{\mathbfup}[1]{%
\begingroup\ensuremath\changegreek\bm{\mathrm{#1}}\endgroup}


\makeatletter
\def\changegreek{\@for\next:={%
        alpha,beta,gamma,delta,epsilon,zeta,eta,theta,iota,kappa,lambda,mu,nu,%
        xi,pi,rho,sigma,tau,upsilon,phi,chi,psi,omega,varepsilon,varpi,%
    varrho,varsigma,varphi}%
\do{\expandafter\let\csname\next\expandafter\endcsname\csname\next up\endcsname}}
\makeatother

% Define vectors in bold, roman, lowercase font
\newcommand{\vct}[1]{\ensuremath{\mathbfup{\MakeLowercase{#1}}}}

% Define unit vectors in bold, roman, lowercase font, with hats
\newcommand{\uvct}[1]{\ensuremath{\mathbfup{\hat{\MakeLowercase{#1}}}}}

% Define matrices in bold, roman, uppercase font
\newcommand{\mtrx}[1]{\ensuremath{\mathbfup{\MakeUppercase{#1}}}}
\usetikzlibrary{%
    shapes,%
    arrows,%
    patterns,%
    positioning,%
    calc%
}




\tdplotsetmaincoords{60}{90}
\tikzset{
    block/.style = {draw, rectangle, stroke = black!80,fill = gray!15, minimum height = 3em, minimum width = 6em}
    arr/.style = {single arrow, draw, ->}
}
\begin{document}
\begin{tikzpicture}[tdplot_main_coords]
    % Define vertices of A:
    \coordinate (a1) at (1,0,0);
    \coordinate (a2) at (3,0,0);
    \coordinate (a3) at (2,2,0);
    % Define vertices of B:
    \coordinate (b1) at (1.5,1,0.5);
    \coordinate (b2) at (2.5,1,0.5);
    \coordinate (b3) at (2,2,-1);

    % Define intersection points
    \coordinate(i1) at (5/3,4/3,0);
    \coordinate(i2) at (7/3,4/3,0);


    % Define help vertices for filling the triangles in a more aesthetically
    % pleasing fashion
    \coordinate (outerb) at (143/64,73/48,0);
    \coordinate (outera) at (63/42,1,0);
    \coordinate (innera) at (137/84,1,0);

    % Shade triangle B
    \draw[pattern = north west lines, pattern color = gray!40, stroke = black!80] (i1) -- (b1) -- (b2) -- (i2);
    \draw[pattern = north west lines, pattern color = gray!40, stroke = black!80] (i2) -- (b3) -- (outerb);
    % Shade triangle A
    \draw[fill=gray!30,stroke=black!80] (a1) -- (a2) -- (i2) -- (innera) -- (outera) -- cycle;
    \draw[fill=gray!30,draw opacity = 0] (i1) -- (i2) -- (a3) -- cycle;
    \draw[stroke=black!80] (i1) -- (a3) -- (i2);
    % Draw vertex segments which are partly obscured
    \draw[stroke=black!80, densely dotted] (i1) -- (outera);
    \draw[stroke=black!80, densely dotted] (i1) -- (outerb);

    % Draw intersection points
    \draw[fill=gray!10,stroke=black!80]  (i1) circle (1pt);
    \draw[fill=gray!10,stroke=black!80]  (i2) circle (1pt);
\end{tikzpicture}
\end{document}
