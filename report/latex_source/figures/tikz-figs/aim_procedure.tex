\documentclass[crop]{standalone}
\usepackage{tikz}
\usepackage[]{tikz-3dplot}
\usepackage{pgfplots}
\usepackage[]{amsmath}
\usepackage[]{libertine}
\usepackage[libertine]{newtxmath}
\usepackage[]{bm}
\usepackage[]{physics}
% Macros for greek letters in roman style, in math mode
\DeclareRobustCommand{\mathup}[1]{%
\begingroup\ensuremath\changegreek\mathrm{#1}\endgroup}
\DeclareRobustCommand{\mathbfup}[1]{%
\begingroup\ensuremath\changegreek\bm{\mathrm{#1}}\endgroup}

\makeatletter
\def\changegreek{\@for\next:={%
        alpha,beta,gamma,delta,epsilon,zeta,eta,theta,iota,kappa,lambda,mu,nu,%
        xi,pi,rho,sigma,tau,upsilon,phi,chi,psi,omega,varepsilon,varpi,%
    varrho,varsigma,varphi}%
\do{\expandafter\let\csname\next\expandafter\endcsname\csname\next up\endcsname}}
\makeatother

% Define vectors in bold, roman, lowercase font
\newcommand{\vct}[1]{\ensuremath{\mathbfup{\MakeLowercase{#1}}}}

% Define unit vectors in bold, roman, lowercase font, with hats
\newcommand{\uvct}[1]{\ensuremath{\mathbfup{\hat{\MakeLowercase{#1}}}}}

% Define matrices in bold, roman, uppercase font
\newcommand{\mtrx}[1]{\ensuremath{\mathbfup{\MakeUppercase{#1}}}}

\usetikzlibrary{%
    angles,%
    arrows.meta,%
    backgrounds,%
    calc,%
    decorations,%
    fit,%
    hobby,%
    patterns,%
    positioning,%
    quotes
}
\makeatletter
\tikzset{
  fitting node/.style={
    inner sep=0pt,
    fill=none,
    draw=none,
    reset transform,
    fit={(\pgf@pathminx,\pgf@pathminy) (\pgf@pathmaxx,\pgf@pathmaxy)}
  },
  reset transform/.code={\pgftransformreset}
}
\makeatother
% A simple empty decoration, that is used to ignore the last bit of the path
\pgfdeclaredecoration{ignore}{final}
{
\state{final}{}
}

% Declare the actual decoration.
\pgfdeclaremetadecoration{middle}{initial}{
    \state{initial}[
        width={0pt},
        next state=middle
    ]
    {\decoration{moveto}}

    \state{middle}[
        width={\pgfdecorationsegmentlength*\pgfmetadecoratedpathlength},
        next state=final
    ]
    {\decoration{curveto}}

    \state{final}
    {\decoration{ignore}}
}


% Create a key for easy access to the decoration
\tikzset{middle segment/.style={decoration={middle},decorate, segment length=#1}}

\def\getangle(#1)(#2)#3{%
  \begingroup%
    \pgftransformreset%
    \pgfmathanglebetweenpoints{\pgfpointanchor{#1}{center}}{\pgfpointanchor{#2}{center}}%
    \expandafter\xdef\csname angle#3\endcsname{\pgfmathresult}%
  \endgroup%
}


\begin{document}


\tdplotsetmaincoords{60}{20}

\begin{tikzpicture}[tdplot_main_coords]
    % Macro for the unit vector scale (i.e., the length of the unit vectors)
    \pgfmathsetmacro{\usclx}{2.15};
    \pgfmathsetmacro{\uscly}{2.35};
    \pgfmathsetmacro{\usclz}{1.55};

    % Macro for the axes parallel to the unit vectors
    \pgfmathsetmacro{\asclx}{7}
    \pgfmathsetmacro{\ascly}{7}
    \pgfmathsetmacro{\asclz}{3}

    % Macro for the vector separating origin point and aim point
    \pgfmathsetmacro{\x}{0.78*\asclx}
    \pgfmathsetmacro{\y}{0.64*\ascly}
    \pgfmathsetmacro{\z}{0.78*\asclz}

    % Set coordinates for origin point
    \coordinate (r) at (0,0,0);

    % Set coordinates for aiming point
    \coordinate (ra) at ($(r) + (\x,\y,\z)$);

    % Set coordinates for end points of unit vectors
    \coordinate (xi1) at ($(r) + (\usclx,0,0)$);
    \coordinate (xi2) at ($(r) + (0,\uscly,0)$);
    \coordinate (xi3) at ($(r) + (0,0,\usclz)$);

    % Set coordinates of orthogonal projection of aiming vector
    \coordinate (rort) at ($(r) + (0,0,\z)$);

    % Set coordinates of parallel projection of aiming vector
    \coordinate (rpar) at ($(r) + (\x,\y,0)$);

    % Shade plane spanned by xi1 and xi2
    \draw[fill=gray!15,draw opacity = 0] (r) -- ($(r)+(\asclx,0,0)$) -- ($(r)+(\asclx,\ascly,0)$) -- ($(r)+(0,\ascly,0)$) -- cycle;

    % Draw axes parallel to unit vectors
    \draw[color=gray!80] (r) -- ($(r) + (\asclx,0,0)$); % xi1
    \draw[color=gray!80] (r) -- ($(r) + (0,\ascly,0)$); % xi2
    \draw[color=gray!80] (r) -- ($(r) + (0,0,\asclz)$); % xi3

    % Draw unit vectors
	\getangle(r)(xi1)b;
    \draw[->,very thick,color=black!90] (r) -- (xi1) node[midway,below,rotate=\angleb]{$\vct{\xi}_{1}(\vct{r})$};
	\getangle(r)(xi2)b;
    \draw[->,very thick,color=black!90] (r) -- (xi2) node[above right=-0.25cm and -0.25cm,rotate=\angleb] {$\vct{\xi}_{2}(\vct{r})$};
    \draw[->,very thick,color=black!90] (r) -- (xi3) node[midway,left]{$\vct{\xi}_{3}(\vct{r})$};

    % Draw guide lines to orthogonal component of aim vector
    \draw[dashed,stroke=black!65] (ra) -- (rort);

    % Draw guide lines to parallel component of aim vector
    \draw[dashed,stroke=black!65] (ra) -- (rpar);


    % Draw parallel component of aim vector
	\getangle(r)(rpar)b;
    \draw[thick, stroke=black!90,dashed] (r) -- (rpar);
    \draw[->,stroke=black!90,thick,middle segment = 0.3] (r) -- (rpar) node[midway,below left,rotate=\angleb]{$\vct{f}(\vct{r},\vct{r}_{\mathrm{aim}})$};

    % Draw aim point
    \draw[fill=gray!10,stroke=black!90] (ra) circle(2.5pt) node[right]{$\vct{r}_{\mathrm{aim}}$};
    % Draw nonmodified aim vector
    \draw[->,stroke=black!90,thick,densely dashed,middle segment = 1] (r) -- (ra);

    % Draw origin point
    \draw[fill=gray!10,stroke=black!90,fill opacity=1] (r) circle(2.5pt) node[below left = 1pt and 1pt] {$\vct{r}$};


\end{tikzpicture}
\end{document}
