Our method for computing repelling LCSs in three-dimensional flow systems
seems to yield reasonable results. Analytically constructed test cases indicate
that our method of generating three-dimensional surfaces, in addition to
extracting repelling LCSs as subsets of computed manifolds which satisfy all
the necessary and sufficient existence criteria, work as intended. Moreover,
the robustness of the LCSs obtained in the two variations of the
Arnold-Beltrami-Childress flow considered here, suggests that the computed LCSs
are not particularly sensitive to imperfect model data --- which is a
characteristic property of \emph{Lagrangian} transport barriers. Accordingly,
our computed LCSs for flow in the Førde fjord likely form reasonable
approximations of the actual LCSs contained within.

There is certainly room for further resarch with regards to the numerical
implementation of one of the LCS existence conditions derived from their
variational theory; in particular, that pertaining to the identification of
locally most repelling material surfaces (the other conditions are quite
unambiguous, in comparison). To our knowledge, a general, robust numerical
routine is yet to be described in the literature. For instance, in a recent
conducted by \textcite{oettinger2016autonomous}, who set out to compute
quasi-three-dimensional hyperbolic LCSs, the authors seemingly did not attempt
to implement the aforementioned criteria numerically, appearing to be satisfied
with reasonable suggestions as to where such LCSs \emph{might} be located. Our
approach, based on identifying which of the individual points constituting each
manifold was locally repelling, involved one parameter which was determined by
use of the  initial grid spacing, and another which filtered away the smallest
LCS  surfaces --- as these were assumed not to impact the overall circulation
significantly (originally suggested by \textcite{farazmand2012computing}).
A suggested alternative approach, which was not investigated as part of this
project, would be to utilize a sort of numerical clustering algorithm to
extract LCSs as the most repelling material surfaces in small neighborhoods,
where each of the considered sets of surfaces ideally would be of similar size
and orientation.

Seeing as our method of computing fully three-dimensional LCSs is significantly
more complex and consumes more computational resources than well established
routines for generating their two-dimensional counterparts (see e.g.\
\textcite{onu2015lcstool}), strong arguments regarding the additional insight
from three-dimensional anlysis are needed in order to substantiate a change in
practices. Moreover, computing fully three-dimensional LCSs might not be
necessary, depending on the type of transport phenomenon under consideration.
For instance, regarding the spread of debris and contaminations such as garbage
patches or oil spill remnants across the ocean surface, the underlying
transport system can reasonably be considered two-dimensional. Meanwhile, for
circulation in rivers or fjords, where the depth is of the same scale as the
width, fully three-dimensional analysis might be more suitable. Simply put, for
systems in which all three dimensions are of similar relevance --- i.e.,\
systems which cannot reasonably be regarded as two-dimensional --- computing
fully three-dimensional LCS surfaces might be prudent; in contrast to the
quasi-three-dimensional approach of \textcite{oettinger2016autonomous} which
loses out on the minute details of the extra dimension, and does not yield
coherent surfaces favorable for further inquiries.

\subsubsection{Suggestions for further work}
\label{ssub:suggestions_for_further_work}

Experimental methods for the verification of Lagrangian flow analysis have
gained traction lately. Investigations are currently being conducted as to the
usefulness of LCSs as predictors for diffusive flow patterns, for which
analyzing tracer trajectories of the overall circulation might be insufficient
\parencite{haller2018material}. Moreover, recent pilot studies indicate that
Lagrangian analysis yields valid predictions for transport by oceanic currents
\parencite{filippi2018detection} --- including the fully three-dimensional
circulation patterns arising in off-shore currents
\parencite{peacock2018targeted}. Accordingly, empirical studies pertaining to
the validity of the LCSs computed by use of our variation of the geodesic level
set method lays within the realm of feasibility.

In addition to the aforementioned issue of implementing the LCS existence
criteria which identifies LCSs as surfaces which are locally most repelling,
further research aiming to enhance our approach for computing repelling LCSs
could result in increased transparency and efficiency. Specifically,
abandoning the strict ordering of mesh points in level sets forming topological
circles could allow for an increase in the resolution of LCS behaviour near
domain boundaries, and potentially yield more accurate surface approximations
overall. Then again, other options for computing invariant manifolds of
three-dimensional vector fields exist. Aside from the method of geodesic level
sets, \textcite{krauskopf2005survey} present four others; investigating each of
them in the context of computing hyperbolic LCSs could provide valuable
insights. Moreover, another method less demanding in terms of resource
consumption than the present approach would render the computation of
three-dimensional LCSs a feasible task without the need of supercomputers.
Courtesy of facilitating three-dimensional Lagrangian analysis for a wider
audience, this could, in turn, accelerate the further development of LCS
computing tools as a whole.

Lastly, to our knowledge, little research has been conducted regarding the
choice of interpolation and integration methods in order to simulate transport
phenomena in real-world systems. High order adaptive step size integration
methods are generally used in conjunction with interpolated velocity fields,
seemingly with little awareness as to whether or not the use of higher-order
methods are warranted. For instance, \textcite{vansebille2018lagrangian}
provide a rigorous discussion of the currently available tools for computer
simulated tracer advection, but lacks a detailed treatment of integration
methods. The significance of the choice of interpolation scheme is generally
recognized, though left largely unexplored. Whereas
\textcite{lekien2005tricubic} developed a (locally) tricubic interpolation
method intended for use in simulating three-dimensional flow (and computing
three-dimensional LCSs) --- motivated by the observation that linear
interpolation is largely insufficient in terms of yielding smooth velocity
fields from model data --- the authors provide no arguments for preferring
cubic methods over even higher order alternatives. A third example is the
article by \textcite{gough2017persistent}, in which the authors report having
used a high order adaptive step size Runge-Kutta solver together with a cubic
interpolation routine --- much like what was done for this project --- in order
to investigate oceanic transport patterns in the northwestern Gulf of Mexico;
yet fail to motivate their choices of integration and interpolation methods.
Based on the above, investigations with regards to to the interaction between
integration and interpolation schemes in the context of computing LCSs are
appealing, due to their innate relation to application.

