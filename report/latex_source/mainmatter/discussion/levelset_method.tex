\section{Comments on the method of geodesic level sets}
\label{sec:comments_on_the_method_of_geodesic_level_sets}
%
As mentioned in \cref{cha:method}, our take on the method of geodesic level
sets (see \cref{sec:revised_approach_to_computing_new_mesh_points}) hinges on
characteristic properties of hyperbolic LCSs (cf.\ \cref{def:hyperbolic_lcs}).
For \emph{repelling} LCSs (see \cref{def:repelling_lcs}), which we concentrated
on for this project, the existence criterion given in \cref{eq:lcs_condition_c}
states that these are everywhere \emph{orthogonal} to the local direction of
strongest repulsion. The extra degree of freedom compared to manifolds defined
as being everywhere tangent to some direction field (such as the strange
attractor in the Lorenz system, as was considered by
\textcite{krauskopf2005survey}) facilitated a more effective (in terms of
computational runtime) and conceptually simpler method of generating such
manifolds, than the more direct adaption of \posscite{krauskopf2005survey}
method (which is outlined in
\cref{sec:legacy_approach_to_computing_new_mesh_points}).
%Note, however, that
%\citeauthor{krauskopf2005survey}'s method retains a greater degree of general
%applicability.

Although useful for managing the mesh accuracy (see
\cref{sec:managing_mesh_accuracy}), and central to our
triangulation algorithm (outlined in detail in \cref{sec:continuously%
_reconstructing_three_dimensional_manifold_surfaces_from_point_meshes}),
exclusively arranging mesh points in closed topological circles has irrefutable
weaknesses. In systems for which periodic boundary conditions are not
applicable, the addition of further level sets is promptly terminated when one
or more of the trajectories used to compute new mesh points (see
\cref{sec:revised_approach_to_computing_new_mesh_points}) exits the
computational domain. This impedes the ability to resolve LCS behaviour near
the domain boundaries, which is evident from inspecting the LCSs in the Førde
fjord (presented in \cref{sec:computed_lcss_in_the_forde_fjord}). Although this
issue could be managed by computing strain eigenvalues and -vectors in a domain
\emph{containing} the domain of interest and expanding from it in all
directions, this workaround is computationally demanding. Depending on to what
extent the underlying flow system is known (or modelled), and the location of
the domain of interest, it might not even be possible.

On a related note, demanding that a new geodesic level set is computed using
a mesh point descending from \emph{each} of the mesh points in the preceding
level set renders computing the underlying manifold in its entirety from a
\emph{single} focal point $\vct{x}_{0}$ (see \cref{sec:preliminaries_for%
_computing_repelling_lcss_in_3d_flow_by_means_of_geodesic_level_sets}) quite
difficult. If a single point strand (that is, the set of mesh points which can
be traced back to a single, common ancestor) is terminated --- either due to
reaching the domain edges, or failure to compute a new mesh point (see
\cref{sub:handling_failures_to_compute_satisfactory_mesh_points_revised}) ---
so too is the addition of further level sets. Thus, unless the manifold as a
whole expands as a perfect topological surface area, as seen from the focal
point $\vct{x}_{0}$, encapsulating it in its entirety by means of geodesic
level sets becomes impossible --- even when not accounting for the possibility
of numerical error.

As mentioned in \cref{sec:revised_approach_to_computing_new_mesh_points}, we
let each computed mesh point \emph{inherit} its unit tangent $\vct{t}$ from
its direct ancestor, rather than computing new unit tangents using the
interpolation curve $\mathcal{C}_{i}$. This was a conscious choice. In
our experience, the computed level sets quickly started to form fully
three-dimensional topological circles, leading to notable local curvature along
the interpolation curves. This rendered selecting how far to either side of a
given mesh point to move, in order to approximate the local tangent vector by a
finite coordinate difference, hard to do in a consistent manner. Simply using
the coordinates of the mesh point's nearest neighbors (as originaly suggested
by \textcite{krauskopf2005survey}) for this purpose was found to be
inconsistent. Failure to compute tangent vectors consistently often lead
neighboring point strands to form intersections, yielding disorderly meshes
which in turn frequently lead the points constituting a level set to
\emph{not} form a topological circle. These issues are not present in our
aforementioned inheritance-based approach. However, if a computed manifold were
to twist itself in such a way that a unit tangent $\vct{t}$ laid
\emph{within} it, this could result in failure to compute one or more new mesh
points, which would then lead to terminating the process of adding new level
sets (alternatively, terminating the process of expanding one or more point
strands, cf.\ the preceding paragraphs) prematurely. We addressed this issue
using the tolerance parameter $\gamma_{\|}$; see
\cref{sec:revised_approach_to_computing_new_mesh_points} and, in particular,
\cref{eq:revised_xi3_tan_parallel} for details.

Our variation of the method of geodesic level sets contains many
degrees of freedom (see e.g.\
\cref{tab:initialconditionparams,tab:abc_manifold_params}). Some of these
parameters, mainly those governing the minimum and maximum allowed separations
between neighboring mesh points (see
\cref{sub:maintaining_mesh_point_density}), could reasonably be chosen based on
considerations pertaining to the spatial extent of the computational domain;
alternatively, to what extent the minute details of the LCSs are to be
resolved. How to determine several other undeniably key parameter values ---
such as the tolerances  for the detection of intersecting manifolds (see
\cref{sec:macroscale_stopping_criteria_for_the_expansion_of_computed%
_manifolds}), and the removal of mesh points which form undesired bulges (which
is outlined in \cref{sub:limiting_the_accumulation_of_numerical_noise}) ---
remains less obvious.

That being said, the parameters related to the curvature-guided approach to
dynamically adjust the interset separations were, in our experience, of less
importance; as briefly mentioned in \cref{sub:a_curvature_based_approach_to%
_determining_interset_separations}, the interset step length was rarely
\emph{increased}. More often than not, the interset step length was quickly
reduced to its lower limit, and remained at that level for the generation of
all subsequent level sets. This is not entirely unexpected, as sufficiently
large curvature within a \emph{single} region of any given level set sufficed
to lower the step length (compare \cref{eq:increase_dist,eq:decrease_dist}).
Moreover, as the geodesic level sets continuously expand, encountering such a
region becomes increasingly likely. As the accuracy of the computed mesh points
is independent of the density of mesh points, the main use of the interset step
size is to manage the interpolation error inherent to our linear triangulation
scheme (see \cref{sec:continuously_reconstructing_three_dimensional_manifold%
_surfaces_from_point_meshes} and \textcite{krauskopf2003computing}). Thus, it
seems reasonable to forego the dynamic interset step length in favor of a
fixed one, in cases where the minute details of LCS surfaces are unimportant.
As tentatively suggested in the above, doing so reduces the overall complexity
of our method for generating mesh points, in addition to reducing the number
of free parameters.

Another way of organizing the mesh points, which could circumvent some of the
aforementioned limitations of the present approach, would be as a group of
point strands, each associated with a particular unit tangent $\vct{t}$ ---
determined using $\mathcal{C}_{1}$, the interpolation curve of the innermost
level set, in similar fashion to that described in
\cref{sub:parametrizing_the_innermost_level_set}. Treating the expansion along
each point strand independently would then permit further expansion of a
manifold even when one or more point strands would reach the domain boundaries;
this would also solve the possible issue of computed trajectories along any
given strand failing to yield acceptable mesh points (see
\cref{sub:handling_failures_to_compute_satisfactory_mesh_points_revised}) ---
in which case only the strands in question need to be terminated, rather than
prohibiting the addition of further geodesic level sets entirely.

Aside from taking a step further away from the method of geodesic level set as
originally proposed by \textcite{krauskopf2005survey}, organizing mesh
points as a group of point strands would necessitate developing the mesh
accuracy management method outlined in \cref{sec:managing_mesh_accuracy}
further. In particular, the present approach utilizes the interpolation curve
$\mathcal{C}_{i}$ in order to insert mesh points inbetween nearest neighbor
mesh points in level set $\mathcal{M}_{i+1}$ which are deemed to lie too far
away from eachother (see \cref{sub:maintaining_mesh_point_density}). A
reasonable approach for the case of point strands could be to, having
identified two strands inbetween which a new mesh point is needed, retrieve
their respective ancestor points in the innermost level set, and then compute
(the trajectory of a) new point strand starting out at a point on
$\mathcal{C}_{1}$, midway inbetween said ancestors, keeping only the first
point along the strand which is required to maintain the point density.

Depending on the extent of the initial level set --- that is, the circumference
of its interpolation curve $\mathcal{C}_{1}$ --- and the required mesh point
density, however, this approach could be prone to errors arising from numerical
round-off errors in computing the start points of new point strands. In order
to maintain an overall mesh structure suitable for triangulation purposes, the
steps along each point strand should be equal; that is, mesh points $i$ and
$i+1$ along all point strands should be separated by the same distance
$\Delta_{i}$, whereas all interpoint step sizes $\{\Delta_{i}\}$ need
not be equal. This way, a quasi-circular structure is maintained as the
point strands expand --- that is, subject to one or more of them reaching the
domain boundaries or being terminated due to ending up in (approximately)
closed orbits (see \cref{sub:computing_pseudoradial_trajectories_directly}).

Lastly, several other methods of computing invariant manifolds of vector fields
exist; of whom some might be well-suited in the context of LCSs. For instance,
the method of geodesic level sets is one of five methods presented by
\textcite{krauskopf2005survey}; none of the others were pursued as part of this
work. Exploring the strengths and weaknesses of other
approaches to computing three-dimensional hyperbolic LCSs remains beyond the
scope of this project.
