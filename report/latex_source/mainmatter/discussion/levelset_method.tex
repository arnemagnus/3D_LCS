\section{Comments on the method of geodesic level sets}
\label{sec:comments_on_the_method_of_geodesic_level_sets}

As mentioned in \cref{cha:method}, our take on the method of geodesic level
sets (see \cref{sec:revised_approach_to_computing_new_mesh_points}) hinges on
characteristic properties of hyperbolic LCSs (cf.\ \cref{def:hyperbolic_lcs}).
For \emph{repelling} LCSs (see \cref{def:repelling_lcs}), which we concentrated
on for this project, existence criterion \cref{eq:lcs_condition_c} states that
these are everywhere \emph{orthogonal} to the local direction of strongest
repulsion. The extra degree of freedom compared to manifolds defined as being
everywhere tangent to some direction field (such as the strange attractor in
the Lorenz system, as was considered by \textcite{krauskopf2005survey})
facilitated a more effective (in terms of computational runtime) and
conceptually simpler method of generating them, than the more direct adaption
of \posscite{krauskopf2005survey} method (which is outlined in
\cref{sec:legacy_approach_to_computing_new_mesh_points}). Note, however, that
\citeauthor{krauskopf2005survey}'s method retains a greater degree of
general applicability.

Although useful for managing the mesh accuracy (see
\cref{sec:managing_mesh_accuracy}), and central to our
triangulation algorithm (outlined in detail in \cref{sec:continuously%
_reconstructing_three_dimensional_manifold_surfaces_from_point_meshes}),
exclusively arranging mesh points in closed topological circles has irrefutable
weaknesses. In systems for which periodic boundary conditions are not
applicable, the addition of further level sets is promptly terminated when one
or more of the trajectories used to compute new mesh points (see
\cref{sec:revised_approach_to_computing_new_mesh_points}) exits the
computational domain. This impedes the ability to resolve LCS behaviour near
the domain boundaries, which is evident from inspecting the LCSs in the Førde
fjord (presented in \cref{sec:computed_lcss_in_the_forde_fjord}). Although this
issue could be managed by computing strain eigenvalues and -vectors in a domain
\emph{containing} the domain of interest and expanding from it in all
directions, this workaround is computationally demanding. Depending on to what
extent the underlying flow system is known (or modelled), and the location of
the domain of interest, it might not even be possible.

On a related note, demanding that a new geodesic level set is computed using
a mesh point descending from \emph{each} of the mesh points in the preceding
level set renders computing the underlying manifold in its entirety from a
\emph{single} focal point $\vct{x}_{0}$ (see \cref{sec:preliminaries_for%
_computing_repelling_lcss_in_3d_flow_by_means_of_geodesic_level_sets}) quite
difficult. If a single point strand (that is, the set of mesh points which can
be traced back to a single, common ancestor) is terminated --- either due to
reaching the domain edges, or failure to compute a new mesh point (see
\cref{sub:handling_failures_to_compute_satisfactory_mesh_points_revised}), so
too is the addition of further level sets. Thus, unless the manifold as a whole
expands as a perfect topological surface area, as seen from the focal point
$\vct{x}_{0}$, encapsulating it in its entirety by means of geodesic level sets
becomes impossible --- even when not accounting for the possibility of
numerical error.

Moreover, our variation of the method of geodesic level sets contains many
degrees of freedom (see e.g.\
\cref{tab:initialconditionparams,tab:abc_manifold_params}). Some of these
parameters, mainly those governing the minimum and maximum allowed separations
between neighboring mesh points, could reasonably be chosen based on
considerations pertaining to the spatial extent of the computational domain;
alternatively, to what extent the minute details of the LCSs are to be
resolved. How to determine several other undeniably key parameter values ---
such as the tolerances  for the detection of intersecting manifolds (see
\cref{sec:macroscale_stopping_criteria_for_the_expansion_of_computed%
_manifolds}), and the removal of mesh points which form undesired bulges (which
is outlined in \cref{sub:limiting_the_accumulation_of_numerical_noise}) ---
remains less obvious.

That being said, the parameters related to the curvature-guided approach to
dynamically adjust the interset separations were, in our experience, of less
importance; as briefly mentioned in \cref{sec:a_curvature_based_approach_to%
_determining_interset_separations}, the interset step length was rarely
\emph{increased}. More often than not, the interset step length was quickly
reduced to its lower limit, and remained at that level for the generation of
all subsequent level sets. This is not entirely unexpected, as sufficiently
large curvature within a \emph{single} region of any given level set sufficed
to lower the step length (compare \cref{eq:increase_dist,eq:decrease_dist}).
Moreover, as the geodesic level sets continuously expand, encoutering such a
region becomes increasingly likely. As the accuracy of the computed mesh points
is independent of the density of mesh points, the main use of the interset step
size is to manage the interpolation error inherent to our linear triangulation
scheme (see \cref{sec:continuously_reconstructing_three_dimensional_manifold%
_surfaces_from_point_meshes}) \parencite{krauskopf2003computing}. Thus, it
seems reasonable to forego the dynamic interset step length in favor of a
fixed one, where the minute details of LCS surfaces are unimportant. As
tentatively suggested in the above, doing so reduces the overall complexity of
our method for generating mesh points, in addition to reducing the number of
free parameters.


\begin{framed}
    \begin{itemize}
        \item Svakheter:
            \item En videre revisjon, hvor en går lengre vekk fra ideen om å
                organisere punkter i topologiske sirkler til fordel for
                \emph{point strains} kan være hensiktsmessig --- spesielt hva
                gjelder beregning av en mangfoldighet som ikke har uniform
                utstrekning i alle retninger, sett fra arnesteded
                \begin{itemize}
                    \item Kan medføre innfløkt logikk hva gjelder å
                        opprettholde meshpunkttetthet
                \end{itemize}
                \item Alternative fremgangsmåter finnes. Blant annet beskriver
                    \textcite{krauskopf2005survey} X andre metoder for å
                    beregne invariante mangfoldigheter av tredimensjonale
                    vektorfelter. Ingen av disse ble undersøkt som en del av
                    dette prosjektet. \emph{Forslag til videre arbeid}.
    \end{itemize}
\end{framed}

