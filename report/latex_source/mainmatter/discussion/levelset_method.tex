\section{Comments on the method of geodesic level sets}
\label{sec:comments_on_the_method_of_geodesic_level_sets}




\begin{framed}
    \begin{itemize}
        \item Vår tilpasning av metoden utnytter underliggende egenskaper ved
            frastøtende LCS-flater. Dermed ikke universelt anvendbar.
        \item Kort resym{\'e} av fordelene ved vår metode
            \begin{itemize}
                \item Konseptuelt enklere
                \item Raskere
            \end{itemize}
        \item Svakheter:
            \begin{itemize}
                \item Å kreve at meshpunktene utgjør lukkede topologiske
                    sirkler vanskeliggjør oppløsning av oppførsel nær
                    domenekanter (for ikke-periodiske grensebetingelser, jf.\
                    LCSene i Førdefjorden, som presentert i
                    \cref{sec:computed_lcss_in_the_forde_fjord})
                \item Å kreve at alle nye meshpunkt skal kunne beregnes fra
                    hver forfader i foregående levelset gjør det vanskelig å
                    beregne `hele' den invariante mangfoldigheten ut fra hvert
                    arnepunkt. Mer spesifikt er dette umulig med mindre
                    mangfoldigheten strekker steg utover som en perfekt
                    sirkulær overflate.
                \item Mange frihetsgrader (se
                    \cref{tab:initialconditionparams,%
                    tab:fjord_manifold_params}), hvorav kun noen kan velges fra
                    direkte betraktninger av domenets romlige utstrekning.
                    Blant annet utgjør intersection-lengde og
                    loop-fjernings-toleranse nøkkelparametre som det ikke
                    nødvendigvis er åpenbart hvordan en best bør velge.
                    \begin{itemize}
                        \item Parameterne tilhørende den krumningsbaserte
                            metoden for å velge neste interset-distanse kan
                            sløyfes; vår erfaring er at disse aldri økte
                            distansen, snarere reduserte den så langt som mulig
                            i løpet av ikke alt for mange levelset.

                            Ikke helt uventet, fordi det holder med mye
                            krumning \emph{ett} sted for å redusere
                            steglengden.
                    \end{itemize}
            \end{itemize}
            \item En videre revisjon, hvor en går lengre vekk fra ideen om å
                organisere punkter i topologiske sirkler til fordel for
                \emph{point strains} kan være hensiktsmessig --- spesielt hva
                gjelder beregning av en mangfoldighet som ikke har uniform
                utstrekning i alle retninger, sett fra arnesteded
                \begin{itemize}
                    \item Kan medføre innfløkt logikk hva gjelder å
                        opprettholde meshpunkttetthet
                \end{itemize}
                \item Alternative fremgangsmåter finnes. Blant annet beskriver
                    \textcite{krauskopf2005survey} X andre metoder for å
                    beregne invariante mangfoldigheter av tredimensjonale
                    vektorfelter. Ingen av disse ble undersøkt som en del av
                    dette prosjektet. \emph{Forslag til videre arbeid}.
    \end{itemize}
\end{framed}

