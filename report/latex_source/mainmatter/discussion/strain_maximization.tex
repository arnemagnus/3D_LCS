\section[Reflections upon the process of identifying locally most
repelling material surfaces]
{Reflections upon the process of identifying locally \\\phantom{5.3} most
repelling material surfaces}
\label{sec:reflections_on_the_process_of_identifying_locally_most_normally%
_repelling_material_surfaces}

To our knowledge, standardized algorithms for the detection of points which
satisfy LCS existence criterion~\eqref{eq:lcs_condition_d} --- which is used to
identify points that might be local repulsion maxima --- have not yet been
found. In particular, numerical round-off error makes detecting the zeros of
inner products, like the one in condition~\eqref{eq:lcs_condition_d},
challenging. The conditions given in \cref{eq:lcs_condition_a,%
eq:lcs_condition_b,eq:lcs_condition_c} are quite unambiguous, in comparison.
Moreover, while the concept of \emph{local} maxima for the normal repulsion is
well-defined for analytical systems, this is not the case for numerical
simulations. In contrast to the infinitesimal neighborhoods one may consider
for analytical flow, the discrete nature of numerics --- coupled with possible
numerical round-off error --- means that the regions within which one looks for
repulsion maxima must have finite extent. Accordingly, the scale at which one
performs \emph{local} comparisons becomes significant. Our approach to finding
points which satisfy LCS existence condition~\eqref{eq:lcs_condition_d} is
outlined in \cref{sub:identifying_suitable_initial_conditions_for_developing%
_lcss}, where we used a small perturbation parameter $\varepsilon$ to define
the extent of  the nearby regions within which we sought local repulsion
maxima. In particular, $\varepsilon$ was chosen to be an order of magnitude
smaller than the grid spacing, in order for the local neighborhoods to be of
the same approximate scale as the smallest level of detail which the cubic
interpolation schemes (see \cref{sec:flow_systems_defined_by_gridded%
_velocity_data,sec:computing_cauchy_green_strain_eigenvalues_and_vectors}) can
reasonably be expected to resolve.

An alternative way of checking if a point satisfies condition~%
\eqref{eq:lcs_condition_d} could be extending the work of
\textcite{farazmand2012computing} from two to three dimensions. A direct
adaption would amount to finding all intersections between the computed
surfaces and a family of planes, then, having organized the surfaces in bundles
based on their intersections with any given plane being sufficiently close,
flagging the most repulsive surface within each bundle as a local strain
maximizer. This would not, however, \emph{fully} solve the challenge of
translating the concept of locality to numerics. Furthermore, there does not
appear to be an unambiguous way of selecting the aforementioned family of
planes --- a notion which is supported by \textcite{farazmand2012computing}
failing to mention any details on the set of lines (the two-dimensional
equivalent of the family of planes) they used for their applications. Lastly,
as the intersection between any material surface and a plane generally forms a
curve, rather than a unique intersection point, the process of identifying a
material surface whose intersections with any given plane lie sufficiently
close to those of any other material surface could easily become expensive in
terms of computational resources.

Another option, somewhat similar to the approach of
\textcite{farazmand2012computing}, would be to simply divide the computational
domain into a set of smaller domains, identifying the computed surfaces which
(partially) lie within each such region and then flagging the most strongly
repelling surface within each subdomain as a local repulsion maximizer. Like
the selection of planes in the aforementioned adaption of the method of
\citeauthor{farazmand2012computing}, however, there does not (to our knowledge)
exist an objective way to select the size nor locations of these subdomains.
Furthermore, the approach of comparisons within smaller sets of the
computational domain does not take the orientation of the material surfaces
into account --- a weakness which is shared with the previously mentioned
adaption of \citeauthor{farazmand2012computing}'s method. Conceptually, using
direct comparisons of material surfaces in order to detect the surfaces which
form local repulsion maxima should not involve comparing surfaces with
disparate orientations. Neither should two surfaces for which only a small
subset of one lies anywhere near the other; such material surfaces would likely
influence the overall flow patterns quite differently. Highly optimized
algorithms would likely be needed in order to check such extra comparison
criteria without excess consumption of the available computational resources.

To our knowledge, computing LCSs in three-dimensional flow has not been
attempted particularly frequently, rendering us without reliable reference
cases. \textcite{blazevski2014hyperbolic} construct three-dimensional LCSs by
dividing their computational domain into a set of planes. After computing LCSs
within each plane as locally most repelling material lines, they consider these
LCS curves as the projections of three-dimensional structures onto the plane
family, whereupon they apply a curve fitting algorithm to connect the LCS
curves and form three-dimensional structures. This approach is not, however,
fully three-dimensional, as it ignores transport orthogonal to the planes;
moreover, \citeauthor{blazevski2014hyperbolic} do not provide any evidence as
to whether or not their approach is robust with regards to the orientation (or
density) of the plane family. \textcite{oettinger2016autonomous} seemingly do
not even attempt to identify local repulsion or attraction maxima in their
considerations of hyperbolic LCSs (see \cref{def:hyperbolic_lcs,%
def:attracting_lcs,def:repelling_lcs}). Notably,
\citeauthor{oettinger2016autonomous} appear to be content with identifying
invariant manifolds of the $\vct{\xi}_{2}$- and $\vct{\xi}_{1}$-direction
fields (in the case of attracting LCSs, the $\vct{\xi}_{1}$-direction field is
replaced with the $\vct{\xi}_{3}$-direction field) as regions where LCSs may
reasonably be expected to exist (see \cref{rmk:invariance_lcs}).

Although outside of the scope of this project, yet another alternative
approach would be to identify all material surfaces which lie reasonably close
to each other, having similar spatial orientation and (preferably) size, by
means of some numerical clustering algorithm. Then, the most strongly repulsive
surface segments within each cluster could reasonably be considered as the
local repulsion maximizer. Possibly geared towards a project pertaining to
machine learning, this sort of approach would benefit greatly from reliable
reference cases. Furthermore, basing the selection process solely on the
computed manifolds' repulsion averages (as defined in
\cref{eq:lcs_lm3_weight}), or other macroscale quantities, need not necessarily
be the best possible approach in terms of extracting the most significant LCSs.
In particular, subsets of large LCSs could exhibit significant repulsion,
without necessarily resulting in large repulsion averages. Similarly,
relatively small yet strongly repelling LCSs need not be particularly
significant for the overall flow pattern.

Compared with the aforementioned alternatives, one could argue that our
approach of checking whether or not each \emph{point} in a computed material
surface satisfies existence criterion~\eqref{eq:lcs_condition_d} by considering
a small neighborhood around them (whose extent is defined by the perturbation
parameter $\varepsilon$, cf.\
\cref{sub:identifying_suitable_initial_conditions_for_developing_lcss}) is more
faithful to the underlying theory. Note in particular that our approach is
based on the local repulsion of small subsets (namely, the points constituting
the parametrization) of the computed manifolds, in contrast to the global
comparison of repulsion averages (or similar quantities) inherent to the other
methods. However, put simply, there is certainly room for further research with
regards to the numerical implementation of LCS existence criterion~%
\eqref{eq:lcs_condition_d}.

