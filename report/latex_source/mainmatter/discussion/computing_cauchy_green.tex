\section[Regarding the variational approach to computing the Cauchy-Green
strain characteristics] {Regarding the variational approach to computing
\\\phantom{4.1} the Cauchy-Green strain characteristics}
\label{sec:regarding_the_variational_approach_to_computing_the_cauchy_green%
_strain_characteristics}

As outlined in
\cref{sec:computing_the_flow_map_and_its_directional_derivatives,%
sec:computing_cauchy_green_strain_eigenvalues_and_vectors}, we used an
SVD decomposition of the flow map Jacobian to find the Cauchy-Green strain
eigenvalues and -vectors, rather than computing these directly from the
Cauchy-Green strain tensor field. This approach, suggested by
\textcite{miron2012anisotropic} and endorsed by
\textcite{oettinger2016autonomous}, boasts superior accuracy compared to the
more conventional approach of approximating the directional derivatives of the
flow map (i.e.,\ the components of the flow map Jacobian) by applying a finite
difference method and then explicitly computing the Cauchy-Green strain
tensor field --- as was done by \textcite{farazmand2012computing}. In addition
to the increased mathematical complexity of directly transporting the flow map
Jacobian field, the approach relies on bounded first spatial derivatives of the
underlying velocity field, as is evident from inspecting
\cref{eq:timederivative_flowmap_jacobian}. This should, however, not be an issue
when considering smooth analytical test cases, or when using a high (quadratic
or higher, cf.\ \cref{sub:spline_interpolation_of_discrete_data}) order
interpolation method for gridded data. Alternatively, the derivatives can be
approximated by e.g.\ a finite difference method. Should any of these
approaches prove impractical, the method of \textcite{farazmand2012computing}
could be sufficient.




