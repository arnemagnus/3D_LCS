\section[Remarks on the overall computation of three-dimensional repelling
LCSs]{Remarks on the overall computation of \\\phantom{5.5} three-dimensional
repelling LCSs}
\label{sec:remarks_on_the_overall_computation_of_three_dimensional_repelling_lcss}

As briefly alluded to in \cref{sec:thoughts_on_the_extraction_of_lcss_as%
_subsets_of_the_computed_manifolds}, all of the LCS surfaces presented in
\cref{cha:results} were generated using the clustering approach, favored over
the carve-out approach due to being less reliant on large underlying manifolds
(in addition to the conceptual advantages outlined in \cref{sec:thoughts_on%
_the_extraction_of_lcss_as_subsets_of_the_computed_manifolds}). This proved
important for LCS analysis in both variants of the ABC flow, for which the
computed manifolds often terminated due to the detection of unphysical
self-intersections (see \cref{sec:macroscale_stopping_criteria_for_the%
_expansion_of_computed_manifolds}). Possibly caused by small numerical errors
perturbing a small number of mesh points onto an adjacent manifold ---
accompanied by the subsequent interpolation curves $\mathcal{C}_{i}$ being
distorted, yielding compound errors when inserting mesh points from ficticious
ancestor points (see \cref{sec:managing_mesh_accuracy}) --- this issue might be
mitigated by organizing mesh points as bundles of point strands rather than
geodesic circles (as outlined in
\cref{sec:comments_on_the_method_of_geodesic_level_sets}).

Following manual bundling of (partly) overlapping surface elements, the LCSs in
either variant of the ABC flow (see \cref{fig:steady_lcss,fig:unsteady_lcss}),
were obtained as structures consisting of between \numprint{3} and
\numprint{23} unique surface elements. Similarly, LCSs obtained in the Førde
fjord (see \cref{fig:fjord_lcss}), were largely grouped in a sequence of
horizontal layers. We elected to color each surface element constituting the
LCSs in the Førde fjord according to their relative repulsion average, rather
than assigning a single color to each layer, in order to investigate whether or
not the repulsion within a given horizontal layer is uniform. From inspection
of \cref{fig:fjord_lcss}, this seems to be a reasonable conclusion.

Demonstrated in \cref{sec:verifying_our_method_of_extracting_repelling_lcss%
_from_the_computed_manifolds}, our method for computing repelling LCSs appears
to work as intended. This notion is further supported by the LCSs obtained for
the ABC flows and flow in the Førde fjord conforming well with the computed
$\mathcal{U}_{0}$ domains (see \cref{sec:computed_lcss_in_the_abc_flow,%
sec:computed_lcss_in_the_forde_fjord}). Interestingly, the conformity between
the $\mathcal{U}_{0}$ domains and LCSs obtained for the steady and unsteady ABC
flows (see \cref{sec:computed_lcss_in_the_abc_flow}) indicates that,
although substantial, the time perturbation in the case of the unsteady flow
(illustrated in \cref{fig:abc_timedep_coeff}) did not significantly alter its
underlying stretch and strain properties. This result is indicative of the
robustness believed to be characteristic to LCSs. Moreover, considering the
model velocity data to provide a reasonable approximation of the \emph{actual}
oceanic circulation, this also suggests that our computed LCSs provide a good
estimate of (some of) the underlying transport barriers in the Førde fjord ---
a conjecture which warrants further investigation.

Seeing as computing LCSs in three dimensions is a considerably more convoluted
process than for two-dimensional flows (compare our method, outlined in
\cref{cha:method}, to that of e.g.\ \textcite{loken2017sensitivity}),
investigating whether or not the conceptually simpler quasi-three-dimensional
approach of e.g.\ \textcite{blazevski2014hyperbolic} (and, to a lesser
extent, \textcite{oettinger2016autonomous}) --- elaborated upon in
greater detail in
\cref{sec:reflections_on_the_process_of_identifying_locally_most_normally%
_repelling_material_surfaces} --- yields similar LCSs could certainly be
worthwile. This could reduce the need for computational resources, which in
turn would make the detection of LCSs in three-dimensional flow systems more
readily available. That being said, the quasi-three-dimensional approaches will
likely never be fully capable of encapsulating the peculiarities of
three-dimensional surfaces, regardless of which curve fitting algorithm is used
in order to extract them.

Overall, strong situational arguments in terms of accuracy requirements or a
desire for insight in the fully three-dimensional structure of transport
barriers in a given system are needed in order to justify computing LCSs in
three dimensions rather than two. Generally, we would advise computing LCSs in
three dimensions only for systems in which all three dimensions are of similar
relevance; flow in the Førde fjord (a subset of which was considered here)
constitutes an example of such a system, as its depth and width are typically
similar in magnitude, and significant vertical transport can be observed.
Conversely, transport along oceanic surface currents --- within which
contaminations such as garbage patches or oil spill remnants are frequently
transferred --- can reasonably be approximated as being two-dimensional.

Pilot studies pertaining to the use of LCSs as predictors for transport by
oceanic currents have recently been conducted. For instance,
\textcite{filippi2018detection} performed field experiments on repelling and
attracting transport barriers along the Scott Reef in Western Australia,
while \textcite{peacock2018targeted} did similar exercises in the vicinity of
Martha's Vineyard. Interestingly, \citeauthor{peacock2018targeted} investigated
the fully three-dimensional transport characteristics of offshore currents,
finding complementary results to that suggested by Lagrangian analysis. This
suggests that designing field experiments in order to verify the repelling LCSs
computed by the method of \cref{cha:method} is not beyond the realms of
possibility --- then again, doing so remains beyond the scope of this project.


