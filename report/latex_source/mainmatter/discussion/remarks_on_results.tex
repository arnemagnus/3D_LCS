\section[Remarks on the overall computation of three-dimensional repelling
LCSs]{Remarks on the overall computation of \\\phantom{5.5} three-dimensional
repelling LCSs}
\label{sec:remarks_on_the_overall_computation_of_three_dimensional_repelling_lcss}

As briefly alluded to in \cref{sec:thoughts_on_the_extraction_of_lcss_as%
_subsets_of_the_computed_manifolds}, all of the LCS surfaces presented in
\cref{cha:results} were generated using the clustering approach, favored over
the carve-out approach for the reasons outlined in the above. For instance, for
the LCSs in either variant of the ABC flow (see
\cref{fig:steady_lcss,fig:unsteady_lcss}), each of the identified LCS
structures consist of between \numprint{3} and \numprint{23} different surface
elements. The LCSs obtained in the Førde fjord (see \ref{fig:fjord_lcss}),
however, were largely grouped in natural horizontal layers. Accordingly, we
elected to color each surface element according to their relative repulsion
average rather instead of assigning a single color to each layer in order to
depict that all of the LCSs within a single layer are more or less equally
repulsive.

Demonstrated in \cref{sec:verifying_our_method_of_extracting_repelling_lcss%
_from_the_computed_manifolds}, our method for computing repelling LCSs appears
to work as intended. This notion is further supported by the LCSs obtained for
the ABC flows and flow in the Førde fjord conforming well with the computed
$\mathcal{U}_{0}$ domains (see \cref{sec:computed_lcss_in_the_abc_flow,%
sec:computed_lcss_in_the_forde_fjord}).




