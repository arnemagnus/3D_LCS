\section{Comments on the method of geodesic level sets}
\label{sec:comments_on_the_method_of_geodesic_level_sets}
%
As mentioned in \cref{cha:method}, our take on the method of geodesic level
sets (see \cref{sec:revised_approach_to_computing_new_mesh_points}) hinges on
characteristic properties of hyperbolic LCSs (cf.\ \cref{def:hyperbolic_lcs}).
For \emph{repelling} LCSs (see \cref{def:repelling_lcs}), which we concentrated
on for this project, existence criterion \cref{eq:lcs_condition_c} states that
these are everywhere \emph{orthogonal} to the local direction of strongest
repulsion. The extra degree of freedom compared to manifolds defined as being
everywhere tangent to some direction field (such as the strange attractor in
the Lorenz system, as was considered by \textcite{krauskopf2005survey})
facilitated a more effective (in terms of computational runtime) and
conceptually simpler method of generating them, than the more direct adaption
of \posscite{krauskopf2005survey} method (which is outlined in
\cref{sec:legacy_approach_to_computing_new_mesh_points}). Note, however, that
\citeauthor{krauskopf2005survey}'s method retains a greater degree of
general applicability.

Although useful for managing the mesh accuracy (see
\cref{sec:managing_mesh_accuracy}), and central to our
triangulation algorithm (outlined in detail in \cref{sec:continuously%
_reconstructing_three_dimensional_manifold_surfaces_from_point_meshes}),
exclusively arranging mesh points in closed topological circles has irrefutable
weaknesses. In systems for which periodic boundary conditions are not
applicable, the addition of further level sets is promptly terminated when one
or more of the trajectories used to compute new mesh points (see
\cref{sec:revised_approach_to_computing_new_mesh_points}) exits the
computational domain. This impedes the ability to resolve LCS behaviour near
the domain boundaries, which is evident from inspecting the LCSs in the Førde
fjord (presented in \cref{sec:computed_lcss_in_the_forde_fjord}). Although this
issue could be managed by computing strain eigenvalues and -vectors in a domain
\emph{containing} the domain of interest and expanding from it in all
directions, this workaround is computationally demanding. Depending on to what
extent the underlying flow system is known (or modelled), and the location of
the domain of interest, it might not even be possible.

On a related note, demanding that a new geodesic level set is computed using
a mesh point descending from \emph{each} of the mesh points in the preceding
level set renders computing the underlying manifold in its entirety from a
\emph{single} focal point $\vct{x}_{0}$ (see \cref{sec:preliminaries_for%
_computing_repelling_lcss_in_3d_flow_by_means_of_geodesic_level_sets}) quite
difficult. If a single point strand (that is, the set of mesh points which can
be traced back to a single, common ancestor) is terminated --- either due to
reaching the domain edges, or failure to compute a new mesh point (see
\cref{sub:handling_failures_to_compute_satisfactory_mesh_points_revised}), so
too is the addition of further level sets. Thus, unless the manifold as a whole
expands as a perfect topological surface area, as seen from the focal point
$\vct{x}_{0}$, encapsulating it in its entirety by means of geodesic level sets
becomes impossible --- even when not accounting for the possibility of
numerical error.

Moreover, our variation of the method of geodesic level sets contains many
degrees of freedom (see e.g.\
\cref{tab:initialconditionparams,tab:abc_manifold_params}). Some of these
parameters, mainly those governing the minimum and maximum allowed separations
between neighboring mesh points, could reasonably be chosen based on
considerations pertaining to the spatial extent of the computational domain;
alternatively, to what extent the minute details of the LCSs are to be
resolved. How to determine several other undeniably key parameter values ---
such as the tolerances  for the detection of intersecting manifolds (see
\cref{sec:macroscale_stopping_criteria_for_the_expansion_of_computed%
_manifolds}), and the removal of mesh points which form undesired bulges (which
is outlined in \cref{sub:limiting_the_accumulation_of_numerical_noise}) ---
remains less obvious.

That being said, the parameters related to the curvature-guided approach to
dynamically adjust the interset separations were, in our experience, of less
importance; as briefly mentioned in \cref{sub:a_curvature_based_approach_to%
_determining_interset_separations}, the interset step length was rarely
\emph{increased}. More often than not, the interset step length was quickly
reduced to its lower limit, and remained at that level for the generation of
all subsequent level sets. This is not entirely unexpected, as sufficiently
large curvature within a \emph{single} region of any given level set sufficed
to lower the step length (compare \cref{eq:increase_dist,eq:decrease_dist}).
Moreover, as the geodesic level sets continuously expand, encoutering such a
region becomes increasingly likely. As the accuracy of the computed mesh points
is independent of the density of mesh points, the main use of the interset step
size is to manage the interpolation error inherent to our linear triangulation
scheme (see \cref{sec:continuously_reconstructing_three_dimensional_manifold%
_surfaces_from_point_meshes}) \parencite{krauskopf2003computing}. Thus, it
seems reasonable to forego the dynamic interset step length in favor of a
fixed one, where the minute details of LCS surfaces are unimportant. As
tentatively suggested in the above, doing so reduces the overall complexity of
our method for generating mesh points, in addition to reducing the number of
free parameters.

Another way of organizing the mesh points, which could circumvent some of the
aforementioned limitations of the present approach, would be as a group of
point strands, each associated with a particular unit tangent $\vct{t}$ ---
determined using $\mathcal{C}_{1}$, the interpolation curve of the innermost
level set, in similar fashion to that described in
\cref{sub:parametrizing_the_innermost_level_set}. Treating the expansion along
each point strand independently would then permit further expansion of a
manifold even when one or more point strands would reach the domain boundaries;
this would also solve the possible issue of computed trajectories along any
given strand failing to yield acceptable mesh points (see
\cref{sub:handling_failures_to_compute_satisfactory_mesh_points_revised}) ---
in which case only the strands in question need to be terminated.

Aside from taking a step further away from the method of geodesic level set as
originally proposed by \textcite{krauskopf2005survey}, organizing mesh
points as a group of point strands would necessitate developing the mesh
accuracy management method outlined in \cref{sec:managing_mesh_accuracy}
further. In particular, the present approach utilizes the interpolation curve
$\mathcal{C}_{i}$ in order to insert mesh points inbetween nearest neighbor
mesh points in level set $\mathcal{M}_{i+1}$ which are deemed to lie too far
away from eachother (see \cref{sub:maintaining_mesh_point_density}). A
reasonable approach for the case of point strands could be to, having
identified two strands inbetween which a new mesh point is needed, retrieve
their respective ancestor points in the innermost level set, and then compute
(the trajectories of a) new point strand starting out at a point on
$\mathcal{C}_{1}$, midway inbetween said ancestors, keeping only the first
point along the strand which is required to maintain the point density.

Depending on the extent of the initial level set --- that is, the circumference
of its interpolation curve $\mathcal{C}_{1}$ --- and the required mesh point
density, however, this approach could be prone to errors arising from numerical
round-off errors in computing the start points of new point strands. Moreover,
in order to maintain an overall mesh structure suitable for triangulation
purposes, the steps along each point strand should be equal; that is, mesh
points $i$ and $i+1$ along all point strands should be separated by the same
distance $\Delta_{i}$, whereas all interpoint step sizes $\{\Delta_{i}\}$ need
not be the same. This way, a quasi-circular structure is maintained as the
point strands expand --- that is, subject to one or more of them reaching the
domain boundaries or being terminated due to ending up in (approximately)
closed orbits (see \cref{sub:computing_pseudoradial_trajectories_directly}).

Lastly, several other methods of computing invariant manifolds of vector fields
exist; of whom some might be well-suited in the context of LCSs. For instance,
the method of geodesic level sets is one of five methods presented by
\textcite{krauskopf2005survey}; none of the others were pursued as part of this
work. Exploring the strengths and weaknesses of other
approaches to computing three-dimensional hyperbolic LCSs remains beyond the
scope of this project.

\section[On our approach to computing the Cauchy-Green
strain characteristics] {On our approach to computing the%
    \\\phantom{5.1} Cauchy-Green strain characteristics}
\label{sec:on_our_approach_to_computing_the_cauchy_green%
_strain_characteristics}

We used an SVD decomposition of the flow map Jacobian to find the Cauchy-Green
strain eigenvalues and -vectors, rather than computing these directly from the
Cauchy-Green strain tensor field --- as described in
\cref{sec:computing_the_flow_map_and_its_directional_derivatives,%
sec:computing_cauchy_green_strain_eigenvalues_and_vectors}. This approach,
suggested by \textcite{miron2012anisotropic} and endorsed by
\textcite{oettinger2016autonomous}, boasts superior accuracy compared to the
more conventional approach of approximating the directional derivatives of the
flow map (i.e.,\ the components of the flow map Jacobian) by applying a finite
difference method and then explicitly computing the Cauchy-Green strain
tensor field (which \textcite{farazmand2012computing} did in order to find LCSs
in two-dimensional flow). In addition to the increased mathematical complexity
of directly transporting the flow map Jacobian field, our approach relies on
bounded first spatial derivatives of the underlying velocity field, as is
evident from inspecting \cref{eq:timederivative_flowmap_jacobian}. This should,
however, not be an issue when considering smooth analytical test cases, or when
using a high (quadratic or higher, cf.\
\cref{sub:spline_interpolation_of_discrete_data}) order interpolation method
for gridded data. Alternatively, the derivatives can be approximated by e.g.\ a
finite difference method. Should any of these approaches prove impractical, the
method of \textcite{farazmand2012computing} could be sufficient.

Note that the resolution of the grid of tracers, on which the Cauchy-Green
strain eigenvalues and -vectors are computed (see
\cref{sec:computing_the_flow_map_and_its_directional_derivatives,%
sec:computing_cauchy_green_strain_eigenvalues_and_vectors}), plays a critically
important role in the successful detection of LCSs. For instance, we were
unable to reproduce the spherical LCS of \cref{sub:an_analytical_lcs_test_case}
using (quite significantly) fewer tracers; which naturally resulted in a
reduced number of initial conditions satisfying LCS existence conditions~
\eqref{eq:lcs_condition_a},~\eqref{eq:lcs_condition_b}
and~\eqref{eq:lcs_condition_d} (see
\cref{sub:identifying_suitable_initial_conditions_for_developing_lcss}).
Difficulties arose with such a sparse grid of tracers that no initial
conditions for the generation of manifolds layed sufficiently close to the
strongly repulsive unit sphere (see \cref{fig:spherical_lm3}). To our
knowledge, there is no way to determine \emph{a priori} what density of tracers
will suffice for any flow system. Thus, even though educated guesses based on
the scale at which one is interested in the microscopic behaviour in the system
might be prudent, we recommend to use as fine a grid of tracers as possible,
within the constraints set by the available computational resources.
based on the scale at which one is interested in the

As mentioned in \cref{sec:flow_systems_defined_by_gridded_velocity_data}, we
interpolated the model velocity field using quadrivariate, cubic B-splines ---
that is, cubic interpolation in time and all spatial direction. This involved
us having to keep the model data pertaining to the region of interest (that is,
the model data for a domain expanding beyond said region in all directions, in
order to resolve the behaviour near the boundaries, cf.\
\cref{sec:macroscale_stopping_criteria_for_the_expansion_of_computed%
_manifolds}) for the entirety of the considered time interval. Because of our
data set's disparate resolution in the horizontal and vertical directions
(as mentioned in \cref{sec:flow_systems_defined_by_gridded_velocity_data}), in
addition to the small spatial region of interest (in comparison to the entire
fjord), the use of quadrivariate interpolation was unproblematic regarding
the consumption of working memory. For other applications, however, this is not
necessarily the case, depending on the problem's scale (temporal and spatial)
and the resolution of the model data.

Should memory consumption be an issue for a discrete dataset, it is possible
to forego temporal interpolation entirely (provided that the sampling rate
is adequate), and instead opt for trivariate interpolation in space, generating
an interpolation object for each time instance. This renders the use of
ODE solvers of adaptive stepsize --- like the Dormand-Prince 8(7) method we
wound up choosing (more on that to follow in
\cref{sec:regarding_our_choice_of_numerical_ode_solver}) --- moot, as the
solution time steps would then have to coincide with the time levels of the
dataset. Using a lower order ODE solution method, such as the explicit
trapezoidal rule (which does not require intermediary samples when moving from
one time level to the next), yields inferior performance; high order embedded
Runge-Kutta solvers are much less prone to numerical error
\parencite{loken2017sensitivity}.




\section{Regarding our choice of numerical ODE solver}
\label{sec:regarding_our_choice_of_numerical_ode_solver}

As mentioned in \cref{sec:on_our_approach_to_computing_the_cauchy_green_strain%
_characteristics}, the Dormand-Prince 8(7) method was used for both the tracer
advection used to compute the Cauchy-Green strain eigenvalues and -vectors, and
computing new mesh points in the expansion of computed manifolds (as described
in  \cref{sec:computing_the_flow_map_and_its_directional_derivatives,%
sec:computing_cauchy_green_strain_eigenvalues_and_vectors} and
\cref{sec:revised_approach_to_computing_new_mesh_points}, respectively). Thus,
any propagation of numerical round-off errors can reasonably be expected to
have occurred in a consistent manner throughout. Moreover, courtesy of the
adaptive step size adjustment routine outlined in
\cref{sub:the_implementation_of_dynamic_runge_kutta_step_size}, using an
embedded Runge-Kutta solver allows for dispensing with a particular numerical
integration step size. Accordingly, the use of an embedded ODE solver --- like
the Dormand-Prince 8(7) method --- involves one less degree of freedom than
traditional singlestep methods such as the classical \nth{4}-order Runge-Kutta
method (that is, subject to limiting the step size from above in order to avoid
overshoots when computing manifold mesh points, briefly mentioned in
\cref{sec:revised_approach_to_computing_new_mesh_points}). This is
significant, because the present approach already involves a lot of free
parameters (see \cref{sec:comments_on_the_method_of_geodesic_%
level_sets}).

\textcite{loken2017sensitivity} showed that the Dormand-Prince 8(7) method
yields very accurate numerical approximations at a very low computational cost
--- at least, that is, for smooth flow systems (see
\cref{sec:the_type_of_flow_systems_considered}). As is apparent from
\cref{def:runge_kutta_order}, pertaining to the order of accuracy of
Runge-Kutta methods, the accuracy of numerical solutions obtained by using
Runge-Kutta solvers depend on the order of the method itself, in addition to
the smoothness of the underlying function. Although generally more accurate,
higher-order ODE solvers yield increasingly diminishing returns compared to
their lower-order siblings when the order of the ODE solver exceeds the
function's number of smooth derivatives.

Thus, for gridded model data --- both regarding the oceanic currents of the
Førde fjord (\cref{sub:oceanic_currents_in_the_forde_fjord}), and the
discretely sampled Cauchy-Green strain eigenvalue and -vector fields
(\cref{sec:computing_the_flow_map_and_its_directional_derivatives,%
sec:computing_cauchy_green_strain_eigenvalues_and_vectors}) --- the
interpolation routine sets an upper bound in terms of the accuracy with which
LCSs can be computed. For more complex systems than the ones investigated here,
the interaction between the integration and interpolation schemes could be
critical; both in terms of numerical precision and computational resource
consumption. Independently of the scales at which well-resolved LCSs are sought
in a given transport system, the aforementioned effects warrant further
investigation, yet remain beyond the scope of this project.

\section[Reflections upon the process of identifying locally most
repelling material surfaces]
{Reflections upon the process of identifying locally \\\phantom{5.3} most
repelling material surfaces}
\label{sec:reflections_on_the_process_of_identifying_locally_most_normally%
_repelling_material_surfaces}

To our knowledge, standardized algorithms for the detection of points which
satisfy LCS existence criterion~\eqref{eq:lcs_condition_d} --- which is used to
identify points that might be local repulsion maxima --- have not yet been
found. In particular, numerical round-off error makes detecting the zeros of
inner products, like the one in condition~\eqref{eq:lcs_condition_d},
challenging. The conditions given in \cref{eq:lcs_condition_a,%
eq:lcs_condition_b,eq:lcs_condition_c} are quite unambiguous, in comparison.
Moreover, while the concept of \emph{local} maxima for the normal repulsion is
well-defined for analytical systems, this is not the case for numerical
simulations. In contrast to the infinitesimal neighborhoods one may consider
for analytical flow, the discrete nature of numerics --- coupled with possible
numerical round-off error --- means that the regions within which one looks for
repulsion maxima must have finite extent. Accordingly, the scale at which one
performs \emph{local} comparisons becomes significant. Our approach to finding
points which satisfy LCS existence condition~\eqref{eq:lcs_condition_d} is
outlined in \cref{sub:identifying_suitable_initial_conditions_for_developing%
_lcss}, where we used a small perturbation parameter $\varepsilon$ to define
the extent of  the nearby regions within which we sought local repulsion
maxima. In particular, $\varepsilon$ was chosen to be an order of magnitude
smaller than the grid spacing, in order for the local neighborhoods to be of
the same scale as the smallest level of detail which the cubic interpolation
schemes (see \cref{sec:flow_systems_defined_by_gridded_velocity_data,%
sec:computing_cauchy_green_strain_eigenvalues_and_vectors}) can reasonably
be expected to resolve.

An alternative way of checking if a point satisfies condition~%
\eqref{eq:lcs_condition_d} could be extending the work of
\textcite{farazmand2012computing} from two to three dimensions. A direct
adaption would amount to finding all intersections between the computed
surfaces and a family of planes, then, having organized the surfaces in bundles
based on their intersections with any given plane being sufficiently close,
flagging the most repulsive surface within each bundle as a local strain
maximizer. This would not, however, \emph{fully} solve the challenge of
translating the concept of locality to numerics. Furthermore, there does not
appear to be an unambiguous way of selecting the aforementioned family of
planes --- a notion which is supported by \textcite{farazmand2012computing}
failing to mention any details on the sets of lines (the two-dimensional
equivalent of the family of planes) they used for their applications. Lastly,
as the intersection between any material surface and a plane generally forms a
curve, rather than a unique intersection point, the process of identifying a
material surface whose intersections with any given plane lie sufficiently
close to those of any other material surface could easily become expensive in
terms of computational resources.

Another option, similar to the approach of \textcite{farazmand2012computing},
would be to simply divide the computational domain into a set of smaller
domains, identifying the computed surfaces which (partially) lie within each
such region and then flagging the most strongly repelling surface within each
subdomain as a local repulsion maximizer. Like the selection of planes in the
aforementioned adaption of the method of \citeauthor{farazmand2012computing},
however, there does not (to our knowledge) exist an objective way to select
the size nor locations of these subdomains. Moreover, the approach of
comparisons within smaller sets of the computational domain does not take the
orientation of the material surfaces into account --- a weakness which is
shared with the adaption of \citeauthor{farazmand2012computing}'s method.
Conceptually, using direct comparisons of material surfaces in order to detect
the surfaces which form local repulsion maxima should not involve comparing
surfaces with disparate orientations. Neither should two surfaces for which
only a small subset of one lies anywhere near the other; such material surfaces
would likely influence the overall flow patterns quite differently. Highly
optimized algorithms would likely be needed in order to check such extra
comparison criteria without excess consumption of the available computational
resources.

To our knowledge, computing LCSs in three-dimensional flow has not been
attempted particularly frequently, rendering us without reliable reference
cases. \textcite{blazevski2014hyperbolic} construct three-dimensional LCSs by
dividing their computational domain into a set of planes. After computing LCSs
within each plane as locally most repelling material lines, they consider these
LCS curves as the projections of three-dimensional structures onto the plane
family, whereupon they apply a curve fitting algorithm to connect the LCS
curves and form three-dimensional structures. This approach is not, however,
fully three-dimensional, as it ignores transport orthogonal to the planes;
moreover, \citeauthor{blazevski2014hyperbolic} do not provide any evidence as
to whether or not their approach is robust with regards to the orientation (or
density) of the plane family. \textcite{oettinger2016autonomous} seemingly do
not even attempt to identify local repulsion or attraction maxima in their
considerations of hyperbolic LCSs (see \cref{def:hyperbolic_lcs,%
def:attracting_lcs,def:repelling_lcs}). Notably,
\citeauthor{oettinger2016autonomous} appear to be content with identifying
invariant manifolds of the $\vct{\xi}_{2}$- and $\vct{\xi}_{1}$-direction
fields (in the case of attracting LCSs, the $\vct{\xi}_{1}$-direction field is
replaced with the $\vct{\xi}_{3}$-direction field) as regions where LCSs may
reasonably be expected to exist (see \cref{rmk:invariance_lcs}).

Although outside of the scope of this project, yet another alternative
approach would be to identify all material surfaces which lie reasonably close
to eachother, having similar spatial orientation and (preferably) size, by
means of some numerical clustering algorithm. Then, the most strongly repellent
surface segments within each cluster could reasonably be considered as the local
repulsion maximizer. Possibly geared towards a project pertaining to machine
learning, this sort of approach would benefit greatly from reliable reference
cases. Moreover, basing the selection process solely on the computed manifolds'
repulsion averages (as defined in \cref{eq:lcs_lm3_weight}), or other
macroscale quantities, need not necessarily be the best possible approach in
terms of extracting the most significant LCSs. In particular, subsets of large
LCSs could exhibit significant repulsion, without necessarily resulting in
large repulsion averages. Similarly, relatively small yet strongly repelling
LCSs need not be particularly significant for the overall flow pattern.

Compared with the aforementioned alternatives, one could argue that our
approach of checking whether or not each \emph{point} in a computed material
surface satisfies existence criterion~\eqref{eq:lcs_condition_d} by considering
a small neighborhood around them (whose extent is defined by the perturbation
parameter $\varepsilon$, cf.\
\cref{sub:identifying_suitable_initial_conditions_for_developing_lcss}) is more
faithful to the underlying theory. Note in particular that our approach is
based on the local repulsion of small subsets (namely, the points constituting
the parametrization) of the computed manifolds, in contrast to the global
comparison of repulsion averages (or similar quantities) inherent to the other
methods. However, put simply, there is certainly room for further research with
regards to the numerical implementation of LCS existence criterion~%
\eqref{eq:lcs_condition_d}.


\section{Identifying LCSs as subsets of computed manifolds}
\label{sec:identifying_lcss_as_subsets_of_computed_manifolds}

The collection of manifolds computed by means of the method
outlined in the preceding sections, starting from an approximately
even distribution of points in the $\mathcal{U}_{0}$ domain
(cf.\
\cref{sec:preliminaries_for_computing_repelling_lcss_in_3d_flow_by_means_of_geodesic_level_sets}
and, in particular, \cref{tab:initialconditionparams}), are all surfaces which
satisfy LCS existence criterion~\eqref{eq:lcs_condition_c} --- that is,
they are everywhere perpendicular to the local direction of maximal repulsion.
In order to extract repelling LCSs from these parametrized surfaces,
we then identified the regions of the manifolds --- represented as a subset
of the mesh points in their parametrization --- which also satisfy the
remaining existence criteria; namely,~\eqref{eq:lcs_condition_a},
\eqref{eq:lcs_condition_b} and~\eqref{eq:lcs_condition_d}. This was done
completely analogously to how we identified the grid points belonging within the
$\mathcal{U}_{0}$ domain, as outlined in
\cref{sub:identifying_suitable_initial_conditions_for_developing_lcss}.
In particular, each mesh point $\mathcal{M}_{i,j}$ of a computed manifold
$\mathcal{M}$ was flagged as to whether or not it satsified all
of the aforementioned existence criteria.

We then proceded to construct a repelling LCS $\mathcal{L}$ from the mesh
points of $\mathcal{M}$. Per
\cref{sub:identifying_suitable_initial_conditions_for_developing_lcss},
the mesh point at the centre of any given manifold always satisfies
all the LCS existence criteria; accordingly, it was added as the first mesh
point $\mathcal{L}_{0}$ of the extracted LCS. Going through the list of
the remanining mesh points $\mathcal{M}_{i,j}$ which satsified all LCS
criteria, we added a manifold mesh point to the set of LCS points provided
that
\begin{equation}
    \label{eq:lcs_from_manifold_separation_criterion}
    \norm{\vct{x}_{i,j}-\tilde{\vct{x}}_{\kappa}} < %
    \gamma_{\square}\Delta_{\max}
\end{equation}
holds for at least one $\kappa$, where $\vct{x}_{i,j}$ and
$\tilde{\vct{x}}_{\kappa}$ denote the coordinates of mesh point
$\mathcal{M}_{i,j}$ and the already accepted LCS point
$\mathcal{L}_{\kappa}\in\{\mathcal{L}_{k}\}$, respectively. $\Delta_{\max}$ is
the maximum allowed mesh point separation used for computing the manifold (cf.\
\cref{sub:maintaining_mesh_point_density}), while the scalar
parameter $\gamma_{\square}\geq1$ allows for extracting smoother LCSs, more
well-suited for visualization purposes --- as will be made clear shortly.
Subsequently, all mesh points \emph{not} satisfying all of the LCS existence
criteria were added to the set of LCS points, provided that they comply with a
similar distance threshold as given in
\cref{eq:lcs_from_manifold_separation_criterion}, where we only computed
mesh point separations relative to the LCS points which satisfied all of
the existence criteria (thus not to any point added to the LCS, for which
this was not the case).

The tolerance parameter $\gamma_{\square}$ thus allowed us to mitigate possible
numerical error; in particular, if any given mesh point was slightly perturbed
away from the underlying manifold, it could still end up being a part of the LCS.
Finally, we looked at all of the surface elements pertaining to the
triangulation of the manifold $\mathcal{M}$ (as described in
\cref{sec:continuously_reconstructing_three_dimensional_manifold_surfaces_from_%
point_meshes}) in conjunction with the set of mesh points which had been
recognized as belonging to the LCS $\mathcal{L}$. If mesh points corresponding
to two of the three vertices defining a triangular surface element had been
recognized as part of $\mathcal{L}$, we then added the mesh point corresponding
to the last remaining vertex to the set of LCS points $\{\mathcal{L}_{k}\}$.
Accordingly, the triangulations of the $\mathcal{M}$ were reused for
$\mathcal{L}$. These slight relaxations of the LCS existence criteria facilitate
the extraction of smoother LCS surfaces are favorable for the visual
representation of LCSs. Moreover, they mitigate the possible effects of
numerical error perturbing any given mesh point $\mathcal{M}_{i,j}$ away from
the \emph{actual} manifold --- leaving it more than sufficiently close to the
manifold for visualization purposes --- by possibly allowing it to be included
as part of the LCS after all. \Cref{fig:manifold_lcs_conversion} shows an
example of extracting a repelling LCS from a computed manifold.

\begin{figure}[htpb]
    \centering
    \begin{subfigure}[b]{0.475\textwidth}
        \centering
        \importpgf{figures/mpl-figs}{conversion-mf-small.pgf}
        \caption[]{{\small A computed manifold in its entirety}}
        \label{fig:mf_conversion_mf}
    \end{subfigure}
    \begin{subfigure}[b]{0.475\textwidth}
        \centering
        \importpgf{figures/mpl-figs}{conversion-lcs-small.pgf}
        \caption[]{{\small The extracted repelling LCS}}
        \label{fig:mf_conversion_lcs}
    \end{subfigure}
    \caption[An example of a repelling LCS extracted as a subset of a computed
    manifold]
    {An example of a repelling LCS extracted as a subset of a computed
        manifold. In (\subref*{fig:mf_conversion_mf}), a sample manifold for
        the steady ABC flow is shown, whereas (\subref*{fig:mf_conversion_lcs})
        shows the subset of the manifold which satisfies the LCS existence
        criteria given in \cref{eq:lcs_conditions} (or lies sufficiently close
        to any point satisfying these criteria, meaning that their inclusion
        facilitates triangulations which provide an overall enhanced visual
        representation).
    }
    \label{fig:manifold_lcs_conversion}
\end{figure}



The extracted LCS surfaces $\mathcal{L}$, parametrized as a set of mesh points
$\{\mathcal{L}_{k}\}$, represent three-dimensional surfaces which --- allowing
for a little numerical error --- comply with all of the existence criteria
for repelling LCSs, as originally proposed by \textcite{haller2011variational}.
Inspired by the work of \textcite{farazmand2012computing}, we then sought to
dispose of the smallest among the computed LCSs, as these are expected to
be the least significant in terms of the overall flow within the system.
In order to obtain a measure of the size of our three-dimensional surfaces,
to each LCS point $\mathcal{L}_{k}$, we assigned a weighting given by the
surface area approximating the region of the underlying manifold $\mathcal{M}$
that is closer to the corresponding mesh point $\mathcal{M}_{i,j}$ than any
others. To the mesh point located at the manifold epicentre
$\vct{x}_{0}$, we assigned the weight
$\mathcal{W}_{0} = \pi(\delta_{\text{init}}/2)^{2}$. The weights of all other
mesh points were computed as
\begin{equation}
    \label{eq:lcs_point_weight}
    \mathcal{W}_{k} := \mathcal{A}_{i,j} \approx %
    \frac{\Delta_{i}+\Delta_{i-1}}{2} \cdot %
    \frac{\norm{\vct{x}_{i,j+1}-\vct{x}_{i,j}}%
                +\norm{\vct{x}_{i,j}-\vct{x}_{i,j-1}}}{2},
\end{equation}
where, as always, $\vct{x}_{i,j}$ denotes the coordinates of mesh point
$\mathcal{M}_{i,j}$. This surface approximation is illustrated in
\cref{fig:lcs_point_weighting}. These weights were also used to compute
a repulsion average $\overline{\lambda}_{3}$; in particular,
\begin{equation}
    \label{eq:lcs_lm3_weight}
    \mathcal{W} = \sum\limits_{k}\mathcal{W}_{k},  \quad%
    \overline{\lambda}_{3} = \frac{1}{\mathcal{W}} %
    \sum\limits_{k}\lambda_{3}(\tilde{\vct{x}}_{k})\mathcal{W}_{k},
\end{equation}
where the summation is over all mesh points in the parametrization of
$\mathcal{L}$, and  $\tilde{\vct{x}}_{k}$ denotes the coordinates of mesh
point $\mathcal{L}_{k}$. Any LCS for which the computed total weight
$\mathcal{W}$ was smaller than some pre-set limit $\mathcal{W}_{\min}$ or
$\overline{\lambda}_{3} < 1$, the latter as a sanity
check to ensure overall repulsion, per existence criterion
\eqref{eq:lcs_condition_a}, were discarded.

\begin{figure}[htpb]
    \centering
    \resizebox{0.9\linewidth}{!}%
    {\includestandalone{figures/tikz-figs/lcsextraction_weighting}}
    \caption[Our way of assigning weights to mesh points in computed LCSs]
    {Our way of assigning weights to mesh points in computed LCSs. To a given
        LCS point $\mathcal{L}_{k}$, we assigned a weight given by a
        rectangular approximation of the surface area closer to the
        corresponding manifold mesh point $\mathcal{M}_{i,j}$ than all other
        points in the parametrization of the manifold $\mathcal{M}$ (shaded).
        Here, $\Delta_{i}$ corresponds to the interset step length used to
        create level set $\mathcal{M}_{i+1}$, based on level set
        $\mathcal{M}_{i}$, (see
        \cref{sec:revised_approach_to_computing_new_mesh_points}), while
    $\vct{x}_{i,j}$ denotes the coordinates of mesh point $\mathcal{M}_{i,j}$.
}
    \label{fig:lcs_point_weighting}
\end{figure}



\section[Remarks on the overall computation of three-dimensional repelling
LCSs]{Remarks on the overall computation of \\\phantom{5.5} three-dimensional
repelling LCSs}
\label{sec:remarks_on_the_overall_computation_of_three_dimensional_repelling_lcss}

As briefly alluded to in \cref{sec:thoughts_on_the_extraction_of_lcss_as%
_subsets_of_the_computed_manifolds}, all of the LCS surfaces presented in
\cref{cha:results} were generated using the clustering approach, favored over
the carve-out approach due to being less reliant on large underlying manifolds
(in addition to the conceptual advantages outlined in \cref{sec:thoughts_on%
_the_extraction_of_lcss_as_subsets_of_the_computed_manifolds}). This proved
important for LCS analysis in both variants of the ABC flow, for which the
computed manifolds often terminated due to the detection of unphysical
self-intersections (see \cref{sec:macroscale_stopping_criteria_for_the%
_expansion_of_computed_manifolds}). Possibly caused by small numerical errors
perturbing a small number of mesh points onto an adjacent manifold ---
accompanied by the subsequent interpolation curves $\mathcal{C}_{i}$ being
distorted, yielding compound errors when inserting mesh points from ficticious
ancestor points (see \cref{sec:managing_mesh_accuracy}) --- this issue might be
mitigated by organizing mesh points as bundles of point strands rather than
geodesic circles (as outlined in
\cref{sec:comments_on_the_method_of_geodesic_level_sets}).

Following manual bundling of (partly) overlapping surface elements, the LCSs in
either variant of the ABC flow (see \cref{fig:steady_lcss,fig:unsteady_lcss}),
were obtained as structures consisting of between \numprint{3} and
\numprint{23} unique surface elements. Similarly, LCSs obtained in the Førde
fjord (see \cref{fig:fjord_lcss}), were largely grouped in a sequence of
horizontal layers. We elected to color each surface element constituting the
LCSs in the Førde fjord according to their relative repulsion average, rather
than assigning a single color to each layer, in order to investigate whether or
not the repulsion within a given horizontal layer is uniform. From inspection
of \cref{fig:fjord_lcss}, this seems to be a reasonable conclusion.

Demonstrated in \cref{sec:verifying_our_method_of_extracting_repelling_lcss%
_from_the_computed_manifolds}, our method for computing repelling LCSs appears
to work as intended. This notion is further supported by the LCSs obtained for
the ABC flows and flow in the Førde fjord conforming well with the computed
$\mathcal{U}_{0}$ domains (see \cref{sec:computed_lcss_in_the_abc_flow,%
sec:computed_lcss_in_the_forde_fjord}). Interestingly, the conformity between
the $\mathcal{U}_{0}$ domains and LCSs obtained for the steady and unsteady ABC
flows (see \cref{sec:computed_lcss_in_the_abc_flow}) indicates that,
although substantial, the time perturbation in the case of the unsteady flow
(illustrated in \cref{fig:abc_timedep_coeff}) did not significantly alter its
underlying stretch and strain properties. This result is indicative of the
robustness believed to be characteristic to LCSs. Moreover, considering the
model velocity data to provide a reasonable approximation of the \emph{actual}
oceanic circulation, this also suggests that our computed LCSs provide a good
estimate of (some of) the underlying transport barriers in the Førde fjord ---
a conjecture which warrants further investigation.

Seeing as computing LCSs in three dimensions is a considerably more convoluted
process than for two-dimensional flows (compare our method, outlined in
\cref{cha:method}, to that of e.g.\ \textcite{loken2017sensitivity}),
investigating whether or not the conceptually simpler quasi-three-dimensional
approach of e.g.\ \textcite{blazevski2014hyperbolic} (and, to a lesser
extent, \textcite{oettinger2016autonomous}) --- elaborated upon in
greater detail in
\cref{sec:reflections_on_the_process_of_identifying_locally_most_normally%
_repelling_material_surfaces} --- yields similar LCSs could certainly be
worthwile. This could reduce the need for computational resources, which in
turn would make the detection of LCSs in three-dimensional flow systems more
readily available. That being said, the quasi-three-dimensional approaches will
likely never be fully capable of encapsulating the peculiarities of
three-dimensional surfaces, regardless of which curve fitting algorithm is used
in order to extract them.

Overall, strong situational arguments in terms of accuracy requirements or a
desire for insight in the fully three-dimensional structure of transport
barriers in a given system are needed in order to justify computing LCSs in
three dimensions rather than two. Generally, we would advise computing LCSs in
three dimensions only for systems in which all three dimensions are of similar
relevance; flow in the Førde fjord (a subset of which was considered here)
constitutes an example of such a system, as its depth and width are typically
similar in magnitude, and significant vertical transport can be observed.
Conversely, transport along oceanic surface currents --- within which
contaminations such as garbage patches or oil spill remnants are frequently
transferred --- can reasonably be approximated as being two-dimensional.

Pilot studies pertaining to the use of LCSs as predictors for transport by
oceanic currents have recently been conducted. For instance,
\textcite{filippi2018detection} performed field experiments on repelling and
attracting transport barriers along the Scott Reef in Western Australia,
while \textcite{peacock2018targeted} did similar exercises in the vicinity of
Martha's Vineyard. Interestingly, \citeauthor{peacock2018targeted} investigated
the fully three-dimensional transport characteristics of offshore currents,
finding complementary results to that suggested by Lagrangian analysis. This
suggests that designing field experiments in order to verify the repelling LCSs
computed by the method of \cref{cha:method} is not beyond the realms of
possibility --- then again, doing so remains beyond the scope of this project.




\begin{framed}
    \begin{itemize}
    \item ABC-LCSenes robusthet under tidsavhengig perturbasjon tyder på at
        metoden er velegnet også for bruk på modelldata. Griddata \emph{minner}
        om den underliggende strømningen $\to$ LCSene er sannsynligvis rimelige
        representasjoner av de faktiske LCSene.
    \item Henvis til Peacock et al.'s haveksperiment (tracers sluppet fra båt)
    \item Avveiningsspørsmål hvorvidt en skal kjøre full 3D-analyse, eller nøye seg med 2D
        \begin{itemize}
            \item For systemer som til god tilærming kan betraktes som todimensjonale, som
                spredning av olje-/plastsøl langs havoverflatestrømninger, vil nok todimensjonal analyse
                kunne være tilstrekkelig --- fordi utstrekningen i planet (generelt) er signifikant større enn i dybderetningen
                \item For systemer som større elver eller fjorder, hvor utstrekningen er
                    i samme størrelsesorden som dybden --- eller, når en betrakter spredningsprosesser som ikke
                    begrenser seg til overflatestrømninger (som feks rester fra gruvedeponi) ---
                    kan 3D-analyse være hensiktsmessig
                \item \emph{Situasjonsavhengig}
        \end{itemize}
    \end{itemize}
\end{framed}
