\section[Regarding our approach to computing the Cauchy-Green
strain characteristics] {Regarding our approach to computing the%
    \\\phantom{5.1} Cauchy-Green strain characteristics}
\label{sec:regarding_our_approach_to_computing_the_cauchy_green%
_strain_characteristics}

We used an SVD decomposition of the flow map Jacobian to find the Cauchy-Green
strain eigenvalues and -vectors, rather than computing these directly from the
Cauchy-Green strain tensor field --- as described in
\cref{sec:computing_the_flow_map_and_its_directional_derivatives,%
sec:computing_cauchy_green_strain_eigenvalues_and_vectors}. This approach,
suggested by \textcite{miron2012anisotropic} and endorsed by
\textcite{oettinger2016autonomous}, boasts superior accuracy compared to the
more conventional approach of approximating the directional derivatives of the
flow map (i.e.,\ the components of the flow map Jacobian) by applying a finite
difference method and then explicitly computing the Cauchy-Green strain
tensor field (which \textcite{farazmand2012computing} did in order to find LCSs
in two-dimensional flow). In addition to the increased mathematical complexity
of directly transporting the flow map Jacobian field, our approach relies on
bounded first spatial derivatives of the underlying velocity field, as is
evident from inspecting \cref{eq:timederivative_flowmap_jacobian}. This should,
however, not be an issue when considering smooth analytical test cases, or when
using a high (quadratic or higher, cf.\
\cref{sub:spline_interpolation_of_discrete_data}) order interpolation method for
gridded data. Alternatively, the derivatives can be approximated by e.g.\ a
finite difference method. Should any of these approaches prove impractical, the
method of \textcite{farazmand2012computing} could be sufficient.

Note that the resolution of the grid of tracers, on which the Cauchy-Green
strain eigenvalues and -vectors are computed (see
\cref{sec:computing_the_flow_map_and_its_directional_derivatives,%
sec:computing_cauchy_green_strain_eigenvalues_and_vectors}), plays a critically
important role in the successful detection of LCSs. For instance, we were
unable to reproduce the spherical LCS of \cref{sub:an_analytical_lcs_test_case}
using (quite significantly) fewer tracers; which naturally resulted in a
reduced number of initial conditions satisfying LCS existence conditions~
\eqref{eq:lcs_condition_a},~\eqref{eq:lcs_condition_b}
and~\eqref{eq:lcs_condition_d} (see
\cref{sub:identifying_suitable_initial_conditions_for_developing_lcss}).
Difficulties arose with such a sparse grid of tracers that no initial
conditions for the generation of manifolds layed sufficiently close to the
strongly repulsive unit sphere (see \cref{fig:spherical_lm3}). To our
knowledge, there is no way to determine \emph{a priori} what density of tracers
will suffice for any flow system. Thus, even though educated guesses based on
the scale at which one is interested in the microscopic behaviour in the system
might be prudent, we recommend to use as fine a grid of tracers as possible,
within the constraints set by the available computational resources.
based on the scale at which one is interested in the

As mentioned in \cref{sec:flow_systems_defined_by_gridded_velocity_data}, we
interpolated the model velocity field using quadrivariate, cubic B-splines ---
that is, cubic interpolation in time and all spatial direction. This involved
us having to keep the model data pertaining to the region of interest (that is,
the model data for a domain expanding beyond said region in all directions, in
order to resolve the behaviour near the boundaries, cf.\
\cref{sec:macroscale_stopping_criteria_for_the_expansion_of_computed_manifolds})
for the entirety of the considered time interval. Because of our data set's
disparate resolution in the horizontal and vertical directions (which is
outlined in \cref{sec:flow_systems_defined_by_gridded_velocity_data}), in
addition to the small spatial region of interest (in comparison to the entire
fjord), the use of quadrivariate interpolation was unproblematic regarding
the consumption of working memory. For other applications, however, this is not
necessarily the case, depending on the problem's scale (temporal and spatial)
and the resolution of the model data.

Should memory consumption be an issue for a discrete dataset, it is possible
to forego temporal interpolation entirely (provided that the sampling rate
is adequate), and instead opt for trivariate interpolation in space, generating
an interpolation object for each time instance. This renders the use of
ODE solvers of adaptive stepsize --- like the Dormand-Prince 8(7) method we
wound up choosing (more on that to follow in
\cref{sec:on_the_choice_of_numerical_ode_solver}) --- moot, as the solution
time steps would then have to coincide with the time levels of the dataset.
Using a lower order ODE solution method, such as the explicit trapezoidal rule
(which does not require intermediary samples when moving from one time level to
the next), yields inferior performance; high order embedded Runge-Kutta solvers
are much less prone to numerical error \parencite{loken2017sensitivity}.




\section{Regarding our choice of numerical ODE solver}
\label{sec:regarding_our_choice_of_numerical_ode_solver}

\section[On the process of identifying locally most repelling material
surfaces]{On the process of identifying locally most \\\phantom{5.3} repelling
    material surfaces}
\label{sec:on_the_process_of_identifying_locally_most_normally_repelling_material_surfaces}

To our knowledge, standardized algorithms for the detection of points which
satisfy LCS existence criterion~\eqref{eq:lcs_condition_d} --- which is used to
identify points that might be local repulsion maxima --- have not yet been
found. The conditions given in \cref{eq:lcs_condition_a,eq:lcs_condition_b,%
eq:lcs_condition_c} are quite unambiguous, in comparison. Numerical round-off
error makes detecting the zeros of inner products, like the one in condition~%
\eqref{eq:lcs_condition_d}, challenging. Moreover, while the concept of
\emph{local} maxima for the normal repulsion is unambiguous for analytical
systems, this is not the case for numerical simulations. In contrast to the
infinitesimal neighborhoods one may consider for analytical flow, the discrete
nature of numerics --- coupled with possible numerical round-off error ---
means that the regions within which one looks for repulsion maxima must have
finite extent. Accordingly, the scale at which one performs \emph{local}
comparisons becomes significant. Our approach to finding points which satisfy
LCS existence condition~\eqref{eq:lcs_condition_d} is outlined in
\cref{sub:identifying_suitable_initial_conditions_for_developing_lcss}, where
we used a small perturbance parameter $\varepsilon$ to define the extent of the
nearby regions within which we sought local repulsion maxima.

An alternative way of checking if a point satisfies condition~%
\eqref{eq:lcs_condition_d} could be extending the work of
\textcite{farazmand2012computing} from two to three dimensions. A direct
adaption would amount to finding all intersections between the computed
surfaces and a family of planes, then, having organized the surfaces in bundles
based on their intersections with any given plane being sufficiently close,
flagging the most repulsive surface within each bundle as a local strain
maximizer. This would not, however, \emph{solve} the challenge of translating
the concept of locality to numerics. Furthermore, there does not appear to be
an unambiguous way of selecting the aforementioned family of planes --- a
notion which is supported by \citeauthor{farazmand2012computing} failing to
mention any details on the sets of lines (the two-dimensional equivalent of
the family of planes) they used for their applications. Lastly, as the
intersection between any material surface and a plane generally forms a
curve, rather than a unique intersection point, the process of identifying a
material surface whose intersections with any given plane lie sufficiently
close to those of any other material surface could easily become expensive in
terms of computational resource consumption.

Another option, similar to the approach of \textcite{farazmand2012computing},
would be to simply divide the computational domain into a set of smaller
domains, identifying the computed surfaces which (partially) lie within each
such region and then flagging the most strongly repelling surface within each
subdomain as a local repulsion maximizer. Like the selection of planes in the
aforementioned adaption of the method of \citeauthor{farazmand2012computing},
however, there does not (to our knowledge) exist an objective way to select
the size nor locations of these subdomains. Moreover, the approach of
comparisons within smaller sets of the computational domain does not take the
orientation of the material surfaces into account --- a weakness which is
shared with the adaption of \citeauthor{farazmand2012computing}'s method.
Conceptually, using direct comparisons of material surfaces in order to detect
the surfaces which form local repulsion maxima should not involve comparing
surfaces with disparate orientations. Neither should two surfaces for which
only a small subset of one lies anywhere near the other; such material surfaces
would likely influence the overall flow patterns quite differently. Highly
optimized algorithms would likely be needed in order to check such extra
comparison criteria without excess consumption of the available computational
resources.

To our knowledge, computing LCSs in three-dimensional flow has not been
attempted particularly frequently, rendering us without reliable reference
cases. \textcite{blazevski2014hyperbolic} construct three-dimensional LCSs by
dividing their computational domain into a set of planes. After computing LCSs
within each plane as locally most repelling material lines, they consider these
LCS curves as the projections of three-dimensional structures onto the plane
family, whereupon they apply a curve fitting algorithm to connect the LCS
curves and form three-dimensional structures. This approach is not, however,
fully three-dimensional; moreover, \citeauthor{blazevski2014hyperbolic} do not
provide any evidence as to whether or not their approach is robust with regards
to the orientation (or density) of the plane family.
\textcite{oettinger2016autonomous} seemingly do not even attempt to identify
local repulsion or attraction maxima in their considerations of hyperbolic
LCSs (see \cref{def:hyperbolic_lcs,def:attracting_lcs,def:repelling_lcs}).
Notably, \citeauthor{oettinger2016autonomous} appear to be content with
identifying invariant manifolds of the $\vct{\xi}_{2}$- and $\vct{\xi}_{1}$-
(or, in the case of attracting LCSs, $\vct{\xi}_{3}$-) direction fields as
regions where LCSs may reasonably be expected to exist (see
\cref{rmk:invariance_lcs}).

Although outside of the scope of this project, yet another alternative
approach would be to identify all material surfaces which lie reasonably close
to eachother, having similar spatial orientation and (preferably) size, by
means of some numerical clustering algorithm. Then, the most strongly repellent
surface segments within each bundle could reasonably be considered as the local
repulsion maximizer. Possibly geared towards a project pertaining to machine
learning, this sort of approach would benefit greatly from reliable reference
cases. Moreover, basing the selection process solely on the computed manifolds'
repulsion averages (as defined in \cref{eq:lcs_lm3_weight}), or other
macroscale quantities, need not necessarily be the best possible approach in
terms of extracting the most significant LCSs. In particular, subsets of large
LCSs could exhibit significant repulsion, without necessarily resulting in large
repulsion averages. Similarly, relatively small yet strongly repelling LCSs
need not be particularly significant for the overall flow pattern.

Compared with the aforementioned alternatives, one could argue that our
approach of checking whether or not each \emph{point} in a computed material
surface satisfies existence criterion~\eqref{eq:lcs_condition_d} by considering
a small neighborhood around them (whose extent is defined by the perturbance
parameter $\varepsilon$, cf.\
\cref{sub:identifying_suitable_initial_conditions_for_developing_lcss}) is more
faithful to the underlying theory. Note in particular that our approach is
based on the local repulsion of small subsets (namely, the points constituting
the parametrization) of the computed manifolds, in contrast to the global
comparison of repulsion averages (or similar quantities) inherent to the other
methods. However, put simply, there is certainly room for further research with
regards to the numerical implementation of LCS existence criterion~%
\eqref{eq:lcs_condition_d}.

\section{About the computation of material surfaces as invariant manifolds}
\label{sec:about_the_computation_of_material_surfaces_as_invariant_manifolds}



\section{Remarks on the computed LCSs}
\label{sec:remarks_on_the_computed_lcss}

\subsection{On the LCSs obtained for flow in (both variants of) the ABC
velocity field}
\label{sub:on_the_lcss_obtained_for_flow_in_both_variants_of_the_abc_velocity_field}

\subsection{Regarding the LCSs obtained for flow in the Førde fjord}
\label{sub:regarding_the_lcss_obtained_for_flow_in_the_forde_fjord}






\begin{framed}
    \begin{itemize}
    \item Bruk av Dormand-Prince-integrator ga oss en frihetsgrad mindre (integrasjonssteglengde)
    \item ABC-LCSenes robusthet under tidsavhengig perturbasjon tyder på at metoden er velegnet også for bruk på modelldata
    \item Henvis til Peacock et al.'s haveksperiment (tracers sluppet fra båt)
    \item Avveiningsspørsmål hvorvidt en skal kjøre full 3D-analyse, eller nøye seg med 2D
        \begin{itemize}
            \item For systemer som til god tilærming kan betraktes som todimensjonale, som
                spredning av olje-/plastsøl langs havoverflatestrømninger, vil nok todimensjonal analyse
                kunne være tilstrekkelig --- fordi utstrekningen i planet (generelt) er signifikant større enn i dybderetningen
                \item For systemer som større elver eller fjorder, hvor utstrekningen er
                    i samme størrelsesorden som dybden --- eller, når en betrakter spredningsprosesser som ikke
                    begrenser seg til overflatestrømninger (som feks rester fra gruvedeponi) ---
                    kan 3D-analyse være hensiktsmessig
                \item \emph{Situasjonsavhengig}
        \end{itemize}
    \end{itemize}
\end{framed}
