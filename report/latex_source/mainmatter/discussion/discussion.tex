\begin{framed}
    \begin{itemize}
    \item Bruk av Dormand-Prince-integrator ga oss en frihetsgrad mindre (integrasjonssteglengde)
    \item ABC-LCSenes robusthet under tidsavhengig perturbasjon tyder på at metoden er velegnet også for bruk på modelldata
    \item Henvis til Peacock et al.'s haveksperiment (tracers sluppet fra båt)
    \item Avveiningsspørsmål hvorvidt en skal kjøre full 3D-analyse, eller nøye seg med 2D
        \begin{itemize}
            \item For systemer som til god tilærming kan betraktes som todimensjonale, som
                spredning av olje-/plastsøl langs havoverflatestrømninger, vil nok todimensjonal analyse
                kunne være tilstrekkelig --- fordi utstrekningen i planet (generelt) er signifikant større enn i dybderetningen
                \item For systemer som større elver eller fjorder, hvor utstrekningen er
                    i samme størrelsesorden som dybden --- eller, når en betrakter spredningsprosesser som ikke
                    begrenser seg til overflatestrømninger (som feks rester fra gruvedeponi) ---
                    kan 3D-analyse være hensiktsmessig
                \item \emph{Situasjonsavhengig}
        \end{itemize}
    \end{itemize}
\end{framed}
