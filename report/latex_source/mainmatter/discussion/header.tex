The main focus of this chapter is the analysis and critique of our method for
computing LCSs in three-dimensional flows (presented in \cref{cha:method}).
\Cref{sec:comments_on_the_method_of_geodesic_level_sets,%
sec:on_our_approach_to_computing_the_cauchy_green_strain_characteristics,%
sec:regarding_our_choice_of_numerical_ode_solver} contain thorough
examinations of our various choices and assumptions made in order to compute
invariant manifolds everywhere orthogonal to the direction of strongest
repulsion. In \cref{sec:reflections_on_the_process_of_identifying_locally_most%
_normally_repelling_material_surfaces,sec:thoughts_on_the_extraction_of_lcss%
_as_subsets_of_the_computed_manifolds}, we review our method of extracting
repelling LCSs from the aforementioned manifolds. Lastly, \cref{sec:remarks_on%
_the_overall_computation_of_three_dimensional_repelling_lcss} assesses
the overall relevance of our method for computing three-dimensional transport
barriers. Throughout this chapter, we present potential topics of
further research, mainly in terms of further method refinement.
