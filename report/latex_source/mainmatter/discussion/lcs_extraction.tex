\section[Thoughts on the extraction of LCSs as subsets of the computed manifolds]
{Thoughts on the extraction of LCSs as subsets of \\\phantom{5.5}
 the computed manifolds}
\label{sec:thoughts_on_the_extraction_of_lcss_as_subsets_of_the_computed%
_manifolds}

While the perturbation parameter $\varepsilon$ --- used in order to identify
local repulsion maxima (discussed in \cref{sec:reflections_on_the_process_of%
_identifying_locally_most_normally_repelling_material_surfaces}) --- may
reasonably be chosen based on the density of advected tracers (see
\cref{sec:computing_the_flow_map_and_its_directional_derivatives}), how to
determine suitable values for the relaxation parameter
$\gamma_{\blacktriangleright}$ and the filtering weight $\mathcal{W}_{\min}$
used to extract repelling LCSs from our computed manifolds (see
\cref{sec:identifying_lcss_as_subsets_of_computed_manifolds}) is less
self-evident. Specifically, these might be selected based on user-determined,
subjective considerations, such as to modify the \emph{number} of ensuing LCSs
in addition to their sizes. This might be useful in some settings, but is hard
to reconcile with the otherwise objective nature of our LCS generation
routines.

Two main regimes exist regarding the choices of
$\gamma_{\blacktriangleright}$ and $\mathcal{W}_{\min}$. Our preference --- in
the following referred to as the \emph{clustering} approach --- involves
selecting (relatively) small values for both $\gamma_{\blacktriangleright}$
and $\mathcal{W}_{\min}$, and then treating clusters of (partly) overlapping
LCS surface elements as single entities. This necessitates a way of sorting the
surface  elements into bundles of interconnected surfaces
(manually or otherwise), yet yields LCSs which are not particularly dependent
on the parameter values. The other way --- in the following referred to as the
\emph{carve-out} approach --- involves the use of large values for both
$\gamma_{\blacktriangleright}$ and $\mathcal{W}_{\min}$, and considering each
resulting surface element as a standalone LCS. While this approach would
generally yield smoother and more aesthetically pleasing LCS surfaces, it may
easily involve significant bias towards the largest material surfaces ---
without there being an obvious way of determining whether or not these form the
most significant barriers to transport \emph{a priori} --- and depends
sensitively on the parameter choices.

Whether the clustering approach is the better choice in the general case, or if
an intermediate approach might be more sensible, remains to be seen. Moreover,
for some applications, the assumption that only sufficiently large repelling
LCSs influence the overall flow patterns significantly (originally suggested
by \textcite{farazmand2012computing}) might not hold; in particular where
small, yet very strongly repelling material surfaces are concerned. In such
cases, the use of $\mathcal{W}_{\min}$ to filter away the supposed least
coherent (i.e.,\ those we believe to be most affected by numerical noise, which
might just not be LCSs at all) LCS surfaces is problematic. Investigations
pertaining to the general case, however, remain beyond the scope of this
project.
