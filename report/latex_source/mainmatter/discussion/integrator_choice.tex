\section{Regarding our choice of numerical ODE solver}
\label{sec:regarding_our_choice_of_numerical_ode_solver}

As previously mentioned, the Dormand-Prince 8(7) method was used for both the
tracer advection used to compute the Cauchy-Green strain eigenvalues and
-vectors, and computing new mesh points in the expansion of computed manifolds
(as described in \cref{sec:computing_the_flow_map_and_its_directional%
_derivatives,sec:computing_cauchy_green_strain_eigenvalues_and_vectors}, and
\cref{sec:revised_approach_to_computing_new_mesh_points}, respectively).
In contrast to traditional singlestep ODE solvers --- such as the classical
\nth{4}-order Runge-Kutta method (see
\cref{sec:solving_systems_of_ordinary_differential_equations}) --- which would
typically require different step sizes for the two cases (as the scales at which
the dynamics occurs in the two systems can reasonably be presumed to be
disparate), using an embedded method with a single set of numerical tolerance
parameters for the integration step adjustment (see
\cref{sub:the_implementation_of_dynamic_runge_kutta_step_size}) means that the
propagation of numerical round-off errors can reasonably be expected to have
occurred in a consistent manner throughout.

Moreover, the aforementioned tolerance levels can be selected independently of
the scales of the system; the results obtained by
\textcite{loken2017sensitivity} suggest that using the Dormand-Prince 8(7)
method with tolerance levels ranging from $10^{-10}$ to $10^{-5}$ are
sufficient in order for numerical round-off error to be the main
concern, compared to pure integration error, when computing hyperbolic LCSs
(albeit in two-dimensional systems). In addition, using an embedded ODE
solver meant that explicitly defining integration step sizes for the
two aforementioned transport processes (for the computation of strain
eigenvalues and -vectors, and for obtaining new mesh points, respectively) was
not required. This is significant, because the present approach already
involves several free parameters (see \cref{sec:comments_on_the_method_of%
_geodesic_level_sets}).

\textcite{loken2017sensitivity} showed that the Dormand-Prince 8(7) method
yields very accurate numerical approximations at a very low computational cost
--- at least, that is, for smooth flow systems (see
\cref{sec:the_type_of_flow_systems_considered}). As is apparent from
\cref{def:runge_kutta_order}, the accuracy of numerical solutions obtained by
using Runge-Kutta solvers depend on the order of the method itself, in addition
to the smoothness of the underlying function. Although generally more accurate,
higher-order ODE solvers yield increasingly diminishing returns compared to
their lower-order siblings when the order of the ODE solver exceeds the
function's number of smooth derivatives.

Thus, for gridded model data --- both regarding the oceanic currents of the
Førde fjord (\cref{sub:oceanic_currents_in_the_forde_fjord}), and the
discretely sampled Cauchy-Green strain eigenvalue and -vector fields
(\cref{sec:computing_the_flow_map_and_its_directional_derivatives,%
sec:computing_cauchy_green_strain_eigenvalues_and_vectors}) --- the
interpolation routine sets an upper bound in terms of the accuracy with which
LCSs can be computed. For more complex systems than the ones investigated here,
the interaction between the integration and interpolation schemes could be
critical; both in terms of numerical precision and computational resource
consumption. Independently of the scales at which well-resolved LCSs are sought
in a given transport system, the aforementioned effects warrant further
investigation, yet remain beyond the scope of this project.
