\section{Regarding our choice of numerical ODE solver}
\label{sec:regarding_our_choice_of_numerical_ode_solver}

As mentioned in \cref{sec:on_our_approach_to_computing_the_cauchy_green_strain%
_characteristics}, the Dormand-Prince 8(7) method was used for both the tracer
advection used to compute the Cauchy-Green strain eigenvalues and -vectors, and
computing new mesh points in the expansion of computed manifolds (as described
in  \cref{sec:computing_the_flow_map_and_its_directional_derivatives,%
sec:computing_cauchy_green_strain_eigenvalues_and_vectors} and
\cref{sec:revised_approach_to_computing_new_mesh_points}, respectively). Thus,
any propagation of numerical round-off errors can reasonably be expected to
have occurred in a consistent manner throughout. Moreover, courtesy of the
adaptive step size adjustment routine outlined in
\cref{sub:the_implementation_of_dynamic_runge_kutta_step_size}, using an
embedded Runge-Kutta solver allows for dispensing with a particular numerical
integration step size. Accordingly, the use of an embedded ODE solver --- like
the Dormand-Prince 8(7) method --- involves one less degree of freedom than
traditional singlestep methods such as the classical \nth{4}-order Runge-Kutta
method (that is, subject to limiting the step size from above in order to avoid
overshoots when computing manifold mesh points, briefly mentioned in
\cref{sec:revised_approach_to_computing_new_mesh_points}). This is
significant, because the present approach already involves a lot of free
parameters (see \cref{sec:comments_on_the_method_of_geodesic_%
level_sets}).

\textcite{loken2017sensitivity} showed that the Dormand-Prince 8(7) method
yields very accurate numerical approximations at a very low computational cost
--- at least, that is, for smooth flow systems (see
\cref{sec:the_type_of_flow_systems_considered}). As is apparent from
\cref{def:runge_kutta_order}, pertaining to the order of accuracy of
Runge-Kutta methods, the accuracy of numerical solutions obtained by using
Runge-Kutta solvers depend on the order of the method itself, in addition to
the smoothness of the underlying function. Although generally more accurate,
higher-order ODE solvers yield increasingly diminishing returns compared to
their lower-order siblings when the order of the ODE solver exceeds the
function's number of smooth derivatives.

Thus, for gridded model data --- both regarding the oceanic currents of the
Førde fjord (\cref{sub:oceanic_currents_in_the_forde_fjord}), and the
discretely sampled Cauchy-Green strain eigenvalue and -vector fields
(\cref{sec:computing_the_flow_map_and_its_directional_derivatives,%
sec:computing_cauchy_green_strain_eigenvalues_and_vectors}) --- the
interpolation routine sets an upper bound in terms of the accuracy with which
LCSs can be computed. For more complex systems than the ones investigated here,
the interaction between the integration and interpolation schemes could be
critical; both in terms of numerical precision and computational resource
consumption. Independently of the scales at which well-resolved LCSs are sought
in a given transport system, the aforementioned effects warrant further
investigation, yet remain beyond the scope of this project.
