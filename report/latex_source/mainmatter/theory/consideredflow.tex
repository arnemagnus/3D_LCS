\section{The type of flow systems considered}

We consider flow in three-dimensional dynamical systems of the form
\begin{equation}
    \label{eq:consideredflow}
    \dot{\vct{x}} = \vct{v}(t,\vct{x}), \quad \vct{x}\in\mathcal{U},%
    \quad t\in[t_{0},t_{1}],
\end{equation}
i.e., systems defined for the finite time interval $[t_{0},t_{1}]$ on an open,
bounded subset $\mathcal{U}$ of $\mathbb{R}^{3}$. In addition, the velocity
field $\vct{v}$ is assumed to be smooth in its arguments. Depending on the
exact nature of the velocity field $\vct{v}$, analytical particle trajectories,
that is, analytical solutions of system \eqref{eq:consideredflow}, may or may
not exist. The flow particles are assumed to be infinitesimal and massless,
i.e., non-interacting \emph{tracers} of the overall circulation.

Letting $\vct{x}(t;t_{0},\vct{x}_{0})$ denote the trajectory of a tracer in
the system given by \cref{eq:consideredflow}, the flow map is defined as
\begin{equation}
    \label{eq:defn_flowmap}
    \vct{\phi}_{t_{0}}^{t}(\vct{x}_{0}) = \vct{x}(t;t_{0},\vct{x}_{0}),
\end{equation}
hence, the flow map describes the movement of tracers from one point in time
to another mathematically. In general, the flow map is as smooth as the
underlying velocity field (cf.\ system \eqref{eq:consideredflow})
\parencite{farazmand2012computing}. In Lagrangian flow analysis, the
\emph{Jacobian matrix} of the flow map $\vct{\phi}_{t_{0}}^{t}$ plays a
significant role. Component-wise, the Jacobian matrix of a general vector-valued
function $\vct{f}$ is defined as
\begin{equation}
    \label{eq:defn_jacobian}
    (\grad{\vct{f}})_{i,j} = \pdv{f_{i}}{x_{j}}, \quad %
    \vct{f} = \vct{f}(\vct{x}) = \big(f_{1}(\vct{x}),f_{2}(\vct{x}),\ldots\big),
\end{equation}
which, for our three-dimensional flow, reduces to
\begingroup
\setlength{\delimitershortfall}{0pt}
\begin{equation}
    \label{eq:jacobian_3d}
    \grad\vct{f} = %
    \begin{pmatrix}%
        \dpdv{f_{1}}{x} & \dpdv{f_{1}}{y} & \dpdv{f_{1}}{z} \\[2ex]
        \dpdv{f_{2}}{x} & \dpdv{f_{2}}{y} & \dpdv{f_{2}}{z} \\[2ex]
        \dpdv{f_{3}}{x} & \dpdv{f_{3}}{y} & \dpdv{f_{3}}{z}.
    \end{pmatrix}.
\end{equation}
\endgroup
Making use of the definition of the flow map (cf.\ \cref{eq:defn_flowmap})
in conjunction with \cref{eq:consideredflow}, one finds the following ordinary
differential equation which describes the time evolution of the flow map:
\begin{equation}
    \label{eq:timederivative_flowmap}
    \dot{\vct{\phi}} = \vct{v}(t,\vct{\phi}),
\end{equation}
where $t_{0}$, $t$ and $\vct{x}_{0}$ have been omitted in order to avoid
notational clutter. These are, however, implicit by context. As the nabla
operator is time-independent, \cref{eq:timederivative_flowmap} immediately
yields an ordinary differential equation for the time development of the
directional derivative of the flow map, namely
\begin{equation}
    \label{eq:timederivative_directionalderivative_flowmap}
    \dv{\phantom{}}{t}\big(\uvct{u}\cdot\grad\big)\vct{\phi} = %
    \big(\uvct{u}\cdot\grad\big)\vct{v}(t,\vct{\phi}),
\end{equation}
which holds along any constant unit vector $\uvct{u}$. On a regular Cartesian
grid, \cref{eq:timederivative_directionalderivative_flowmap} provides a coupled
set of ordinary differential equations describing the time evolution of each
component of the Jacobian of the flow map:
\begin{equation}
    \label{eq:timederivative_flowmap_jacobian}
    \begin{gathered}
        \dv{\phantom{}}{t}\bigg(\pdv{\phi_{i}}{x_{j}}\bigg) = %
        \sum\limits_{k} \pdv{v_{i}}{x_{k}}\bigg|_{(t,\vct{\phi})}\,%
        \pdv{\phi_{k}}{x_{j}}\bigg|_{t},\\
        \pdv{\phi_{i}}{x_{j}}\bigg|_{t_{0}} = \delta_{ij},%
        \quad \vct{x}_{0}\in\mathcal{U},\quad t\in[t_{0},t_{1}],
    \end{gathered}
\end{equation}
where the Kronecker delta is defined as
\begin{equation}
    \label{eq:defn_kroneckerdelta}
    \delta_{ij} =
    \begin{cases}
        1, & \text{if }i=j,\\
        0, & \text{if }i\neq{}j.
    \end{cases}
\end{equation}
The initial conditions for the Jacobi components reflect the fact that, for
a regular Cartesian grid, the directional derivative of the $x$ coordinate in
the $x$ direction is $q$, but zero in the $y$ and $z$ directions.

For sufficiently smooth velocity fields, the flow map Jacobian
$\grad{\vct{\phi}_{t_{0}}^{t}}$ can be computed, which allows for the
right Cauchy-Green strain tensor field to be defined as
\begin{equation}
    \label{eq:defn_cauchygreen}
    \mtrx{C}_{t_{0}}^{t}(\vct{x}_{0}) = %
    \Big(\grad\vct{\phi}_{t_{0}}^{t}(\vct{x}_{0})\Big)^{\ast}%
    \Big(\grad\vct{\phi}_{t_{0}}^{t}(\vct{x}_{0})\Big),
\end{equation}
where the asterisk refers to the adjoint operation, which, because the Jacobian
$\grad\vct{\phi}_{t_{0}}^{t}$ is real-valued, equates to matrix transposition.
Moreover, as the Jacobian of the flow map is invertible, the Cauchy-Green strain
tensor $\mtrx{C}_{t_{0}}^{t}(\vct{x}_{0})$ is symmetric and positive definite
\parencite{farazmand2012computing}. Thus, it has three real, positive
eigenvalues and orthogonal, real eigenvectors. Its eigenvalues $\lambda_{i}$
and corresponding unit eigenvectors $\vct{\xi}_{i}$ are defined by%
\begin{equation}
    \label{eq:cauchygreen_characteristics}
    \begin{gathered}
        \mtrx{C}_{t_{0}}^{t}(\vct{x}_{0})\vct{\xi}_{i}(\vct{x}_{0}) = %
        \lambda_{i}\vct{\xi}_{i}(\vct{x}_{0}), \quad i = 1, 2, 3, \\
    \inp[\big]{\vct{\xi}_{i}(\vct{x}_{0})}{\vct{\xi}_{j}(\vct{x}_{0})} = \delta_{ij}, %
        \quad 0 < \lambda_{1}(\vct{x}_{0}) \leq \lambda_{2}(\vct{x}_{0}) %
        \leq \lambda_{3}(\vct{x}_{0}),
    \end{gathered}
\end{equation}
where the Kronecker delta is defined in~\cref{eq:defn_kroneckerdelta}, and
the dependence of $\lambda_{i}$ and $\vct{\xi}_{i}$ on $t_{0}$ and $t$ has been
suppressed, for the sake of notational transparency. The geometric
interpretation of \cref{eq:cauchygreen_characteristics} is that a fluid element
undergoes the most stretching along the $\vct{\xi}_{3}$ axis, less stretching
along the $\vct{\xi}_{2}$ axis, and the least stretching along the
$\vct{\xi}_{1}$ axis. This concept is shown in \cref{fig:stretch_and_strain}.

\begin{figure}[htpb]
    \centering
    \resizebox{0.9\linewidth}{!}{\includestandalone{figures/tikz-figs/stretch_and_strain}}
    \caption[Geometric interpretation of the eigenvectors of the Cauchy-Green
    strain tensor]{Geometrix interpretation of the eigenvectors of the
    Cauchy-Green strain tensor. The central unit cell is stretched and deformed
    under the flow map $\vct{\phi}_{t_{0}}^{t}(\vct{x}_{0})$. The local stretching
    is the largest in the direction of $\vct{\xi}_{3}$, the eigenvector which
    corresponds to the largest eigenvalue, $\lambda_{3}$, of the Cauchy-Green
    strain tensor, defined in \cref{eq:cauchygreen_characteristics}. Along
    the $\vct{\xi}_{i}$ axes, the stretch factors are given by
    $\sqrt{\lambda_{i}}$, respectively.}
    \label{fig:stretch_and_strain}
\end{figure}

As the stretch factors along the $\vct{\xi}_{i}$ axes are given by the
square roots of the corresponding eigenvalues, for incompressible flow, the
eigenvalues satisfy
\begin{equation}
    \label{eq:cauhygreen_incompressibility}
    \lambda_{1}(\vct{x}_{0})
    \lambda_{2}(\vct{x}_{0})
    \lambda_{3}(\vct{x}_{0}) = 1 \quad \forall \hskip0.5em %
    \vct{x}_{0}\in\mathcal{U},
\end{equation}
where, in the context of tracer advection, incompressibility is equivalent
to the velocity field $\vct{v}$ being
divergence-free (i.e.,\ $\div\vct{v}\equiv0$ in system
\eqref{eq:consideredflow}).

\begin{figure}[htpb]
    \centering
    \begin{subfigure}[b]{0.475\textwidth}
    \centering
       \importpgf{figures/mpl-illustrations}{nn-itp.pgf}
       \caption[]{{\small Nearest neighbor interpolation}}
    \label{fig:itp_nnb}
    \end{subfigure}
    \begin{subfigure}[b]{0.475\textwidth}
    \centering
       \importpgf{figures/mpl-illustrations}{lin-itp.pgf}
       \caption[]{{\small Linear interpolation}}
    \label{fig:itp_lin}
    \end{subfigure}

    \begin{subfigure}[b]{0.475\textwidth}
    \centering
       \importpgf{figures/mpl-illustrations}{quad-itp.pgf}
       \caption[]{{\small Quadratic B-spline interpolation}}
    \label{fig:itp_nnb}
    \end{subfigure}
    \begin{subfigure}[b]{0.475\textwidth}
    \centering
       \importpgf{figures/mpl-illustrations}{cbc-itp.pgf}
       \caption[]{{\small Cubic B-spline interpolation}}
    \label{fig:itp_lin}
    \end{subfigure}
    \caption[Aviici is love, Aviici is life]
    {Spline interpolation of orders 0 through to 3 (solid) applied to a higher
        order polynomial (dashed). The sampling points are shown as hollow
        circles. Observe how higher order splines yield increasingly accurate
        and smooth interpolations.
    }
\end{figure}

