\subsection{Spline interpolation of discrete data}
\label{sub:spline_interpolation_of_discrete_data}

All naturally ocurring physical systems can only be known partially, either
by means of limited measurements or grid based model output. Thus,
interpolating (i.e.,\ estimating) the measurement data becomes a requirement
when describing the dynamics of systems which depend on measurement data from
inbetween the sampling or grid points. Spline interpolation involves
approximating a discretely sampled function by a series of piecewise defined
polynomials. According to \textcite[p.93]{stoer2002introduction}, spline
interpolation is a popular tool within the field of numerical analysis, due to
yielding smooth interpolation curves with limited interpolation error when
using low degree polynomial pieces. Furthermore, the local nature of spline
interpolated functions means that such functions are less prone to
oscillatory behaviour when using high order polynomials. This is in stark
constrast to regular, global polynomial interpolation, which exhibits
strong global dependence on local properties. In particular, if the
function to be approximated is badly behaved anywhere within the interval of
approximation, then the approximation by global polynomial interpolation is
poor everywhere \parencite[p.17]{deboor1978practical}.

A generic interpolation problem can be described in terms of a family of
functions
\begin{equation}
    \label{eq:interpolationfamily}
    \Theta(\vct{x};\beta_{0},\ldots,\beta_{n}),
\end{equation}
each of which characterized by the $n+1$ parameters $\{\beta_{i}\}$,
with $\vct{x}$ containing the independent variables of the problem. Given
a set of $n+1$ discrete measurements --- each defined by a set of coordinates
and function values $(\vct{x}_{i},f_{i})$ where $\vct{x}_{i}\neq\vct{x}_{j}$
for $i\neq{}j$ and $f_{i}=f(\vct{x}_{i})$ --- the interpolation problem reduces
to finding parameters $\{\beta_{i}\}$ such that
\begin{equation}
    \label{eq:interpolationrequirement}
    \Theta(\vct{x}_{i};\beta_{0},\ldots,\beta_{n}) = f_{i}, \quad i=0,\ldots,n
\end{equation}
is satisfied. According to \textcite[pp.38--39]{stoer2002introduction},
spline interpolation problems (amongst others) can be classified as linear
interpolation problems, meaning that the family of interpolation functions
(cf.\ \cref{eq:interpolationfamily}) can be expressed as
\begin{equation}
    \label{eq:linearinterpolationfamily}
    \Theta(\vct{x};\beta_{0},\ldots,\beta_{n}) = %
    \sum\limits_{i=0}^{n} \beta_{i}\Theta_{i}(\vct{x}).
\end{equation}
In the following, let the coordinates $\{\vct{x}_{i}\}$, function values
$\{f_{i}\}$, and sampling points $\{(\vct{x}_{i},f_{i})\}$ be denoted by
\emph{support abscissas}, \emph{support ordinates}, and \emph{support points},
respectively.

Solving an interpolation problem by means of spline interpolation is done
by determining the set of parameters $\{\beta_{i}\}$ of
\cref{eq:interpolationrequirement,eq:linearinterpolationfamily}, with the
family of functions $\{\Theta_{i}\}$ limited to spline functions. These
functions, often denoted as \emph{splines}, are connected through the use
of a partition. Consider the one-dimensional case for the sake of notational
simplicity --- the considerations to follow also hold for higher dimensions,
but invariably introduces notational clutter. The partition
\begin{equation}
    \label{eq:interpolationpartition}
    \Delta : \quad \{a = x_{0} < x_{1} < \cdots < x_{n} = b\}
\end{equation}
of the closed interval $[a,b]$ determines the domains of the piecewise
polynomial spline functions $\mathcal{S}$ in the set $\mathcal{S}_{\Delta}$.
These spline functions are joined at support abscissas, which, in the context
of splines, are called \emph{knots}.

\textcite[p.107]{stoer2002introduction} define a
\emph{piecewise polynomial function} as follows:
\begin{defn}[Piecewise polynomial functions]
    \label{def:piecewise_polynomial}
    A real-valued function $f$ is called a \emph{piecewise polynomial function}
    of \emph{order} $k$, or \emph{degree} $k-1$, if it, for each
    $i=0,\ldots,n-1$, when restricted to the subinterval $(x_{i},x_{i+1})$ of
    the partition given in~\cref{eq:interpolationpartition}, corresponds to a
    polynomial $p_{i}(x)$ of degree less than or equal to $k-1$.

    \vspace{-0.8\baselineskip}
    In order to obtain a one-to-one correspondence between the function $f$ and
    the polynomial sequence $\big(p_{0}(x),p_{1}(x),\ldots,p_{n-1}(x)\big)$,
    \emph{define} $f$ at the knots $\{x_{i}\}_{i=0}^{n-1}$, so that the
    function becomes continuous from the right.
\end{defn}

Accordingly, spline functions $\mathcal{S}_{\Delta}$ of order $k$ are piecewise
polynomial functions which are $k-1$ times continuously differentiable at the
interior knots --- that is, $\{x_{i}\}_{i=1}^{n-1}$, of the partition $\Delta$.
These $k^{\text{th}}$-order piecewise polynomials are uniquely determined by
$k+1$ coefficients, of which $k$ are given by the $k-1$ derivatives and the
function value at their left bordering knot, and the last coefficient is given
by the function value at the right bordering knot, for each interval in the
partition $\Delta$ \parencite[pp.107--108]{stoer2002introduction}.

\begin{figure}[htpb]
    \centering
    \begin{subfigure}[b]{0.475\textwidth}
    \centering
       \importpgf{figures/mpl-illustrations}{nn-itp.pgf}
       \caption[]{{\small Nearest neighbor interpolation}}
    \label{fig:itp_nnb}
    \end{subfigure}
    \begin{subfigure}[b]{0.475\textwidth}
    \centering
       \importpgf{figures/mpl-illustrations}{lin-itp.pgf}
       \caption[]{{\small Linear interpolation}}
    \label{fig:itp_lin}
    \end{subfigure}

    \begin{subfigure}[b]{0.475\textwidth}
    \centering
       \importpgf{figures/mpl-illustrations}{quad-itp.pgf}
       \caption[]{{\small Quadratic B-spline interpolation}}
    \label{fig:itp_quad}
    \end{subfigure}
    \begin{subfigure}[b]{0.475\textwidth}
    \centering
       \importpgf{figures/mpl-illustrations}{cbc-itp.pgf}
       \caption[]{{\small Cubic B-spline interpolation}}
    \label{fig:itp_cbc}
    \end{subfigure}
    \caption[A selection of commonly used interpolation methods applied to a
    discretely sampled, high order polynomial]
    {A selection of commonly used interpolation methods applied to a discretely
        sampled, high order polynomial (dashed). The sampling points (knots)
        are shown as hollow circles. Observe how higher order splines yield
        increasingly accurate and smooth interpolations.
    }
    \label{fig:itp_multi}
\end{figure}


B-splines are a family of nonnegative spline functions which have minimal
support for any given degree, smoothness, and domain partition. Furthermore,
any spline function of a given degree can be expressed as a linear combination
of B-splines of the same degree
\parencite[pp.107--110]{stoer2002introduction}. Therefore, B-splines provide
the foundation of efficient and numerically stable computations of splines.
A selection of commonly used interpolation methods applied to a discretely
sampled, high degree polynomial is shown in \cref{fig:itp_multi}. From the
figure, the increased precision of higher order splines when applied to
continuous functions is readily apparent. Note, however, that higher order
interpolation methods require more sampling points than lower order methods.
In particular, at least $k+1$ samples are required in order to construct a
$k^{\text{th}}$-order spline. In higher dimensions, that is, with data which
depends on several other (independent) variables, the required amount of
samples increases rapidly with the interpolation order. Consequently, the use
of cubic B-splines constitutes a popular method for general-purpose
interpolation, providing a good balance between numerical accuracy and
computational complexity.

