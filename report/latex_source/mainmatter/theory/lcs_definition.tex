\section[Definition of Lagrangian coherent structures for three-dimensional
flows]{Definition of Lagrangian coherent structures for\\\phantom{2.2}
three-dimensional flows}

Lagrangian coherent structures (henceforth abbreviated to LCSs) can be described
as time-evolving surfaces which shape coherent trajectory patterns in dynamical
systems, defined over a finite time interval \parencite{haller2010variational}.
There are three main types of LCSs, namely \emph{elliptic}, \emph{hyperbolic}
and \emph{parabolic}. Rougly speaking, parabolic LCSs outline cores of jet-like
trajectories, elliptic LCSs describe vortex boundaries, whereas hyperbolic LCSs
are comprised of overall attractive or repelling manifolds. As such, hyperbolic
LCSs practically act as organizing centers of observable tracer patterns
\parencite{onu2015lcstool}. Because hyperbolic LCSs provide the most readily
applicable insight in terms of forecasting flow in e.g.\ oceanic currents,
such structures have been the focus of this project.

\subsection{Hyperbolic LCSs}
\label{sub:hyperbolic_lcss}

The identification of LCSs for reliable forecasting requires sufficiency and
necessity conditions, supported by mathematical theorems.
\textcite{haller2010variational} derived a variational LCS theory based on
the Cauchy-Green strain tensor, defined by \cref{eq:defn_cauchygreen}, from
which the aforementioned conditions follow. The immediately relevant parts
of \citeauthor{haller2010variational}'s theory are given in
\cref{def:normal_repellence,def:repelling_lcs,def:attracting_lcs,%
def:hyperbolic_lcs} \parencite{haller2010variational}.

\begin{defn}[Normally repellent material surfaces]
    \label{def:normal_repellence}
    A \emph{normally repellent material surface} over the time interval
    $[t_{0},t_{0}+T]$ is a compact material surface segment $\mathcal{M}(t)$
    which is overall repelling, and on which the normal repulsion rate is
    greater than the tangential repulsion rate.
\end{defn}

A \emph{material surface} is a smooth surface $\mathcal{M}(t_{0})$ at time $t_{0}$,
which is advected by the flow map, given by \cref{eq:defn_flowmap}, into a
dynamic material line
$\mathcal{M}(t) = \vct{F}_{t_{0}}^{t}\big(\mathcal{M}(t_{0})\big)$. The
required \emph{compactness} of the material surface segment signifies that, in
some sense, it must be topologically well-behaved. That the material surface is
\emph{overall repelling} means that nearby trajectories are repelled from,
rather than attracted towards, the material surface. Lastly, requiring that the
\emph{normal} repulsion rate is greater than the \emph{tangential} repulsion
rate means that nearby trajectories are in fact driven away from the material
surface, rather than being stretched along with it due to shear stress.

\begin{defn}[Repelling LCS]
    \label{def:repelling_lcs}
    A \emph{repelling LCS} over the time interval $[t_{0},t_{0}+T]$ is a
    normally repelling material surface $\mathcal{M}(t_{0})$ whose normal repulsion
    admits a pointwise non-degenerate maximum relative to any nearby material
    surface $\widehat{\mathcal{M}}(t_{0})$.
\end{defn}

\begin{defn}[Attracting LCS]
    \label{def:attracting_lcs}
    An \emph{attracting LCS} over the time interval $[t_{0},t_{0}+T]$ is defined
    as a repelling LCS over the \emph{backward} time interval $[t_{0}+T,t_{0}]$.
\end{defn}

\begin{defn}[Hyperbolic LCS]
    \label{def:hyperbolic_lcs}
    A \emph{hyperbolic LCS} over the time interval $[t_{0},t_{0}+T]$ is a
    \emph{repelling} or \emph{attracting} LCS over the same time interval.
\end{defn}

Note that the above definitions associate LCSs with the time interval $I$ over
which the dynamical system under consideration is known, or, at the very least,
where information regarding the behaviour of tracers, is sought. Generally,
LCSs obtained over a time interval $I$ do not necessarily exist over different
time intervals \parencite{farazmand2012computing}.

For sufficiently smooth three-dimensional flow, the above definitions can be
summarized as a set of mathematical existence criteria, based on the
Cauchy-Green strain tensor
\parencite{haller2010variational,farazmand2012computing,karrasch2012comment,%
farazmand2011erratum}.
These are given in \cref{thm:lcs_conditions}.

\begin{thm}[Sufficient and necessary conditions for LCSs in
    three-dimensional flows]
    \label{thm:lcs_conditions}
    Consider a compact material surface $\mathcal{M}(t)\subset\mathcal{U}$
    evolving over the time interal $[t_{0},t_{0}+T]$. Then $\mathcal{M}(t)$
    is a repelling LCS over $[t_{0},t_{0}+T]$ if and only if all of the
    following holds for all initial conditions
    $\vct{x}_{0}\in\mathcal{M}(t_{0})$:
    \begin{subequations}
        \label{eq:lcs_conditions}
        \begin{align}
            \label{eq:lcs_condition_a}
            &\lambda_{2}(\vct{x}_{0}) \neq \lambda_{3}(\vct{x}_{0}) > 1,\\
            \label{eq:lcs_condition_b}
            &\Big\langle\vct{\xi}_{3}(\vct{x}_{0},%
        \mtrx{H}_{\lambda_{3}}(\vct{x}_{0})\vct{\xi}_{3}(\vct{x}_{0})\Big\rangle
        < 0,\\
            \label{eq:lcs_condition_c}
            &\vct{\xi}_{3}(\vct{x}_{0}) \perp \mathcal{M}(t_{0}),\\
            \label{eq:lcs_condition_d}
            & \big\langle\grad\lambda_{3}(\vct{x}_{0}),%
        \vct{\xi}_{3}(\vct{x}_{0})\big\rangle = 0.
        \end{align}
    \end{subequations}
\end{thm}

In \cref{thm:lcs_conditions}, $\langle\mathord{\cdot},\mathord{\cdot}\rangle$
denotes the Euclidean inner product, and $\mtrx{H}_{\lambda_{3}}$ denotes the
Hessian matrix of the largest eigenvalues of the Cauchy-Green strain tensor
field. Component-wise, the Hessian matrix of a general, smooth, scalar-valued
function $f$ is defined as
\begin{equation}
    \label{eq:defn_hessian}
    {(\mtrx{H}_{f})}_{i,j} = \pdv[2]{f}{x_{i}}{x_{j}},
\end{equation}
which, for our three-dimensional flow, reduces to
\begingroup
\setlength{\delimitershortfall}{0pt}
\begin{equation}
    \label{eq:hessian_3d}
    \mtrx{H}_{f} = %
    \begin{pmatrix}
        \dfrac{\partial{}^{2}f}{\partial{}x^{2}} &
        \dfrac{\partial{}^{2}f}{\partial{}x\partial{}y} &
        \dfrac{\partial{}^{2}f}{\partial{}x\partial{}z} \\[2ex]
        \dfrac{\partial{}^{2}f}{\partial{}y\partial{}x} &
        \dfrac{\partial{}^{2}f}{\partial{}y^{2}} &
        \dfrac{\partial{}^{2}f}{\partial{}y\partial{}z} \\[2ex]
        \dfrac{\partial{}^{2}f}{\partial{}z\partial{}x} &
        \dfrac{\partial{}^{2}f}{\partial{}z\partial{}y} &
        \dfrac{\partial{}^{2}f}{\partial{}z^{2}}
    \end{pmatrix}.
\end{equation}
\endgroup

Condition~\eqref{eq:lcs_condition_a} ensures that the normal repulsion rate is
larger than the tangential stretch due to shear strain along the LCS, in
accordance with \cref{def:normal_repellence}. Conditions%
~\eqref{eq:lcs_condition_c} and~\eqref{eq:lcs_condition_d} suffice to enforce
that the normal repulsion rate attains a local extremum along the LCS, relative
to all nearby material surfaces. Lastly, condition~\eqref{eq:lcs_condition_b}
ensures that this is a strict local maximum.
