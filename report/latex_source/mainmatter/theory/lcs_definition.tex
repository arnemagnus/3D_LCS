\section[Definition of Lagrangian coherent structures for three-dimensional
flows]{Definition of Lagrangian coherent structures for\\\phantom{2.2}
three-dimensional flows}
\label{sec:definition_of_lagrangian_coherent_structures_for_three_dimensional_flows}

A necessary prerequisite for three-dimensional Lagrangian flow analysis is the
concept of \emph{material surfaces}, which \textcite{oettinger2016autonomous}
define as

\begin{defn}[Material surfaces]
    \label{def:material_surface}
    Consider a set of initial positions forming a two-dimensional surface
    $\mathcal{M}(t_{0})$ at time $t_{0}$ in $\mathcal{U}$. Its time-$t$ image,
    $\mathcal{M}(t)$, is obtained under the flow map as
    \begin{equation}
        \label{eq:advected_materialsurface}
        \mathcal{M}(t) = \vct{\phi}_{t_{0}}^{t}\big(\mathcal{M}(t_{0})\big).
    \end{equation}
    The union of \emph{all} time-$t$ images,
    $\cup_{t\in[t_{0},t_{1}]}\mathcal{M}(t)$, is a hypersurface in the
    extended phase space $\mathcal{U}\times\mathcal{I}$, called a
    \emph{material surface}.
\end{defn}
In the following, the entire material surface will be referred to by the
notation $\mathcal{M}(t)$. Although no material surfaces can be crossed by
tracers, only special material surfaces create coherence in the phase space
$\mathcal{U}$ and thus act as observable transport barriers. Such material
surfaces are generally referred to as \emph{Lagrangian coherent structures}
(henceforth abbreviated to LCSs).

Subsequently, LCSs can be described as time-evolving surfaces which shape
coherent trajectory patterns in dynamical systems, defined over a finite time
interval \parencite{haller2011variational}. There are three main types of LCSs,
namely \emph{elliptic}, \emph{hyperbolic} and \emph{parabolic}. Rougly speaking,
parabolic LCSs outline cores of jet-like trajectories, elliptic LCSs describe
vortex boundaries, whereas hyperbolic LCSs are comprised of overall attractive
or repelling manifolds. As such, hyperbolic LCSs practically act as organizing
centers of observable tracer patterns \parencite{onu2015lcstool}. Because
hyperbolic LCSs provide the most readily applicable insight in terms of
forecasting flow in e.g.\ oceanic currents, such structures have been the focus
of this project.

\subsection{Hyperbolic LCSs in three dimensions}
\label{sub:hyperbolic_lcss_in_three_dimensions}

The identification of LCSs for reliable forecasting requires sufficiency and
necessity conditions, supported by mathematical theorems.
\textcite{haller2011variational} derived a variational LCS theory based on
the Cauchy-Green strain tensor, defined by \cref{eq:defn_cauchygreen}, from
which the aforementioned conditions follow. The immediately relevant parts
of \citeauthor{haller2011variational}'s theory are given in
\cref{def:normal_repellence,def:repelling_lcs,def:attracting_lcs,%
def:hyperbolic_lcs} \parencite{haller2011variational}.

\begin{defn}[Normally repellent material surfaces]
    \label{def:normal_repellence}
    A \emph{normally repellent material surface} over the time interval
    $[t_{0},t_{1}]$ is a compact material surface segment $\mathcal{M}(t)$
    which is overall repelling, and on which the normal repulsion rate is
    greater than the tangential repulsion rate.
\end{defn}

The required \emph{compactness} of the material surface segment signifies that,
in some sense, it must be topologically well-behaved. That the material surface
is \emph{overall repelling} means that nearby trajectories are repelled from,
rather than attracted towards, the material surface. Lastly, requiring that the
\emph{normal} repulsion rate is greater than the \emph{tangential} repulsion
rate means that nearby trajectories are in fact driven away from the material
surface, rather than being stretched along with it due to shear stress.

\begin{defn}[Repelling LCS]
    \label{def:repelling_lcs}
    A \emph{repelling LCS} over the time interval $[t_{0},t_{1}]$ is a
    normally repelling material surface $\mathcal{M}(t_{0})$ whose normal repulsion
    admits a pointwise non-degenerate maximum relative to any nearby material
    surface $\widehat{\mathcal{M}}(t_{0})$.
\end{defn}

\begin{defn}[Attracting LCS]
    \label{def:attracting_lcs}
    An \emph{attracting LCS} over the time interval $[t_{0},t_{1}]$ is defined
    as a repelling LCS over the \emph{backward} time interval $[t_{1},t_{0}]$.
\end{defn}

\begin{defn}[Hyperbolic LCS]
    \label{def:hyperbolic_lcs}
    A \emph{hyperbolic LCS} over the time interval $[t_{0},t_{1}]$ is a
    \emph{repelling} or \emph{attracting} LCS over the same time interval.
\end{defn}

Note that the above definitions associate LCSs with the time interval
$\mathcal{I}$ over which the dynamical system under consideration is known, or,
at the very least, where information regarding the behaviour of tracers, is
sought. Generally, LCSs obtained over a time interval $\mathcal{I}$ do not
necessarily exist over different time intervals
\parencite{farazmand2012computing}.

For sufficiently smooth three-dimensional flow, the above definitions can be
summarized as a set of mathematical existence criteria, based on the
Cauchy-Green strain tensor (cf.\ \cref{eq:defn_cauchygreen})
\parencite{haller2011variational,farazmand2012computing,karrasch2012comment,%
farazmand2012erratum}.
These are given in \cref{thm:lcs_conditions}.

\begin{thm}[Sufficient and necessary conditions for LCSs in
    three-dimensional flows]
    \label{thm:lcs_conditions}
    Consider a compact material surface $\mathcal{M}(t)\subset\mathcal{U}$
    evolving over the time interval $[t_{0},t_{1}]$. Then $\mathcal{M}(t)$
    is a repelling LCS over $[t_{0},t_{1}]$ if and only if all of the
    following holds for all initial conditions
    $\vct{x}_{0}\in\mathcal{M}(t_{0})$:
    \begin{subequations}
        \label{eq:lcs_conditions}
        \begin{align}
            \label{eq:lcs_condition_a}
            &\lambda_{2}(\vct{x}_{0}) \neq \lambda_{3}(\vct{x}_{0}) > 1,\\
            \label{eq:lcs_condition_b}
            &\inp[\Big]{\vct{\xi}_{3}(\vct{x}_{0})}%
            {\mtrx{H}_{\lambda_{3}}(\vct{x}_{0})\vct{\xi}_{3}(\vct{x}_{0})} < 0 \\
            \label{eq:lcs_condition_c}
            &\vct{\xi}_{3}(\vct{x}_{0}) \perp \mathcal{M}(t_{0}),\\
            \label{eq:lcs_condition_d}
            &\inp[\big]{\grad\lambda_{3}(\vct{x}_{0})}%
            {\vct{\xi}_{3}(\vct{x}_{0})} = 0.
        \end{align}
    \end{subequations}
\end{thm}

In \cref{thm:lcs_conditions}, $\inp{\cdot}{\cdot}$ signifies the Euclidean inner
product, and $\mtrx{H}_{\lambda_{3}}$ denotes the Hessian matrix of the largest
eigenvalues of the Cauchy-Green strain tensor field. Component-wise, the Hessian
matrix of a general, smooth, scalar-valued function $f$ is defined as
\begin{equation}
    \label{eq:defn_hessian}
    {(\mtrx{H}_{f})}_{i,j} = \pdv[2]{f}{x_{i}}{x_{j}},
\end{equation}
which, for our three-dimensional flow, reduces to
\begingroup
\setlength{\delimitershortfall}{0pt}
\begin{equation}
    \label{eq:hessian_3d}
    \mtrx{H}_{f} = %
    \begin{pmatrix}
        \dpdv[2]{f}{x} & \dpdv{f}{x}{y} & \dpdv{f}{x}{z} \\[2ex]
        \dpdv{f}{y}{x} & \dpdv[2]{f}{y} & \dpdv{f}{y}{z} \\[2ex]
        \dpdv{f}{z}{x} & \dpdv{f}{z}{y} & \dpdv[2]{f}{z}
    \end{pmatrix}.
\end{equation}
\endgroup

Condition~\eqref{eq:lcs_condition_a} ensures that the normal repulsion rate is
larger than the tangential stretch due to shear strain along the LCS, in
accordance with \cref{def:normal_repellence}. Conditions%
~\eqref{eq:lcs_condition_c} and~\eqref{eq:lcs_condition_d} suffice to enforce
that the normal repulsion rate attains a local extremum along the LCS, relative
to all nearby material surfaces. Lastly, condition~\eqref{eq:lcs_condition_b}
ensures that this is a strict local maximum.

From condition~\eqref{eq:lcs_condition_c} and the orthormality of the
Cauchy-Green strain eigenvectors (cf.\ \cref{eq:cauchygreen_characteristics}),
it follows that any initial LCS surface is tangent to the planes locally
spanned by $\vct{\xi}_{1}(\vct{x}_{0})$ and $\vct{\xi}_{2}(\vct{x}_{0})$.
Thus, an integral curve of any (normalized) linear combination of the
$\vct{\xi}_{1}$- and $\vct{\xi}_{2}$-direction fields, launched from an
arbitrary point of the surface $\mathcal{M}(t_{0})$, will lever leave
$\mathcal{M}(t_{0})$. Hence:

\begin{rmrk}[Invariance of initial positions of repelling LCSs]
    \label{rmk:invariance_lcs}
    The initial position $\mathcal{M}(t_{0})$ of any repelling LCS
    (\cref{def:repelling_lcs}) is an invariant manifold of the autonomous
    dynamical system
    \begin{equation}
        \label{eq:dynamialsystem_initialposition}
        \vct{r}^{\prime} = a\vct{\xi}_{1}(\vct{r})+b\vct{\xi}_{2}(\vct{r}),%
        \quad a^{2}+b^{2}=1.
    \end{equation}
\end{rmrk}

\begingroup
\crefname{equation}{}{} % Suppress cref's <<equations>> locally.
Note that the converse of~\cref{rmk:invariance_lcs} does not hold. That is,
a material surface $\Xi(t_{0})$ which is an invariant manifold of all
(normalized) linear combinations of $\vct{\xi}_{1}$ and $\vct{\xi}_{2}$ does not
necessarily correspond to a repelling LCS $\mathcal{M}(t_{0})$ --- that is,
unless $\Xi(t_{0})$ also satisfies conditions%
\cref{eq:lcs_condition_a,eq:lcs_condition_b,eq:lcs_condition_d}.
\endgroup
