\subsection{The Runge-Kutta family of numerical ODE solvers}
\label{sub:the_runge_kutta_family_of_numerical_ode_solvers}

In numerical analysis, the Runge-Kutta family of methods is a popular collection
of implicit and explicit iterative methods, used in temporal discretization in
order to obtain numerical approximations of the \emph{true} solutions of systems
like~\eqref{eq:ivpsystem}. The German mathematicians C. Runge and M.W. Kutta
developed the first of the family's methods at the turn of the twentieth
century~\parencite[p.134]{hairer1993solving}. The general outline of what
is now known as a Runge-Kutta method is as follows:

\begin{defn}[Runge-Kutta methods]
    \label{def:runge_kutta_methods}
    Let $s$ be an integer and $\{a_{i,j}\}_{i,j=1}^{s}$, %_{\substack{i=1\\j=1}}^{s}$,
    $\{b_{i}\}_{i=1}^{s}$ and $\{c_{i}\}_{i=1}^{s}$ be real coefficients.\hfill\newline
    Let $h$ be the numerical step length used in the temporal discretization.\hfill\newline
    Then, the method
    \begin{equation}
        \label{eq:generalrungekutta}
        \begin{aligned}
        k_{i} &= f\bigg(t_{n} + c_{i}h, x_{n} + h\sum\limits_{j=1}^{s}a_{i,j}k_{j}\bigg), \quad{}i=1,\ldots,s,\\
        x_{n+1} &= x_{n} + h\sum\limits_{i=1}^{s}b_{i}k_{i},
        \end{aligned}
    \end{equation}
    is called an \emph{s-stage Runge-Kutta method} for the system
    \eqref{eq:ivpsystem}.
\end{defn}

The main reason to include multiple stages in a Runge-Kutta method is to
improve the numerical accuracy of the computed solutions. The \emph{order} of a
Runge-Kutta method can be defined as follows:

\begin{defn}[Order of Runge-Kutta methods]
    \label{def:runge_kutta_order}
    A Runge-Kutta method, given by~\cref{eq:generalrungekutta}, is of
    \emph{order p} if, for sufficiently smooth systems~\eqref{eq:ivpsystem},
    the local error $e_{n}$ scales as $h^{p+1}$. That is:
    \begin{equation}
        \label{eq:rungekuttaorder}
        e_{n} = \norm{x_{n}-u_{n-1}(t_{n})} \leq K\hspace{0.5ex}h^{p+1}
    \end{equation}
    where $u_{n-1}(t)$ is the exact solution of the ODE in system
    \eqref{eq:ivpsystem} at time $t$, subject to the initial condition
    $u_{n-1}(t_{n-1}) = x_{n-1}$, and $K$ is a numerical constant. This is true,
    if the Taylor series for the exact solution $u_{n-1}(t_{n})$ and the
    numerical solution $x_{n}$ coincide up to (and including) the term $h^{p}$.
\end{defn}%
The \emph{global} error
\begin{equation}
    \label{eq:globalrungekuttaerror}
    E_{n} = x_{n}-x(t_{n}),
\end{equation}
where $x(t)$ is the exact solution of system~\eqref{eq:ivpsystem} at time $t$,
accumulated by $n$ repeated applications of the numerical method, can be
estimated by
\begin{equation}
    \label{eq:globalrungekuttaerrorapprox}
    \abs{E_{n}} \leq C\sum\limits_{l=1}^{n}\abs{e_{l}},
\end{equation}
where $C$ is a numerical constant, depending on both the right hand side of the
ODE in system~\eqref{eq:ivpsystem} and the difference $t_{n}-t_{0}$. Making
use of~\cref{def:runge_kutta_order}, the global error is limited from above by
\begin{equation}
    \label{eq:globalrungekuttaerrorestimate}
    \begin{aligned}
        \abs{E_{n}} &\leq C\sum\limits_{l=1}^{n}\abs{e_{l}} %
        \leq C\sum\limits_{l=1}^{n}\abs{K_{l}}\hspace{0.5ex}h^{p+1}%
        \leq C\hspace{0.5ex}\max\limits_{l}\big\{\abs{K_{l}}\big\}\hspace{0.5ex}n\hspace{0.5ex}h^{p+1}\\
        &\leq C\hspace{0.5ex}\max\limits_{l}\big\{\abs{K_{l}}\big\}\hspace{0.5ex}\frac{t_{n}-t_{0}}{h}\hspace{0.5ex}h^{p+1}
        \leq \widetilde{K}\hspace{0.5ex}h^{p},
    \end{aligned}
\end{equation}
where $\widetilde{K}$ is a numerical constant.
\Cref{eq:globalrungekuttaerrorestimate} demonstrates that, for a \emph{p}-th
order Runge-Kutta method, the global error can be expected to scale as $h^{p}$.

In \cref{def:runge_kutta_methods}, the matrix $(a_{i,j})$ is commonly called
the \emph{Runge-Kutta matrix}, while the coefficients $\{b_{i}\}$ and
$\{c_{i}\}$ are known as the \emph{weights} and \emph{nodes}, respectively.
Since the 1960s, it has been customary to represent Runge-Kutta methods, given
by~\cref{eq:generalrungekutta}, symbolically, by means of mnemonic devices known
as Butcher tableaus \parencite[p.134]{hairer1993solving}. The Butcher tableau
for a general \emph{s}-stage Runge-Kutta method, as introduced in
\cref{def:runge_kutta_methods}, is presented in \cref{tab:generalbutcher}.
For explicit Runge-Kutta methods, the Runge-Kutta matrix $(a_{i,j})$ is lower
triangular. Similarly, for fully implicit Runge-Kutta methods, the Runge-Kutta
matrix is upper triangular. The difference between explicit and implicit
methods is outlined in~\cref{eq:exim}.

\begin{table}[htpb]
    \centering
    \caption[Butcher tableau representing a generic $s$-stage Runge-Kutta
    method]{Butcher tableau representing a generic $s$-stage Runge-Kutta
    method.}
    \label{tab:generalbutcher}
    \[\renewcommand{\arraystretch}{1.25}
        \begin{array}{c|cccc}
            \toprule
            c_{1} & a_{1,1} & a_{1,2} & \ldots & a_{1,s}\\
            c_{2} & a_{2,1} & a_{2,2} & \ldots & a_{2,s}\\
            \vdots & \vdots & \vdots & \ddots & \vdots \\
            c_{s} & a_{s,1} & a_{s,2} & \ldots & a_{s,s}\\
            \hline
            & b_{1} & b_{2} & \ldots & b_{s}\\
            \bottomrule
    \end{array}
\]
\end{table}


During the first half of the twentieth century, a substantial amount of research
was conducted in order to develop numerically robust, high-order, explicit
Runge-Kutta methods. The idea was that using such methods would mean one could
resort to larger time increments $h$ without sacrificing precision in the
computed solution. However, the required number of stages $s$ grows quicker than
linearly as a function of the required order $p$. It has been proven that, for
$p\geq5$, no explicit Runge-Kutta method of order $p$ with $s=p$ stages exists
\parencite[p.173]{hairer1993solving}. This is one of the reasons for the
attention shift from the latter half of the 1950s and onwards, towards so-called
\emph{embedded} Runge-Kutta methods.

The basic idea of embedded Runge-Kutta methods is that they, aside from the
numerical approximation $x_{n+1}$, yield a second approximation
$\widehat{x}_{n+1}$. The difference between the two approximations then
provides an estimate of the local error of the less precise result, which can
be used for automatic step size control
\parencite[pp.167--168]{hairer1993solving}. The trick is to construct two
independent, explicit Runge-Kutta methods which both use the \emph{same}
function evaluations. This results in practically obtaining the two solutions
for the price of one, in terms of computational complexity. The Butcher tableau
of a generic, embedded, explicit Runge-Kutta method is illustrated in
\cref{tab:genericembeddedbutcher}.

For embedded methods, the coefficients are tuned such that
\begin{subequations}
    \label{eq:embedded_runge_kutta_solutions}
    \begin{equation}
        \label{eq:embeddedsol}
        x_{n+1} = x_{n} + h\sum\limits_{i=1}^{s}b_{i}k_{i}
    \end{equation}
    is of order $p$, and
    \begin{equation}
        \label{eq:embeddedinterp}
        \widehat{x}_{n+1} = x_{n} + h\sum\limits_{i=1}^{s}\widehat{b}_{i}k_{i}
    \end{equation}
\end{subequations}
is of order $\widehat{p}$, typically with $\widehat{p} = p + 1$. Which of the
solutions is used to continue the numerical integration, depends on the
integration method in question. In the following, the solution which is
\emph{not} used to continue the integration, will be referred to as the
\emph{interpolant} solution.

\begin{table}[htpb]
    \centering
    \caption[Butcher tableau representation of a generic, embedded, explicit
    Runge-Kutta method]{Butcher tableau representation a generic, embedded, explicit Runge-Kutta method.}
    \label{tab:genericembeddedbutcher}
    \[\renewcommand{\arraystretch}{1.25}
    \begin{array}{c|ccccc}
    \toprule
    0 \\
    c_{2} & a_{2,1} \\
    c_{3} & a_{3,1} & a_{3,2} \\
    \vdots & \vdots & \vdots & \ddots\\
    c_{s} & a_{s,1} & a_{s,2} & \ldots & a_{s,s-1}\\
    \hline
    & b_{1} & b_{2} & \ldots & b_{s-1} & b_{s} \\
    \hline
    & \widehat{b}_{1} & \widehat{b}_{2} & \ldots & \widehat{b}_{s-1}& \widehat{b}_{s}\\
    \bottomrule
    \end{array}
\]
\end{table}


There exists an abundance of Runge-Kutta methods; many of which are fine-tuned
for specific constraints, such as problems of varying degrees of stiffness.
Based on prior investigations --- such as the work done by
\textcite{loken2017sensitivity} --- using explicit, high order, embedded
Runge-Kutta methods to compute Lagrangian coherent structures
(which will be elaborated upon in
\cref{sec:definition_of_lagrangian_coherent_structures_for_three_dimensional_flows})
consistently yields accurate solutions at lower computational cost than
the most common fixed stepsize methods. Accordingly, the Dormand-Prince 8(7)
method --- consisting of an eighth order solution with a seventh order
interpolant --- was chosen as the single, multipurpose, numerical ODE solver
for this project.

Note that the concept of \emph{order} is less well-defined for embedded
methods than for fixed stepsize methods, as a direct consequence of the
adaptive time step. Although the \emph{local} errors of each integration
step scale as per~\cref{eq:rungekuttaorder}, the bound on the \emph{global}
(i.e.,\ observable) error suggested in~\cref{eq:globalrungekuttaerrorestimate}
is invalid, as the time step is, in principle, different for each integration
step.

Butcher tableau representations of the classical \nth{4}-order Runge-Kutta
method and the embedded Dormand-Prince 8(7) method are available in~
\cref{tab:butcherrk4,tab:butcherdopri87}; where the latter has been typeset in
landscape mode for the reader's convenience. Details on how the dynamic time
step of the Dormand-Prince 8(7) method was implemented will be presented in
\cref{sub:the_implementation_of_dynamic_runge_kutta_step_size}.

\begin{table}[htpb]
    \centering
    \caption[Butcher tableau representation of the explicit, classical
    Runge-Kutta method]
    {Butcher tableau representation of the explicit, classical
        Runge-Kutta method.}
    \label{tab:butcherrk4}
    \renewcommand{\arraystretch}{2.5}
    \begin{tabular}{C|CCCC}
    \toprule
    0 \\
    \dfrac{1}{2} & \dfrac{1}{2} \\
    \dfrac{1}{2} & 0 & \dfrac{1}{2} \\
    1 & 0 & 0 & 1 \\
    \hline
    & \dfrac{1}{6} & \dfrac{1}{3} & \dfrac{1}{3} & \dfrac{1}{6}\\
    \bottomrule
    \end{tabular}
\end{table}

\begingroup
\captionsetup[table]{skip=3pt}
\begin{sidewaystable}
    \centering
    \tiny
    \caption[Butcher tableau for the Dormand-Prince 8(7) embedded
    Runge-Kutta method]
    {Butcher tableau for the Dormand-Prince 8(7) embedded Runge-Kutta method.
        The $\widehat{b}$ coefficients give an \nth{8}-order accurate solution
        used to continue the integration. The ${b}$ coefficients yield a
        \nth{7}-order interpolant, which can be used to estimate the error of
        the numerical approximation, and to dynamically adjust the time step.
        The provided coefficients are rational approximations, accurate to
        about 24 significant decimal digits. For reference, see
    \textcite{prince1981highorder}.}
    \label{tab:butcherdopri87}
    \resizebox*{0.95\textwidth}{!}{%
        \renewcommand{\arraystretch}{2.25}
        \begin{tabular}{C|CCCCCCCCCCCCC}
            \toprule
            0 \\
            \dfrac{1}{18} & \dfrac{1}{18} \\
            \dfrac{1}{12} & \dfrac{1}{48} & \dfrac{1}{16} \\
            \dfrac{1}{8} & \dfrac{1}{32} & 0 & \dfrac{3}{32} \\
            \dfrac{5}{16} & \dfrac{5}{16} & 0 & \dfrac{-75}{64} & \dfrac{75}{64} \\
            \dfrac{3}{8} & \dfrac{3}{80} & 0 & 0 & \dfrac{3}{16} & \dfrac{3}{20} \\
            \dfrac{59}{400} & \dfrac{29443841}{614563906} & 0 & 0 %
                            & \dfrac{77736358}{692538347} %
                            & \dfrac{-28693883}{1125000000} %
                            & \dfrac{23124283}{1800000000} \\
            \dfrac{93}{200} & \dfrac{16016141}{946692911} & 0 & 0 %
                            & \dfrac{61564180}{158732637} %
                            & \dfrac{22789713}{633445777} %
                            & \dfrac{545815736}{2771057229} %
                            & \dfrac{-180193667}{1043307555} \\
            \dfrac{5490023248}{9719169821} & \dfrac{39632708}{573591083} & 0 & 0 %
                                           & \dfrac{-433636366}{683701615} %
                                           & \dfrac{-421739975}{2616292301} %
                                           & \dfrac{100302831}{723423059} %
                                           & \dfrac{790204164}{839813087} %
                                           & \dfrac{800635310}{3783071287} \\
            \dfrac{13}{20} & \dfrac{246121993}{1340847787} & 0 & 0 %
                           & \dfrac{-37695042795}{15268766246} %
                           & \dfrac{-309121744}{1061227803} %
                           & \dfrac{-12992083}{490766935} %
                           & \dfrac{6005943493}{2108947689} %
                           & \dfrac{393006217}{1396673457} %
                           & \dfrac{123872331}{1001029789} \\
            \dfrac{1201146811}{1299019798} & \dfrac{-1028468189}{846180014} & 0 & 0 %
                                           & \dfrac{8478235783}{508512851} %
                                           & \dfrac{1311729495}{1432422823} %
                                           & \dfrac{-10304129995}{1701304382} %
                                           & \dfrac{-48777925059}{3047939560} %
                                           & \dfrac{15336726248}{1032824649} %
                                           & \dfrac{-45442868181}{339846796} %
                                           & \dfrac{3065993473}{5917172653} \tabularnewline
            1 & \dfrac{185892177}{718116043} & 0 & 0 & \dfrac{-3185094517}{667107341} %
              & \dfrac{-477755414}{1098053517} & \dfrac{-703635378}{230739211} %
              & \dfrac{5731566787}{1027545527} & \dfrac{5232866602}{850066563} %
              & \dfrac{-4093664535}{808688257} & \dfrac{3962137247}{1805957418} %
              & \dfrac{65686358}{487910083} \tabularnewline
            1 & \dfrac{403863854}{491063109} & 0 & 0 & \dfrac{-5068492393}{434740067} %
              & \dfrac{-411421997}{543043805} & \dfrac{652783627}{914296604} %
              & \dfrac{11173962825}{925320556} & \dfrac{-13158990841}{6184727034} %
              & \dfrac{3936647629}{1978049680} & \dfrac{-160528059}{685178525} %
              & \dfrac{248638103}{1413531060} & 0 \tabularnewline
            \hline
            & \dfrac{13451932}{455176623} & 0 & 0 & 0 & 0 %
            & \dfrac{-808719486}{976000145} & \dfrac{1757004468}{5645159321} %
            & \dfrac{656045339}{265891186} & \dfrac{-3867574721}{1518517206} %
            & \dfrac{465885868}{322736535} & \dfrac{53011238}{667516719} %
            & \dfrac{2}{45} & 0\tabularnewline
            \hline
            & \dfrac{1400451}{335480064} & 0 & 0 & 0 & 0 %
            & \dfrac{-59238493}{1068277825} & \dfrac{181606767}{758867731} %
            & \dfrac{56129285}{797845732} & \dfrac{-1041891430}{1371343529} %
            & \dfrac{760417239}{1151165299} & \dfrac{118820643}{751138087} %
            & \dfrac{-528747749}{2220607170} & \dfrac{1}{4}    \tabularnewline
            \bottomrule
    \end{tabular}}
\end{sidewaystable}
\endgroup

