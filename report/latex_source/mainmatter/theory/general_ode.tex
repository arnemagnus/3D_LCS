\section[Solving systems of ordinary differential equations]%
{Solving systems of ordinary differential eq\null{}uations}%
\label{sec:solving_systems_of_ordinary_differential_equations}

In physics, like other sciences, modeling a system often equates to solving an
initial value problem. An initial value problem can be described in terms of an
ordinary differential equation (hereafter abbreviated to ODE) of the form
\begin{equation}
    \label{eq:ivpsystem}
    \dot{x}(t) = f\big(t,x(t)\big), \quad x(t_{0}) = x_{0},
\end{equation}
where $x$ is an unknown function (scalar or vector) of time $t$. The function
$f$ is defined on an open subset $\Omega$ of $\mathbb{R}\times\mathbb{R}^{n}$,
where $n$ is the number of spatial dimensions; that is, the number of
components of $x$. The initial condition $(t_{0},x_{0})$ is a point in the
domain of $f$, i.e.,\ $(t_{0},x_{0})\in\Omega$. In higher dimensions (namely,
$n>1$), the differential \cref{eq:ivpsystem} generally extends to a coupled
family of ODEs
\begin{equation}
    \label{eq:ivpsystemhigherdimensions}
    \dot{x}_{i}(t) = f_{i}\big(t,x_{1}(t),x_{2}(t),\ldots,x_{n}(t)\big),%
    \quad x_{i}(t_{0}) = x_{i,0}, \quad i = 1,\ldots,n.
\end{equation}
The system is nonlinear if the function $f$ in \cref{eq:ivpsystem}, or, if at
least one of the functions $\{f_{i}\}$ in \cref{eq:ivpsystemhigherdimensions},
is nonlinear in one or more of its arguments. For the sake of notational
simplicity, the discussion to follow in the rest of this section is based
on the one-dimensional case, that is, system~\eqref{eq:ivpsystem}, for $n=1$.
However, all of the considerations also hold for $n>1$.

Say that the solution of system \eqref{eq:ivpsystem} is sought at some time
$t_{f}$. In order to approximate said solution numerically, the time variable
must first be discretized. This is frequently done by defining
\begin{equation}
    \label{eq:discretime}
    t_{j} = t_{0}+j\cdot{}h,
\end{equation}
where $t_{j}$ is the time level $j$ for integer $j$, and $h$ is some increment
which is smaller than $t_{f}-t_{0}$. Typically, the time increment is chosen
such that an integer number of step lengts $h$ equals the difference
$t_{f}-t_{0}$. With the discretized time, the numerical solution of
system~\eqref{eq:ivpsystem} is found by successive applications of some
numerical integration method. The Runge-Kutta family of numerical methods for
ODE systems is a common choice, and will be elaborated upon in greater detail
in~\cref{sub:the_runge_kutta_family_of_numerical_ode_solvers}.

All numerical integration schemes fall into one of two categories; explicit
and implicit methods. Explicit methods are characterized by computing the state
of the system at a later time, based on the state of the system at the current
time (in some cases, the state at earlier times are also considered). Implicit
methods, however, involve the solution of an equation in which both the
current and the later state of the system are involved. Thus, a generic,
explicit method for computing the state of the system at time $t+h$, given its
state at $t$, can be expressed as
\begin{subequations}
    \label{eq:exim}
    \begin{align}
        \label{eq:exim_ex}
        x(t+h) &= F\big(x(t)\big),\\
        \intertext{while, for implicit methods, an equation of the sort}
        \label{eq:exim_im}
        G\big(x(t),x(t+h)\big)&=0,
    \end{align}
\end{subequations}
is solved to find $x(t+h)$.

In solving linear ODEs, implicit methods require the solution of a linear
system at every time step. Typically, implicit methods are more computationally
demanding than explicit methods. The main selling point of implicit methods is
that they are more numerically stable than explicit methods. This property
means that implicit methods are particularly well-suited for \emph{stiff}
systems, i.e., physical systems with highly disparate time scales
\parencite[p.2]{hairer1996solving}. For such systems, most explicit methods are
unstable, unless the time step $h$ is made exceptionally small, rendering these
methods practically useless. For \emph{nonstiff} systems, however, implicit
methods behave similarly to their explicit analogues in terms of numerical
accuracy and convergence properties.

Irrespective of which numerical integration method is employed, one obtains
an approximation of the true solution of the system~\eqref{eq:ivpsystem}
\emph{at} the discrete time levels, that is,
\begin{equation}
    \label{eq:num_int_approx_sol}
    x_{j} \approx x(t_{j}),
\end{equation}
where $x(t)$ is the exact solution at time $t$. The accuracy of the
approximation, however, depends on both the numerical integration method and
the time step length $h$ used for the temporal discretization. One way
of obtaining approximations of the true solution \emph{inbetween} the discrete
time levels is by means of interpolation --- a numerical technique which will
be elaborated upon in~\cref{sub:spline_interpolation_of_discrete_data}. For
nonlinear systems, analytical solutions usually do not exist. Thus, such
systems are often analyzed by means of numerical methods.
