In three dimensions, the matrix $\mtrx{L}$ is defined as
\begin{equation}
    \renewcommand{\arraystretch}{2.5}
\mtrx{L} = \begin{pmatrix}
        \vct{\nabla}^{2}\mtrx{C}^{-1}\big[\vct{\xi}_{3},\vct{\xi}_{3},\vct{\xi}_{3},\vct{\xi}_{3}\big] %
        & 2\dfrac{\lambda_{3}-\lambda_{1}}{\lambda_{1}\lambda_{3}}\big\langle{\vct{\xi}_{1}},{(\vct{\nabla}\vct{\xi}_{3})\vct{\xi}_{3}}\big\rangle %
        & 2\dfrac{\lambda_{3}-\lambda_{2}}{\lambda_{2}\lambda_{3}}\big\langle{\vct{\xi}_{2}},{(\vct{\nabla}\vct{\xi}_{3})\vct{\xi}_{3}}\big\rangle \\
        2\dfrac{\lambda_{3}-\lambda_{1}}{\lambda_{1}\lambda_{3}}\big\langle{\vct{\xi}_{1}},{(\vct{\nabla}\vct{\xi}_{3})\vct{\xi}_{3}}\big\rangle %
        & 2\dfrac{\lambda_{3}-\lambda_{1}}{\lambda_{1}\lambda_{3}} %
        & 0 \\
        2\dfrac{\lambda_{3}-\lambda_{2}}{\lambda_{2}\lambda_{3}}\big\langle{\vct{\xi}_{2}},{(\vct{\nabla}\vct{\xi}_{3})\vct{\xi}_{3}}\big\rangle %
        & 0 %
        & 2\dfrac{\lambda_{3}-\lambda_{2}}{\lambda_{2}\lambda_{3}}
    \end{pmatrix},
\end{equation}

where the dependence of $\mtrx{L}$ and its constituent parts on $t$, $t_{0}$
and $\vct{x}_{0}$ has been omitted for brevity. The distinction between
so-called strong and weak hyperbolic LCSs is based upon whether or not
$\mtrx{L}$ is positive definite
\parencite{haller2010variational,farazmand2011erratum}. By Sylvester's
theorem, a square matrix is positive definite if and only if all of the
leading principal minors of $\mtrx{L}$ are positive. The term
\emph{leading principal minors} is defined as follows
\parencite{gilbert1991positive}:
\begin{defn}[Principals of square matrices]
    \label{def:leadingprincipalminor}
    Let $\mtrx{A}$ be an $n\times{}n$ matrix. For $1\leq{}k\leq{}n$, the
    $k^{\textnormal{th}}$ \emph{principal submatrix} of $\mtrx{A}$ is the $k\times{}k$
    submatrix formed from the first $k$ rows and first $k$ columns of $\mtrx{A}$.
    Its determinant is the $k^{\textnormal{th}}$ principal minor.
\end{defn}
Thus, in three dimensions, this amounts to the simultaneous fulfillment of the
three requirements
\begin{subequations}
    \label{eq:positive_definiteness_of_l}
    \begin{align}
        \label{eq:principal_minor_one}
        \vct{\nabla}^{2}\mtrx{C}\big[\vct{\xi}_{3},\vct{\xi}_{3},\vct{\xi}_{3},,\vct{\xi}_{3}\big] &> 0 \\
        \label{eq:principal_minor_two}
        2\dfrac{\lambda_{3}-\lambda_{1}}{\lambda_{1}\lambda_{3}}\vct{\nabla}^{2}\mtrx{C}\big[\vct{\xi}_{3},\vct{\xi}_{3},\vct{\xi}_{3},\vct{\xi}_{3}\big]  - 4\Big(\dfrac{\lambda_{3}-\lambda_{1}}{\lambda_{1}\lambda_{3}}\Big)^{2}\big\langle\vct{\xi}_{1},(\vct{\nabla}\vct{\xi}_{3})\vct{\xi}_{3}\big\rangle^{2} &> 0 \\
        \label{eq:principal_minor_three}
        \det(\mtrx{L}) &> 0
    \end{align}
\end{subequations}

