In three dimensions, the matrix $\mtrx{L}$ is defined as
\begingroup
\renewcommand{\arraystretch}{2.5}
\begin{equation}
\mtrx{L} = \begin{pmatrix}
        \vct{\nabla}^{2}\mtrx{C}^{-1}\big[\vct{\xi}_{3},\vct{\xi}_{3},\vct{\xi}_{3},\vct{\xi}_{3}\big] %
        & 2\dfrac{\lambda_{3}-\lambda_{1}}{\lambda_{1}\lambda_{3}}\big\langle{\vct{\xi}_{1}},{(\vct{\nabla}\vct{\xi}_{3})\vct{\xi}_{3}}\big\rangle %
        & 2\dfrac{\lambda_{3}-\lambda_{2}}{\lambda_{2}\lambda_{3}}\big\langle{\vct{\xi}_{2}},{(\vct{\nabla}\vct{\xi}_{3})\vct{\xi}_{3}}\big\rangle \\
        2\dfrac{\lambda_{3}-\lambda_{1}}{\lambda_{1}\lambda_{3}}\big\langle{\vct{\xi}_{1}},{(\vct{\nabla}\vct{\xi}_{3})\vct{\xi}_{3}}\big\rangle %
        & 2\dfrac{\lambda_{3}-\lambda_{1}}{\lambda_{1}\lambda_{3}} %
        & 0 \\
        2\dfrac{\lambda_{3}-\lambda_{2}}{\lambda_{2}\lambda_{3}}\big\langle{\vct{\xi}_{2}},{(\vct{\nabla}\vct{\xi}_{3})\vct{\xi}_{3}}\big\rangle %
        & 0 %
        & 2\dfrac{\lambda_{3}-\lambda_{2}}{\lambda_{2}\lambda_{3}}
    \end{pmatrix},
\end{equation}
\endgroup
where the
dependence of $\mtrx{L}$ and its constituent parts on $t$, $t_{0}$ and
$\vct{x}_{0}$ has been omitted for brevity. The distinction between
so-called strong and weak hyperbolic LCSs is based upon whether or not
$\mtrx{L}$ is positive definite
\parencite{haller2010variational,farazmand2011erratum}. By Sylvester's
theorem, a square matrix is positive definite if and only if all of the
leading principal minors of $\mtrx{L}$ are positive. The term
\emph{leading principal minors} is defined as follows
\parencite{gilbert1991positive}:
\begin{defn}[Principals of square matrices]
    \label{def:leadingprincipalminor}
    Let $\mtrx{A}$ be an $n\times{}n$ matrix. For $1\leq{}k\leq{}n$, the
    $k^{\textnormal{th}}$ \emph{principal submatrix} of $\mtrx{A}$ is the $k\times{}k$
    submatrix formed from the first $k$ rows and first $k$ columns of $\mtrx{A}$.
    Its determinant is the $k^{\textnormal{th}}$ principal minor.
\end{defn}
Thus, in three dimensions, the matrix $\mtrx{L}$ being positive definite
amounts to the simultaneous fulfillment of the three requirements
\begin{subequations}
    \label{eq:positive_definiteness_of_l}
    \begin{align}
        \label{eq:principal_minor_one}
        \vct{\nabla}^{2}\mtrx{C}^{-1}\big[\vct{\xi}_{3},%
                                          \vct{\xi}_{3},%
                                          \vct{\xi}_{3},%
                                          \vct{\xi}_{3}\big] &> 0, \\
        \label{eq:principal_minor_two}
        2\dfrac{\lambda_{3}-\lambda_{1}}{\lambda_{1}\lambda_{3}}%
        \bigg\{\vct{\nabla}^{2}\mtrx{C}^{-1}\big[\vct{\xi}_{3},%
                                              \vct{\xi}_{3},%
                                              \vct{\xi}_{3},%
                                              \vct{\xi}_{3}\big] %
        -2\dfrac{\lambda_{3}-\lambda_{1}}{\lambda_{1}\lambda_{3}}%
                \big\langle\vct{\xi}_{1},%
                    (\vct{\nabla}\vct{\xi}_{3})\vct{\xi}_{3}\big\rangle^{2}\bigg\} &> 0, \\
        \label{eq:principal_minor_three}
        \det(\mtrx{L}) &> 0.
    \end{align}
\end{subequations}
By straightforward algebraic manipulations, the inequality
\eqref{eq:principal_minor_three} is equivalent to
\begin{equation}
    \label{eq:determinant_inequality}
    \begin{aligned}
    4 \frac{(\lambda_{3}-\lambda_{2})(\lambda_{3}-\lambda_{1})}%
           {\lambda_{1}\lambda_{2}\lambda_{3}^{2}}%
           \bigg\{\vct{\nabla}^{2}\mtrx{C}^{-1}\big[\vct{\xi}_{3},%
                                                  \vct{\xi}_{3},
                                                  \vct{\xi}_{3},
                                                  \vct{\xi}_{3}\big]
                - \Big( &2\frac{\lambda_{3}-\lambda_{1}}{\lambda_{1}\lambda_{3}}%
                        \big\langle\vct{\xi}_{1},%
                        (\vct{\nabla}\vct{\xi}_{3})\vct{\xi}_{3}\big\rangle^{2}  \\%
                      +  &2\frac{\lambda_{3}-\lambda_{2}}{\lambda_{2}\lambda_{3}}%
                        \big\langle\vct{\xi}_{2},%
                        (\vct{\nabla}\vct{\xi}_{3})\vct{\xi}_{3}\big\rangle^{2} %
                   \Big)%
           \bigg\} > 0.
    \end{aligned}
\end{equation}
Now, we make use of the following result from \textcite{haller2010variational}:
\begin{lemm}
    At each point of a weak LCS in three dimensions, the following identity holds:
    \begin{equation}
        \label{eq:hess_inverse_cg_tensor_identity}
        \begin{aligned}
            \vct{\nabla}^{2}\mtrx{C}^{-1}\big[\vct{\xi}_{3},%
                                              \vct{\xi}_{3},%
                                              \vct{\xi}_{3},%
                                              \vct{\xi}_{3}%
                                          \big] %
                = - \frac{1}{\lambda_{3}^{2}} %
                          \big\langle\vct{\xi}_{3},%
                              \vct{\nabla}^{2}\lambda_{3}\vct{\xi}_{3} %
                          \big\rangle %
                          &+ 2\frac{\lambda_{3}-\lambda_{1}}{\lambda_{1}\lambda_{3}}%
                        \big\langle\vct{\xi}_{1},%
                        (\vct{\nabla}\vct{\xi}_{3})\vct{\xi}_{3}\big\rangle^{2}  \\%
                        &+  2\frac{\lambda_{3}-\lambda_{2}}{\lambda_{2}\lambda_{3}}%
                        \big\langle\vct{\xi}_{2},%
                        (\vct{\nabla}\vct{\xi}_{3})\vct{\xi}_{3}\big\rangle^{2}.
        \end{aligned}
    \end{equation}
    \begin{proof}
        See theorem 7 in \textcite{haller2010variational}.
    \end{proof}
\end{lemm}
Using \cref{eq:hess_inverse_cg_tensor_identity} in conjunction with
\cref{eq:determinant_inequality} and the relationships
$0\leq\lambda_{1}\leq\lambda_{2}\leq\lambda_{3}$, the inequalities
\eqref{eq:positive_definiteness_of_l} can be expressed as follows:
\begin{subequations}
    Foo
\end{subequations}
