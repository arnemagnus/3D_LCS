\begin{figure}[htpb]
    \centering
    \resizebox{0.9\linewidth}{!}{\includestandalone%
    {figures/tikz-figs/stretch_and_strain}}
    \caption[Geometric interpretation of the eigenvectors of the Cauchy-Green
    strain \newline{}tensor]
    {Geometric interpretation of the eigenvectors of the
        Cauchy-Green strain tensor. The central unit cell is stretched and
        deformed under the flow map $\vct{\phi}_{t_{0}}^{t}(\vct{x}_{0})$. The
        local stretching is largest in the direction of $\vct{\xi}_{3}$,
        the eigenvector which corresponds to the largest eigenvalue,
        $\lambda_{3}$, of the Cauchy-Green strain tensor, defined in
        \cref{eq:cauchygreen_characteristics}. Along the $\vct{\xi}_{i}$ axes,
        the stretch factors are given by $\sqrt{\lambda_{i}}$, respectively.
    }
    \label{fig:stretch_and_strain}
\end{figure}


