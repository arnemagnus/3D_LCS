\begin{figure}[htpb]
    \centering
    \begin{subfigure}[b]{0.475\textwidth}
    \centering
       \importpgf{figures/mpl-illustrations}{nn-itp.pgf}
       \caption[]{{\small Nearest neighbor interpolation}}
    \label{fig:itp_nnb}
    \end{subfigure}
    \begin{subfigure}[b]{0.475\textwidth}
    \centering
       \importpgf{figures/mpl-illustrations}{lin-itp.pgf}
       \caption[]{{\small Linear interpolation}}
    \label{fig:itp_lin}
    \end{subfigure}

    \begin{subfigure}[b]{0.475\textwidth}
    \centering
       \importpgf{figures/mpl-illustrations}{quad-itp.pgf}
       \caption[]{{\small Quadratic B-spline interpolation}}
    \label{fig:itp_quad}
    \end{subfigure}
    \begin{subfigure}[b]{0.475\textwidth}
    \centering
       \importpgf{figures/mpl-illustrations}{cbc-itp.pgf}
       \caption[]{{\small Cubic B-spline interpolation}}
    \label{fig:itp_cbc}
    \end{subfigure}
    \caption[A selection of commonly used interpolation methods applied to a
    discretely sampled, high degree polynomial]
    {A selection of commonly used interpolation methods applied to a discretely
        sampled, high degree polynomial (dashed). The sampling points (knots)
        are shown as hollow circles. The splines in (\subref*{fig:itp_quad})
        and (\subref*{fig:itp_cbc}) are of second and third order, respectively
        (see \cref{def:piecewise_polynomial}). Observe how higher order splines
        yield increasingly accurate and smooth interpolations.
    }
    \label{fig:itp_multi}
\end{figure}
