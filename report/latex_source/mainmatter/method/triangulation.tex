\section[Continuously reconstructing manifold surfaces from expanding point
meshes in 3D]
{Continuously reconstructing manifold surfaces\\\phantom{3.9}  from point
meshes in 3D}
\label{sec:continuously_reconstructing_three_dimensional_manifold_surfaces%
_from_point_meshes}

While expanding the point mesh parametrization of a manifold $\mathcal{M}$
in bundles constituting geodesic level sets, we attempt to reproduce its
fully three-dimensional structure by simultaneously interpolating inbetween the
mesh points. Our main objective for representing manifolds as continuous
interpolation objects is visual representation (in addition to the detection
of self-intersections, as will be outlined in
\cref{sub:continuous_self_intersection_checks}) --- accordingly, we do not
provide any form of analytical expression for $\mathcal{M}$'s surface.
Moreover, as high order interpolation schemes are complicated considerably by
the irregular structure inherent to the parametrization of $\mathcal{M}$ as a
sequence of level sets, linear interpolation became our method of choice.

To our knowledge, all three-dimensional plotting algorithms rely on some sort
of triangulation method to regularize a pointwise parametrized surface.
General-purpose routines for triangulation generation were found to be
unsuitable, due to the specific mesh structure of $\mathcal{M}$, arising from
the parametrization by geodesic level sets. For instance, Delaunay
triangulation not only resulted in omitting triangles which, to the naked eye,
were crucial for the overall manifold structure, but also generated a lot of
undesirable surface triangles --- a problem which became increasingly prominent
near creases. Accordingly, we made use of the fact that the
\texttt{plot\_trisurf} routine from the Python plotting library
\texttt{Matplotlib} accepts an optional input argument specifying a list of
triangles to plot, given by their vertices, and made our own triangulation
scheme based on the specific structure of the mesh point parametrization of the
computed manifolds.

We begin by fixing a traversion direction, specifying the order in which
the triangles are created. Starting with the innermost level set, we then
specify the sets of vertices for a set of triangles which together cover the
(approximate) surface between the manifold epicentre $\vct{x}_{0}$
and $\mathcal{C}_{1}$ (see \cref{fig:innermost_levelset}). When a new geodesic
level set $\mathcal{M}_{i+1}$ satisfies the accuracy constraints outlined in
\cref{sec:managing_mesh_accuracy}, we move along the mesh points
constituting $\mathcal{C}_{i+1}$ in the selected direction, adding new
triangles covering the (approximate) surface area between the interpolation
curves $\mathcal{C}_{i}$ and $\mathcal{C}_{i+1}$.

The surface area enclosed by the innermost level set was simply reconstructed
by forming the triangles whose vertices are given as
$\{\vct{x}_{0},\vct{x}_{1,j},\vct{x}_{1,j+1}\}$. Our treatment of the ensuing
level sets is best described in terms of the local triangles formed around
a single mesh point $\mathcal{M}_{i,j}$. The base case --- that is, when no
(nearby) points in the level set $\mathcal{M}_{i+1}$ have been removed nor
added inbetween direct descendants in order to maintain the overall mesh point
quality (cf.\  \cref{sub:maintaining_mesh_point_density,%
sub:limiting_the_accumulation_of_numerical_noise}) --- is handled by demanding
that the triangles associated with $\mathcal{M}_{i,j}$ cover the tetragonal
surface element with vertices given by
$\{\vct{x}_{i,j},\vct{x}_{i,j+1},\vct{x}_{i+1,j+1},\vct{x}_{i+1,j}\}$.
Accordingly, we add two triangles with vertices given by
$\{\vct{x}_{i,j},\vct{x}_{i+1,j},\vct{x}_{i+1,j+1}\}$ and
$\{\vct{x}_{i,j},\vct{x}_{i,j+1},\vct{x}_{i+1,j}\}$, respectively. This is
shown in \cref{fig:triangulation_basecase}.

The cases not involving a one-to-one correspondence between the mesh points in
$\mathcal{M}_{i}$ and $\mathcal{M}_{i+1}$ require special attention. This
occurs whenever mesh points are inserted by making use of ficticious ancestors,
or when mesh points are removed, in order to maintain the mesh point density
(cf.\ \cref{sub:maintaining_mesh_point_density}) --- or, when removing unwanted
bulges in order to dampen the effects of compound numerical noise
(as described in \cref{sub:limiting_the_accumulation_of_numerical_noise}).
The treatment of these special cases is shown in
\cref{fig:triangulation_pointinserted,fig:triangulation_pointremoved}, and will
be outlined in greater detail in the upcoming paragraph. In particular, note
how all of $\mathcal{M}_{i,j}$'s nearest neighbors in the surrounding level set
$\mathcal{M}_{i+1}$ are used in the triangulations --- regardless of whether
these are computed from the mesh points in level set $\mathcal{M}_{i}$, or
have ficticious ancestors. This ensures (approximate) coverage of the entire
surface area inbetween each level set, and thus the computed manifold as a
whole.

When an extra mesh point $\mathcal{M}_{i+1,j+1/2}$ is
inserted inbetween $\mathcal{M}_{i+1,j}$ and $\mathcal{M}_{i+1,j+1}$, the
tetragonal surface element whose vertices are located at
$\{\vct{x}_{i,j},\vct{x}_{i,j+1},%
\vct{x}_{i+1,j},\vct{x}_{i+1,j+1/2}\}$ is approximated
by means of two triangles. Again expressed in terms of their vertices,
these are $\{\vct{x}_{i,j},\vct{x}_{i+1,j},\vct{x}_{i+1,j+1/2}\}$ and
$\{\vct{x}_{i,j},\vct{x}_{i,j+1},\vct{x}_{i+1,j+1/2}\}$. In the event
that the mesh point $\mathcal{M}_{i+1,j}$ was removed, either as part of an
undesired bulge or in order to preserve mesh density, the tetragonal surface
with vertices at
$\{\vct{x}_{i,j},\vct{x}_{i,j+1},\vct{x}_{i+1,j-1},\vct{x}_{i+1,j+1}\}$ is
constructed using two triangles, with vertices at
$\{\vct{x}_{i,j},\vct{x}_{i+1,j-1},\vct{x}_{i+1,j+1}\}$ and
$\{\vct{x}_{i,j},\vct{x}_{i,j+1},\vct{x}_{i+1,j+1}\}$, respectively.
If more than one intermediate mesh point is removed, the treatment is
completely analogous, occasionally resulting in some triangle elements
being significantly larger than their neighbors.

\section[Continuously reconstructing manifold surfaces from expanding point
meshes in 3D]
{Continuously reconstructing manifold surfaces\\\phantom{3.9}  from point
meshes in 3D}
\label{sec:continuously_reconstructing_three_dimensional_manifold_surfaces%
_from_point_meshes}

While expanding the point mesh parametrization of a manifold $\mathcal{M}$
in bundles of mesh points constituting geodesic level sets, we attempted to
reproduce its fully three-dimensional structure by simultaneously interpolating
inbetween the mesh points. Our main objective for representing manifolds as
continuous interpolation objects is visual representation (in addition to the
detection of self-intersections, as will be outlined in
\cref{sub:continuous_self_intersection_checks}) --- accordingly, we do not
provide any form of analytical expression for $\mathcal{M}$'s surface.
Moreover, as high order interpolation schemes are complicated considerably by
the irregular structure inherent to the parametrization of $\mathcal{M}$ as a
sequence of level sets, linear interpolation became our method of choice.

To our knowledge, all three-dimensional surface plotting algorithms rely on
some sort of triangulation method to regularize a pointwise parametrized
surface. General-purpose routines for triangulation generation were found to be
unsuitable, due to the specific mesh structure of $\mathcal{M}$, arising from
the parametrization by geodesic level sets. For instance, Delaunay
triangulation not only resulted in omitting triangles which, to the naked eye,
were crucial for the overall manifold structure, but also generated a lot of
undesirable surface triangles --- a problem which became increasingly prominent
near creases. Accordingly, we made use of the fact that the
\texttt{plot\_trisurf} routine from the Python plotting library
\texttt{Matplotlib} accepts an optional input argument specifying a list of
triangles to plot, given by their vertices, and made our own triangulation
scheme based on the specific structure of the mesh point parametrization of the
computed manifolds.

We begin by fixing a traversion direction, specifying the order in which
the triangles are created. Starting with the innermost level set, we then
specify the vertices for a set of triangles which together cover the
(approximate) surface between the manifold epicentre $\vct{x}_{0}$
and $\mathcal{C}_{1}$ (see \cref{fig:innermost_levelset}). When a new geodesic
level set $\mathcal{M}_{i+1}$ satisfies the accuracy constraints outlined in
\cref{sec:managing_mesh_accuracy}, we move along the mesh points
constituting $\mathcal{C}_{i+1}$ in the selected direction, adding new
triangles covering the (approximate) surface area between the interpolation
curves $\mathcal{C}_{i}$ and $\mathcal{C}_{i+1}$.

The surface area enclosed by the innermost level set was simply reconstructed
by forming the triangles whose vertices are given as
$\{\vct{x}_{0},\vct{x}_{1,j},\vct{x}_{1,j+1}\}$. Our treatment of the ensuing
level sets is best described in terms of the local triangles formed around
a single mesh point $\mathcal{M}_{i,j}$. The base case --- that is, when no
(nearby) points in the level set $\mathcal{M}_{i+1}$ have been removed nor
added inbetween direct descendants in order to maintain the overall mesh point
quality (cf.\  \cref{sub:maintaining_mesh_point_density,%
sub:limiting_the_accumulation_of_numerical_noise}) --- is handled by demanding
that the triangles associated with $\mathcal{M}_{i,j}$ cover the tetragonal
surface element with vertices given by
$\{\vct{x}_{i,j},\vct{x}_{i,j+1},\vct{x}_{i+1,j+1},\vct{x}_{i+1,j}\}$.
Accordingly, we add two triangles with vertices given by
$\{\vct{x}_{i,j},\vct{x}_{i+1,j},\vct{x}_{i+1,j+1}\}$ and
$\{\vct{x}_{i,j},\vct{x}_{i,j+1},\vct{x}_{i+1,j}\}$, respectively. This is
shown in \cref{fig:triangulation_basecase}.

The cases not involving a one-to-one correspondence between the mesh points in
$\mathcal{M}_{i}$ and $\mathcal{M}_{i+1}$ require special attention. This
occurs whenever mesh points are inserted by making use of ficticious ancestors,
or when mesh points are removed, in order to maintain the mesh point density
(cf.\ \cref{sub:maintaining_mesh_point_density}) --- alternatively, when
removing unwanted bulges in order to dampen the effects of compound numerical
noise (as described in
\cref{sub:limiting_the_accumulation_of_numerical_noise}). The treatment of
these special cases is shown in
\cref{fig:triangulation_pointinserted,fig:triangulation_pointremoved}, and will
be outlined in greater detail in the upcoming paragraph. In particular, note
how all of $\mathcal{M}_{i,j}$'s nearest neighbors in the surrounding level set
$\mathcal{M}_{i+1}$ are used in the triangulations --- regardless of whether
these are computed from the mesh points in level set $\mathcal{M}_{i}$, or
have ficticious ancestors. This ensures (approximate) coverage of the entire
surface area inbetween each level set, and thus the computed manifold as a
whole.

When an extra mesh point $\mathcal{M}_{i+1,j+1/2}$ is
inserted inbetween $\mathcal{M}_{i+1,j}$ and $\mathcal{M}_{i+1,j+1}$, the
tetragonal surface element whose vertices are located at
$\{\vct{x}_{i,j},\vct{x}_{i,j+1},%
\vct{x}_{i+1,j},\vct{x}_{i+1,j+1/2}\}$ is approximated
by means of two triangles. Again expressed in terms of their vertices,
these are $\{\vct{x}_{i,j},\vct{x}_{i+1,j},\vct{x}_{i+1,j+1/2}\}$ and
$\{\vct{x}_{i,j},\vct{x}_{i,j+1},\vct{x}_{i+1,j+1/2}\}$. In the event
that the mesh point $\mathcal{M}_{i+1,j}$ was removed, either as part of an
undesired bulge or in order to preserve mesh density, the tetragonal surface
with vertices at
$\{\vct{x}_{i,j},\vct{x}_{i,j+1},\vct{x}_{i+1,j-1},\vct{x}_{i+1,j+1}\}$ is
constructed using two triangles, with vertices at
$\{\vct{x}_{i,j},\vct{x}_{i+1,j-1},\vct{x}_{i+1,j+1}\}$ and
$\{\vct{x}_{i,j},\vct{x}_{i,j+1},\vct{x}_{i+1,j+1}\}$, respectively.
If more than one intermediate mesh point is removed, the treatment is
completely analogous, occasionally resulting in some triangle elements
being significantly larger than their neighbors.

\section[Continuously reconstructing manifold surfaces from expanding point
meshes in 3D]
{Continuously reconstructing manifold surfaces\\\phantom{3.9}  from point
meshes in 3D}
\label{sec:continuously_reconstructing_three_dimensional_manifold_surfaces%
_from_point_meshes}

While expanding the point mesh parametrization of a manifold $\mathcal{M}$
in bundles of mesh points constituting geodesic level sets, we attempted to
reproduce its fully three-dimensional structure by simultaneously interpolating
inbetween the mesh points. Our main objective for representing manifolds as
continuous interpolation objects is visual representation (in addition to the
detection of self-intersections, as will be outlined in
\cref{sub:continuous_self_intersection_checks}) --- accordingly, we do not
provide any form of analytical expression for $\mathcal{M}$'s surface.
Moreover, as high order interpolation schemes are complicated considerably by
the irregular structure inherent to the parametrization of $\mathcal{M}$ as a
sequence of level sets, linear interpolation became our method of choice.

To our knowledge, all three-dimensional surface plotting algorithms rely on
some sort of triangulation method to regularize a pointwise parametrized
surface. General-purpose routines for triangulation generation were found to be
unsuitable, due to the specific mesh structure of $\mathcal{M}$, arising from
the parametrization by geodesic level sets. For instance, Delaunay
triangulation not only resulted in omitting triangles which, to the naked eye,
were crucial for the overall manifold structure, but also generated a lot of
undesirable surface triangles --- a problem which became increasingly prominent
near creases. Accordingly, we made use of the fact that the
\texttt{plot\_trisurf} routine from the Python plotting library
\texttt{Matplotlib} accepts an optional input argument specifying a list of
triangles to plot, given by their vertices, and made our own triangulation
scheme based on the specific structure of the mesh point parametrization of the
computed manifolds.

We begin by fixing a traversion direction, specifying the order in which
the triangles are created. Starting with the innermost level set, we then
specify the vertices for a set of triangles which together cover the
(approximate) surface between the manifold epicentre $\vct{x}_{0}$
and $\mathcal{C}_{1}$ (see \cref{fig:innermost_levelset}). When a new geodesic
level set $\mathcal{M}_{i+1}$ satisfies the accuracy constraints outlined in
\cref{sec:managing_mesh_accuracy}, we move along the mesh points
constituting $\mathcal{C}_{i+1}$ in the selected direction, adding new
triangles covering the (approximate) surface area between the interpolation
curves $\mathcal{C}_{i}$ and $\mathcal{C}_{i+1}$.

The surface area enclosed by the innermost level set was simply reconstructed
by forming the triangles whose vertices are given as
$\{\vct{x}_{0},\vct{x}_{1,j},\vct{x}_{1,j+1}\}$. Our treatment of the ensuing
level sets is best described in terms of the local triangles formed around
a single mesh point $\mathcal{M}_{i,j}$. The base case --- that is, when no
(nearby) points in the level set $\mathcal{M}_{i+1}$ have been removed nor
added inbetween direct descendants in order to maintain the overall mesh point
quality (cf.\  \cref{sub:maintaining_mesh_point_density,%
sub:limiting_the_accumulation_of_numerical_noise}) --- is handled by demanding
that the triangles associated with $\mathcal{M}_{i,j}$ cover the tetragonal
surface element with vertices given by
$\{\vct{x}_{i,j},\vct{x}_{i,j+1},\vct{x}_{i+1,j+1},\vct{x}_{i+1,j}\}$.
Accordingly, we add two triangles with vertices given by
$\{\vct{x}_{i,j},\vct{x}_{i+1,j},\vct{x}_{i+1,j+1}\}$ and
$\{\vct{x}_{i,j},\vct{x}_{i,j+1},\vct{x}_{i+1,j}\}$, respectively. This is
shown in \cref{fig:triangulation_basecase}.

The cases not involving a one-to-one correspondence between the mesh points in
$\mathcal{M}_{i}$ and $\mathcal{M}_{i+1}$ require special attention. This
occurs whenever mesh points are inserted by making use of ficticious ancestors,
or when mesh points are removed, in order to maintain the mesh point density
(cf.\ \cref{sub:maintaining_mesh_point_density}) --- alternatively, when
removing unwanted bulges in order to dampen the effects of compound numerical
noise (as described in
\cref{sub:limiting_the_accumulation_of_numerical_noise}). The treatment of
these special cases is shown in
\cref{fig:triangulation_pointinserted,fig:triangulation_pointremoved}, and will
be outlined in greater detail in the upcoming paragraph. In particular, note
how all of $\mathcal{M}_{i,j}$'s nearest neighbors in the surrounding level set
$\mathcal{M}_{i+1}$ are used in the triangulations --- regardless of whether
these are computed from the mesh points in level set $\mathcal{M}_{i}$, or
have ficticious ancestors. This ensures (approximate) coverage of the entire
surface area inbetween each level set, and thus the computed manifold as a
whole.

When an extra mesh point $\mathcal{M}_{i+1,j+1/2}$ is
inserted inbetween $\mathcal{M}_{i+1,j}$ and $\mathcal{M}_{i+1,j+1}$, the
tetragonal surface element whose vertices are located at
$\{\vct{x}_{i,j},\vct{x}_{i,j+1},%
\vct{x}_{i+1,j},\vct{x}_{i+1,j+1/2}\}$ is approximated
by means of two triangles. Again expressed in terms of their vertices,
these are $\{\vct{x}_{i,j},\vct{x}_{i+1,j},\vct{x}_{i+1,j+1/2}\}$ and
$\{\vct{x}_{i,j},\vct{x}_{i,j+1},\vct{x}_{i+1,j+1/2}\}$. In the event
that the mesh point $\mathcal{M}_{i+1,j}$ was removed, either as part of an
undesired bulge or in order to preserve mesh density, the tetragonal surface
with vertices at
$\{\vct{x}_{i,j},\vct{x}_{i,j+1},\vct{x}_{i+1,j-1},\vct{x}_{i+1,j+1}\}$ is
constructed using two triangles, with vertices at
$\{\vct{x}_{i,j},\vct{x}_{i+1,j-1},\vct{x}_{i+1,j+1}\}$ and
$\{\vct{x}_{i,j},\vct{x}_{i,j+1},\vct{x}_{i+1,j+1}\}$, respectively.
If more than one intermediate mesh point is removed, the treatment is
completely analogous, occasionally resulting in some triangle elements
being significantly larger than their neighbors.

\section[Continuously reconstructing manifold surfaces from expanding point
meshes in 3D]
{Continuously reconstructing manifold surfaces\\\phantom{3.9}  from point
meshes in 3D}
\label{sec:continuously_reconstructing_three_dimensional_manifold_surfaces%
_from_point_meshes}

While expanding the point mesh parametrization of a manifold $\mathcal{M}$
in bundles of mesh points constituting geodesic level sets, we attempted to
reproduce its fully three-dimensional structure by simultaneously interpolating
inbetween the mesh points. Our main objective for representing manifolds as
continuous interpolation objects is visual representation (in addition to the
detection of self-intersections, as will be outlined in
\cref{sub:continuous_self_intersection_checks}) --- accordingly, we do not
provide any form of analytical expression for $\mathcal{M}$'s surface.
Moreover, as high order interpolation schemes are complicated considerably by
the irregular structure inherent to the parametrization of $\mathcal{M}$ as a
sequence of level sets, linear interpolation became our method of choice.

To our knowledge, all three-dimensional surface plotting algorithms rely on
some sort of triangulation method to regularize a pointwise parametrized
surface. General-purpose routines for triangulation generation were found to be
unsuitable, due to the specific mesh structure of $\mathcal{M}$, arising from
the parametrization by geodesic level sets. For instance, Delaunay
triangulation not only resulted in omitting triangles which, to the naked eye,
were crucial for the overall manifold structure, but also generated a lot of
undesirable surface triangles --- a problem which became increasingly prominent
near creases. Accordingly, we made use of the fact that the
\texttt{plot\_trisurf} routine from the Python plotting library
\texttt{Matplotlib} accepts an optional input argument specifying a list of
triangles to plot, given by their vertices, and made our own triangulation
scheme based on the specific structure of the mesh point parametrization of the
computed manifolds.

We begin by fixing a traversion direction, specifying the order in which
the triangles are created. Starting with the innermost level set, we then
specify the vertices for a set of triangles which together cover the
(approximate) surface between the manifold epicentre $\vct{x}_{0}$
and $\mathcal{C}_{1}$ (see \cref{fig:innermost_levelset}). When a new geodesic
level set $\mathcal{M}_{i+1}$ satisfies the accuracy constraints outlined in
\cref{sec:managing_mesh_accuracy}, we move along the mesh points
constituting $\mathcal{C}_{i+1}$ in the selected direction, adding new
triangles covering the (approximate) surface area between the interpolation
curves $\mathcal{C}_{i}$ and $\mathcal{C}_{i+1}$.

The surface area enclosed by the innermost level set was simply reconstructed
by forming the triangles whose vertices are given as
$\{\vct{x}_{0},\vct{x}_{1,j},\vct{x}_{1,j+1}\}$. Our treatment of the ensuing
level sets is best described in terms of the local triangles formed around
a single mesh point $\mathcal{M}_{i,j}$. The base case --- that is, when no
(nearby) points in the level set $\mathcal{M}_{i+1}$ have been removed nor
added inbetween direct descendants in order to maintain the overall mesh point
quality (cf.\  \cref{sub:maintaining_mesh_point_density,%
sub:limiting_the_accumulation_of_numerical_noise}) --- is handled by demanding
that the triangles associated with $\mathcal{M}_{i,j}$ cover the tetragonal
surface element with vertices given by
$\{\vct{x}_{i,j},\vct{x}_{i,j+1},\vct{x}_{i+1,j+1},\vct{x}_{i+1,j}\}$.
Accordingly, we add two triangles with vertices given by
$\{\vct{x}_{i,j},\vct{x}_{i+1,j},\vct{x}_{i+1,j+1}\}$ and
$\{\vct{x}_{i,j},\vct{x}_{i,j+1},\vct{x}_{i+1,j}\}$, respectively. This is
shown in \cref{fig:triangulation_basecase}.

The cases not involving a one-to-one correspondence between the mesh points in
$\mathcal{M}_{i}$ and $\mathcal{M}_{i+1}$ require special attention. This
occurs whenever mesh points are inserted by making use of ficticious ancestors,
or when mesh points are removed, in order to maintain the mesh point density
(cf.\ \cref{sub:maintaining_mesh_point_density}) --- alternatively, when
removing unwanted bulges in order to dampen the effects of compound numerical
noise (as described in
\cref{sub:limiting_the_accumulation_of_numerical_noise}). The treatment of
these special cases is shown in
\cref{fig:triangulation_pointinserted,fig:triangulation_pointremoved}, and will
be outlined in greater detail in the upcoming paragraph. In particular, note
how all of $\mathcal{M}_{i,j}$'s nearest neighbors in the surrounding level set
$\mathcal{M}_{i+1}$ are used in the triangulations --- regardless of whether
these are computed from the mesh points in level set $\mathcal{M}_{i}$, or
have ficticious ancestors. This ensures (approximate) coverage of the entire
surface area inbetween each level set, and thus the computed manifold as a
whole.

When an extra mesh point $\mathcal{M}_{i+1,j+1/2}$ is
inserted inbetween $\mathcal{M}_{i+1,j}$ and $\mathcal{M}_{i+1,j+1}$, the
tetragonal surface element whose vertices are located at
$\{\vct{x}_{i,j},\vct{x}_{i,j+1},%
\vct{x}_{i+1,j},\vct{x}_{i+1,j+1/2}\}$ is approximated
by means of two triangles. Again expressed in terms of their vertices,
these are $\{\vct{x}_{i,j},\vct{x}_{i+1,j},\vct{x}_{i+1,j+1/2}\}$ and
$\{\vct{x}_{i,j},\vct{x}_{i,j+1},\vct{x}_{i+1,j+1/2}\}$. In the event
that the mesh point $\mathcal{M}_{i+1,j}$ was removed, either as part of an
undesired bulge or in order to preserve mesh density, the tetragonal surface
with vertices at
$\{\vct{x}_{i,j},\vct{x}_{i,j+1},\vct{x}_{i+1,j-1},\vct{x}_{i+1,j+1}\}$ is
constructed using two triangles, with vertices at
$\{\vct{x}_{i,j},\vct{x}_{i+1,j-1},\vct{x}_{i+1,j+1}\}$ and
$\{\vct{x}_{i,j},\vct{x}_{i,j+1},\vct{x}_{i+1,j+1}\}$, respectively.
If more than one intermediate mesh point is removed, the treatment is
completely analogous, occasionally resulting in some triangle elements
being significantly larger than their neighbors.

\input{mainmatter/method/figures/triangulation}



