\section[Computing the flow map and its directional derivatives]
{Computing the flow map and its directional\\ \phantom{3.2} derivatives}
\label{sec:computing_the_flow_map_and_its_directional_derivatives}

Computing the flow map Jacobian field is crucial, as it is used in our
definition of the Cauchy-Green strain tensor field (cf.\
\cref{eq:defn_cauchygreen}) --- whose eigenvalues and -vectors, in turn, form
the basis of the LCS existence criteria given in \cref{eq:lcs_conditions}. An
outline of how we solved \cref{eq:consideredflow} in conjunction with
\cref{eq:timederivative_flowmap_jacobian} in order to obtain the final state
of the flow map Jacobian field is presented in
\cref{sub:advecting_a_set_of_tracers}. Furthermore,
\cref{sub:the_implementation_of_dynamic_runge_kutta_step_size} contains a
detailed description of how the dynamic Runge-Kutta solver step size (see
\cref{sub:the_runge_kutta_family_of_numerical_ode_solvers}) was
implemented. How we then extracted the Cauchy-Green strain eigenvalues and
-vectors from the final state flow map Jacobian field is the topic of
\cref{sec:computing_cauchy_green_strain_eigenvalues_and_vectors}.

\subsection{Advecting a set of tracers}
\label{sub:advecting_a_set_of_tracers}

The variational framework for computing LCSs is based upon the advection of
non-interacting tracers, as described in
\cref{sec:the_type_of_flow_systems_considered}, by the systems mentioned in
\cref{sec:flow_systems_defined_by_analytical_velocity_fields,%
sec:flow_systems_defined_by_gridded_velocity_data}. The computational domains
$\mathcal{U}$ were discretized by a set equidistant tracers, effectively
creating a uniform grid with tracers placed on and within the domain boundaries
of $\mathcal{U}$. The grid parameters are summarized in~\cref{tab:gridparams}.

\begin{table}[htpb]
    \centering
    \caption[Grid parameters for advection in the considered flow systems]
    {Grid parameters for advection in the considered flow systems. For the
        fjord system, the domain extents and grid spacings are given in
        units of metre. Also note that the grid spacings have been truncated
        to two significant decimal digits.
    }
    \label{tab:gridparams}
    \begin{tabular}{ccc}
        \toprule
        & Analytical ABC flow & Fjord model data\\
        \midrule
        %
        Computational domain %
        & $[0,2\pi]^{3}$ %
        & $[0,500]\times[0,500]\times[50,300]$\\
        %
        $N_{x}$, $N_{y}$, $N_{z}$ %
        & $256$, $256$, $256$ %
        & $200$, $200$, $100$\\
        %
        $\Delta{x} = \Delta{y} = \Delta{z}$ %
        & $2.5\cdot10^{-2}$ %
        & $2.5$\\
        \bottomrule
    \end{tabular}
\end{table}


In order to increase the precision of the computed Cauchy-Green strain tensor
field, it is necessary to increase the accuracy with which one computes the
Jacobian of the flow map, as their accuracies are intrinsically linked.
This follows from \cref{eq:defn_cauchygreen}. Accordingly, the flow map
Jacobian was computed directly, by means of simultaneously solving the
twelve coupled ODEs given by
\cref{eq:consideredflow,eq:timederivative_flowmap_jacobian}, letting the
underlying velocity field transport the tracers. All twelve ODEs were solved
simultaneously, using the Dormand-Prince 8(7) method (see
\cref{sub:the_runge_kutta_family_of_numerical_ode_solvers} and, in particular,
\cref{tab:butcherdopri87}). The dynamic step length adjustment procedure
will be outlined in detail in
\cref{sub:the_implementation_of_dynamic_runge_kutta_step_size}.

In this framework, the tracer advection takes second stage to the ``advection''
of the components of the flow map Jacobian. As it turns out, the increase in
mathematical complexity which the coupling terms introduces is a small price to
pay for the increased precision compared to the straightforward approach of
applying a finite diffference scheme to the advected flow map
\parencite{oettinger2016autonomous}. This is also evident from previous
``finite difference-based'' LCS computing endeavors, in which the use of
several grids of tracers was necessitated in order to accurately compute the
flow map Jacobian \parencite{loken2017sensitivity,farazmand2012computing}.

\subsection{The implementation of dynamic Runge-Kutta step size}
\label{sub:the_implementation_of_dynamic_runge_kutta_step_size}

In order to implement automatic step size control, the procedure suggested by
\textcite[pp.167--168]{hairer1993solving} was followed closely. A starting step
size $h$ needs to be prescribed; this generally differs based upon the
(pseudo-)time scale of the underlying system. For the first solution step,
the embedded Dormand-Prince 8(7) method, as described in
\cref{sub:the_runge_kutta_family_of_numerical_ode_solvers,tab:butcherdopri87},
yields the two approximations $x_{1}$ and $\widehat{x}_{1}$, from which the
difference $x_{1}-\widehat{x}_{1}$ can be used as an estimate of the error
of the less precise result. The idea is to force the error of the numerical
solution to satisfy, componentwise:
\begin{equation}
    \label{eq:adaptivetimestep_baseline}
    \abs{x_{1,i}-\widehat{x}_{1,i}}\leq\mathrm{sc}_{i}, \quad%
    \mathrm{sc}_{i} = \mathrm{Atol}_{i}+\max\Big(\abs{x_{1,i}},%
    \abs{\widehat{x}_{1,i}}\Big)\cdot\mathrm{Rtol}_{i},
\end{equation}
where $\mathrm{Atol}_{i}$ and $\mathrm{Rtol}_{i}$ are the desired absolute
and relative tolerances. For this project, the tolerance values
\begin{equation}
    \label{eq:adaptivetimestep_tolerances}
    \mathrm{Atol}_{i} = 10^{-7}, \quad \mathrm{Rtol}_{i} = 10^{-7}
\end{equation}
were used throughout.

As a measure of the numerical error,
\begin{equation}
    \label{eq:adaptivetimestep_errorestimate}
    \mathrm{err} = \sqrt{\frac{1}{n}\sum\limits_{i=1}^{n}%
    \bigg(\frac{x_{1,i}-\widehat{x}_{1,i}}{\mathrm{sc}_{i}}\bigg)^{2}}
\end{equation}
is used. Then, $\mathrm{err}$ is compared to unity in order to find an optimal
step size. From \cref{def:runge_kutta_order}, it follows that $\mathrm{err}$
scales like $h^{q+1}$, where $q = \min(p,\widehat{p}\,)$. Thus, under
the expected scaling $\mathrm{err}\approx{}Kh^{q+1}$, and the assumption
$1\approx{}Kh_{\mathrm{opt}}^{q+1}$, one finds the optimal step size according
to
\begin{equation}
    \label{eq:adaptivetimestep_optimalstepsize}
    h_{\mathrm{opt}} = h\cdot%
    \bigg(\frac{1}{\mathrm{err}}\bigg)^{\frac{1}{q+1}}.
\end{equation}
If $\mathrm{err}\leq1$, the suggested solution step is accepted, the
(pseudo-)time variable $t$ is increased by $h$, and the step length is modified
according to
\cref{eq:adaptivetimestep_optimalstepsize,eq:adaptivetimestep_timestepupdate}.
Which of the two approximations $x_{n+1}$ or $\widehat{x}_{n+1}$ is used to
continue the integration generally depends on the embedded Runge-Kutta method
in question. Continuing the integration with the higher order result is
commonly referred to as \emph{local extrapolation}. The Dormand-Prince 8(7)
method is tuned in order to minimize the error of the higher order result;
accordingly, local extrapolation was used throughout. If $\mathrm{err} > 1$, the
solution step is rejected, and the step length decreased before attempting
another step. The procedure for updating the time step can be summarized
as follows:
\begin{equation}
    \label{eq:adaptivetimestep_timestepupdate}
    h_{\mathrm{new}} = %
    \begin{cases}
        \min(\mathrm{fac}_{\mathrm{max}}\cdot{}h,%
        \mathrm{fac}\cdot{}h_{\mathrm{opt}}) %
        & \text{if the solution step is accepted,}\\
        \mathrm{fac}\cdot{}h_{\mathrm{opt}}, %
        & \text{if the solution step is rejected,}
    \end{cases}
\end{equation}
where $\mathrm{fac}$ and $\mathrm{fac}_{\mathrm{max}}$ are numerical safety
factors, intended to prevent increasing the step size \emph{too} much, in order
to make it more likely that the next step is accepted. Here, the parameter
values
\begin{equation}
    \label{eq:adaptivetimetep_safetyfactors}
    \mathrm{fac}=0.8,\quad \mathrm{fac}_{\mathrm{max}} = 2.0,
\end{equation}
were used throughout.
