\section[Revised approach to computing new mesh points]
{Revised approach to computing new mesh points}
\label{sec:revised_approach_to_computing_new_mesh_points}

The version of the method of geodesic level sets presented by
\textcite{krauskopf2005survey} is centered around computing manifolds
defined by being everywhere tangent to some three-dimensional vector field
(as a concrete example of application, \citeauthor{krauskopf2005survey}
seek to compute the strange attractor of the Lorenz system). As noted by
\textcite{oettinger2016autonomous}, repelling LCSs in 3D consist of subsets
of manifolds defined by being everywhere orthgonal to the $\vct{\xi}_{3}$
direction field --- which is reflected in existence criterion
\eqref{eq:lcs_condition_c}. Thus, compared to the type of systems considered
by \citeauthor{krauskopf2005survey}, we have an extra degree of freedom
when seeking to identify repelling LCSs in 3D; in particular, moving along
a manifold in arbitrary directions within a planes orthogonal to the local
$\vct{\xi}_{3}$ vector is allowed, in contrast to being constrained to moving
back and forth along a curve. The paragraphs (and sections) to follow will
reveal how we utilized the additional degree of freedom to reduce the number of
calculations necessary to compute new mesh points.

The overarching principles, nevertheless, remain the same. Like for the legacy
approach presented in
\cref{sec:legacy_approach_to_computing_new_mesh_points}, mesh points in
level set $\mathcal{M}_{i+1}$ are computed from the points in the prior level
set $\mathcal{M}_{i}$. Similarly, the considerations to follow rely on
each mesh point $\mathcal{M}_{i,j}$ having inherited its tangential vector
from its direct ancestor; namely, $\vct{t}_{i,j}:=\vct{t}_{i-1,j}$. How
the special cases of this inheritance-based approach were treated, will be
described in greater detail in
\emph{\textbf{SETT INN REF TIL AVSN OM ARV AV TANGENSVEKTOR; EV AVSN OM INNSETTING OG FJERNING AV PUNKTER}}.
Moreover, from the mesh point $\mathcal{M}_{i,j}$, we wish to place a new
mesh point $\mathcal{M}_{i+1,j}$ at an intersection of the manifold
$\mathcal{M}$ and the half-plane $\mathcal{H}_{i,j}$ located a distance
$\Delta_{i}$ from $\mathcal{M}_{i,j}$. As usual, $\mathcal{H}_{i,j}$ is defined
by the coordinate $\vct{x}_{i,j}$, the unit normal $\vct{t}_{i,j}$ and
the guidance vector $\vct{\rho}_{i,j}$ (the latter of which is defined in
\cref{eq:innermost_prevvec,eq:general_prevvec}).
