\section[Revised approach to computing new mesh points]
{Revised approach to computing new mesh points}
\label{sec:revised_approach_to_computing_new_mesh_points}

The version of the method of geodesic level sets presented by
\textcite{krauskopf2005survey} is centered around computing manifolds
defined by being everywhere tangent to some three-dimensional vector field
(as a concrete example of application, \citeauthor{krauskopf2005survey}
seek to compute the strange attractor of the Lorenz system). As noted by
\textcite{oettinger2016autonomous}, repelling LCSs in 3D consist of subsets
of manifolds defined by being everywhere orthgonal to the $\vct{\xi}_{3}$
direction field --- which is reflected in existence criterion
\eqref{eq:lcs_condition_c}. Thus, compared to the type of systems considered
by \citeauthor{krauskopf2005survey}, we have an extra degree of freedom
when seeking to identify repelling LCSs in 3D; in particular, moving along
a manifold in arbitrary directions within a planes orthogonal to the local
$\vct{\xi}_{3}$ vector is allowed, in contrast to being constrained to moving
back and forth along a curve. The paragraphs (and sections) to follow will
reveal how we utilized the additional degree of freedom to reduce the number of
calculations necessary to compute new mesh points.

The overarching principles, nevertheless, remain the same. Like for the legacy
approach presented in
\cref{sec:legacy_approach_to_computing_new_mesh_points}, mesh points in
level set $\mathcal{M}_{i+1}$ are computed from the points in the prior level
set $\mathcal{M}_{i}$. Similarly, the considerations to follow rely on
each mesh point $\mathcal{M}_{i,j}$ having inherited its tangential vector
from its direct ancestor; namely, $\vct{t}_{i,j}:=\vct{t}_{i-1,j}$. How
the special cases of this inheritance-based approach were treated, will be
described in greater detail in \cref{sub:maintaining_mesh_point_density}.
Moreover, from the mesh point $\mathcal{M}_{i,j}$, we wish to place a new
mesh point $\mathcal{M}_{i+1,j}$ at an intersection of the manifold
$\mathcal{M}$ and the half-plane $\mathcal{H}_{i,j}$ located a distance
$\Delta_{i}$ from $\mathcal{M}_{i,j}$. As usual, $\mathcal{H}_{i,j}$ is defined
by the coordinate $\vct{x}_{i,j}$, the unit normal $\vct{t}_{i,j}$ and
the guidance vector $\vct{\rho}_{i,j}$ (the latter of which is defined in
\cref{eq:innermost_prevvec,eq:general_prevvec}).

\subsection{Computing pseudoradial trajectories directly}
\label{sub:computing_pseudoradial_trajectories_directly}

Just like in the legacy approach outlined in
\cref{sec:legacy_approach_to_computing_new_mesh_points}, we define a local
direction field as
\begin{equation}
    \label{eq:revised_direction_field}
    \vct{\psi}(\vct{x}) = %
    \frac{\vct{\xi}_{3}(\vct{x})\times\vct{t}_{i,j}}%
    {\norm{\vct{\xi}_{3}(\vct{x})\times\vct{t}_{i,j}}},
\end{equation}
where $\vct{t}_{i,j}$ is the unit tangent associated with the mesh point
$\mathcal{M}_{i,j}$. Here, however, we make explicit use of our previously
mentioned additional degree of freedom; namely, that trajectories within
the parametrized manifold are allowed arbitrary movements within the
planes which are locally orthogonal to the $\vct{\xi}_{3}$ direction field,
rather than being constrained to moving along a three-dimensional curve. Thus,
we compute the coordinates of the new mesh point $\mathcal{M}_{i+1,j}$ as the
end point of a \emph{single} trajectory.

Any trajectory starting out within the half-plane $\mathcal{H}_{i,j}$ and
moving in the direction field given by \cref{eq:revised_direction_field} is
certain to remain within the half-plane, as the direction field is everywhere
orthogonal to its unit normal $\vct{t}_{i,j}$. That is, as long as the
direction field used in computing said trajectory is everywhere oriented
radially outwards. Accordingly, we computed a single trajectory in the
aforementioned direction field, starting out at
$\vct{x}_{\text{init}}=\vct{x}_{i,j}$, using the Dormand-Prince 8(7) adaptive
ODE solver (see \cref{tab:butcherdopri87,%
sub:the_implementation_of_dynamic_runge_kutta_step_size}), where all of
the vectors of the intermediary Runge-Kutta evaulations were corrected, if
necessary, by continuous comparison with $\vct{\rho}_{i,j}$ and
direction-reversion if an intermediary vector was pointing radially inwards.

Should the $\vct{\xi}_{3}$ direction be parallel to the unit tangent
$\vct{t}_{i,j}$ locally along the trajectory, the direction field
\cref{eq:revised_direction_field} would become undefined. In such regions,
we allowed the Runge-Kutta solver to step in the direction used for the
immediately preceding step. Numerically, such regions were recognized by
\begin{equation}
    \label{eq:revised_xi3_tan_parallel}
    \norm{\vct{\xi}_{3}(\vct{x})\times\vct{t}_{i,j}} < \gamma_{\|},
\end{equation}
where $\gamma_{\|}$ is a small tolerance parameter. Like for the legacy
approach (outlined in \cref{sec:legacy_approach_to_computing_new_mesh_points}),
the self-correcting integration step length meant that we did not treat the
integration step as a degree of freedom, and, in order to avoid overstepping,
the step length of the Dormand-Prince solver was continuously limited from
above by $\Delta_{i}-\norm{\vct{x}-\vct{x}_{i,j}}$. Moreover, the total allowed
integration arclength was limited to an integer multiple $\gamma_{\text{arc}}$
of the interset step $\Delta_{i}$, allowing for the termination of any
(hypothetical) trajectory which would end up in a stable orbit.

The trajectory integration was immediately interrupted upon reaching a point
$\vct{x}_{\text{fin}}$ separated from $\vct{x}_{i,j}$ by a distance
$\Delta_{i}$. Like for the legacy approach, this criterion was checked by means
of a tolerance parameter, seeing as directly comparing floating-point numbers
for equality is prone to numerical round-off error. In particular, if a point
$\vct{x}$ satisfied
\begin{equation}
    \label{eq:revised_dist_tol}
    \abs{\frac{\norm{\vct{x}-\vct{x}_{i,j}}}{\Delta_{i}}-1} < \gamma_{\Delta},
\end{equation}
with $\gamma_{\Delta}$ being a small number, the point was flagged as laying a
distance $\Delta_{i}$ from $\vct{x}_{i,j}$. Thus, upon reaching a point
satisfying \cref{eq:revised_dist_tol}, the trajectory was terminated, and the
new mesh point $\mathcal{M}_{i+1,j}$ was placed at the trajectory end point
$\vct{x}_{\text{fin}}$. \Cref{fig:revised_point_generation} depicts a
typical trajectory used to compute new mesh points in the fashion discussed
in the above.

\documentclass[crop]{standalone}
\usepackage{tikz}
\usepackage[]{tikz-3dplot}
\usepackage{pgfplots}
\pgfplotsset{compat=1.15}
\usepackage[]{amsmath}
\usepackage[]{libertine}
\usepackage[libertine]{newtxmath}
\usepackage[]{bm}
\usepackage[]{physics}
% Macros for greek letters in roman style, in math mode
\DeclareRobustCommand{\mathup}[1]{%
\begingroup\ensuremath\changegreek\mathrm{#1}\endgroup}
\DeclareRobustCommand{\mathbfup}[1]{%
\begingroup\ensuremath\changegreek\bm{\mathrm{#1}}\endgroup}


\makeatletter
\def\changegreek{\@for\next:={%
        alpha,beta,gamma,delta,epsilon,zeta,eta,theta,iota,kappa,lambda,mu,nu,%
        xi,pi,rho,sigma,tau,upsilon,phi,chi,psi,omega,varepsilon,varpi,%
    varrho,varsigma,varphi}%
\do{\expandafter\let\csname\next\expandafter\endcsname\csname\next up\endcsname}}
\makeatother

% Define vectors in bold, roman, lowercase font
\newcommand{\vct}[1]{\ensuremath{\mathbfup{\MakeLowercase{#1}}}}

% Define unit vectors in bold, roman, lowercase font, with hats
\newcommand{\uvct}[1]{\ensuremath{\mathbfup{\hat{\MakeLowercase{#1}}}}}

% Define matrices in bold, roman, uppercase font
\newcommand{\mtrx}[1]{\ensuremath{\mathbfup{\MakeUppercase{#1}}}}
\usetikzlibrary{%
    angles,%
    arrows.meta,%
    backgrounds,%
    calc,%
    decorations,%
    fit,%
    hobby,%
    patterns,%
    positioning,%
    quotes
}

\tdplotsetmaincoords{70}{120}

\begin{document}
\begin{tikzpicture}[tdplot_main_coords]
    \pgfmathsetmacro{\radius}{5}
    \pgfmathsetmacro{\size}{2.5}
%    % Place the set of initial points
    \foreach [count = \i] \ang in {205,240,260,280,165}%
    {%
        \coordinate (\i) at ( {\radius*cos(\ang-90)}, {\radius*sin(\ang-90)},{0} );
    }

    % Place coordinates giving the circle of positions at which one aims
    \coordinate (tp) at ({\radius*cos(150)},{\radius*sin(150)},1);
    \coordinate (bt) at ({\radius*cos(150)},{\radius*sin(150)},-1);


    % Define coordinates of half-plane $\mathcal{H}_{i,j}$
    \coordinate (lu) at ($(2)!2!(tp)$);
    \coordinate (ld) at ($(2)!2!(bt)$);
    \coordinate (ru) at ($(lu)+7*({-0.866},{0.5},0)$);
    \coordinate (rd) at ($(ld)+7*({-0.866},{0.5},0)$);
    % Shade half-plane
    \draw[fill=gray!30,draw opacity=0] (lu) -- (ld) -- (rd) -- (ru) -- cycle;
    % Draw interpolation curve
    \draw[stroke=black!80,thin,dotted] (70:\radius) arc (70:200:\radius);
    % Draw $\vct{rho}_{i,j}$ at $x_{i,j}$
    \path[draw,stroke=black!65,->,thick,dashdotted ] (2) to ($(2) + 5*({-0.866},{0.5},0)$) coordinate (rhoend);
    \node[below left = 7.5pt and 35pt of rhoend,rotate=12.5] {$\vct{\rho}_{i,j}$};
    \node[above left = 0pt and 0pt of rd] {$\mathcal{H}_{i,j}$};


    % Define coordinates of pivot for attaching $\mathcal{C}_{i}$ label
    \coordinate (mrk) at ({\radius*cos(195)},{\radius*sin(195)},0);
    % Attach label
    \coordinate [above = 10pt of mrk] (lbl);
    \node at (lbl) {$\mathcal{C}_{i}$};
    \draw[stroke=black!80] ($(mrk)!0.1!(lbl)$) to[in= 270,out=20] ($(mrk)!0.55!(lbl)$);

    % Draw points in initial level set
    \foreach \i in {1,2,3,4,5}%
    {%
        \draw[fill=gray!10,stroke=black!65] (\i) circle (\size pt);
    }
    % Draw semicircle of radius $\Delta_{i}$, centered at $x_{i,j}$
    \pic[densely dashed,draw,stroke=grey!80,angle radius=39] {angle = bt--2--tp};



    % Label ancestor point
    \coordinate[below left = 0pt and 30pt of 2] (plbl);
    \node[below left = 0pt and 0pt of 2]  {$\mathcal{M}_{i,j}$};


    % Define aim point
    \coordinate (aim) at ($(2) + 1.4*({-0.866},{0.5},0) - (0,0,0.9)$);

    % Draw aim point
    \draw[stroke=black!65,fill=gray!10] (aim) circle (\size pt);
    \node[right = 0pt of aim]  {$\vct{x}_{\text{fin}}$};

    % Draw some trajectories, the last of which manages to find aim point
    \path[draw,stroke=black!80,->] (2) to[out = -80, in = -180+30]  (aim);
    % Redraw start point
    \draw[fill=gray!10,stroke=black!65] (2) circle (\size pt);
    % Suggest initial radius
    \draw[decorate,decoration={brace,amplitude=5pt,raise=0pt},yshift=0pt] (2) -- ($(2)!1.46!(tp)$)   node [midway,left=2pt] {\footnotesize{$\Delta_{i}$}};

\end{tikzpicture}
\end{document}




\subsection{Handling failures to compute satisfactory mesh points}
\label{sub:handling_failures_to_compute_satisfactory_mesh_points_revised}

Much like the legacy approach outlined in
\cref{sec:legacy_approach_to_computing_new_mesh_points}, it can never be
guaranteed that all of the computed trajectories will yield new, acceptable
mesh points --- moreover, missing out on points in a level set prohibits
the generation of further level sets. In particular, our procedure for
maintaining mesh accuracy (which will be described in
\cref{sub:maintaining_mesh_point_density}) makes extensive use of the
interpolation curves $\{\mathcal{C}_{i}\}$. As the smoothness of the
interpolation curve $\mathcal{C}_{i+1}$ depends on \emph{all} of the points
used for its creation, proper handling of tricky mesh points remains critically
important.

Seeing as the only tolerance parameter involved in this method pertains to
achieving the desired separation between a meshpoint and its direct descendant
--- see \cref{eq:revised_dist_tol} --- we elected not to progressively relax
this constraint. Instead, we interpreted the failure of any trajectory to reach
a point sufficiently far away, as the trajectory being coiled such that the
required integration arc length to yield a point sufficiently far away from
the ancestor mesh point, exceeded the maximum allowed integration path
length (governed by $\gamma_{\text{arc}}$, cf.\
\cref{sub:computing_pseudoradial_trajectories_directly}). Accordingly, the
incomplete level set was discarded, and attempted to be recomputed with
$\Delta_{i}$ reduced to $\Delta_{\min}$ (our general dynamic adjustment
procedure for $\Delta_{i}$ will be described in detail in
\cref{sub:a_curvature_based_approach_to_determining_interset_separations}).
Should an entire, new geodesic level set prove incomputable even at the
minimum permitted step length, attempts at further expansion of the computed
manifold were abandoned. In such cases, this variant of the method of geodesic
level sets simply did not suffice, for the given set of development parameters.
Details on other stopping criteria for the generation of manifolds will be
presented in \cref{sec:macroscale_stopping_criteria_for_the_expansion_of%
_computed_manifolds}.



\subsection{Key improvements of the revised algorithm for computing new mesh
points}
\label{sub:key_improvements_of_the_revised_algorithm_for_computing_new_mesh%
_points}

When compared to the convoluted way of computing new mesh points presented in
\cref{sec:legacy_approach_to_computing_new_mesh_points}, it is readily apparent
that the revised approach is significantly less complex. In particular, note
how launching a single trajectory starting at the ancestor mesh point results
in the legacy algorithm for computing new trajectory start points iteratively
(cf. \cref{fig:s_update_flowchart}) being rendered entirely superfluous.
Moreover, as all computed trajectories necessarily remain within the target
half-planes, no tolerance parameter for detecting intersections between
trajectories and half-planes is needed. In short, the revised algorithm is
conceptually simpler, and involves a lesser number of free (tolerance)
parameters.

Simple numerical experiments confirmed that, given the same initial
conditions and underlying direction fields, the two approaches yielded the
same mesh points --- at least, within rounding error. Moreover, these tests
also revealed that the revised algorithm is usually two orders of
magnitude faster (in terms of computational runtime). Unsurprisingly, the
main time expenditure of the legacy approach turned out to be the computation
of elusive mesh points; often, several thousand computed trajectories were
needed before a satisfactory new point was found. In paticular, the probability
of finding an acceptable mesh point depends sensitively on choosing an
appropriate aim point --- leading to significant slowdowns in regions wherein
the underlying manifold behaves erratically.

For its superior speed and simplicity, the revised approach of forced
pseudoradial trajectories became our method of choice. Accordingly, all
of our results (which will be presented in \cref{cha:results}) were generated
with this method. Note that, while deemed inferior in the context of
identifying repelling LCSs in three-dimensional flow, the legacy approach of
guided trajectories remains a valid way of computing three-dimensional
manifolds. Thus, it can be used as a baseline for computing other kinds of
three-dimensional manifolds defined in a different manner than repelling LCSs
--- which remains beyond the scope of this project.

