\section{Identifying LCSs as subsets of computed manifolds}
\label{sec:identifying_lcss_as_subsets_of_computed_manifolds}

The collection of manifolds computed by means of the method
outlined in the preceding sections, starting from an approximately
even distribution of points in the $\mathcal{U}_{0}$ domain
(cf.\ \cref{sec:preliminaries_for_computing_repelling_lcss_in_3d_flow_by_means%
_of_geodesic_level_sets} and, in particular,
\cref{tab:initialconditionparams}), are all surfaces which satisfy LCS
existence criterion~\eqref{eq:lcs_condition_c} --- that is, they are everywhere
perpendicular to the local direction of maximal repulsion. In order to extract
repelling LCSs from these parametrized surfaces, we then identified the regions
of the manifolds --- represented as a subset of the mesh points in their
parametrization --- which also satisfy the remaining existence criteria;
namely,~\eqref{eq:lcs_condition_a},~\eqref{eq:lcs_condition_b} and~%
\eqref{eq:lcs_condition_d}. This was done completely analogously to how we
identified the grid points belonging within the $\mathcal{U}_{0}$ domain, as
outlined in \cref{sub:identifying_suitable_initial_conditions_for_developing%
_lcss}. In particular, each mesh point $\mathcal{M}_{i,j}$ of a computed
manifold $\mathcal{M}$ was flagged as to whether or not it satsified all
of the aforementioned existence criteria.

We then proceded to construct a repelling LCS $\mathcal{L}$ from the mesh
points of $\mathcal{M}$. Per
\cref{sub:identifying_suitable_initial_conditions_for_developing_lcss},
the mesh point at the centre of any given manifold always satisfies
all the LCS existence criteria; accordingly, it was added as the first mesh
point $\mathcal{L}_{0}$ of the extracted LCS. We then went through the list of
the mesh points $\mathcal{M}_{i,j}$ which satisfied all LCS criteria,
traversing each level set in the order in which their mesh points were added.
In order to avoid isolated points in the ensuing parametrization of the LCS, we
added a manifold mesh point to the set of LCS points provided that
\begin{equation}
    \label{eq:lcs_from_manifold_separation_criterion}
    \norm{\vct{x}_{i,j}-\tilde{\vct{x}}_{\kappa}} < %
    \gamma_{\blacktriangleright}\Delta_{\max}
\end{equation}
held for at least one $\kappa$, where $\vct{x}_{i,j}$ and
$\tilde{\vct{x}}_{\kappa}$ denote the coordinates of mesh point
$\mathcal{M}_{i,j}$ and the already accepted LCS point
$\mathcal{L}_{\kappa}\in\{\mathcal{L}_{k}\}$, respectively. $\Delta_{\max}$ is
the maximum allowed mesh point separation used for computing the manifold (cf.\
\cref{sub:maintaining_mesh_point_density}), while the scalar
parameter $\gamma_{\blacktriangleright}\geq1$ allows for extracting smoother
LCSs, more well-suited for visualization purposes --- as will be made clear
shortly. Subsequently, the mesh points in the parametrization of $\mathcal{M}$
which did \emph{not} satisfy all of the LCS existence criteria were added to
the set of LCS points for the purpose of enhanced visual representation ---
provided that they complied with a similar distance threshold as the one given
in \cref{eq:lcs_from_manifold_separation_criterion}. Note, however, that these
mesh points were only added if they were sufficiently close to a point
in $\{\mathcal{L}_{k}\}$ which satisfied \emph{all} of the LCS existence
criteria.

The tolerance parameter $\gamma_{\blacktriangleright}$ thus allowed us to mitigate
possible numerical error; in particular, if any given mesh point was slightly
perturbed away from the underlying manifold, it could still end up being a part
of the LCS. Finally, we looked at all of the surface elements pertaining to the
triangulation of the manifold $\mathcal{M}$ (as described in
\cref{sec:continuously_reconstructing_three_dimensional_manifold_surfaces_from_%
point_meshes}) in conjunction with the set of mesh points which had been
recognized as belonging to the LCS $\mathcal{L}$. If mesh points corresponding
to two of the three vertices defining a triangular surface element had been
recognized as part of $\mathcal{L}$, we then added the mesh point corresponding
to the last remaining vertex to the set of LCS points $\{\mathcal{L}_{k}\}$.
Accordingly, the triangulations of the manifold $\mathcal{M}$ were reused for
$\mathcal{L}$. These slight relaxations of the LCS existence criteria
facilitate the extraction of smoother LCS surfaces and are favorable for the
visual representation of LCSs. Moreover, they mitigate the possible effects of
numerical error perturbing any given mesh point $\mathcal{M}_{i,j}$ away from
the \emph{actual} manifold --- leaving it more than sufficiently close to the
manifold for visualization purposes --- by possibly allowing it to be included
as part of the LCS after all. \Cref{fig:manifold_lcs_conversion} shows an
example of extracting a repelling LCS from a computed manifold.

\begin{figure}[htpb]
    \centering
    \begin{subfigure}[b]{0.475\textwidth}
        \centering
        \importpgf{figures/mpl-figs}{conversion-mf-small.pgf}
        \caption[]{{\small A computed manifold in its entirety}}
        \label{fig:mf_conversion_mf}
    \end{subfigure}
    \begin{subfigure}[b]{0.475\textwidth}
        \centering
        \importpgf{figures/mpl-figs}{conversion-lcs-small.pgf}
        \caption[]{{\small The extracted repelling LCS}}
        \label{fig:mf_conversion_lcs}
    \end{subfigure}
    \caption[An example of a repelling LCS extracted as a subset of a computed
    manifold]
    {An example of a repelling LCS extracted as a subset of a computed
        manifold. In (\subref*{fig:mf_conversion_mf}), a sample manifold for
        the steady ABC flow is shown, whereas (\subref*{fig:mf_conversion_lcs})
        shows the subset of the manifold which satisfies the LCS existence
        criteria given in \cref{eq:lcs_conditions} (or lies sufficiently close
        to any point satisfying these criteria, meaning that their inclusion
        facilitates triangulations which provide an overall enhanced visual
        representation).
    }
    \label{fig:manifold_lcs_conversion}
\end{figure}



The extracted LCS surfaces $\mathcal{L}$, parametrized as a set of mesh points
$\{\mathcal{L}_{k}\}$, represent three-dimensional surfaces which --- allowing
for a little numerical error --- comply with all of the existence criteria
for repelling LCSs given in \cref{eq:lcs_conditions}, as originally proposed by
\textcite{haller2011variational}. Inspired by the work of
\textcite{farazmand2012computing}, we then sought to dispose of the smallest
among the computed LCSs, as these are expected to be the least significant in
terms of influencing the overall flow within the system.

In order to obtain a measure of the size of our three-dimensional surfaces,
to each LCS point $\mathcal{L}_{k}$, we assigned a weighting given by the
surface area approximating the region of the underlying manifold $\mathcal{M}$
that is closer to the corresponding mesh point $\mathcal{M}_{i,j}$ than any
others. To the mesh point located at the manifold epicentre
$\vct{x}_{0}$, we assigned the weight
$\mathcal{W}_{0} = \pi(\delta_{\text{init}}/2)^{2}$. The weights of all other
mesh points were computed as
\begin{equation}
    \label{eq:lcs_point_weight}
    \mathcal{W}_{k} := \mathcal{A}_{i,j} \approx %
    \frac{\Delta_{i}+\Delta_{i-1}}{2} \cdot %
    \frac{\norm{\vct{x}_{i,j+1}-\vct{x}_{i,j}}%
                +\norm{\vct{x}_{i,j}-\vct{x}_{i,j-1}}}{2},
\end{equation}
where, as always, $\vct{x}_{i,j}$ denotes the coordinates of mesh point
$\mathcal{M}_{i,j}$. This surface approximation is illustrated in
\cref{fig:lcs_point_weighting}. These weights were also used to compute
a repulsion average $\overline{\lambda}_{3}$; in particular,
\begin{equation}
    \label{eq:lcs_lm3_weight}
    \mathcal{W} = \sum\limits_{k}\mathcal{W}_{k},  \quad%
    \overline{\lambda}_{3} = \frac{1}{\mathcal{W}} %
    \sum\limits_{k}\lambda_{3}(\tilde{\vct{x}}_{k})\mathcal{W}_{k},
\end{equation}
where the summation is over all mesh points in the parametrization of
$\mathcal{L}$, and $\tilde{\vct{x}}_{k}$ denotes the coordinates of mesh
point $\mathcal{L}_{k}$.

In practice, we found that some mesh points exhibited $\lambda_{3}$-values
which were significantly different to others in their vicinities. We
interpreted these disparate $\lambda_{3}$-values as spurious interpolation
artefacts. Therefore, in order to limit biasing from outlier mesh points
(measured in terms of their $\lambda_{3}$ value) in the computed repulsion
average, we alternately removed the least and most repulsive mesh points from
the averaging process, as long as this altered the resulting repulsion average
significantly. In particular, denoting the adjusted repulsion average by
$\widehat{\lambda}_{3}$, we iteratively removed outlier extrema provided that
\begin{subequations}
\begin{equation}
    \label{eq:lcs_lm3_adjusted}
    \abs{1-\frac{\widehat{\lambda}_{3}}{\overline{\lambda}_{3}}} > 0.1,
\end{equation}
whereupon we continuously accepted the most recently computed adjusted
repulsion average as the new reference repulsion average --- that is,
\begin{equation}
    \label{eq:lcs_lm3_accept_iterate}
    \overline{\lambda}_{3} := \widehat{\lambda}_{3}.
\end{equation}
\end{subequations}
This process was repeated until the removal of the least or most strongly
repelling mesh points did not alter the resulting repulsion average enough to
trigger condition~\eqref{eq:lcs_lm3_adjusted}.

Any LCS for which the computed total weight $\mathcal{W}$ (a measure of
its surface area) was smaller than some pre-set limit $\mathcal{W}_{\min}$, or
where $\overline{\lambda}_{3} < 1$ --- the latter as a sanity
check to ensure overall repulsion --- per existence criterion
\eqref{eq:lcs_condition_a}, were discarded. As an aside, note that this method
of identifying repelling LCSs can easily be adapted to the identification of
\emph{attracting} LCSs, by computing Cauchy-Green strain eigenvalues and
-vectors for the \emph{reversed} time interval $[t_{1},t_{0}]$ (see
\cref{def:attracting_lcs}) and otherwise proceeding as discussed.

\begin{figure}[htpb]
    \centering
    \resizebox{0.9\linewidth}{!}%
    {\includestandalone{figures/tikz-figs/lcsextraction_weighting}}
    \caption[Our way of assigning weights to mesh points in computed LCSs]
    {Our way of assigning weights to mesh points in computed LCSs. To a given
        LCS point $\mathcal{L}_{k}$, we assigned a weight given by a
        rectangular approximation of the surface area closer to the
        corresponding manifold mesh point $\mathcal{M}_{i,j}$ than all other
        points in the parametrization of the manifold $\mathcal{M}$ (shaded).
        Here, $\Delta_{i}$ corresponds to the interset step length used to
        create level set $\mathcal{M}_{i+1}$, based on level set
        $\mathcal{M}_{i}$, (see
        \cref{sec:revised_approach_to_computing_new_mesh_points}), while
    $\vct{x}_{i,j}$ denotes the coordinates of mesh point $\mathcal{M}_{i,j}$.
}
    \label{fig:lcs_point_weighting}
\end{figure}


