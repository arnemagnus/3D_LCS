\section{Identifying LCSs as subsets of computed manifolds}
\label{sec:identifying_lcss_as_subsets_of_computed_manifolds}

\begin{framed}
    \begin{itemize}
        \item Gjør lokal sjekk; Ligger et meshpunkt i $\mathcal{U}_{0}$ tas
            det med direkte
        \item For alle punkter som ikke ble lagt til i første omgang;
            ligger de tilstrekkelig nærme et allerede akseptert punkt tas
            de med, fordi (1) de muligens ligger en \emph{tanke} over eller
            under den faktiske mangfoldigheten og (2) de gir flere trekanter
            for triangulering --- dermed en mer koherent representasjon
        \item Henvis til figur som viser ekstrahering av mangfoldighet-del
            som tilfredsstiller kravene for å være frastøtende LCS
        \item Til hvert enkelt meshpunkt, tilegn en $\lambda_{3}$-verdi
            og en vekting, gitt ved det omtrentlige arealet av $\mathcal{M}$
            som ligger nærmere $\mathcal{M}_{i,j}$ enn alle andre punkter.


        \item Analogt til Haller \& Farazmand, 2012computing, filtrerer vi ut
            de minste LCSene (målt i sum av pseudo-areal for de aksepterte
            meshpunktene), ettersom disse ventes å ha minst merkbar påvirkning
            på makroskala-forhold i systemet --- og eventuelle LCSer som
            i (pseudo-areal-)veiet middel har $\lambda_{3}$ mindre enn 1.
    \end{itemize}
\end{framed}
\begin{figure}[htpb]
    \centering
    \begin{subfigure}[b]{0.475\textwidth}
        \centering
        \importpgf{figures/mpl-figs}{conversion-mf-small.pgf}
        \caption[]{{\small A computed manifold in its entirety}}
        \label{fig:mf_conversion_mf}
    \end{subfigure}
    \begin{subfigure}[b]{0.475\textwidth}
        \centering
        \importpgf{figures/mpl-figs}{conversion-lcs-small.pgf}
        \caption[]{{\small The extracted repelling LCS}}
        \label{fig:mf_conversion_lcs}
    \end{subfigure}
    \caption[An example of a repelling LCS extracted as a subset of a computed
    manifold]
    {An example of a repelling LCS extracted as a subset of a computed
        manifold. In (\subref*{fig:mf_conversion_mf}), a sample manifold for
        the steady ABC flow is shown, whereas (\subref*{fig:mf_conversion_lcs})
        shows the subset of the manifold which satisfies the LCS existence
        criteria given in \cref{eq:lcs_conditions} (or lies sufficiently close
        to any point satisfying these criteria, meaning that their inclusion
        facilitates triangulations which provide an overall enhanced visual
        representation).
    }
    \label{fig:manifold_lcs_conversion}
\end{figure}




