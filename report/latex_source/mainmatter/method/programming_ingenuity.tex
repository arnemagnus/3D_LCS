\section[Making the most of the available computational resources]
{Making the most of the available computational \\\phantom{3.12} resources}
\label{sec:making_the_most_of_the_available_computational_resources}

The simultaneous solution of twelve coupled ODEs involved in the advection of
tracers in order to computed the (directional deriatives of the) flow map
(cf.\ \cref{sec:computing_the_flow_map_and_its_directional_derivatives})
quickly proved an unreasonably strenuous task for the author's own personal
laptop. Seeing as the available memory was the main limitation, we parallelized
this computation by means of MPI, and ran it on NTNU's supercomputer, Vilje.
In spite of the tracers being independent, we elected to utilize MPI over
alternative multiprocessing tools in order to access multiple nodes within
the Vilje cluster. Due to the problem's pleasingly parallel nature,
the parallelization process consisted of distributing an appeoximately even
amount of tracers across all ranks, whereupon each rank advected (that is,
simultaneously solved the twelve coupled ODEs for all of) its allocated tracers.
In the end, all of the final state flow map Jacobians were collected by the
designated main process (i..e,\ $\text{rank}=0$), whereupon the Cauchy-Green
strain eigenvalues and -vectors were extracted by means of a SVD decomposition
(as outlined in
\cref{sec:computing_cauchy_green_strain_eigenvalues_and_vectors}).

Regarding the generation of manifolds, code profiling (unsurprisingly) revealed
that the generation of new mesh points by computing (pseudo-)radial trajectories
orthogonal to the $\vct{\xi}_{3}$-direction field was a great source of
time expenditure. Accordingly, we rewrote all numerical routines pertaining to
the generation of new mesh points in Cython. In particular, we made use of
highly optimized, low-level BLAS\footnote{See \url{www.netlib.org/blas} and
\url{https://docs.scipy.org/doc/scipy/reference/linalg.cython_blas.html}}
routines whenever possible. Because of the ever increasing number of necessary
triangle comparisons in our algorithm of detectng manifolds which self-intersect
(as described in \cref{sub:continuous_self_intersection_checks}), all
of the accompanying numerical methods were expressed in Cython, too. Similarly
to how we exposed the Bspline-Fortran library to Python (cf.\
\cref{sub:interpolating_gridded_velocity_data}), we also consistently used
calls by reference in order to avoid unneccesary memory duplication. Overall,
the transition from (NumPy-based) Python to Cython for the most significant
tasks reduced the overall runtime by two orders of magnitude.

Analogously to the advection of tracers, we made use of the mutual independence
of the computed manifolds to accelerate their computation by means of MPI
parallelization across the Vilje cluster. We elected to make a one-to-one
correspondence between the number of MPI threads and the number of manifolds to
generate (as described in
\cref{sub:identifying_suitable_initial_conditions_for_developing_lcss}),
thus allowing each manifold to grow as large as possible before being stopped
due to the one of the criteria imposed in
\cref{sec:macroscale_stopping_criteria_for_the_expansion_of_computed_manifolds}.
Compared to the advection of tracers, or the expansion of manifolds,
the extraction of repelling LCSs as subsets of the computed manifolds
(cf.\ \cref{sec:identifying_lcss_as_subsets_of_computed_manifolds}) was not
a particularly laborious task. Thus, we chose to parallelize this
selection process by making use of the Python \texttt{multiprocessing} library,
as access to a single node in the Vilje cluster sufficed to complete it in
less than a minute.

%
%
%
%The extraction of repelling LCSs as subsets of the computed manifolds
%(as described in \cref{sec:identifying_lcss_as_subsets_of_computed_manifolds})
%was not a particularly laborious task, compared to the advection of tracers
%and the generation of said manifolds. Accordingly, we did not utilize MPI
%
%
%\begin{framed}
%    Poengter bruk av MPI for (triviell) parallellisering av mangfoldighetsgenerering,
%    og videre Pythons \texttt{multiprocessing} for seleksjon.
%
%    `Kodeprofilering avslørte at ... var den store tidstyven, derfor ...'
%
%    Understrek at alle numeriske rutiner som har med generering av nye punkter å
%    gjøre, ble skrevet i Cython for å minimere kjøretid. Spesifikt
%    benyttet vi høyt optimerte, lavnivå BLAS-rutiner når dette lot seg gjøre.
%
%    Likeså progget vi intersection-deteksjonsalgoritmen i samme språk,
%    pga jevnt økende behov for sammenligninger når mangfoldigheten spredte seg
%    utover.
%
%    Totalt sett medførte omskrivingen til Cython to størrelsesordeners reduksjon
%    i kjøretid.
%\end{framed}
