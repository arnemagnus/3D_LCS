\subsection{Handling failures to compute satisfactory mesh points}
\label{sub:handling_failures_to_compute_satisfactory_mesh_points_legacy}

As mentioned in
\cref{sub:selecting_initial_conditions_from_which_to_compute_new_mesh_points},
it can never be guaranteed that any trajectories which start out at some point
along the smooth curve $\mathcal{C}_{i}$ will be able to generate a new
mesh point located within $\mathcal{H}_{i,j}$ which simultaneously satisfies
all of our defined constraints. Missing out on points in any freshly generated
level set prohibits the generation of more level sets --- in particular,
partly dependent on the number of successfully computed points, the
smoothness of the interpolation curve $\mathcal{C}_{i+1}$ exhibits a critical
dependence on \emph{all} of the points used for its creation. Accordingly,
handling tricky mesh points is crucial.

Although it remains impossible to \emph{ensure} that the trajectory-based
approach to compute new mesh points comes to fruition, the success rate can be
increased significantly by responding appropriately to an initially failed
search. Our strategy of choice is based upon adjusting the computed aim point
incrementally. Specifically, given the initial angular offset $\alpha_{i,j}$ of
$\vct{x}_{\text{aim}}$ along the semicircle of radius $\Delta_{i}$, with
regards to the guidance vector $\vct{\rho}_{i,j}$ (see
\cref{fig:faithful_point_generation}), we perturb
$\vct{x}_{\text{aim}}$ along said semicircle by the forced alteration
\begin{subequations}
    \label{eq:angular_adjustment}
    \begin{equation}
        \label{eq:angular_adjustment_part_one}
        \alpha_{i,j} := \alpha_{i,j} + \delta\alpha,
    \end{equation}
    with
    \begin{equation}
        \label{eq:angular_adjustment_part_two}
        \abs{\delta\alpha} \leq \delta\alpha_{\max}.
    \end{equation}
\end{subequations}
Note that the range of offsets --- determined by $\delta\alpha_{\max}$ ---
should be chosen based on the expected geometry of the manifold as a whole;
in contrast to the number of attempted angular offsets, which should be
guided by considerations pertaining to the availability of computational
resources.

Note that reattempting the iterative point search algorithm outlined in
the entirety of the current section (that is,
\cref{sec:legacy_approach_to_computing_new_mesh_points}) for all possible
perturbation configurations of $\vct{x}_{\text{aim}}$ (given by
\cref{eq:angular_adjustment}) does not \emph{guarantee} that an
acceptable mesh point is found. In an attempt to find particularly elusive mesh
points, the entire algorithm --- including angular perturbations as outlined in
the above --- was then reran with simultaneously and progressively relaxed
point acceptance criteria. That is, the numerical tolerance parameters
$\gamma_{\mathcal{H}}$ and $\gamma_{\Delta}$ (cf.\
\cref{eq:plane_tolerance,eq:dist_tolerance}) were gradually increased up to
pre-set maximum values $\gamma_{\mathcal{H}}^{\max}$ and
$\gamma_{\Delta}^{\max}$.

If, after having increased said tolerance parameters to their maximal
permitted values, we were unable able to find an ``acceptable'' mesh point, the
incomplete level set was promptly discarded, and attempted to be computed
anew with reduced $\Delta_{i}$ (more on the adjustment procedure for
$\Delta_{i}$ to follow in
\cref{sub:a_curvature_based_approach_to_determining_interset_separations}).
Should an entire, new geodesic level set not be computable even with the
minimum permitted step length, attempts at expanding the computed manifold
further were abandoned; constrained by the given (maximal) tolerance
parameters, the method simply would not be able to expand the computed manifold
any further. The various other stopping criteria for the generation of
manifolds will be described in detail in \cref{sec:macroscale_stopping%
_criteria_for_the_expansion_of_computed_manifolds}.
