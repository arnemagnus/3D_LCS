
\subsection{Identifying suitable initial conditions for developing LCSs}
\label{sub:identifying_suitable_initial_conditions_for_developing_lcss}

Inspired by the two-dimensional approach of \textcite{farazmand2012computing},
in order to identify repelling LCSs, the first step was to identify the
subdomain $\mathcal{U}_{0}\subset\mathcal{U}$ in which the existence conditions
\eqref{eq:lcs_condition_a},~\eqref{eq:lcs_condition_b} and
\eqref{eq:lcs_condition_d} are satisfied --- as these conditions can be
verified for individual points, unlike criterion~\eqref{eq:lcs_condition_c}.
All grid points which lie within $\mathcal{U}_{0}$ would then be valid initial
conditions for repelling LCSs. Of the aforementioned criteria,
condition~\eqref{eq:lcs_condition_d} is the least straightforward to implement
numerically, as identifying the zeros of inner products is prone to numerical
round-off error. Our approach is based on comparing the value of $\lambda_{3}$
at a given grid point $\vct{x}_{0}$ to the values of $\lambda_{3}$ at the two
points $\vct{x}_{0}\pm\varepsilon\vct{\xi}_{3}(\vct{x}_{0})$, where
$\varepsilon$ is a number one order of magnitude smaller than the grid spacing.
Should $\lambda_{3}(\vct{x}_{0})$ be the largest of the three, the point
$\vct{x}_{0}$ would be flagged as satisfying criterion
\eqref{eq:lcs_condition_d}.

Using all of the points in $\mathcal{U}_{0}$ domain would invariably involve
computing alot of LCSs several times over --- in particular, if two neighboring
grid points are both located in the $\mathcal{U}_{0}$ domain, then they likely
belong to the same manifold. In order to reduce the number of redundant
calculations, the set of considered initial conditions was further reduced,
by only checking whether every $\nu^{\text{th}}$ grid point along each axis
belonged to $\mathcal{U}_{0}$, that is, only considering one in every
$\nu^{3}$ grid points in the entire domain as possible initial conditions.
Because the number of grid points was different for the different types of
flow (cf.\ \cref{tab:gridparams}), so too was the pseudo-sampling frequency
$\nu$. The values for $\varepsilon$, $\nu$ and the resulting number of initial
conditions are given in \cref{tab:initialconditionparams}. Note that, using
the given filtering parameters, the initial conditions reduced to a far more
manageable number of grid points, than all of the grid points which satisfy
the LCS conditions~\eqref{eq:lcs_condition_a},~\eqref{eq:lcs_condition_b} and~%
\eqref{eq:lcs_condition_d}.

\begin{table}[htpb]
    \centering
    \caption[Parameter choices for selecting LCS initial conditions]
    {Parameter choices for selecting LCS initial conditions. $\varepsilon$ was
    made to be one order of magnitude smaller than the grid spacing, and
    $\nu$ was selected to be a common divisor of the number of grid points
    along each Cartesian abscissa, cf.\ \cref{tab:gridparams}. Note that
    $\varepsilon$ for the fjord model data is given in units of metre. Observe
    how the reduced set of initial conditions contains orders of magnitude
    fewer points than the total number of grid points which satisfy the LCS
    existence criteria~\eqref{eq:lcs_condition_a},~\eqref{eq:lcs_condition_b}
    and~\eqref{eq:lcs_condition_d}.}
    \label{tab:initialconditionparams}
    \begin{tabular}{ccc}
        \toprule
        & Analytical ABC flow & Fjord model data\\
        \midrule
        $\varepsilon$ & $5\cdot10^{.-3}$ & $10^{-1}$\\
        $\nu$ & $8$ & $5$ \\[3pt]
        \makecell[c]{\# initial conditions\\without $\nu$-filtering} & \makecell[c]{$340951$ (steady) \\ $361461$ (unsteady)} & $209945$\\[9pt]
        \makecell[c]{\# initial conditions\\with $\nu$-filtering} & \makecell[c]{$618$ (steady)\\$676$ (unsteady)} & $1631$\\
        \bottomrule
    \end{tabular}
\end{table}

