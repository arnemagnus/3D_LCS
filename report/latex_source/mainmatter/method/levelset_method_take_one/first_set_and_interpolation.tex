\subsection{Parametrizing the innermost level set}
\label{sub:pametrizing_the_innermost_level_set}

For an initial condition $\vct{x}_{0}$ identified by means of the method
outlined in
\cref{sub:identifying_suitable_initial_conditions_for_developing_lcss}, the
corresponding LCS must locally be tangent to the plane with unit normal given
by $\vct{\xi}_{3}(\vct{x}_{0})$, as a consequence of LCS existence criterion~%
\eqref{eq:lcs_condition_c}. Accordingly, the first geodesic level set is
approximated by a set of $n$ points $\{\mathcal{M}_{1,j}\}_{j=1}^{n}$, which all
lie within the aforementioned tangent plane, separated from $\vct{x}_{0}$ by a
distance $\delta$; all of which assumed to lie within the same manifold. The
parameter $\delta$ was chosen small compared to the grid spacing, in order to
limit the inherent errors of this linearization.

An interpolation curve $\mathcal{C}_{1}$ was then made, with a view to
representing the innermost level set in a smoother fashion. In particular,
this interpolation curve was designed as a parametric spline. To this end,
the points $\{\mathcal{M}_{1,j}\}_{j=1}^{n}$ were first ordered in clockwise
or counterclockwise --- merely a matter of perspective --- fashion. Then,
each mesh point was assigned an independent variable $s$ based upon the
cumulative interpoint distance along the ordered list of points, starting
at $j=1$; estimated by means of the Euclidean norm, then normalized by dividing
$s$ through by the total interpoint distance around the entire initial level
set.

In an intermediary step, the coordinates of the mesh point
$\mathcal{M}_{1,1}$ was appended to a list containing the ordered coordinates
of the mesh points $\{\mathcal{M}_{1,j}\}_{j=1}^{n}$, such that, aside from
$\mathcal{M}_{1,1}$, which corresponded to both $s=0$ and $s=1$, there was a
one-to-one correspondence between the $s$-values and the coordinates of the
mesh points. Then, considering each of the lists containing the Cartesian
coordinates of the mesh points as univariate functions of the pseudo-arclength
parameter $s$, separate cubic B-splines were made for each set of coordinates,
making use of the \texttt{bspline\_1d} extension type from the Bspline-Fortran
library \parencite{williams2018bspline}, which was ported to Python as outlined
in \cref{sub:interpolating_gridded_velocity_data}. The constructed innermost
level set and its associated interpolation curve is illustrated in
\cref{fig:innermost_levelset}.



\begin{figure}[htpb]
    \centering
    \resizebox{0.9\linewidth}{!}{\includestandalone{figures/tikz-figs/initial_levelset_and_interpolation}}
    \caption[The construction of the innermost geodesic level set]
    {The construction of the innermost geodesic level set. An initial
        condition $\vct{x}_{0}$ is found by means of the method outlined in
        \cref{sub:identifying_suitable_initial_conditions_for_developing_lcss}.
        A set of $n$ meshpoints $\{\mathcal{M}_{1,j}\}_{j=1}^{n}$ is then evenly
        distributed within the plane defined by the point $\vct{x}_{0}$ and the
        unit normal $\vct{\xi}_{3}(\vct{x}_{0})$, which is shaded. Each mesh
        point is separated from $\vct{x}_{0}$ by a small distance
        $\delta_{\text{init}}$ (dashed). Using a normalized pseudo-arclength
        parameter $s$, the mesh point coordinates are interpolated using cubic
        B-splines, forming the smooth curve $\mathcal{C}_{1}$ (dotted).}
    \label{fig:innermost_levelset}
\end{figure}

