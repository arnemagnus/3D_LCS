\subsection{Selecting initial conditions from which to compute new mesh points}
\label{sub:selecting_initial_conditions_from_which_to_compute_new_mesh_points}

As suggested by \textcite{krauskopf2005survey}, the starting point for all
trajectories intended to reach $\vct{x}_{\text{aim}}$ could be chosen as any
point not equal to $\vct{x}_{i,j}$ along the parametrized curve
$\mathcal{C}_{i}$. However, trajectories starting out at points along
$\mathcal{C}_{i}$ which are far removed from $\vct{x}_{i,j}$ are likely to
require a long integration path, which would result in an increase in the
accumulated numerical round-off error. Subject to such errors, these
trajectories might not even get close to $\vct{x}_{\text{aim}}$ with
a reasonable computational resource consumption. Accordingly, we limited the
potential number of trajectories to compute by only considering a subset of the
interpolation curve $\mathcal{C}_{i}$ as initial conditions, as follows:
\begin{equation}
    \label{eq:faithful_initialcondition_interval}
    \vct{x}_{\text{init}} = \mathcal{C}_{i}(\breve{s}),\quad %
    \breve{s}\in\big\{[s_{j}-\varsigma)\cup(s_{j}+\varsigma]\big\},\quad%
    0<\varsigma\leq\frac{1}{2},
\end{equation}
where the inherent periodicity of the (quasi-)arclength parametrization of
$\mathcal{C}_{i}$ is implicitly applied. Here, $\varsigma$ was set to $0.1$,
ensuring that $20$ \% of all possible initial conditions along $\mathcal{C}_{i}$
were considered.

For computing trajectories whose initial conditions are given by
\cref{eq:faithful_initialcondition_interval} and direction fields given by
\cref{eq:faithful_local_directionfield}, the Dormand-Prince 8(7) adaptive
ODE solver (cf.\ \cref{tab:butcherdopri87}) was the method of choice. In order
to ensure that any trajectory did not overstep the half-plane
$\mathcal{H}_{i,j}$ in passing, the step length was continuously limited from
above by $\norm{\vct{x}_{\text{aim}}-\vct{x}}$. Moreover, in order to avoid
spending unreasonable computational resources on trajectories which for
practical purposes never would result in acceptable, new mesh points, the
total allowed integration arclength was limited by a scalar multiple of
the initial separation $\norm{\vct{x}_{\text{init}}-\vct{x}_{\text{aim}}}$.
In particular, this limitation meant that trajectories which ended in
stable orbits around $\vct{x}_{\text{aim}}$ were not allowed to keep going
indefinitely.

If any trajectory terminated in a point $\vct{x}_{\text{fin}}$
which lied within the half-plane $\mathcal{H}_{i,j}$ at a distance
$\Delta_{i}$ from $\vct{x}_{i,j}$, then $\vct{x}_{\text{fin}}$ was used as
coordinates for a new mesh point $\mathcal{M}_{i+1,j}$. Numerically, these
checks were implemented by means of tolerance parameters, as comparing
floating-point numbers for equivalence is prone to numerical round-off error.
More precisely, a point $\vct{x}$ was said to lie within $\mathcal{H}_{i,j}$
provided that
\begin{equation}
    \label{eq:plane_tolerance}
    \vct{\chi} = \frac{\vct{x}-\vct{x}_{i,j}}{\norm{\vct{x}-\vct{x}_{i,j}}};%
    \quad \abs{\inp[]{\vct{\chi}}{\vct{t}_{i,j}}} < \gamma_{\mathcal{H}},
\end{equation}
whereas
\begin{equation}
    \label{eq:dist_tolerance}
    \abs{\frac{\norm{\vct{x}-\vct{x}_{i,j}}}{\Delta_{i}}-1} < \gamma_{\Delta}
\end{equation}
sufficed for it to be flagged as lying a distance $\Delta_{i}$ from
$\vct{x}_{i,j}$, with $\gamma_{\mathcal{H}}$ and $\gamma_{\Delta}$ small
numbers. When a trajectory first intersected $\mathcal{H}_{i,j}$, the
integration was stopped abrubtly, leaving its endpoint $\vct{x}_{\text{fin}}$
as its suggested coordinates for the new mesh point. As briefly mentioned in
\cref{sub:parametrizing_the_innermost_level_set}, the unit tangent vectors
$\vct{t}_{i,j}$ were generally inherited --- the treatment of special cases
will be explained in greater detail in
\emph{\textbf{SETT INN REF TIL AVSN OM INNSETTING OG FJERNING AV PKT}}.
\Cref{fig:faithful_point_generation} depicts a few typical trajectory patterns
which commonly occured when searching for new mesh points in the fashion
discussed in the above.

\documentclass[crop]{standalone}
\usepackage{tikz}
\usepackage[]{tikz-3dplot}
\usepackage{pgfplots}
\pgfplotsset{compat=1.15}
\usepackage[]{amsmath}
\usepackage[]{libertine}
\usepackage[libertine]{newtxmath}
\usepackage[]{bm}
\usepackage[]{physics}
% Macros for greek letters in roman style, in math mode
\DeclareRobustCommand{\mathup}[1]{%
\begingroup\ensuremath\changegreek\mathrm{#1}\endgroup}
\DeclareRobustCommand{\mathbfup}[1]{%
\begingroup\ensuremath\changegreek\bm{\mathrm{#1}}\endgroup}


\makeatletter
\def\changegreek{\@for\next:={%
        alpha,beta,gamma,delta,epsilon,zeta,eta,theta,iota,kappa,lambda,mu,nu,%
        xi,pi,rho,sigma,tau,upsilon,phi,chi,psi,omega,varepsilon,varpi,%
    varrho,varsigma,varphi}%
\do{\expandafter\let\csname\next\expandafter\endcsname\csname\next up\endcsname}}
\makeatother

% Define vectors in bold, roman, lowercase font
\newcommand{\vct}[1]{\ensuremath{\mathbfup{\MakeLowercase{#1}}}}

% Define unit vectors in bold, roman, lowercase font, with hats
\newcommand{\uvct}[1]{\ensuremath{\mathbfup{\hat{\MakeLowercase{#1}}}}}

% Define matrices in bold, roman, uppercase font
\newcommand{\mtrx}[1]{\ensuremath{\mathbfup{\MakeUppercase{#1}}}}
\usetikzlibrary{%
    angles,%
    arrows.meta,%
    backgrounds,%
    calc,%
    decorations,%
    fit,%
    hobby,%
    patterns,%
    positioning,%
    quotes
}


\tdplotsetmaincoords{70}{120}
\begin{document}
\begin{tikzpicture}[tdplot_main_coords]
    \pgfmathsetmacro{\radius}{5}
    \pgfmathsetmacro{\size}{2.5}
%    % Place the set of initial points
    \foreach [count = \i] \ang in {205,240,260,280,165}%
    {%
        \coordinate (\i) at ( {\radius*cos(\ang-90)}, {\radius*sin(\ang-90)},{0} );
    }

    % Place coordinates giving the circle of positions at which one aims
    \coordinate (tp) at ({\radius*cos(150)},{\radius*sin(150)},1);
    \coordinate (bt) at ({\radius*cos(150)},{\radius*sin(150)},-1);


    % Define coordinates of half-plane $\mathcal{H}_{i,j}$
    \coordinate (lu) at ($(2)!2!(tp)$);
    \coordinate (ld) at ($(2)!2!(bt)$);
    \coordinate (ru) at ($(lu)+7*({-0.866},{0.5},0)$);
    \coordinate (rd) at ($(ld)+7*({-0.866},{0.5},0)$);
    % Shade half-plane
    \draw[fill=gray!30,draw opacity=0] (lu) -- (ld) -- (rd) -- (ru) -- cycle;
    % Draw interpolation curve
    \draw[stroke=black!80,thin,dotted] (70:\radius) arc (70:200:\radius);
    % Draw $\vct{rho}_{i,j}$ at $x_{i,j}$
    \path[draw,stroke=black!65,->,thick,dashdotted ] (2) to ($(2) + 5*({-0.866},{0.5},0)$) coordinate (rhoend);
    \node[below left = 7.5pt and 35pt of rhoend,rotate=12.5] {$\vct{\rho}_{i,j}$};
    \node[above left = 0pt and 0pt of rd] {$\mathcal{H}_{i,j}$};

    % Define coordinates of pivot for attaching $\mathcal{C}_{i}$ label
    \coordinate (mrk) at ({\radius*cos(195)},{\radius*sin(195)},0);
    % Attach label
    \coordinate [above = 10pt of mrk] (lbl);
    \node at (lbl) {$\mathcal{C}_{i}$};
    \draw[stroke=black!80] ($(mrk)!0.1!(lbl)$) to[in= 270,out=20] ($(mrk)!0.55!(lbl)$);

    % Draw points in initial level set
    \foreach \i in {1,2,3,4,5}%
    {%
        \draw[fill=gray!10,stroke=black!65] (\i) circle (\size pt);
    }
    % Draw semicircle of radius $\Delta_{i}$, centered at $x_{i,j}$
    \pic[densely dashed,draw,stroke=grey!80,angle radius=39] {angle = bt--2--tp};



    % Label ancestor point
    \coordinate[below left = 0pt and 30pt of 2] (plbl);
    \node at (plbl) {$\mathcal{M}_{i,j}$};
    \draw[stroke=black!80] ($(plbl)!0.4!(2)$) to[out = 20, in = 160]  ($(plbl)!0.85!(2)$);


    % Define aim point
    \coordinate (aim) at ($(2) + 1.2*({-0.866},{0.5},0) + (0,0,0.75)$);

    % Define support point in semicircle, to indicate radius
    \coordinate (supp) at ($(2) + 1.57*({-0.866},{0.5},0)$);
    % Draw semicircle to indicate angle offset of aim point
    \pic[draw,stroke=gray!80,angle eccentricity = 1.7,angle radius=10] {angle = supp--2--aim};
    % Add text node to explain the angle
    \node[below right = 10pt and -2.5pt of 2] (mrk) {$\alpha_{i,j}$};
    % Add anchor for line from node to within angle
    \coordinate (inang) at ($(2) + 0.25*({-0.866},{0.5},0) + (0,0,0.25)$);
    \path[draw,stroke=black!80,thin] ($(mrk)!0.1!(inang)$) to [out = 60, in = -180+60] ($(mrk)!0.8!(inang)$);
    % Draw vector from start point to aim point
    \draw[stroke=black!65,thin,dashed] (2) -- (aim);
    % Redraw start point
    \draw[fill=gray!10,stroke=black!65] (2) circle (\size pt);
    % Draw aim point
    \draw[stroke=black!65,fill=gray!10] (aim) circle (\size pt);
    \node[above right = 0pt and 0pt of aim]  {$\vct{x}_{\text{aim}}$};
    % Suggest initial radius
    \draw[decorate,decoration={brace,amplitude=5pt,raise=0pt},yshift=0pt] ($(2)!1.46!(bt)$) -- (2)   node [midway,left=2pt] {\footnotesize{$\Delta_{i}$}};

    % Draw some trajectories, the last of which manages to find aim point
    \coordinate (p1i) at ($({\radius*cos(180)},{\radius*sin(180)},0)$);
    \coordinate (p1f) at ($(2) + 2.8*({-0.866},{0.5},0) + (0,0,0.75)$);
    \coordinate (p2i) at ($({\radius*cos(160)},{\radius*sin(160)},0)$);
    \coordinate (p2f) at ($(2) + 1.1*({-0.866},{0.5},0) + 1.7*(0,0,0.75)$);
    \coordinate (p3i) at ($({\radius*cos(140)},{\radius*sin(140)},0)$);
    \coordinate (p3f) at ($(2) + 0.7*({-0.866},{0.5},0) + (0,0,0.75)$);
    \coordinate (p4i) at ($({\radius*cos(125)},{\radius*sin(125)},0)$);
    \path[draw,stroke=black!80,->] (p1i) to[out = 30, in = 150]  (p1f);
    \path[draw,stroke=black!80,->] (p2i) to[out = 30, in = 140]  (p2f);
    \path[draw,stroke=black!80,->] (p3i) to[out = 40, in = 30]  (p3f);
    \path[draw,stroke=black!80,->] (p4i) to[out = 40, in = -70]  (aim);

\end{tikzpicture}
\end{document}


