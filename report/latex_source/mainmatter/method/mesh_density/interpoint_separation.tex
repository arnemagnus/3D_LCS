\subsection{Maintaining mesh point density}
\label{sub:maintaining_mesh_point_density}

When expanding a manifold by computing new geodesic level sets, the distance
separating neighboring mesh points within each subsequent level set generally
increases. This is due to the mesh points being a parametrization of what is
essentially an expanding topological circle. Having successfully computed
a new geodesic level set --- that is, having found a descendant point
$\mathcal{M}_{i+1,j}$ for each of the ancestor points $\{\mathcal{M}_{i,j}\}$
--- by use of the method outlined in
\cref{sec:revised_approach_to_computing_new_mesh_points}
(or \cref{sec:legacy_approach_to_computing_new_mesh_points}), we then inspected
all of the distances separating nearest neighbors.

If any of the separations between nearest neighbors exceeded $\Delta_{\max}$,
we sought to insert a new mesh point inbetween. Specifically, if
$\norm{\vct{x}_{i+1,j}-\vct{x}_{i+1,j+1}}>\Delta_{\max}$, a new mesh point
$\mathcal{M}_{i+1,j+\frac{1}{2}}$ was computed, using the method described
in \cref{sec:revised_approach_to_computing_new_mesh_points} (or
\cref{sec:legacy_approach_to_computing_new_mesh_points}), by launching
a trajectory starting from a ficticious ancestor point
$\mathcal{M}_{i,j+\frac{1}{2}}$, located midway inbetween $\mathcal{M}_{i,j}$
and $\mathcal{M}_{i,j+1}$ along the interpolation curve $\mathcal{C}_{i}$. As
the ficticious mesh point $\mathcal{M}_{i,j+\frac{1}{2}}$ does not itself have
a direct ancestor from which to inherit a unit tangent,
$\vct{t}_{i,j+\frac{1}{2}}$ was instead constructed by normalizing the
arithmetic average of $\vct{t}_{i,j}$ and $\vct{t}_{i,j+1}$, and passed
on to $\mathcal{M}_{i,j+\frac{1}{2}}$. This way, the interpolation error
is limited; no new mesh points are generated using interpolations (in
intermediary computations) over intervals of length exceeding
$\Delta_{\max}$.

Conversely, if any of the nearest neighbor separations became smaller than
$\Delta_{\min}$, we sought to remove one of them, as long as the
distance separating mesh points which would then become nearest neighbors
did not exceed $\Delta_{\max}$. Accordingly, if any one of a pair of neighboring
mesh points was to be removed, we chose to discard the one which would result
in the smallest separation between the ensuing, \emph{new} pair of nearest
neighbors. Our principles for inserting new mesh points inbetween others,
and removing grid points which are too close together, are illustrated
in \emph{\textbf{SETT INN REF TIL FIG}}.




