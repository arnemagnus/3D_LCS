\subsection{Maintaining mesh point density}
\label{sub:maintaining_mesh_point_density}

When expanding a manifold by computing new geodesic level sets, the distance
separating neighboring mesh points within each subsequent level set generally
increases. This is due to the mesh points being a parametrization of what is
essentially an expanding topological circle, for which an increasing amount of
sampling points are needed in order to maintain the point density. This concept
is illustrated in \cref{fig:rationale_for_inserting_new_points}. Thus, having
successfully computed a new geodesic level set --- that is, having found a
descendant point $\mathcal{M}_{i+1,j}$ for each of the ancestor points
$\{\mathcal{M}_{i,j}\}$ --- by use of the method outlined in
\cref{sec:revised_approach_to_computing_new_mesh_points},
we then inspected all of the distances separating nearest neighbors.

\begin{figure}[htpb]
    \centering
    \resizebox{0.9\linewidth}{!}%
    {\includestandalone{figures/tikz-figs/concept_of_inserting_new_points}}
    \caption[Conceptual illustration of the rationale behind the insertion of
    new mesh points as the geodesic level sets expand]
    {Conceptual illustration of the rationale behind the insertion of new mesh
        points as the geodesic level sets expand. The shown subset of the
        innermost topological circle is parametrized by three mesh points,
        shown as dark gray circles. As the topological circle is expanded, an
        increasing amount of mesh points must be inserted in order to maintain
        the (approximate) mesh point density. In the intermediate-sized circle
        shown in the figure, the mesh points with no direct analogue in the
        smallest circle are shown in a lighter shade of gray. Similarly, in the
        largest circle shown, the mesh points with no direct analogue in the
        intermediate-sized circle are drawn without fill.
    }
    \label{fig:rationale_for_inserting_new_points}
\end{figure}



Specifically, if any of the separations between nearest neighbors exceeded
$\Delta_{\max}$, we sought to insert a new mesh point between them. That is, if
$\norm{\vct{x}_{i+1,j}-\vct{x}_{i+1,j+1}}>\Delta_{\max}$, a new mesh point
$\mathcal{M}_{i+1,j+\frac{1}{2}}$ was computed, using the method described
in \cref{sec:revised_approach_to_computing_new_mesh_points} by launching a
trajectory starting from a ficticious ancestor point
$\mathcal{M}_{i,j+\frac{1}{2}}$, located midway inbetween $\mathcal{M}_{i,j}$
and $\mathcal{M}_{i,j+1}$ along the interpolation curve $\mathcal{C}_{i}$. As
the ficticious mesh point $\mathcal{M}_{i,j+\frac{1}{2}}$ does not itself have
a direct ancestor from which to inherit a unit tangent,
$\vct{t}_{i,j+\frac{1}{2}}$ was instead constructed by normalizing the
arithmetic average of $\vct{t}_{i,j}$ and $\vct{t}_{i,j+1}$, and passed
on to $\mathcal{M}_{i,j+\frac{1}{2}}$. The ficticious mesh point's guidance
vector $\vct{\rho}_{i,j+\frac{1}{2}}$ was constructed in similar fashion ---
its role in the dynamic update of the interset separation $\Delta_{i}$ will
be described in detail in
\cref{sub:a_curvature_based_approach_to_determining_interset_separations}.
This way, the interpolation error is limited; no new mesh points are generated
using interpolations (in intermediary computations) over intervals of length
exceeding $\Delta_{\max}$.

Conversely, if any of the nearest neighbor separations became smaller than
$\Delta_{\min}$, we sought to remove one of the points, as long as the
distance separating mesh points which would then become nearest neighbors
did not exceed $\Delta_{\max}$. Accordingly, if any one of a pair of
neighboring mesh points was to be removed, we chose to discard the one which
would result in the smallest separation between the ensuing, \emph{new} pair of
nearest neighbors. In our experience, the removal of mesh points rarely
occurred; however, it was crucial for generating the spherical LCS surface
which will be presented in \cref{sub:an_analytical_lcs_test_case}.
Our principles for inserting new mesh points inbetween
others, and removing grid points which are too close together, are illustrated
in \cref{fig:mesh_management_insertion_and_deletion}.

\begin{figure}[htpb]
    \centering
    \begin{subfigure}[b]{0.475\textwidth}
        \centering
        \resizebox{0.9\linewidth}{!}{\includestandalone{figures/tikz-figs/density_management_inserting_new_point}}
        \caption[]{{\small Inserting a new mesh point inbetween a \\\phantom{(a)} pair of mesh points which are too far apart}}
        \label{fig:mesh_management_pure_insertion}
    \end{subfigure}
    \begin{subfigure}[b]{0.475\textwidth}
        \centering
        \resizebox{0.9\linewidth}{!}{\includestandalone{figures/tikz-figs/density_management_removing_point}}
        \caption[]{{\small Removing a mesh point too close to another,\\\phantom{(b)} if the ensuing separations are acceptable}}
        \label{fig:mesh_management_pure_deletion}
    \end{subfigure}
    \caption[Our approach to inserting new, or removing, mesh points to maintain
    mesh point density]
    {Our approach to inserting new, or removing, mesh points to maintain
        mesh point density. When two neighboring mesh points in a freshly
        computed level set are too far apart with regards to the given mesh
        parameter $\Delta_{\max}$, we attempt to insert a new mesh point
        between them. As shown in
        (\subref*{fig:mesh_management_pure_insertion}), this is done by using
        the method described in
        \cref{sec:revised_approach_to_computing_new_mesh_points}, using a
        ficticious initial condition midway inbetween the two ancestor mesh
        points $\mathcal{M}_{i,j}$ and $\mathcal{M}_{i,j+1}$ along the
        interpolated curve $\mathcal{C}_{i}$, indicated in the figure by
        a lighter shade of gray. Should two neighboring mesh points
        be too close together, and one of the two can be removed without the
        resulting sets of neighboring points being too far apart, we remove
        the one which results in the shortest interpoint separation (as shown
        in (\subref*{fig:mesh_management_pure_deletion}), where the point
        which is removed is indicated with a lighter shade of gray).
    }
    \label{fig:mesh_management_insertion_and_deletion}
\end{figure}


Having completed the process of inserting and removing mesh points, the
points in the level set were assigned consecutive integer indices. In
particular, if a mesh point $\mathcal{M}_{i,j+\frac{1}{2}}$ was inserted
inbetween $\mathcal{M}_{i,j}$ and $\mathcal{M}_{i,j+1}$, the respective indices
were updated as $\mathcal{M}_{i,j+\frac{1}{2}}\to\mathcal{M}_{i,j+1}$,
$\mathcal{M}_{i,j+1}\to\mathcal{M}_{i,j+2}$ and so forth. Conversely, if mesh
point $\mathcal{M}_{i,j+1}$ was removed, then
$\mathcal{M}_{i,j+2}\to\mathcal{M}_{i,j+1}$,
$\mathcal{M}_{i,j+3}\to\mathcal{M}_{i,j+2}$ and so on.
