\subsection{Limiting the accumulation of numerical noise}
\label{sub:limiting_the_accumulation_of_numerical_noise}

Early tests with regards to the generation of manifolds revealed that
the compound numerical error over the course of many level sets often resulted
in irregular behaviour. Specifically, small bulges in the mesh easily became
amplified in the subsequent level sets, which frequently caused unwanted
loops in the interpolation curves $\{\mathcal{C}_{i}\}$ which extended
far from the manifold epicentre (namely $\vct{x}_{0}$, cf.\
\cref{sub:identifying_suitable_initial_conditions_for_developing_lcss}).
Occasionally, this would lead to the generated manifold to fold into itself.
Per their definition, this is never permitted for repelling LCSs. In
particular, self-intersecting manifolds indicate that, in a small neighborhood
of  any intersection, there would not be a well-defined direction of strongest
repulsion, which would violate LCS existence criterion
\eqref{eq:lcs_condition_a}.

As a countermeasure against the formation of undesired loops within a level
set, we reviewed a recently computed geodesic level set, as follows: Consider a
point $\mathcal{M}_{i+1,j}$ located at $\vct{x}_{i+1,j}$. If, for any point
$\mathcal{M}_{i+1,j+k}$ with $k>1$, the separation between the mesh points
$\mathcal{M}_{i+1,j}$ and $\mathcal{M}_{i+1,j+k}$ satisfies the pre-set bounds
for the mesh point density (cf.\ \cref{sub:maintaining_mesh_point_density}),
and is significantly smaller than the cumulative nearest neighbor separations
in the mesh point sequence $\{\mathcal{M}_{i+1,j+\kappa}\}_{\kappa=0}^{k}$,
we would remove the intermittent mesh points. In more mathematical terms; if
the interpoint distances satisfy
\begin{subequations}
    \label{eq:loop_removal_conditions}
    \begin{equation}
        \label{eq:loop_removal_condition_one}
        \Delta_{\min} < \norm{\vct{x}_{i+1,j+k}-\vct{x}_{i+1,j}} < %
        \Delta_{\max}
    \end{equation}
    and
    \begin{equation}
        \label{eq:loop_removal_condition_two}
        \norm{\vct{x}_{i+1,j+k}-\vct{x}_{i+1,j}} < %
        \gamma_{\circlearrowleft} \sum\limits_{\kappa=0}^{k-1}%
        \norm{\vct{x}_{i+1,j+\kappa+1}-\vct{x}_{i+1,j+\kappa}},
    \end{equation}
\end{subequations}
then all mesh points ${\{\mathcal{M}_{i+1,j+\kappa}\}}_{\kappa=1}^{k-1}$
are removed. Here, $0 \leq \gamma_{\circlearrowleft} \leq 1$ is a bulge
tolerance parameter, which essentially determines an upper limit for the extent
(measured in cumulative arclength) of loop-like segments of any given
interpolation curve $\mathcal{C}_{i}$. Specifically, a large
$\gamma_{\circlearrowleft}$ facilitates removal of rounded loops, whereas a
small $\gamma_{\circlearrowleft}$ restricts the removal process to sharp
bulges.

A characteristic example demonstrating this method of removing unwanted
loops is shown in \cref{fig:loop_removal}. While possibly sacrificing some
resolution of the manifold as a whole, adhering to criterion
\eqref{eq:loop_removal_conditions} prevents the removal of any \emph{reasonable}
bulge formations. The removal of several consecutive mesh
points in the manner outlined may cause some triangle elements in the
reconstruction of manifold surfaces from the resulting point meshes (see
\cref{sec:continuously_reconstructing_three_dimensional_manifold_surfaces%
_from_point_meshes}) to be somewhat larger than their neighbors, and, in some
cases, some triangles may partly overlap. However, as no new mesh points are
\emph{added} as a direct consequence of this loop-removal algorithm, it does
not introduce new errors. It is also worth mentioning that some of the removed
mesh points were recovered in the ensuing level sets, then as constituents of
an overall smoother level set. In our experience, the removal of
loop-forming mesh points was rarely required.

\begin{figure}[htpb]
    \centering
    \resizebox{0.9\linewidth}{!}{\includestandalone{figures/tikz-figs/loop_removal}}
    \caption[Our approach to limit the compound numerical error, by continuously
    removing unwanted loops in the computed level sets]
    {Our approach to limit the compound numerical error, by continuously
    removing unwanted loops in the computed level sets. As the separation
between the mesh points $\mathcal{M}_{i,j}$ and $\mathcal{M}_{i,j+k}$
satisfies the restrictions regarding mesh point density
(cf.\ \cref{sub:maintaining_mesh_point_density}), the mesh points
${\{\mathcal{M}_{i,j+\kappa}\}}_{\kappa=1}^{k-1}$ (shown in a lighter
shade of gray) are removed, provided that the cumulative neighbor separation
around the bulge is sufficiently large (per the conditions presented in
\cref{eq:loop_removal_conditions}) --- which is the case in the above.
The resulting interpolation curve $\mathcal{C}_{i}$ becomes significantly
more well-behaved without the superfluous mesh points.
}
    \label{fig:loop_removal}
\end{figure}

