\subsection{A curvature-based approach to determining interset separations}
\label{sub:a_curvature_based_approach_to_determining_interset_separations}

Given the interpoint separation constraints described in
\cref{sub:maintaining_mesh_point_density}, we have some flexibility regarding
the choice of interset step length $\Delta_{i}$. In an approach closely
mirroring that of \textcite{krauskopf2005survey}, we used the (approximate)
local curvatures along each point strain (i.e.,\ the set of mesh points
which can be traced back to a common ancestor) in order to determine
whether or not the local manifold dynamics were resolved to a satisfactory
level of detail. Starting out with an initial interset separation
$\Delta_{1}=2\Delta_{\min}$ for the second innermost level set (i.e.,\, the
first level set which was computed using the method described in
\cref{sec:revised_approach_to_computing_new_mesh_points}), we sought to
ensure that the subsequent $\Delta_{i}$ resulted in the encapsulation of the
finer details of the growing manifolds.

Specifically, once all mesh points which constitute a geodesic level set
$\mathcal{M}_{i+1}$ have been identified, all a distance $\Delta_{i}$ away from
their direct ancestor points in the previous level set $\mathcal{M}_{i}$, we
computed the angular offsets $\alpha_{i,j}$ of each pair of guidance vectors
$\vct{\rho}_{i,j}$ and $\vct{\rho}_{i+1,j}$ (as defined in
\cref{eq:innermost_prevvec,eq:general_prevvec}). This is sketched in
\cref{fig:angular_adjustment}. If
\begin{equation}
    \label{eq:decrease_dist}
    \alpha_{i,j} > \alpha_{\downarrow} \quad \text{or} \quad %
    \Delta_{i}\cdot\alpha_{i,j} > (\Delta\alpha)_{\downarrow} \quad %
    \text{for at least one } j,
\end{equation}
where $\alpha_{\downarrow}$ and $(\Delta\alpha)_{\downarrow}$ are upper
curvature tolerance parameters, was satisfied, the level set $\mathcal{M}_{i+1}$
was discarded and recomputed with reduced $\Delta_{i}$. If $\Delta_{i}$ was
already as low as the pre-set $\Delta_{\min}$, the manifold generation
process was terminated, as a new level set $\mathcal{M}_{i+1}$ could not be
computed under the given constraints on interpoint separation (cf.\
\cref{sub:maintaining_mesh_point_density}). Conversely, if
\begin{equation}
    \label{eq:increase_dist}
    \alpha_{i,j} < \alpha_{\uparrow} \quad \text{and} \quad %
    \Delta_{i}\cdot\alpha_{i,j} < (\Delta\alpha)_{\uparrow} \quad %
    \text{for all } j,
\end{equation}
where $\alpha_{\uparrow}$ and $(\Delta\alpha)_{\uparrow}$ are lower curvature
tolerance parameters, was satisfied, the interset distance for computing the
\emph{next} level set, $\Delta_{i+1}$, was made bigger than $\Delta_{i}$
(although never beyond the pre-set $\Delta_{\max}$).

\documentclass[crop]{standalone}
\usepackage{tikz}
\usepackage[]{tikz-3dplot}
\usepackage{pgfplots}
\usepackage[]{amsmath}
\usepackage[]{libertine}
\usepackage[libertine]{newtxmath}
\usepackage[]{bm}
\usepackage[]{physics}
% Macros for greek letters in roman style, in math mode
\DeclareRobustCommand{\mathup}[1]{%
\begingroup\ensuremath\changegreek\mathrm{#1}\endgroup}
\DeclareRobustCommand{\mathbfup}[1]{%
\begingroup\ensuremath\changegreek\bm{\mathrm{#1}}\endgroup}

\makeatletter
\def\changegreek{\@for\next:={%
        alpha,beta,gamma,delta,epsilon,zeta,eta,theta,iota,kappa,lambda,mu,nu,%
        xi,pi,rho,sigma,tau,upsilon,phi,chi,psi,omega,varepsilon,varpi,%
    varrho,varsigma,varphi}%
\do{\expandafter\let\csname\next\expandafter\endcsname\csname\next up\endcsname}}
\makeatother

% Define vectors in bold, roman, lowercase font
\newcommand{\vct}[1]{\ensuremath{\mathbfup{\MakeLowercase{#1}}}}

% Define unit vectors in bold, roman, lowercase font, with hats
\newcommand{\uvct}[1]{\ensuremath{\mathbfup{\hat{\MakeLowercase{#1}}}}}

% Define matrices in bold, roman, uppercase font
\newcommand{\mtrx}[1]{\ensuremath{\mathbfup{\MakeUppercase{#1}}}}
\usetikzlibrary{%
    angles,%
    arrows.meta,%
    backgrounds,%
    calc,%
    decorations,%
    fit,%
    hobby,%
    patterns,%
    positioning,%
    quotes
}
\makeatletter
\tikzset{
  fitting node/.style={
    inner sep=0pt,
    fill=none,
    draw=none,
    reset transform,
    fit={(\pgf@pathminx,\pgf@pathminy) (\pgf@pathmaxx,\pgf@pathmaxy)}
  },
  reset transform/.code={\pgftransformreset}
}
\makeatother
% A simple empty decoration, that is used to ignore the last bit of the path
\pgfdeclaredecoration{ignore}{final}
{
\state{final}{}
}

% Declare the actual decoration.
\pgfdeclaremetadecoration{middle}{initial}{
    \state{initial}[
        width={0pt},
        next state=middle
    ]
    {\decoration{moveto}}

    \state{middle}[
        width={\pgfdecorationsegmentlength*\pgfmetadecoratedpathlength},
        next state=final
    ]
    {\decoration{curveto}}

    \state{final}
    {\decoration{ignore}}
}


% Create a key for easy access to the decoration
\tikzset{middle segment/.style={decoration={middle},decorate, segment length=#1}}

\def\getangle(#1)(#2)#3{%
  \begingroup%
    \pgftransformreset%
    \pgfmathanglebetweenpoints{\pgfpointanchor{#1}{center}}{\pgfpointanchor{#2}{center}}%
    \expandafter\xdef\csname angle#3\endcsname{\pgfmathresult}%
  \endgroup%
}


\begin{document}

\tdplotsetmaincoords{70}{20}

\begin{tikzpicture}[tdplot_main_coords]
	\pgfmathsetmacro{\innerxscl}{2.5}
    \pgfmathsetmacro{\inneryscl}{2.5}
	\pgfmathsetmacro{\outerxscl}{4.5}
	\pgfmathsetmacro{\outeryscl}{4.5}
    \pgfmathsetmacro{\innernum}{11} % = Degree span / separation, outer for loop
    \pgfmathsetmacro{\outernum}{21}
    % Inner geodesic level set
    \foreach [count = \i] \a in {15,48,...,345}%
    {%
        \coordinate (i\i) at ( {\innerxscl*cos(\a)} , {\inneryscl*sin(\a)} , 0  ) ;
    }%
    \draw[stroke=black!80,thin,dotted] (i1) to [ curve through ={(i2) .. (i3) .. (i4) .. (i5) .. (i6) .. (i7) .. (i8) .. (i9) .. (i10) .. (i11) }] (i1);
    \foreach \i in {1,...,\innernum}%
    {%
        \draw[stroke=black!80, fill=gray!60] (i\i) circle (3pt);
    }%

    % Outer geodesic level set
    \foreach [count = \i] \a in {10,27,...,350}%
    {%
        \coordinate (o\i) at ( {\outerxscl*cos(\a)} , {\outeryscl*sin(\a)} , 0 ) ;
    }%
    \draw[stroke=black!80,thin,dotted] (o1) to [ tension = 0.3, curve through ={(o2)  .. (o3)  .. (o4)  .. (o5)  .. (o6)  .. (o7)  ..
                                                                 (o8)  .. (o9)  .. (o10) .. (o11) .. (o12) .. (o13) ..
                                                             (o14) .. (o15) .. (o16) .. (o17) .. (o18) .. (o19) .. (o20) .. (o21)}] (o1);

    \foreach \i in {1,...,\outernum}%
    {%
        \draw[stroke=black!80, fill=gray!60] (o\i) circle (3pt);
    }%
    \begin{scope}[on background layer]
    % Draw line parallel to prev_vec
    \draw[stroke=black!80,->,densely dashed] (i1) to ($(i1)!2.25!(o1)$) coordinate (vf); % Draw 225 % of the path from (i1) to (o1)
    % Suggested new point
    \coordinate (n1) at ( {5.2*cos(10)} , {5.1*cos(10)}, 0.1 );
    \draw[stroke=black!80,fill=gray!60] (n1) circle (3pt) ;
    % Draw prev_vec for new point
    \draw[stroke=black!80,->,dashed] (o1) to ($(o1)!0.95!(n1)$);

    \pic [draw,-{>[scale=1.1]}, stroke=black!80, angle radius = 15,angle eccentricity = 1.5] {angle= vf--o1--n1};
    \node at (o1) [above right = -3pt and 15pt] {$\theta$};

    \end{scope}

    % Add nodes
    \node at (i1) [below right] {$\mathcal{P}_{i-1,j}$};
    \node at (o1) [below right] {$\mathcal{P}_{i,j}$};
    \node at (n1) [below right= 3pt and -5pt] {$\mathcal{P}_{i+1,j}$};

    % Denote inner and outer parametrizations
    \node at ($(i3)!0.6!(i4)$) (innerpt) {};
    \node [below left =6pt and 1pt of innerpt] (innermrk) {$\mathcal{C}_{i-1}$};

    \draw[stroke=black!80] ($(innerpt)!0.7!(innermrk)$) to [out=30,in=-60] ($(innermrk)!0.9!(innerpt)$);

    \node at ($(o4)!0.6!(o5)$) (outerpt) {};
    \node [below right=5pt and 1pt of outerpt] (outermrk) {$\mathcal{C}_{i}$};
    \draw[stroke=black!80] ($(outerpt)!0.6!(outermrk)$) to [out=135,in=-70] ($(outermrk)!0.9!(outerpt)$);



\end{tikzpicture}
\end{document}


As is evident from \cref{eq:decrease_dist,eq:increase_dist}, the parameters
$\alpha_{\downarrow}$, $\alpha_{\uparrow}$, $(\Delta\alpha)_{\downarrow}$ and
$(\Delta\alpha)_{\uparrow}$ determine the mesh adaption along point strains.
Choosing appropriate bounds for the offsets $\{\Delta_{i}\cdot\alpha_{i,j}\}$
allows for stricter requirements on angular offsets for large interstep lengths.
Similarly, level sets computed using small interset lenghts are allowed
(comparatively) larger angular offsets in general. In our experience, the
interset step lenghts were rarely \emph{increased}  --- typically, there would
be one or more subsets of the mesh points constituting any given level set which
underwent sufficient curvature such that condition~\eqref{eq:increase_dist} did
not hold.
