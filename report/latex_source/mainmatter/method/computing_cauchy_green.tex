\section[Computing Cauchy-Green strain eigenvalues and -vectors]
{Computing Cauchy-Green strain eigenvalues and \newline\phantom{3.2} -vectors}
\label{sec:computing_cauchy_green_strain_eigenvalues_and_vectors}

Computing the Cauchy-Green strain tensor field directly, by performing a series
of matrix products per its definition in \cref{eq:defn_cauchygreen}, and then
solving for its eigenvalues and -vectors turns out to be numerically
disadvantageous \parencite{oettinger2016autonomous}. In particular, this
method leaves the smallest eigenvalues quite susceptible to numerical
round-off error. A fully equivalent, more numerically sound way of identifying
the Cauchy-Green strain eigenvalues and -vectors is based on performing an
SVD decomposition of the Jacobian field of the flow map, i.e.,
\begin{equation}
    \label{eq:jacobi_flow_map_svd}
    \grad\vct{\phi}_{t_{0}}^{t}(\vct{x}_{0}) = %
    \mtrx{U}\mtrx{\Sigma}\mtrx{V}^{\ast},
\end{equation}
where the asterisk refers to the adjoint operation, $\mtrx{U}$ and $\mtrx{V}$
are unitary matrices and $\mtrx{\Sigma}$ is a diagonal matrix
with nonnegative real numbers --- the \emph{singular values} of
$\grad\vct{\phi}$ --- on the diagonal. Because the flow map Jacobian is square,
so too are the matrices $\mtrx{U}$, $\mtrx{\Sigma}$ and $\mtrx{V}$. Moreover,
as the flow map Jacobian is real-valued, so too are the matrices $\mtrx{U}$ and
$\mtrx{V}$. The eigenvalues of the right Cauchy-Green strain tensor (cf.
\cref{eq:defn_cauchygreen,eq:cauchygreen_characteristics}) are given by the
squares of the singular values, that is, $\lambda_{i}(\vct{x}_{0}) =%
\big(\sigma_{i}(\vct{x}_{0})\big)^{2}$, and the corresponding orthonormal
eigenvectors are found in the columns of $\mtrx{V}$.

\subsubsection{Interpolating the Cauchy-Green strain eigenvalues}
\label{ssub:interpolating_the_cauchy_green_strain_eigenvalues_and_vectors}

For computing LCSs, the Cauchy-Green strain eigenvalues frequently need to
be evaluated inbetween the grid points. Moreover, as suggested by
the existence criterion given in \cref{eq:lcs_condition_b}, all of the second
derivatives of $\lambda_{3}(\vct{x}_{0})$ are also needed. Accordingly, the
eigenvalues were interpolated by means of cubic trivariate B-splines, in order
to ensure continuous second derivatives. For this purpose, the
\texttt{bspline\_3d} derived type from the Bspline-Fortran library
\parencite{williams2018bspline} was ported to Python using the techniques
described in \cref{sub:interpolating_gridded_velocity_data}.

\subsubsection{Interpolating the Cauchy-Green strain eigenvectors}
\label{ssub:interpolating_the_cauchy_green_strain_eigenvectors}

Just like the eigenvalues of the Cauchy-Green strain tensor field, its
eigenvectors frequently need to be evaluated between the grid points in order
to compute LCSs. Like the strain eigenvalues, the strain eigenvectors were
interpolated by means of cubic trivariate B-splines, through the
\texttt{bspline\_3d} derived type from the Bspline-Fortran library
\parencite{williams2018bspline} --- albeit with a twist, in
order to remove local orientational discontinuities. In particular, the
local stretch is equal in magnitude along any given negative $\vct{\xi}_{i}$
axis as that of its positive counterpart, and there is no a priori reason to
expect the SVD decomposition (cf.\ \cref{eq:jacobi_flow_map_svd}) to follow
any particular convention regarding the ``sign'' of the computed eigenvectors.

The interpolation routine outlined here is a generalization of a similar
special-purpose linear interpolation routine which has previously been
utilized to compute LCSs in two spatial dimensions
\parencite{onu2015lcstool,loken2017sensitivity}. The routine is based upon
careful monitoring and local reorientation prior to cubic interpolation, and
its two-dimensional equivalent is illustrated in \cref{fig:special_interp}
--- the principles are similar in three dimensions, but illustrating a
two-dimensional projection of the three-dimensional case simply became
cluttered beyond comprehension.

\begin{figure}[htpb]
    \centering
    \resizebox{0.9\linewidth}{!}%
    {\includestandalone{figures/tikz-figs/special_interpolation}}
    \caption[Conceptual illustration of the special-purpose cubic interpolation
    routine for the Cauchy-Green strain eigenvectors]
    {Conceptual illustration of the special-purpose cubic interpolation routine
        for the Cauchy-Green strain eigenvectors. The 64 nearest grid points
        (16 in two dimensions; here shown in gray) to any given coordinate
        $\vct{x}$ are identified. As the local stretch is equal in magnitude
        along the negative $\vct{\xi}_{i}$ axis as that of its positive
        counterpart, we are free to reverse any $\vct{\xi}_{i}$ vector which is
        rotated more than $90\si{\degree}$ with regards to the chosen pivot
        vector (at the grid point denoted by $\mathcal{P}$) prior to
        interpolating the components of $\vct{\xi}_{i}$, making use of cubic
        B-splines. The vectors used in the local cubic interpolation are
        dashed, wherease the vectors which had to be reversed are dotted.
    }
    \label{fig:special_interp}
\end{figure}


First, the 64 (in two dimensions: 16) nearest neighboring
grid points to any given coordinate $\vct{x}$ are identified. Choosing a
vector at a corner of this local interpolation voxel, orientational
discontinuities between the grid elements are found by inspecting the
inner products of the $\vct{\xi}_{i}$ vectors of the remaining grid points
with the pivot. Rotations exceeding $90\si{\degree}$ are identified by
inner products with the pivot vector being negative, labelled as orientational
discontinuities, and then corrected by reversing the direction of the
corresponding vectors. For each of $\vct{\xi}_{i}$'s three components, cubic
B-spline interpolation is used within the interpolation voxel in order to find
$\vct{\xi}_{i}(\vct{x})$, which is then normalized, like the $\vct{\xi}_{i}$
vectors defined at the grid points are a priori.


