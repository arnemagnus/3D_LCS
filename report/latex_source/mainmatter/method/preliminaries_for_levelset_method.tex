\section[Preliminaries for computing repelling LCSs in 3D flow by means of
geodesic level sets]
{Preliminaries for computing repelling LCSs in 3D\\\phantom{3.5} flow by means
of geodesic level sets}
\label{sec:preliminaries_for_computing_repelling_lcss_in_3d_flow_by_means_of%
_geodesic_level_sets}

Repelling LCSs in three spatial dimensions are quite challenging to compute.
Straightforward numerical integration of the flow in a strain eigendirection
field suffices for Lagrangian analysis of two-dimensional systems
\parencite{farazmand2012computing,loken2017sensitivity}. In three dimensions,
however, this is not the case; LCS existence criterion~%
\eqref{eq:lcs_condition_c} implies that three-dimensional repelling LCSs are
everywhere simultaneously tangent to the $\vct{\xi}_{1}$- \emph{and}
$\vct{\xi}_{2}$-direction fields (see \cref{rmk:invariance_lcs}). Another way
to interpret this extra degree of freedom --- compared to the two-dimensional
case --- is that everywhere within three-dimensional repelling LCS, one is
allowed to move ``freely'' within a plane which is orthogonal to
$\vct{\xi}_{3}(\vct{x})$. Thus, more sophisticated algorithms are needed in
order to compute three-dimensional LCSs, than their two-dimensional
counterparts. Here, we consider a variation of the method of geodesic level
sets for computing repelling LCSs as invariant manifolds of the
$\vct{\xi}_{1}$ and $\vct{\xi}_{2}$ direction fields (cf.\ %
\cref{rmk:invariance_lcs}), as presented by \textcite{krauskopf2005survey}. The
short presentation to follow in the next paragraph will be explained further in
depth in the subsequent sections.

The method is based on the concept of developing an unstable manifold from a
local neighborhood of an initial condition $\vct{x}_{0}$ (how the initial
conditions are selected will be described in
\cref{sub:identifying_suitable_initial_conditions_for_developing_lcss}).
In particular, a small, closed curve $\mathcal{C}_{1}$ forming a geodesic
circle, consisting of mesh points which all lie within the tangent plane
defined by the coordinate $\vct{x}_{0}$ and the unit normal
$\vct{\xi}_{3}(\vct{x}_{0})$, separated from $\vct{x}_{0}$ by a distance
$\delta_{\text{init}}$, is assumed to lie within the same manifold as
$\vct{x}_{0}$. The idea is then to compute the next geodesic circle in a local,
dynamic coordinate system, defined by hyperplanes which are orthogonal to the
most recently computed geodesic circle. A set of accuracy parameters governs
the number of next points by which the next geodesic circle is approximated, in
solving a set of boundary value problems. During the computation, the
interpolation error stays limited by the density of mesh points, so that the
overall quality of the mesh is preserved \parencite{krauskopf2003computing}.

\subsection{Identifying suitable initial conditions for developing LCSs}
\label{sub:identifying_suitable_initial_conditions_for_developing_lcss}

Inspired by the two-dimensional approach of \textcite{farazmand2012computing},
in order to identify repelling LCSs, the first step was to identify the
subdomain $\mathcal{U}_{0}\subset\mathcal{U}$ in which the existence conditions
\eqref{eq:lcs_condition_a},~\eqref{eq:lcs_condition_b} and
\eqref{eq:lcs_condition_d} are satisfied --- as these conditions can be
verified for individual points, unlike criterion~\eqref{eq:lcs_condition_c}.
All grid points which lie within $\mathcal{U}_{0}$ would then be valid initial
conditions for repelling LCSs. Of the aforementioned criteria,
condition~\eqref{eq:lcs_condition_d} is the least straightforward to implement
numerically, as identifying the zeros of inner products is prone to numerical
round-off error. Our approach is based on comparing the value of $\lambda_{3}$
at a given grid point $\vct{x}_{0}$ to the values of $\lambda_{3}$ at the two
points $\vct{x}_{0}\pm\varepsilon\vct{\xi}_{3}(\vct{x}_{0})$, where
$\varepsilon$ is a number one order of magnitude smaller than the grid spacing.
Should $\lambda_{3}(\vct{x}_{0})$ be the largest of the three, signifying that
$\vct{x}_{0}$ could be an approximate maximum of the $\lambda_{3}$ field
along the (local) $\vct{\xi}_{3}$ direction (which compliance with condition
\eqref{eq:lcs_condition_b} would confirm), the point $\vct{x}_{0}$ would be
flagged as satisfying criterion~\eqref{eq:lcs_condition_d}.

Using all of the points in $\mathcal{U}_{0}$ would invariably involve computing
a lot of LCSs several times over --- in particular, if two neighboring grid
points are both part of $\mathcal{U}_{0}$, then they likely belong to the same
manifold. In order to reduce the number of redundant calculations, the set of
considered initial conditions was further reduced, by only checking whether
every $\nu^{\text{th}}$ grid point along each axis belonged to
$\mathcal{U}_{0}$, that is, only considering one in every $\nu^{3}$ grid points
in the entire domain as possible initial conditions. Because the number of grid
points was different for the different types of flow (cf.\
\cref{tab:gridparams}), so too was the pseudo-sampling frequency $\nu$. The
values for $\varepsilon$, $\nu$ and the resulting number of initial conditions
are given in \cref{tab:initialconditionparams}. Note that, using the given
filtering parameters, the initial conditions reduced to a far more manageable
number of grid points, than all of the grid points which satisfy the LCS
conditions~\eqref{eq:lcs_condition_a},~\eqref{eq:lcs_condition_b} and~%
\eqref{eq:lcs_condition_d}.

\begin{table}[htpb]
    \centering
    \caption[Parameter choices for selecting LCS initial conditions]
    {Parameter choices for selecting LCS initial conditions. $\varepsilon$ was
    made to be one order of magnitude smaller than the grid spacing, and
    $\nu$ was selected to be a common divisor of the number of grid points
    along each Cartesian abscissa, cf.\ \cref{tab:gridparams}. Note that
    $\varepsilon$ for the fjord model data is given in units of metre. Observe
    how the reduced set of initial conditions contains orders of magnitude
    fewer points than the total number of grid points which satisfy the LCS
    existence criteria~\eqref{eq:lcs_condition_a},~\eqref{eq:lcs_condition_b}
    and~\eqref{eq:lcs_condition_d}.}
    \label{tab:initialconditionparams}
    \begin{tabular}{ccc}
        \toprule
        & Analytical ABC flow & Fjord model data\\
        \midrule
        $\varepsilon$ & $5\cdot10^{.-3}$ & $10^{-1}$\\
        $\nu$ & $8$ & $5$ \\[3pt]
        \makecell[c]{\# initial conditions\\without $\nu$-filtering} & \makecell[c]{\numprint{340951} (steady) \\ \numprint{361461} (unsteady)} & \numprint{209945}\\[9pt]
        \makecell[c]{\# initial conditions\\with $\nu$-filtering} & \makecell[c]{\numprint{618} (steady)\\\numprint{676} (unsteady)} & \numprint{1631}\\
        \bottomrule
    \end{tabular}
\end{table}


\subsection{Parametrizing the innermost level set}
\label{sub:parametrizing_the_innermost_level_set}

For an initial condition $\vct{x}_{0}$, identified by means of the method
outlined in
\cref{sub:identifying_suitable_initial_conditions_for_developing_lcss}, the
corresponding LCS must locally be tangent to the plane with unit normal given
by $\vct{\xi}_{3}(\vct{x}_{0})$, as a consequence of LCS existence criterion~%
\eqref{eq:lcs_condition_c}. Accordingly, the first geodesic level set is
approximated by a set of $n$ mesh points $\{\mathcal{M}_{1,j}\}_{j=1}^{n}$,
which all lie within the aforementioned tangent plane, evenly distributed along
the circle centered at $\vct{x}_{0}$ with radius $\delta_{\text{init}}$. All
of these points are assumed to lie within the same manifold.
\Cref{fig:innermost_levelset} shows where the points in the innermost level set
are located, in relation to $\vct{x}_{0}$. The parameter $\delta_{\text{init}}$
was chosen small compared to the grid spacing, in order to limit the inherent
errors of this linearization.

An interpolation curve $\mathcal{C}_{1}$ was then made, with a view to
representing the innermost level set in a smoother fashion. In particular,
this interpolation curve was designed as a parametric spline. To this end,
the points $\{\mathcal{M}_{1,j}\}_{j=1}^{n+1}$, where
$\mathcal{M}_{1,n+1}=\mathcal{M}_{1,1}$, were ordered in clockwise
or counterclockwise --- merely a matter of perspective --- fashion. Then,
each mesh point was assigned an independent variable $s$ based upon the
cumulative interpoint distance along the ordered list of points, starting
at $j=1$; estimated by means of the Euclidean norm, then normalized by dividing
through by the total interpoint distance around the entire initial level set.
Consequently, there was a one-to-one correspondence between the $s$-values and
the mesh points, aside from $\mathcal{M}_{1,1}$, to which both $s=0$ and $s=1$
were mapped.

Next, we made lists containing the coordinates of the ordered mesh points
$\{\mathcal{M}_{i,j}\}_{j=1}^{n+1}$; i.e.,\ one list for each of the three
Cartesian coordinates. Considering each of the lists of the mesh points'
Cartesian coordinates as univariate functions of the pseudo-arclength
parameter $s$, separate cubic B-splines were then made for each set of
coordinates, making use of the \texttt{bspline\_1d} extension type from the
Bspline-Fortran library \parencite{williams2018bspline}, which was ported to
Python as outlined in \cref{sub:interpolating_gridded_velocity_data}. The
constructed innermost level set and its associated interpolation curve is
illustrated in \cref{fig:innermost_levelset}. Interpolation curves for all
subsequent level sets (the computation of which will be described in detail
in the sections to follow) were made completely analogously to
$\mathcal{C}_{1}$.

\begin{figure}[htpb]
    \centering
    \resizebox{0.9\linewidth}{!}{\includestandalone{figures/tikz-figs/initial_levelset_and_interpolation}}
    \caption[The construction of the innermost geodesic level set]
    {The construction of the innermost geodesic level set. An initial
        condition $\vct{x}_{0}$ is found by means of the method outlined in
        \cref{sub:identifying_suitable_initial_conditions_for_developing_lcss}.
        A set of $n$ meshpoints $\{\mathcal{M}_{1,j}\}_{j=1}^{n}$ is then evenly
        distributed within the plane defined by the point $\vct{x}_{0}$ and the
        unit normal $\vct{\xi}_{3}(\vct{x}_{0})$, which is shaded. Each mesh
        point is separated from $\vct{x}_{0}$ by a small distance
        $\delta_{\text{init}}$ (dashed). Using a normalized pseudo-arclength
        parameter $s$, the mesh point coordinates are interpolated using cubic
        B-splines, forming the smooth curve $\mathcal{C}_{1}$ (dotted).}
    \label{fig:innermost_levelset}
\end{figure}


In the following, let $\vct{x}_{i,j}$ denote the location of mesh point
$\mathcal{M}_{i,j}$. As suggested by \textcite{krauskopf2005survey}, we next
sought to develop a new level set, parametrized by a new set of points
$\{\mathcal{M}_{2,j}\}$ located in the family of half-planes
$\{\mathcal{H}_{1,j}\}_{j=1}^{n}$, extending radially outwards from the
corresponding points $\{\mathcal{M}_{1,j}\}_{j=1}^{n}$ in the initial level set
while being orthogonal to $\mathcal{C}_{1}$. These half-planes are generally
defined by the points $\vct{x}_{i,j}$ and the (unit) tangent vectors
$\vct{t}_{i,j}$. For the innermost level set, these tangent vectors were
computed as
\begin{equation}
    \label{eq:innermost_tanvec}
    \vct{t}_{1,j} = %
    \frac{\vct{\xi}_{3}(\vct{x}_{0})\times(\vct{x}_{1,j}-\vct{x}_{0})}%
    {\norm{\vct{\xi}_{3}(\vct{x}_{0})\times(\vct{x}_{1,j}-\vct{x}_{0})}}.
\end{equation}
For the subsequent level sets, \textcite{krauskopf2005survey} suggest
determining the tangents $\vct{t}_{i,j}$ using the interpolation curve
$\mathcal{C}_{i}$, by drawing a vector between two points equidistant to
$\mathcal{M}_{i,j}$ in either direction along $\mathcal{C}_{i}$. However,
an inheritance-based approach was found to yield more smoothly parametrized
manifolds, which were less sensitive to numerical noise. This approach,
along with the treatment of special cases, will be explained in greater detail
in the sections to follow (in particular,
\cref{sub:maintaining_mesh_point_density}).

A \emph{guidance} vector $\vct{\rho}_{i,j}$ was computed for each of the
mesh points $\{\mathcal{M}_{i,j}\}$, in order to keep track of the local
(quasi-)radial direction. The guidance vectors for the innermost level set were
computed as
\begin{equation}
    \label{eq:innermost_prevvec}
    \vct{\rho}_{1,j} = %
    \frac{\vct{x}_{1,j}-\vct{x}_{0}}{\norm{\vct{x}_{1,j}-\vct{x}_{0}}}.
\end{equation}
For each mesh point in all ensuing level sets, the guidance vectors were
computed relative to the coordinates of the point in the immediately
preceding level set, from which the new point was computed. That is,
\begin{equation}
    \label{eq:general_prevvec}
    \vct{\rho}_{i,j} = %
    \frac{\vct{x}_{i,j}-\vct{x}_{i-1,\hat{\jmath}}}%
    {\norm{\vct{x}_{i,j}-\vct{x}_{i-1,\hat{\jmath}}}},
\end{equation}
where the indices $j$ and $\hat{\jmath}$ generally need not be the same,
as there is generally not a one-to-one correspondence between points
in subsequent level sets; see \cref{sub:maintaining_mesh_point_density}
for details.

Note that, in computing new mesh points, organized in level sets, a descendant
point $\mathcal{M}_{i+1,j}$ has to be computed for each ancestor point
$\mathcal{M}_{i,j}$. This is due to the method being based on parametrizing
manifolds as a series of smooth topological circles. Should this prove
not to be possible, given a set of tolerance parameters which will be
described in greater detail in the sections to come, the computation is stopped
abruptly, leaving the manifold parametrized by however many of its
geodesic level sets were successfully completed. Further details on the
stopping criteria for the generation of new geodesic level sets will be
presented in \cref{sec:macroscale_stopping_criteria_for_the_expansion_of%
_computed_manifolds}.
