This chapter contains a complete description of our method of computing
repelling LCSs. Details regarding the specific flow systems we used as test
cases will be presented in
\cref{sec:flow_systems_defined_by_analytical_velocity_fields,%
sec:flow_systems_defined_by_gridded_velocity_data}. Our way of extracting
the Cauchy-Green strain eigenvalues and -vectors (cf.\
\cref{eq:cauchygreen_characteristics}) from the aforementioned flow systems
then follows in
\cref{sec:computing_the_flow_map_and_its_directional_derivatives,%
sec:computing_cauchy_green_strain_eigenvalues_and_vectors}. As will be outlined
in \cref{sec:preliminaries_for_computing_repelling_lcss_in_3d_flow_by_means%
_of_geodesic_level_sets,%
sec:legacy_approach_to_computing_new_mesh_points,%
sec:revised_approach_to_computing_new_mesh_points,%
sec:managing_mesh_accuracy,%
sec:continuously_reconstructing_three_dimensional_manifold_surfaces_from%
_point_meshes,%
sec:macroscale_stopping_criteria_for_the_expansion_of_computed_manifolds,%
sec:identifying_lcss_as_subsets_of_computed_manifolds}, the Cauchy-Green
strain eigenvalues and -vectors were then used to compute manifolds as
three-dimensional surfaces --- parametrized in terms of points organized in
\emph{geodesic level sets} --- of which subsets were extracted as repelling
LCSs. Lastly,
\cref{sec:making_the_most_of_the_available_computational_resources} contains a
succint description of some optimization tweaks we introduced in order to limit
the consumption of computational resources.

Note that we present two variants of the method of geodesic level sets for
expanding a manifold by the addition of mesh points.
\Cref{sec:legacy_approach_to_computing_new_mesh_points} contains the framework
for a level set approach which closely resembles the method introduced by
\textcite{krauskopf2005survey} (see also \textcite{krauskopf2003computing}).
However, by making use of the properties defining repelling LCSs (see
\cref{eq:lcs_conditions}), we were able to simplify the method of generating
new mesh points considerably, resulting in a great speedup in terms of
computation runtimes. Our adaption of the method of geodesic level sets for the
specific case of repelling LCSs will hereafter be referred to as the
\emph{revised} approach to computing new mesh points, and is described in detail
in \cref{sec:revised_approach_to_computing_new_mesh_points}. Although it was
not used to generate any of our results (to follow in \cref{cha:results}), the
method of \cref{sec:legacy_approach_to_computing_new_mesh_points} --- from now
on denoted as the \emph{legacy} approach to computing new mesh points ---
is included here, as our \emph{revised} approach is not necessarily applicable
to analyses pertaining to different kinds of LCSs.


