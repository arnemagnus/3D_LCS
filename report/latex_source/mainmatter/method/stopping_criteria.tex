\section[Macroscale stopping criteria for the expansion of computed manifolds]
{Macroscale stopping criteria for the expansion of \\\phantom{3.10} computed
manifolds}
\label{sec:macroscale_stopping_criteria_for_the_expansion_of_computed%
_manifolds}

In principle, the process of developing manifold approximations by adding more
and more mesh points, organized in geodesic level sets, would continue as long
as the overall mesh quality was conserved (cf.\
\cref{sec:managing_mesh_accuracy}). The enforced (pseudo-)uniform expansion
radially outwards, inherent to our take on the method of geodesic level sets,
would then yield a mesh providing a conservative estimate of the size of the
\emph{actual} manifold, as there is no particular reason to expect the manifold
to appear homogeneous, when viewed from the epicentre of the computed level
sets (i.e.\ the initial position $\vct{x}_{0}$, cf.\
\cref{sub:identifying_suitable_initial_conditions_for_developing_lcss}). With
reasonable parameter choices, we became able to generate large manifolds quite
quickly. This lead us to enforce two additional types of stopping criteria,
based on what would happen if the computed manifolds reached the edges of the
computational domain, or folded into themselves --- where the latter of the two
will be described in \cref{sub:continuous_self_intersection_checks}.

The trajectories which are computed in order to identify new mesh points
(per the method described in
\cref{sec:revised_approach_to_computing_new_mesh_points}) frequently overstep
the domains within which the Cauchy-Green strain eigenvalues and -vectors
are defined in order to compute mesh points on or near the domain
boundaries. Thus, in order to resolve the behaviour of manifolds near the
edges of the domain of interest, the aforementioned strain characteristics
need to be computed in a region which \emph{contains} the domain of interest,
expanding beyond it in all directions. For perfectly periodic flow systems,
such as (either variant of) the ABC flow --- described in
\cref{sec:flow_systems_defined_by_analytical_velocity_fields} --- this reduces
to the trivial exercise of utilizing the inherent periodicity (that is,
provided that the computational domain is sufficiently large to encompass at
least one cycle along each direction).

\subsection{Continuous self-intersection checks}
\label{sub:continuous_self_intersection_checks}

Per LCS existence criterion \eqref{eq:lcs_condition_a}, there must be a
uniquely defined direction of strongest repulsion everywhere along a repelling
LCS. Furthermore, our method of expanding manifolds by adding mesh points
organized in geodesic level sets (cf.\
\cref{sec:revised_approach_to_computing_new_mesh_points}) is based on
continuous expansions (pseudo-)radially outwards from the manifold centre.
Accordingly, the intersection of any manifold with itself was interpreted as
a nonphysical artefact of accumulated numerical error. For that reason, we
sought to terminate the expansion of a manifold when self-intersections
were detected.

Our method of detecting manifold self-intersections is based on comparing the
continuously computed interpolation triangles (as described in
\cref{sec:continuously_reconstructing_three_dimensional_manifold_surfaces%
_from_point_meshes}) --- in particular, we compared each triangle which was
added with the most recently computed level set, to all of the
triangles which had been added with the preceding level sets. If at least one
pair of triangles intersected, the newest level set was flagged as
self-intersecting. Our way of determining whether or not two triangles
intersect, is based on the Möller-Trumbore ray-triangle intersection algorithm,
with a detection sensitivity parameter $\epsilon=10^{-8}$
\parencite{moller1997fast}. A visual representation of our
triangle-intersection detection algorithm is available in
\cref{fig:intersection_flowchart}.

\begin{figure}[htpb]
    \centering
    \resizebox{0.9\linewidth}{!}{\includestandalone{figures/tikz-figs/intersection_flowchart}}
    \caption[The construction of the innermost geodesic level set]
    {The construction of the innermost geodesic level set. An initial
        condition $\vct{x}_{0}$ is found by means of the method outlined in
        \cref{sub:identifying_suitable_initial_conditions_for_developing_lcss}.
        A set of $n$ meshpoints $\{\mathcal{M}_{1,j}\}_{j=1}^{n}$ is then evenly
        distributed within the plane defined by the point $\vct{x}_{0}$ and the
        unit normal $\vct{\xi}_{3}(\vct{x}_{0})$, which is shaded. Each mesh
        point is separated from $\vct{x}_{0}$ by a small distance $\delta$
        (dashed). Using a normalized pseudo-arclength parameter $s$, the mesh
        points are interpolated using cubic B-splines, forming the smooth curve
    $\mathcal{C}_{1}$ (dotted).}
    \label{fig:intersection_flowchart}
\end{figure}


Some cases of intersecting triangles warrant special treatment. In particular,
as our triangulation method (outlined in
\cref{sec:continuously_reconstructing_three_dimensional_manifold_surfaces%
_from_point_meshes}) is based on generating triangles which share sides with
its neighbors, we decided to allow triangles to intersect along the edges.
Similarly, we allowed for two triangles to be identical --- which might happen
if a manifold folds onto itself perfectly. Our algorithmic way of treating
these cases resulted in a theoretical false negative; namely, the case of two
triangles intersecting in exactly two points, laying along the edges of
\emph{both} triangles. In our experience, however, this never proved
problematic --- should one pair of triangles happen to intersect in this exact
fashion (which is a rarity in itself due to numerical round-off errors),
another pair of triangles would intersect in such a way that the most recently
added set as a whole would be flagged as self-intersecting. Two of the
aforementioned special cases --- that is, two triangles sharing an edge, and
two triangles intersecting in exactly two points laying along the edges of both
triangles --- are shown in \cref{fig:mollertrumbore_specialcases}.

\begin{figure}[htpb]
    \centering
    \begin{subfigure}[b]{0.45\textwidth}
        \centering
        \resizebox{0.9\textwidth}{!}{\includestandalone{figures/tikz-figs/mollertrumbore_vertexshared}}
        \caption[]{Two triangles sharing an edge, normally\\\phantom{(a) }
        treated as a continuous intersection}
        \label{fig:mollertrumbore_vertexshared}
    \end{subfigure}
    \begin{subfigure}[b]{0.45\textwidth}
        \centering
        \resizebox{0.9\textwidth}{!}{\includestandalone{figures/tikz-figs/mollertrumbore_dualisect}}
        \caption[]{Two triangles whose edges intersect in\\\phantom{(b) } exactly
            two unique points
        }
        \label{fig:mollertrumbore_dualisect}
    \end{subfigure}
    \caption[How the intersection-detection algorithm handles special cases]
    {How the intersection-detection algorithm handles special cases.
        Triangles which share a common edge, as illustrated in
        (\subref*{fig:mollertrumbore_vertexshared}), form the premise of our
        triangulation algorithm; accordingly, this scenario does not get flagged
        as an intersection. Neither does the case when two triangles are
        identical (or one is contained within the other) --- which is not shown
        here --- nor the case when the edges of two triangles intersect in
        exactly two unique points, shown in
        (\subref*{fig:mollertrumbore_dualisect}). The latter is a very marginal
        case which hardly ever occurs, and, for our purposes, would invariably
        coincide with another (nearby) pair of triangles intersecting in a
        different manner than these special cases; ensuring that the computed
        level set as a whole would be flagged as self-intersecting.}
    \label{fig:mollertrumbore_specialcases}
\end{figure}



Initial tests revealed that some self-intersections were quite innocuous, in
that, if the computed manifold was expanded by an additional level set,
the newly added triangulations need not necessarily intersect with any of the
preceding ones. This kind of insipid intersection could be a consequence of
the linear nature of our triangulation method, possibly compounded by round-off
error. Accordingly, we decided to stop the manifold expansion process
if \emph{several consecutive} geodesic level sets introduced intersection
triangulations. This was done by computing the sum of the interset distances
$\{\Delta_{i}\}$ for each consecutive level set $\{\mathcal{M}_{i}\}$ which
introduced new intersections. Whenever this pseudo-intersection length
exceeded a scalar multiple $\gamma_{\cap}$ of $\Delta_{\min}$ (cf.\
\cref{sub:maintaining_mesh_point_density}),
we terminated the manifold computation process; empirical trials suggested
that the intersection issue would then only worsen if more mesh points were
added.

