\subsection{Handling failures to compute satisfactory mesh points}
\label{sub:handling_failures_to_compute_satisfactory_mesh_points_revised}

Much like for the legacy approach outlined in
\cref{sec:legacy_approach_to_computing_new_mesh_points}, it can never be
guaranteed that all of the computed trajectories will yield new, acceptable
mesh points --- moreover, missing out on points in a level set prohibits
the generation of further level sets. In particular, our procedure for
maintaining mesh accuracy (which will be described in
\cref{sub:maintaining_mesh_point_density}) makes extensive use of the
interpolation curves $\{\mathcal{C}_{i}\}$. As the smoothness of the
interpolation curve $\mathcal{C}_{i+1}$ depends on \emph{all} of the points
used for its creation, proper handling of tricky mesh points remains critically
important.

Seeing as the only tolerance parameter involved in this method pertains to
achieving the desired separation between a meshpoint and its direct descendant
--- see \cref{eq:revised_dist_tol} --- we elected not to progressively relax
this constraint. Instead, we interpreted the failure of any trajectory to reach
a point sufficiently far away, as the trajectory being coiled such that the
required integration arc length to yield a point sufficiently far away from
the ancestor mesh point, exceeded the maximum allowed integration path
length (governed by $\gamma_{\text{arc}}$, cf.\
\cref{sub:computing_pseudoradial_trajectories_directly}). Accordingly, the
incomplete level set was discarded, and attempted to be recomputed with
$\Delta_{i}$ reduced to $\Delta_{\min}$ (our general dynamic adjustment
procedure for $\Delta_{i}$ will be described in detail in
\cref{sub:a_curvature_based_approach_to_determining_interset_separations}).
Should an entire, new geodesic level set prove incomputable even at the
minimum permitted step length, attempts at further expansion of the computed
manifold were abandoned. In such cases, this variant of the method of geodesic
level sets simply did not suffice, for the given set of development parameters.
Details on other stopping criteria for the generation of manifolds will be
presented in \cref{sec:macroscale_stopping_criteria_for_the_expansion_of%
_computed_manifolds}.

