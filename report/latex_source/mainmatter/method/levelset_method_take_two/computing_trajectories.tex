\subsection{Computing pseudoradial trajectories directly}
\label{sub:computing_pseudoradial_trajectories_directly}

Just like in the legacy approach outlined in
\cref{sec:legacy_approach_to_computing_new_mesh_points}, we define a local
direction field as
\begin{equation}
    \label{eq:revised_direction_field}
    \vct{\psi}(\vct{x}) = %
    \frac{\vct{\xi}_{3}(\vct{x})\times\vct{t}_{i,j}}%
    {\norm{\vct{\xi}_{3}(\vct{x})\times\vct{t}_{i,j}}},
\end{equation}
where $\vct{t}_{i,j}$ is the unit tangent associated with the mesh point
$\mathcal{M}_{i,j}$. Here, however, we make explicit use of our previously
mentioned additional degree of freedom; namely, that trajectories within
the parametrized manifold are allowed arbitrary movements within the
planes which are locally orthogonal to the $\vct{\xi}_{3}$ direction field,
rather than being constrained to moving along a three-dimensional curve. Thus,
we compute the coordinates of the new mesh point $\mathcal{M}_{i+1,j}$ as the
end point of a \emph{single} trajectory.

Any trajectory starting out within the half-plane $\mathcal{H}_{i,j}$ and
moving in the direction field given by \cref{eq:revised_direction_field} is
certain to remain within the half-plane, as the direction field is everywhere
orthogonal to its unit normal $\vct{t}_{i,j}$. That is, as long as the
direction field used in computing said trajectory is everywhere oriented
radially outwards. Accordingly, we computed a single trajectory in the
aforementioned direction field, starting out at
$\vct{x}_{\text{init}}=\vct{x}_{i,j}$, using the Dormand-Prince 8(7) adaptive
ODE solver (see \cref{tab:butcherdopri87,%
sub:the_implementation_of_dynamic_runge_kutta_step_size}), where all of
the vectors of the intermediary Runge-Kutta evaulations were corrected, if
necessary, by continuous comparison with $\vct{\rho}_{i,j}$ and
direction-reversion if an intermediary vector was pointing radially inwards.

Should the $\vct{\xi}_{3}$ direction be parallel to the unit tangent
$\vct{t}_{i,j}$ locally along the trajectory, the direction field
\cref{eq:revised_direction_field} would become undefined. In such regions,
we allowed the Runge-Kutta solver to step in the direction used for the
immediately preceding step. Numerically, such regions were recognized by
\begin{equation}
    \label{eq:revised_xi3_tan_parallel}
    \norm{\vct{\xi}_{3}(\vct{x})\times\vct{t}_{i,j}} < \gamma_{\|},
\end{equation}
where $\gamma_{\|}$ is a small tolerance parameter. Like for the legacy
approach (outlined in \cref{sec:legacy_approach_to_computing_new_mesh_points}),
the self-correcting integration step length meant that we did not treat the
integration step as a degree of freedom, and, in order to avoid overstepping,
the step length of the Dormand-Prince solver was continuously limited from
above by $\Delta_{i}-\norm{\vct{x}-\vct{x}_{i,j}}$. Moreover, the total allowed
integration arclength was limited to an integer multiple $\gamma_{\text{arc}}$
of the interset step $\Delta_{i}$, allowing for the termination of any
(hypothetical) trajectory which would end up in a stable orbit.

The trajectory integration was immediately interrupted upon reaching a point
$\vct{x}_{\text{fin}}$ separated from $\vct{x}_{i,j}$ by a distance
$\Delta_{i}$. Like for the legacy approach, this criterion was checked by means
of a tolerance parameter, seeing as directly comparing floating-point numbers
for equality is prone to numerical round-off error. In particular, if a point
$\vct{x}$ satisfied
\begin{equation}
    \label{eq:revised_dist_tol}
    \abs{\frac{\norm{\vct{x}-\vct{x}_{i,j}}}{\Delta_{i}}-1} < \gamma_{\Delta},
\end{equation}
with $\gamma_{\Delta}$ being a small number, the point was flagged as laying a
distance $\Delta_{i}$ from $\vct{x}_{i,j}$. Thus, upon reaching a point
satisfying \cref{eq:revised_dist_tol}, the trajectory was terminated, and the
new mesh point $\mathcal{M}_{i+1,j}$ was placed at the trajectory end point
$\vct{x}_{\text{fin}}$. \Cref{fig:revised_point_generation} depicts a
typical trajectory used to compute new mesh points in the fashion discussed
in the above.

\documentclass[crop]{standalone}
\usepackage{tikz}
\usepackage[]{tikz-3dplot}
\usepackage{pgfplots}
\pgfplotsset{compat=1.15}
\usepackage[]{amsmath}
\usepackage[]{libertine}
\usepackage[libertine]{newtxmath}
\usepackage[]{bm}
\usepackage[]{physics}
% Macros for greek letters in roman style, in math mode
\DeclareRobustCommand{\mathup}[1]{%
\begingroup\ensuremath\changegreek\mathrm{#1}\endgroup}
\DeclareRobustCommand{\mathbfup}[1]{%
\begingroup\ensuremath\changegreek\bm{\mathrm{#1}}\endgroup}


\makeatletter
\def\changegreek{\@for\next:={%
        alpha,beta,gamma,delta,epsilon,zeta,eta,theta,iota,kappa,lambda,mu,nu,%
        xi,pi,rho,sigma,tau,upsilon,phi,chi,psi,omega,varepsilon,varpi,%
    varrho,varsigma,varphi}%
\do{\expandafter\let\csname\next\expandafter\endcsname\csname\next up\endcsname}}
\makeatother

% Define vectors in bold, roman, lowercase font
\newcommand{\vct}[1]{\ensuremath{\mathbfup{\MakeLowercase{#1}}}}

% Define unit vectors in bold, roman, lowercase font, with hats
\newcommand{\uvct}[1]{\ensuremath{\mathbfup{\hat{\MakeLowercase{#1}}}}}

% Define matrices in bold, roman, uppercase font
\newcommand{\mtrx}[1]{\ensuremath{\mathbfup{\MakeUppercase{#1}}}}
\usetikzlibrary{%
    angles,%
    arrows.meta,%
    backgrounds,%
    calc,%
    decorations,%
    fit,%
    hobby,%
    patterns,%
    positioning,%
    quotes
}

\tdplotsetmaincoords{70}{120}

\begin{document}
\begin{tikzpicture}[tdplot_main_coords]
    \pgfmathsetmacro{\radius}{5}
    \pgfmathsetmacro{\size}{2.5}
%    % Place the set of initial points
    \foreach [count = \i] \ang in {205,240,260,280,165}%
    {%
        \coordinate (\i) at ( {\radius*cos(\ang-90)}, {\radius*sin(\ang-90)},{0} );
    }

    % Place coordinates giving the circle of positions at which one aims
    \coordinate (tp) at ({\radius*cos(150)},{\radius*sin(150)},1);
    \coordinate (bt) at ({\radius*cos(150)},{\radius*sin(150)},-1);


    % Define coordinates of half-plane $\mathcal{H}_{i,j}$
    \coordinate (lu) at ($(2)!2!(tp)$);
    \coordinate (ld) at ($(2)!2!(bt)$);
    \coordinate (ru) at ($(lu)+7*({-0.866},{0.5},0)$);
    \coordinate (rd) at ($(ld)+7*({-0.866},{0.5},0)$);
    % Shade half-plane
    \draw[fill=gray!30,draw opacity=0] (lu) -- (ld) -- (rd) -- (ru) -- cycle;
    % Draw interpolation curve
    \draw[stroke=black!80,thin,dotted] (70:\radius) arc (70:200:\radius);
    % Draw $\vct{rho}_{i,j}$ at $x_{i,j}$
    \path[draw,stroke=black!65,->,thick,dashdotted ] (2) to ($(2) + 5*({-0.866},{0.5},0)$) coordinate (rhoend);
    \node[below left = 7.5pt and 35pt of rhoend,rotate=12.5] {$\vct{\rho}_{i,j}$};
    \node[above left = 0pt and 0pt of rd] {$\mathcal{H}_{i,j}$};


    % Define coordinates of pivot for attaching $\mathcal{C}_{i}$ label
    \coordinate (mrk) at ({\radius*cos(195)},{\radius*sin(195)},0);
    % Attach label
    \coordinate [above = 10pt of mrk] (lbl);
    \node at (lbl) {$\mathcal{C}_{i}$};
    \draw[stroke=black!80] ($(mrk)!0.1!(lbl)$) to[in= 270,out=20] ($(mrk)!0.55!(lbl)$);

    % Draw points in initial level set
    \foreach \i in {1,2,3,4,5}%
    {%
        \draw[fill=gray!10,stroke=black!65] (\i) circle (\size pt);
    }
    % Draw semicircle of radius $\Delta_{i}$, centered at $x_{i,j}$
    \pic[densely dashed,draw,stroke=grey!80,angle radius=39] {angle = bt--2--tp};



    % Label ancestor point
    \coordinate[below left = 0pt and 30pt of 2] (plbl);
    \node[below left = 0pt and 0pt of 2]  {$\mathcal{M}_{i,j}$};


    % Define aim point
    \coordinate (aim) at ($(2) + 1.4*({-0.866},{0.5},0) - (0,0,0.9)$);

    % Draw aim point
    \draw[stroke=black!65,fill=gray!10] (aim) circle (\size pt);
    \node[right = 0pt of aim]  {$\vct{x}_{\text{fin}}$};

    % Draw some trajectories, the last of which manages to find aim point
    \path[draw,stroke=black!80,->] (2) to[out = -80, in = -180+30]  (aim);
    % Redraw start point
    \draw[fill=gray!10,stroke=black!65] (2) circle (\size pt);
    % Suggest initial radius
    \draw[decorate,decoration={brace,amplitude=5pt,raise=0pt},yshift=0pt] (2) -- ($(2)!1.46!(tp)$)   node [midway,left=2pt] {\footnotesize{$\Delta_{i}$}};

\end{tikzpicture}
\end{document}


