\subsection{Computing pseudoradial trajectories directly}
\label{sub:computing_pseudoradial_trajectories_directly}

Completely analogously to the legacy approach outlined in
\cref{sec:legacy_approach_to_computing_new_mesh_points}, we define a local
direction field as
\begin{equation}
    \label{eq:revised_direction_field}
    \vct{\psi}(\vct{x}) = %
    \frac{\vct{\xi}_{3}(\vct{x})\times\vct{t}_{i,j}}%
    {\norm{\vct{\xi}_{3}(\vct{x}_{0})\times\vct{t}_{i,j}}},
\end{equation}
where $\vct{t}_{i,j}$ is the unit tangent associated with the mesh point
$\mathcal{M}_{i,j}$. Here, however, we make explicit use of our previously
mentioned additional degree of freedom --- namely that trajectories within
the parametrized manifold are allowed arbitrary movements within the
planes which are locally orthogonal to the $\vct{\xi}_{3}$ direction field,
rather than being constrained to moving along a three-dimensional curve ---
by computing the coordinates of the new mesh point $\mathcal{M}_{i+1,j}$
as the end point of a \emph{single} trajectory.

Any trajectory starting out within the half-plane $\mathcal{H}_{i,j}$ and
moving in the direction field given by \cref{eq:revised_direction_field} is
certain to remain within it, as the direction field is everywhere orthogonal
to its unit normal $\vct{t}_{i,j}$. That is, as long as the direction field
used in computing said trajectory is everywhere oriented radially outwards.
Accordingly, we computed a single trajectory in the aforementioned
direction field, starting out at $\vct{x}_{\text{init}}=\vct{x}_{i,j}$, using
the Dormand-Prince 8(7) adaptive ODE solver (see \cref{tab:butcherdopri87,%
sub:the_implementation_of_dynamic_runge_kutta_step_size}), where all of
the vectors of the intermediary Runge-Kutta evaulations were corrected, if
necessary, by continuous comparison with $\vct{\rho}_{i,j}$ and
direction-reversion if an intermediary vector was pointing radially inwards.

Should the $\vct{\xi}_{3}$ direction be parallel to the unit tangent
$\vct{t}_{i,j}$ locally along the trajectory, the direction field
\cref{eq:revised_direction_field} would become undefined. In such regions,
we allowed the Runge-Kutta solver to take a step in the direction used for the
immediately preceding step. Numerically, such regions were recognized by
\begin{equation}
    \label{eq:revised_xi3_tan_parallel}
    \norm{\vct{\xi}_{3}(\vct{x})\times\vct{t}_{i,j}} < \gamma_{\|},
\end{equation}
where $\gamma_{\|}$ is a small tolerance parameter. Like for the legacy
approach (as outlined in
\cref{sec:legacy_approach_to_computing_new_mesh_points}, the self-correcting
integration step length meant that we did not treat the integration
step as a degree of freedom, and, in order to avoid overstepping, the step
length of the Dormand-Prince solver was continuously limited from above by
$\norm{\vct{x}-\vct{x}_{i,j}}-\Delta_{i}$. Moreover, the total allowed
integration arclength was limited to an integer multiple $\gamma_{\text{arc}}$
of the interset step $\Delta_{i}$ allowing for the termination of any
(hypothetical) trajectory which would end up in a stable orbit.

The trajectory integration was immediately interrupted upon reaching a point
$\vct{x}_{\text{fin}}$ separated from $\vct{x}_{i,j}$ by a distance
$\Delta_{i}$. Like for the legacy approach, this criterion was checked by means
of a tolerance parameter, seeing as directly comparing floating-point numbers
is prone to numerical round-off error. In particular, if a point $\vct{x}$
satisfied
\begin{equation}
    \label{eq:revised_dist_tol}
    \abs{\frac{\norm{\vct{x}-\vct{x}_{i,j}}}{\Delta_{i}}-1} < \gamma_{\Delta},
\end{equation}
with $\gamma_{\Delta}$ being a small number, the point was flagged as lying a
distance $\Delta_{i}$ from $\vct{x}_{i,j}$. Thus, upon reaching a point
satisfying \cref{eq:revised_dist_tol}, the trajectory was terminated, and the
new mesh point $\mathcal{M}_{i+1,j}$ was placed at the trajectory end point
$\vct{x}_{\text{fin}}$. \Cref{fig:revised_point_generation} depicts a
typical trajectory used to compute new mesh points in the fashion discussed
in the above.

\begin{figure}[htpb]
    \centering
    \resizebox{0.9\linewidth}{!}%
    {\includestandalone{figures/tikz-figs/revised_levelset_point_generation}}
    \caption[Visualization of a typical trajectory used to compute a new mesh
    point, using the revised approach]
    {Visualization of a typical trajectory used to compute a new mesh point,
        using the revised  approach. A trajectory is computed, starting
        at $\vct{x}_{i,j}$ (that is, the coordinates corresponding to the mesh
        point $\mathcal{M}_{i,j}$) and moving in the direction field given by
        \cref{eq:revised_direction_field}, in order to find a new mesh point
        at the intersection between the manifold $\mathcal{M}$ and the
        half-plane $\mathcal{H}_{i,j}$ (shaded), located a distance
        $\Delta_{i}$ from $\mathcal{M}_{i,j}$. $\mathcal{H}_{i,j}$ is defined
        by the point $\vct{x}_{i,j}$, its unit normal $\vct{t}_{i,j}$ (not
        shown) and the (quasi-)radial unit vector $\vct{\rho}_{i,j}$
        (dashdotted). The half-circle of radius $\Delta_{i}$,
        centered in $\vct{x}_{i,j}$ and lying within $\mathcal{H}_{i,j}$, on
        which we seek to find a new mesh point, is dashed. As soon as a
        trajectory reaches a point separated from $\mathcal{M}_{i,j}$ by
        a distance $\Delta_{i}$ (per \cref{eq:revised_dist_tol}), the
        integration is stopped, and a new mesh point $\mathcal{M}_{i+1,j}$
        (not shown) is placed at the trajectory end point
        $\vct{x}_{\text{fin}}$.
    }
    \label{fig:revised_point_generation}
\end{figure}

