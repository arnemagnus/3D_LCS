\subsection{Key improvements of the revised algorithm for computing new mesh
points}
\label{sub:key_improvements_of_the_revised_algorithm_for_computing_new_mesh%
_points}

When compared to the convoluted way of computing new mesh points presented in
\cref{sec:legacy_approach_to_computing_new_mesh_points}, it is readily apparent
that the revised approach is significantly less complex. In particular, note
how launching a single trajectory starting at the ancestor mesh point results
in the legacy algorithm for computing new trajectory start points iteratively
(cf. \cref{fig:s_update_flowchart}) being rendered entirely superfluous.
Moreover, as all computed trajectories necessarily remain within the target
half-planes, no tolerance parameter for detecting intersections between
trajectories and half-planes is needed. In short, the revised algorithm is
conceptually simpler, and involves a lesser number of free (tolerance)
parameters.

Simple numerical experiments confirmed that, given the same initial
conditions and underlying direction fields, the two approaches yielded the
same mesh points --- at least, within rounding error. Moreover, these tests
also revealed that the revised algorithm is usually two orders of
magnitude faster (in terms of computational runtime). Unsurprisingly, the
main time expenditure of the legacy approach turned out to be the computation
of elusive mesh points; often, several thousand computed trajectories were
needed before a satisfactory new point was found. In paticular, the probability
of finding an acceptable mesh point depends sensitively on choosing an
appropriate aim point --- leading to significant slowdowns in regions wherein
the underlying manifold behaves erratically.

For its superior speed and simplicity, the revised approach of forced
pseudoradial trajectories became our method of choice. Accordingly, all
of our results (which will be presented in \cref{cha:results}) were generated
with this method. Note that, while deemed inferior in the context of
identifying repelling LCSs in three-dimensional flow, the legacy approach of
guided trajectories remains a valid way of computing three-dimensional
manifolds. Thus, it can be used as a baseline for computing other kinds of
three-dimensional manifolds defined in a different manner than repelling LCSs
--- which remains beyond the scope of this project.
