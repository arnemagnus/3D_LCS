In order to identify LCSs in three-dimensional flow by means of geodesic level
set approximations, a system which has been studied extensively in the
literature was chosen. The system is a simple example of a fluid flow which can
exhibit chaotic behaviour \parencite[p.204]{frisch1995turbulence}.

\section{Arnold-Beltrami-Childress flow}
\label{sec:abc_flow}

The Arnold-Beltrami-Childress (hereafter abbreviated to ABC) flow is a
three-dimensional incompressible velocity field which solves the Euler equations
exactly. In terms of the Cartesian coordinate vector $\vct{x}=(x,y,z)$, the
system can be expressed mathematically as
\begin{equation}
    \label{eq:abc_flow}
    \dot{\vct{x}} = \vct{v}(t,\vct{x}) =%
    \begin{pmatrix}
        A\sin(z)+C\cos(y)\\
        B\sin(x)+A\cos(z)\\
        C\sin(y)+B\cos(x)
    \end{pmatrix},
\end{equation}
where $A$, $B$ and $C$ are spatially invariant parameters which dictate the
nature of the flow pattern. The inherent periodicity with regards to the
Cartesian coordinates naturally leads to a domain of interest
$\mathcal{U} = [0,\hspace{0.5ex}2\pi]^{3}$ with periodic boundary conditions
imposed along all three Cartesian axes.

For stationary flow, the parameter values
\begin{equation}
    \label{eq:abc_params_stationary}
    A = \sqrt{3}, \quad B=\sqrt{2}, \quad C=1
\end{equation}
were used, as has been common in the literature, for instance in the article
by \textcite{oettinger2016autonomous}, as these values are known for exhibiting
chaotic trajectories \parencite{zhao1993chaotic}.

\begin{framed}
    Rundt her kan sikkert språket forbedres / brødtekst genereres.
    Lar det være et problem for fremtidige A.
\end{framed}

In order to enforce chaotic behaviour, the parameters $A$, $B$ and $C$ can be
modified to be temporally aperiodic as follows:
\begin{nalign}
    \label{eq:abc_params_nonstationary}
        &A = \sqrt{3}, \\
        &B = \sqrt{2}\big[1 + k_{0}\tanh(k_{1}t)\cos({(k_{2}t)}^{2})\big], \\
        &C = 1 + k_{0}\tanh(k_{1}t)\sin({(k_{3}t)}^{2}),
\end{nalign}
where the parameter values
\begin{equation}
    \label{eq:abc_params_nonstationary_frequencies}
    k_{0} = 0.3, \quad k_{1} = 0.5, \quad k_{2} = 1.5, \quad k_{3} = 1.8
\end{equation}
were used, like in \textcite{oettinger2016autonomous}.

\subsection{(Normal) vector field for a sinusoidal surface}
\label{sub:_normal_vector_field_for_a_sinusoidal_surface}

For the surface implicitly defined as the zeros of the function

\begin{equation}
    f(\vct{r}) = A\sin(\omega_{x}x)\sin(\omega_{y}y) + (z_{0}-z),
\end{equation}

one possible choice for its normal vector field is

\begin{equation}
    \vct{n}(\vct{r}) = %
    \begin{pmatrix}
        A\omega_{x}\cos(\omega_{x}x)\sin(\omega_{y}y)\\
        A\omega_{y}\sin(\omega_{x}x)\cos(\omega_{y}y)\\
        -1
    \end{pmatrix}.
\end{equation}
