\begin{table}[htpb]
    \centering
    \caption[Parameter choices for selecting LCS initial conditions]
    {Parameter choices for selecting LCS initial conditions. $\varepsilon$ was
    made to be one order of magnitude smaller than the grid spacing, and
    $\nu$ was selected to be a common divisor of the number of grid points
    along each Cartesian abscissa, cf.\ \cref{tab:gridparams}. Note that
    $\varepsilon$ for the fjord model data is given in units of metre. Observe
    how the reduced set of initial conditions contains orders of magnitude
    fewer points than the total number of grid points which satisfy the LCS
    existence criteria~\eqref{eq:lcs_condition_a},~\eqref{eq:lcs_condition_b}
    and~\eqref{eq:lcs_condition_d}.}
    \label{tab:initialconditionparams}
    \begin{tabular}{ccc}
        \toprule
        & Analytical ABC flow & Fjord model data\\
        \midrule
        $\varepsilon$ & $5\cdot10^{.-3}$ & $10^{-1}$\\
        $\nu$ & $8$ & $5$ \\[3pt]
        \makecell[c]{\# initial conditions\\without $\nu$-filtering} & \makecell[c]{$340951$ (steady) \\ $361461$ (unsteady)} & $209945$\\[9pt]
        \makecell[c]{\# initial conditions\\with $\nu$-filtering} & \makecell[c]{$618$ (steady)\\$676$ (unsteady)} & $1631$\\
        \bottomrule
    \end{tabular}
\end{table}
