\begin{figure}[htpb]
    \centering
    \resizebox{0.9\linewidth}{!}%
    {\includestandalone{figures/tikz-figs/angular_adjustment}}
    \caption[The principles of curvature-guided interset step length adjustment]
    {The principles of curvature-guided interset step length adjustment. For
        each of the mesh points $\{\mathcal{M}_{i+1,j}\}$ constituting the
        level set $\mathcal{M}_{i+1}$ which were computed from the mesh points
        constituting the most recently completed level set $\mathcal{M}_{i}$
        (i.e.,\ all but the mesh points which were computed from ficticious
        ancestors, cf.\ \cref{sub:maintaining_mesh_point_density}), the angular
        offsets $\alpha_{i,j}$ between the guidance vectors $\vct{\rho}_{i,j}$
        (dashdotted) and $\vct{\rho}_{i+1,j}$ (not shown, but parallel to the
        vector separating $\mathcal{M}_{i,j}$ and $\mathcal{M}_{i+1,j}$, shown
        in solid) were computed. These were used in conjunction with
        $\Delta_{i}$, the interset step used to  compute the new mesh points,
        in order to determine whether or not the suggested level set would have
        to be discarded due to the local curvature along at least one point
        strand being sufficiently large to comply with criterion~%
        \eqref{eq:decrease_dist}. If the level set was deemed acceptable, local
        curvature estimates $\{\alpha_{i,j}\}$ and
        $\{\Delta_{i}\cdot\alpha_{i,j}\}$ were then used to determine if the
        \emph{subsequent} level set $\mathcal{M}_{i+2}$ could be computed
        using $\Delta_{i+1}>\Delta_{i}$; namely, if criterion
        \eqref{eq:increase_dist} was satisfied.
    }
    \label{fig:angular_adjustment}
\end{figure}
