\begin{figure}[htpb]
    \centering
    \begin{subfigure}[b]{0.45\textwidth}
        \centering
        \resizebox{0.9\textwidth}{!}{\includestandalone{figures/tikz-figs/mollertrumbore_vertexshared}}
        \caption[]{Two triangles with a common edge,\\\phantom{(a) } yielding
        an `infinite' no.\ of intersections}
        \label{fig:mollertrumbore_vertexshared}
    \end{subfigure}
    \begin{subfigure}[b]{0.45\textwidth}
        \centering
        \resizebox{0.9\textwidth}{!}{\includestandalone{figures/tikz-figs/mollertrumbore_dualisect}}
        \caption[]{Two triangles whose edges intersect in\\\phantom{(b) } exactly
            two unique points
        }
        \label{fig:mollertrumbore_dualisect}
    \end{subfigure}
    \caption[How the intersection-detection algorithm handles special cases]
    {How the intersection-detection algorithm handles special cases.
        Triangles which share a common edge, as illustrated in
        (\subref*{fig:mollertrumbore_vertexshared}), form the premise of our
        triangulation algorithm; accordingly, this scenario does not get flagged
        as an intersection. Neither does the case when two triangles are
        identical --- which is not shown here --- nor the case when the edges of
        two triangles intersect in exactly two unique points, shown in
        (\subref*{fig:mollertrumbore_dualisect}). The latter is a very marginal
        case which hardly ever occurs, and, for our purposes, would invariably
        coincide with another (nearby) pair of triangles intersecting in a
        different manner than these special cases; ensuring that the computed
        level set as a whole would be flagged as self-intersecting.}
    \label{fig:mollertrumbore_specialcases}
\end{figure}

