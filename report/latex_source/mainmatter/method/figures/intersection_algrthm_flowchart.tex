\begin{figure}[htpb]
    \centering
    \resizebox{0.9\linewidth}{!}{\includestandalone{figures/tikz-figs/intersection_flowchart}}
    \caption[The construction of the innermost geodesic level set]
    {The construction of the innermost geodesic level set. An initial
        condition $\vct{x}_{0}$ is found by means of the method outlined in
        \cref{sub:identifying_suitable_initial_conditions_for_developing_lcss}.
        A set of $n$ meshpoints $\{\mathcal{M}_{1,j}\}_{j=1}^{n}$ is then evenly
        distributed within the plane defined by the point $\vct{x}_{0}$ and the
        unit normal $\vct{\xi}_{3}(\vct{x}_{0})$, which is shaded. Each mesh
        point is separated from $\vct{x}_{0}$ by a small distance $\delta$
        (dashed). Using a normalized pseudo-arclength parameter $s$, the mesh
        points are interpolated using cubic B-splines, forming the smooth curve
    $\mathcal{C}_{1}$ (dotted).}
    \label{fig:intersection_flowchart}
\end{figure}
