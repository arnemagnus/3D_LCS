\begin{figure}[htpb]
    \centering
    \resizebox{0.9\linewidth}{!}%
    {\includestandalone{figures/tikz-figs/revised_levelset_point_generation}}
    \caption[Visualization of a typical trajectory used to compute a new mesh
    point, using the revised approach]
    {Visualization of a typical trajectory used to compute a new mesh point,
        using the revised  approach. A trajectory is computed, starting
        at $\vct{x}_{i,j}$ (that is, the coordinates corresponding to the mesh
        point $\mathcal{M}_{i,j}$) and moving in the direction field given by
        \cref{eq:revised_direction_field}, in order to find a new mesh point
        at the intersection between the manifold $\mathcal{M}$ and the
        half-plane $\mathcal{H}_{i,j}$ (shaded), located a distance
        $\Delta_{i}$ from $\mathcal{M}_{i,j}$. $\mathcal{H}_{i,j}$ is defined
        by the point $\vct{x}_{i,j}$, its unit normal $\vct{t}_{i,j}$ (not
        shown) and the (quasi-)radial unit vector $\vct{\rho}_{i,j}$
        (dashdotted). The half-circle of radius $\Delta_{i}$,
        centered in $\vct{x}_{i,j}$ and lying within $\mathcal{H}_{i,j}$, on
        which we seek to find a new mesh point, is dashed. As soon as a
        trajectory reaches a point separated from $\mathcal{M}_{i,j}$ by
        a distance $\Delta_{i}$ (per \cref{eq:revised_dist_tol}), the
        integration is stopped, and a new mesh point $\mathcal{M}_{i+1,j}$
        (not shown) is placed at the trajectory end point
        $\vct{x}_{\text{fin}}$.
    }
    \label{fig:revised_point_generation}
\end{figure}
