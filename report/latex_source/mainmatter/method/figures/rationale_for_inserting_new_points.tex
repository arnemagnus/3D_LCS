\begin{figure}[htpb]
    \centering
    \resizebox{0.9\linewidth}{!}%
    {\includestandalone{figures/tikz-figs/concept_of_inserting_new_points}}
    \caption[Conceptual illustration of the rationale behind the insertion of
    new mesh points as the geodesic level sets expand]
    {Conceptual illustration of the rationale behind the insertion of new mesh
        points as the geodesic level sets expand. The shown subset of the
        innermost topological circle is parametrized by three mesh points,
        shown as dark gray circles. As the topological circle is expanded, an
        increasing amount of mesh points must be inserted in order to maintain
        the (approximate) mesh point density. In the intermediate-sized circle
        shown in the figure, the mesh points with no direct analogue in the
        smallest circle are shown in a lighter shade of gray. Similarly, in the
        largest circle shown, the mesh points with no direct analogue in the
        intermediate-sized circle are drawn without fill.
    }
    \label{fig:rationale_for_inserting_new_points}
\end{figure}

