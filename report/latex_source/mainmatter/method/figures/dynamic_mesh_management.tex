\begin{figure}[htpb]
    \centering
    \begin{subfigure}[b]{0.475\textwidth}
        \centering
        \resizebox{0.9\linewidth}{!}{\includestandalone{figures/tikz-figs/density_management_inserting_new_point}}
        \caption[]{{\small Inserting a new mesh point inbetween a \\\phantom{(a)} pair of mesh points which are too far apart}}
        \label{fig:mesh_management_pure_insertion}
    \end{subfigure}
    \begin{subfigure}[b]{0.475\textwidth}
        \centering
        \resizebox{0.9\linewidth}{!}{\includestandalone{figures/tikz-figs/density_management_removing_point}}
        \caption[]{{\small Removing a mesh point too close to another,\\\phantom{(b)} if the ensuing separations are acceptable}}
        \label{fig:mesh_management_pure_deletion}
    \end{subfigure}
    \caption[Our approach to inserting new, or removing, mesh points to maintain
    mesh point density]
    {Our approach to inserting new, or removing, mesh points to maintain
        mesh point density. When two neighboring mesh points in a freshly
        computed level set are too far apart with regards to the given mesh
        parameter $\Delta_{\max}$, we attempt to insert a new mesh point
        between them. As shown in
        (\subref*{fig:mesh_management_pure_insertion}), this is done by using
        the method described in
        \cref{sec:revised_approach_to_computing_new_mesh_points}, using a
        ficticious initial condition midway inbetween the two ancestor mesh
        points $\mathcal{M}_{i,j}$ and $\mathcal{M}_{i,j+1}$ along the
        interpolated curve $\mathcal{C}_{i}$, indicated in the figure by
        a lighter shade of gray. Should two neighboring mesh points
        be too close together, and one of the two can be removed without the
        resulting sets of neighboring points being too far apart, we remove
        the one which results in the shortest interpoint separation (as shown
        in (\subref*{fig:mesh_management_pure_deletion}), where the point
        which is removed is indicated with a lighter shade of gray).
    }
    \label{fig:mesh_management_insertion_and_deletion}
\end{figure}
