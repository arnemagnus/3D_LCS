\begin{figure}[htpb]
    \centering
    \begin{subfigure}[b]{0.475\textwidth}
        \centering
        \resizebox{0.9\linewidth}{!}{\includestandalone{figures/tikz-figs/triangulation_pointinserted}}
        \caption[]{{\small Including mesh points with ficticious \\\phantom{(a)} ancestors in the triangulation}}
        \label{fig:triangulation_pointinserted}
    \end{subfigure}
    \begin{subfigure}[b]{0.475\textwidth}
        \centering
        \resizebox{0.9\linewidth}{!}{\includestandalone{figures/tikz-figs/triangulation_pointremoved}}
        \caption[]{{\small Triangulating with the resulting nearest \\\phantom{(b)} neighbors, if a mesh point is removed}}
        \label{fig:triangulation_pointremoved}
    \end{subfigure}
    \caption[How our triangulation algorithm handled the special cases arising
    when the one-to-one correspondence between the points in subsequent level
    sets was broken]
    {How our triangulation algorithm handled the special cases arising when
        the one-to-one correspondence between the points in subsequent level
        sets was broken. In (\subref*{fig:triangulation_pointinserted}),
        a mesh point is inserted inbetween the points $\mathcal{M}_{i+1,j}$
        and $\mathcal{M}_{i+1,j+1}$, while, in (\subref*{fig:triangulation_pointremoved}),
        the mesh point $\mathcal{M}_{i+1,j}$ is removed. Both of these
        adjustments were made in order to maintain the density of mesh points
        (cf.\ \cref{sub:maintaining_mesh_point_density}). The ensuing pair of
        triangles which are computed when considering the triangulations from
        $\mathcal{M}_{i,j}$, are shaded. The remaining triangles (patterned)
        arise when triangulating from different points in level set
        $\mathcal{M}_{i}$. Note that the removal of bundles of mesh points
        in order to dampen the effect of numerical noise (which introduces
        bulges, as described in
        \cref{sub:limiting_the_accumulation_of_numerical_noise}) is treated
        analogously to the case of single missing points.
    }
    \label{fig:triangulation_specialcases}
\end{figure}
