\begin{figure}[htpb]
    \centering
    \begin{subfigure}[b]{0.55\textwidth}
        \centering
        \resizebox{0.9\linewidth}{!}%
        {\includestandalone{figures/tikz-figs/triangulation_basecase}}
        \caption[]{{\small The base case for our triangulation scheme}}
        \label{fig:triangulation_basecase}
    \end{subfigure}

    \begin{subfigure}[b]{0.475\textwidth}
        \centering
        \resizebox{0.9\linewidth}{!}%
        {\includestandalone{figures/tikz-figs/triangulation_pointinserted}}
        \caption[]{{\small Including mesh points with ficticious
        \\\phantom{(b)} ancestors in the triangulation}}
        \label{fig:triangulation_pointinserted}
    \end{subfigure}
    \begin{subfigure}[b]{0.475\textwidth}
        \centering
        \resizebox{0.9\linewidth}{!}%
        {\includestandalone{figures/tikz-figs/triangulation_pointremoved}}
        \caption[]{{\small Triangulating with the resulting nearest
        \\\phantom{(c)} neighbors, if a mesh point is removed}}
        \label{fig:triangulation_pointremoved}
    \end{subfigure}
    \caption[Conceptual illustrations of our triangulation algorithm]
    {Conceptual illustrations of our triangulation algorithm. The standard
        approach, which applies when there is a one-to-one correspondence
        between the nearest neighbors of the point $\mathcal{M}_{i,j}$ in the
        level set $\mathcal{M}_{i}$, and the and $\mathcal{M}_{i,j}$'s nearest
        neighbors in the level set $\mathcal{M}_{i+1}$, is shown in
        (\subref*{fig:triangulation_basecase}). Meanwhile,
        (\subref*{fig:triangulation_pointinserted}) and
        (\subref*{fig:triangulation_pointremoved}) illustrate our handling of
        the special cases which arise when the one-to-one correspondence
        between points in subsequent level sets is broken. In particular,
        (\subref*{fig:triangulation_pointinserted}) shows how we include
        a mesh point inserted inbetween the points $\mathcal{M}_{i+1,j}$
        and $\mathcal{M}_{i+1,j+1}$ in the triangulation, while
        (\subref*{fig:triangulation_pointremoved}) demonstrates how the removal
        of mesh point $\mathcal{M}_{i+1,j}$ affects the triangulation.
        Both of these kinds of adjustments were made in order to maintain the
        density of mesh points (cf.\
        \cref{sub:maintaining_mesh_point_density}). In all of the
        illustrations, the pair of triangles which arise when triangulating
        outwards from $\mathcal{M}_{i,j}$, are shaded. The remaining triangles
        (patterned) originate from the triangulation of other points in level
        set $\mathcal{M}_{i}$. Note that the removal of bundles of mesh points
        in order to dampen the effect of numerical noise (which introduces
        bulges, as described in
        \cref{sub:limiting_the_accumulation_of_numerical_noise}) is treated
        analogously to the case of single missing points.
    }
    \label{fig:triangulation_specialcases}
\end{figure}
