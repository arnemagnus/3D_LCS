\begin{figure}[htpb]
    \centering
    \resizebox{0.9\linewidth}{!}%
    {\includestandalone{figures/tikz-figs/faithful_levelset_point_generation}}
    \caption[Visualization of typical trajectories used to compute a new mesh
    point, using the legacy approach]
    {Visualization of typical trajectories used to compute a new mesh point,
        using the legacy approach. The aim point $\vct{x}_{\text{aim}}$ is used
        to guide trajectories which are locally orthogonal to the
        $\vct{\xi}_{3}$ direction field (cf.\
        \cref{eq:faithful_local_directionfield}) towards the intersection
        between the manifold $\mathcal{M}$ and the half-plane
        $\mathcal{H}_{i,j}$ (shaded), in order to find a new mesh point
        $\mathcal{M}_{i+1,j}$ located a distance $\Delta_{i}$ from
        $\mathcal{M}_{i,j}$. $\mathcal{H}_{i,j}$ is defined by the point
        $\vct{x}_{i,j}$ (that is, the coordinates corresponding to the mesh
        point $\mathcal{M}_{i,j}$), its unit normal $\vct{t}_{i,j}$ (not
        shown), and the (quasi-)radial unit vector $\vct{\rho}_{i,j}$
        (dashdotted). The half-circle of radius $\Delta_{i}$, centered in
        $\vct{x}_{i,j}$ and laying within $\mathcal{H}_{i,j}$, on which we seek
        to find a new mesh point, is dashed. All permitted trajectory initial
        positions lie along the smooth parametrized curve $\mathcal{C}_{i}$
        (dotted), and are given by
        \cref{eq:faithful_initialcondition_interval}. A select few trajectories
        are shown, where the arrowheads indicate the first intersection with
        $\mathcal{H}_{i,j}$ (as per \cref{eq:plane_tolerance}), at which point
        the integration was terminated.
    }
    \label{fig:faithful_point_generation}
\end{figure}
