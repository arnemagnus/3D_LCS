\cleartorecto
\begin{framed}
    \begin{itemize}
        \item ABC-strømning (begge former), fjordstrømning (vist v/ kartprojeksjon(er)),
            oppsett av grids og beregning av $\lambda$, $\xi$ v/ SVD
        \item Interpolasjon av skalare- og vektorstørrelser
        \item Fullstendig beskrivelse av opprinnelig metode (lett modifisert GLS; tanvec, prevvec, levelsets etc.),
            muligens med flowchart
        \item Vis/beskriv eksplisitt noen av problemene/utfordringene med opprinnelig metode
            \begin{itemize}
                \item Veldig mange frihetsgrader
                \item (Tidvis) tilfeldighetspreget oppførsel i regioner med rasktvarierende $\xi_{3}$,
                    hvor godt vi traff avh. av hvilket vinkeloffset vi traff med først (opp eller ned)
                \item Langsom metode; Trenger gjerne mange forsøk om det først går galt
            \end{itemize}
        \item Utfyllende meskrivelse av ny metode, med hovedfokus på forskjellene fra opprinnelig versjon.
            Flowchart?
            \begin{itemize}
                \item Kan tillate os dette fordi vi har en ekstra frihetsgrad sml med system fra levelset-paper
                \item Utvikling langs <<strains>> er en tredimensjonal videreføring av strainlines (siter Løken/Nordgreen 2017; F\&H 2012)
                \item Mer konsekvent oppførsel nær sterke diskontinuiteter, god konvergens
            \end{itemize}
        \item Nytt av versjon 2 er kontinuerlige self-intersection-checks
            \begin{itemize}
                \item Beskriv bruk av Möller-Trumbore osv.
            \end{itemize}
        \item Seleksjon av lokalt sterkest repulsive materialoverflater v/ lokal sjekk
            \begin{itemize}
                \item Sjekk hvert punkt på mangfoldigheten ift ABD
                \item Behold også punkt som ikke er i ABD, så fremt de er nærme nok minst et annet punkt i ABD
                \item Filtrer vekk LCS-kandidater som er for små
            \end{itemize}
    \end{itemize}
\end{framed}
