\section{The considered flow systems}
\label{sec:the_considered_flow_systems}

\subsection{Flow systems defined by analytical velocity fields}
\label{sub:flow_systems_defined_by_analytical_velocity_fields}

\subsubsection{Steady Arnold-Beltrami-Childress flow}
\label{ssub:steady_arnold_beltrami_childress_flow}

The Arnold-Beltrami-Childress (henceforth abbreviated to ABC) flow is a
three-dimensional, incompressible velocity field which solves the Euler
equations exactly. It is a simple example of a fluid flow which can exhibit
chaotic behaviour \parencite[p.204]{frisch1995turbulence}. In terms of the
Cartesian coordinate vector $\vct{x}=(x,y,z)$, the system can be expressed
mathematically as
\begin{equation}
    \label{eq:abc_flow}
    \dot{\vct{x}} = \vct{v}(t,\vct{x}) = %
    \begin{pmatrix}
        A\sin(z) + C\cos(y)\\
        B\sin(x) + A\cos(z)\\
        C\sin(y) + B\cos(x)
    \end{pmatrix},
\end{equation}
where $A$, $B$ and $C$ are parameters which dictate the nature of the flow
pattern. The inherent periodicity with regards to the Cartesian axes naturally
leads to a domain of interest $\mathcal{U} = [0,\hspace{0.5ex}2\pi]^{3}$, with
periodic boundary conditions imposed in $x$, $y$ and $z$.

Here, the parameter values
\begin{equation}
    \label{eq:abc_params_stationary}
    A = \sqrt{3},\quad B = \sqrt{2},\quad C = 1
\end{equation}
were used, as has been common in litterature (e.g.\ by
\textcite{oettinger2016autonomous}), as these values are known to result in
chaotic tracer trajectories \parencite{zhao1993chaotic}.

\subsubsection{Unsteady Arnold-Beltrami-Childress flow}%
\label{ssub:unsteady_arnold_beltrami_childress_flow}

Inspired by~\textcite{oettinger2016autonomous}, a temporally aperiodic
modification of the ABC flow (\cref{eq:abc_flow}) was made by the replacements
\begin{equation}
    \label{eq:abc_params_nonstationary}
    \begin{gathered}
    B\to{}\widetilde{B}(t) = B + B\cdot{}k_{0}\tanh(k_{1}t)\cos({({k_{2}t})}^{2}),\\
    C\to{}\widetilde{C}(t) = C + C\cdot{}k_{0}\tanh(k_{1}t)\sin({({k_{3}t})}^{2}),
    \end{gathered}
\end{equation}
with $A$, $B$ and $C$ given by \cref{eq:abc_params_stationary}, where the
parameters values
\begin{equation}
    \label{eq:abc_params_nonstationary_frequencies}
    k_{0}=0.3,\quad k_{1}=0.5,\quad k_{2}=1.5,\quad k_{3}=1.8,
\end{equation}
were used. The time dependence of the $\widetilde{B}$ and $\widetilde{C}$
coefficients is illustrated in \cref{fig:abc_timedep_coeff}.

\begin{figure}[htpb]
    \centering
    \resizebox{0.9\textwidth}{!}{
        \importpgf{figures/mpl-figs}{dep-bc-coeff.pgf}
    }
    \caption[Aviici is love, Aviici is life]{Time evolution of nonstationary ABC coefficients}
    \label{fig:u0_dom_errs}
\end{figure}



\subsection{Flow systems defined by gridded velocity data}
\label{sub:flow_systems_defined_by_gridded_velocity_data}

\subsubsection{Oceanic currents in the Førde Fjord}
\label{ssub:oceanic_currents_in_the_forde_fjord}

\begin{framed}
    Motiver med potensielt sjødeponi for gruveavfall
\end{framed}




In order to identify LCSs in three-dimensional flow by means of geodesic level
set approximations, a system which has been studied extensively in the
literature was chosen. The system is a simple example of a fluid flow which can
exhibit chaotic behaviour \parencite[p.204]{frisch1995turbulence}.


\subsection{(Normal) vector field for a sinusoidal surface}
\label{sub:_normal_vector_field_for_a_sinusoidal_surface}

For the surface implicitly defined as the zeros of the function

\begin{equation}
    f(\vct{r}) = A\sin(\omega_{x}x)\sin(\omega_{y}y) + (z_{0}-z),
\end{equation}

one possible choice for its normal vector field is

\begin{equation}
    \vct{n}(\vct{r}) = %
    \begin{pmatrix}
        A\omega_{x}\cos(\omega_{x}x)\sin(\omega_{y}y)\\
        A\omega_{y}\sin(\omega_{x}x)\cos(\omega_{y}y)\\
        -1
    \end{pmatrix}.
\end{equation}
