%\section{The considered flow systems}
%\label{sec:the_considered_flow_systems}

\section{Flow systems defined by analytical velocity fields}
\label{sec:flow_systems_defined_by_analytical_velocity_fields}

\subsection{Steady Arnold-Beltrami-Childress flow}
\label{sub:steady_arnold_beltrami_childress_flow}

The Arnold-Beltrami-Childress (henceforth abbreviated to ABC) flow is a
three-dimensional, incompressible velocity field which solves the Euler
equations exactly. It is a simple example of a fluid flow which can exhibit
chaotic behaviour \parencite[p.204]{frisch1995turbulence}. In terms of the
Cartesian coordinate vector $\vct{x}=(x,y,z)$, the system can be expressed
mathematically as
\begin{equation}
    \label{eq:abc_flow}
    \dot{\vct{x}} = \vct{v}(t,\vct{x}) = %
    \begin{pmatrix}
        A\sin(z) + C\cos(y)\\
        B\sin(x) + A\cos(z)\\
        C\sin(y) + B\cos(x)
    \end{pmatrix},
\end{equation}
where $A$, $B$ and $C$ are parameters which dictate the nature of the flow
pattern. The inherent periodicity with regards to the Cartesian axes naturally
leads to a domain of interest $\mathcal{U} = [0,2\pi]^{3}$, with
periodic boundary conditions imposed in $x$, $y$ and $z$.

Here, the parameter values
\begin{equation}
    \label{eq:abc_params_stationary}
    A = \sqrt{3},\quad B = \sqrt{2},\quad C = 1
\end{equation}
were used, as has been common in litterature (e.g.\ by
\textcite{oettinger2016autonomous}), as these values are known to result in
chaotic tracer trajectories \parencite{zhao1993chaotic}. The time domain of
interest for this system was $\mathcal{I}=[0,5]$.

\subsection{Unsteady Arnold-Beltrami-Childress flow}%
\label{sub:unsteady_arnold_beltrami_childress_flow}

Inspired by~\textcite{oettinger2016autonomous}, a temporally aperiodic
modification of the ABC flow (\cref{eq:abc_flow}) was made by the replacements
\begin{equation}
    \label{eq:abc_params_nonstationary}
    \begin{gathered}
    B\to{}\widetilde{B}(t) = B + B\cdot{}k_{0}\tanh(k_{1}t)\cos({({k_{2}t})}^{2}),\\
    C\to{}\widetilde{C}(t) = C + C\cdot{}k_{0}\tanh(k_{1}t)\sin({({k_{3}t})}^{2}),
    \end{gathered}
\end{equation}
with $A$, $B$ and $C$ given by \cref{eq:abc_params_stationary}, where the
parameters values
\begin{equation}
    \label{eq:abc_params_nonstationary_frequencies}
    k_{0}=0.3,\quad k_{1}=0.5,\quad k_{2}=1.5,\quad k_{3}=1.8,
\end{equation}
were used. The fundamental idea of this modification is to further enhance
the chaotic nature of the resulting flow patterns. Similarly modified
ABC flow has previously been at the centre of other three-dimensional transport
barrier investigations --- including hyperbolic LCSs --- such as the work of
\textcite{blazevski2014hyperbolic}; albeit with quite different methods of
computing said LCSs than the one considered here. Like for its stationary
sibling, the time domain of interest for this system was $\mathcal{I}=[0,5]$.
The time dependence of the $\widetilde{B}$ and  $\widetilde{C}$ coefficients is
illustrated in \cref{fig:abc_timedep_coeff}.

\begin{figure}[htpb]
    \centering
    \resizebox{0.9\textwidth}{!}{%% Creator: Matplotlib, PGF backend
%%
%% To include the figure in your LaTeX document, write
%%   \input{<filename>.pgf}
%%
%% Make sure the required packages are loaded in your preamble
%%   \usepackage{pgf}
%%
%% Figures using additional raster images can only be included by \input if
%% they are in the same directory as the main LaTeX file. For loading figures
%% from other directories you can use the `import` package
%%   \usepackage{import}
%% and then include the figures with
%%   \import{<path to file>}{<filename>.pgf}
%%
%% Matplotlib used the following preamble
%%   \usepackage{fontspec}
%%   \setmainfont{DejaVu Serif}
%%   \setsansfont{DejaVu Sans}
%%   \setmonofont{DejaVu Sans Mono}
%%
\begingroup%
\makeatletter%
\begin{pgfpicture}%
\pgfpathrectangle{\pgfpointorigin}{\pgfqpoint{6.000000in}{4.000000in}}%
\pgfusepath{use as bounding box, clip}%
\begin{pgfscope}%
\pgfsetbuttcap%
\pgfsetmiterjoin%
\definecolor{currentfill}{rgb}{1.000000,1.000000,1.000000}%
\pgfsetfillcolor{currentfill}%
\pgfsetlinewidth{0.000000pt}%
\definecolor{currentstroke}{rgb}{1.000000,1.000000,1.000000}%
\pgfsetstrokecolor{currentstroke}%
\pgfsetdash{}{0pt}%
\pgfpathmoveto{\pgfqpoint{0.000000in}{0.000000in}}%
\pgfpathlineto{\pgfqpoint{6.000000in}{0.000000in}}%
\pgfpathlineto{\pgfqpoint{6.000000in}{4.000000in}}%
\pgfpathlineto{\pgfqpoint{0.000000in}{4.000000in}}%
\pgfpathclose%
\pgfusepath{fill}%
\end{pgfscope}%
\begin{pgfscope}%
\pgfsetbuttcap%
\pgfsetmiterjoin%
\pgfsetlinewidth{0.000000pt}%
\definecolor{currentstroke}{rgb}{0.000000,0.000000,0.000000}%
\pgfsetstrokecolor{currentstroke}%
\pgfsetstrokeopacity{0.000000}%
\pgfsetdash{}{0pt}%
\pgfpathmoveto{\pgfqpoint{0.360000in}{0.400000in}}%
\pgfpathlineto{\pgfqpoint{5.940000in}{0.400000in}}%
\pgfpathlineto{\pgfqpoint{5.940000in}{3.960000in}}%
\pgfpathlineto{\pgfqpoint{0.360000in}{3.960000in}}%
\pgfpathclose%
\pgfusepath{}%
\end{pgfscope}%
\begin{pgfscope}%
\pgfsetbuttcap%
\pgfsetroundjoin%
\definecolor{currentfill}{rgb}{0.000000,0.000000,0.000000}%
\pgfsetfillcolor{currentfill}%
\pgfsetlinewidth{0.803000pt}%
\definecolor{currentstroke}{rgb}{0.000000,0.000000,0.000000}%
\pgfsetstrokecolor{currentstroke}%
\pgfsetdash{}{0pt}%
\pgfsys@defobject{currentmarker}{\pgfqpoint{0.000000in}{-0.048611in}}{\pgfqpoint{0.000000in}{0.000000in}}{%
\pgfpathmoveto{\pgfqpoint{0.000000in}{0.000000in}}%
\pgfpathlineto{\pgfqpoint{0.000000in}{-0.048611in}}%
\pgfusepath{stroke,fill}%
}%
\begin{pgfscope}%
\pgfsys@transformshift{0.613636in}{0.400000in}%
\pgfsys@useobject{currentmarker}{}%
\end{pgfscope}%
\end{pgfscope}%
\begin{pgfscope}%
\pgftext[x=0.613636in,y=0.302778in,,top]{\sffamily\fontsize{10.000000}{12.000000}\selectfont \(\displaystyle 0\)}%
\end{pgfscope}%
\begin{pgfscope}%
\pgfsetbuttcap%
\pgfsetroundjoin%
\definecolor{currentfill}{rgb}{0.000000,0.000000,0.000000}%
\pgfsetfillcolor{currentfill}%
\pgfsetlinewidth{0.803000pt}%
\definecolor{currentstroke}{rgb}{0.000000,0.000000,0.000000}%
\pgfsetstrokecolor{currentstroke}%
\pgfsetdash{}{0pt}%
\pgfsys@defobject{currentmarker}{\pgfqpoint{0.000000in}{-0.048611in}}{\pgfqpoint{0.000000in}{0.000000in}}{%
\pgfpathmoveto{\pgfqpoint{0.000000in}{0.000000in}}%
\pgfpathlineto{\pgfqpoint{0.000000in}{-0.048611in}}%
\pgfusepath{stroke,fill}%
}%
\begin{pgfscope}%
\pgfsys@transformshift{1.628182in}{0.400000in}%
\pgfsys@useobject{currentmarker}{}%
\end{pgfscope}%
\end{pgfscope}%
\begin{pgfscope}%
\pgftext[x=1.628182in,y=0.302778in,,top]{\sffamily\fontsize{10.000000}{12.000000}\selectfont \(\displaystyle 1\)}%
\end{pgfscope}%
\begin{pgfscope}%
\pgfsetbuttcap%
\pgfsetroundjoin%
\definecolor{currentfill}{rgb}{0.000000,0.000000,0.000000}%
\pgfsetfillcolor{currentfill}%
\pgfsetlinewidth{0.803000pt}%
\definecolor{currentstroke}{rgb}{0.000000,0.000000,0.000000}%
\pgfsetstrokecolor{currentstroke}%
\pgfsetdash{}{0pt}%
\pgfsys@defobject{currentmarker}{\pgfqpoint{0.000000in}{-0.048611in}}{\pgfqpoint{0.000000in}{0.000000in}}{%
\pgfpathmoveto{\pgfqpoint{0.000000in}{0.000000in}}%
\pgfpathlineto{\pgfqpoint{0.000000in}{-0.048611in}}%
\pgfusepath{stroke,fill}%
}%
\begin{pgfscope}%
\pgfsys@transformshift{2.642727in}{0.400000in}%
\pgfsys@useobject{currentmarker}{}%
\end{pgfscope}%
\end{pgfscope}%
\begin{pgfscope}%
\pgftext[x=2.642727in,y=0.302778in,,top]{\sffamily\fontsize{10.000000}{12.000000}\selectfont \(\displaystyle 2\)}%
\end{pgfscope}%
\begin{pgfscope}%
\pgfsetbuttcap%
\pgfsetroundjoin%
\definecolor{currentfill}{rgb}{0.000000,0.000000,0.000000}%
\pgfsetfillcolor{currentfill}%
\pgfsetlinewidth{0.803000pt}%
\definecolor{currentstroke}{rgb}{0.000000,0.000000,0.000000}%
\pgfsetstrokecolor{currentstroke}%
\pgfsetdash{}{0pt}%
\pgfsys@defobject{currentmarker}{\pgfqpoint{0.000000in}{-0.048611in}}{\pgfqpoint{0.000000in}{0.000000in}}{%
\pgfpathmoveto{\pgfqpoint{0.000000in}{0.000000in}}%
\pgfpathlineto{\pgfqpoint{0.000000in}{-0.048611in}}%
\pgfusepath{stroke,fill}%
}%
\begin{pgfscope}%
\pgfsys@transformshift{3.657273in}{0.400000in}%
\pgfsys@useobject{currentmarker}{}%
\end{pgfscope}%
\end{pgfscope}%
\begin{pgfscope}%
\pgftext[x=3.657273in,y=0.302778in,,top]{\sffamily\fontsize{10.000000}{12.000000}\selectfont \(\displaystyle 3\)}%
\end{pgfscope}%
\begin{pgfscope}%
\pgfsetbuttcap%
\pgfsetroundjoin%
\definecolor{currentfill}{rgb}{0.000000,0.000000,0.000000}%
\pgfsetfillcolor{currentfill}%
\pgfsetlinewidth{0.803000pt}%
\definecolor{currentstroke}{rgb}{0.000000,0.000000,0.000000}%
\pgfsetstrokecolor{currentstroke}%
\pgfsetdash{}{0pt}%
\pgfsys@defobject{currentmarker}{\pgfqpoint{0.000000in}{-0.048611in}}{\pgfqpoint{0.000000in}{0.000000in}}{%
\pgfpathmoveto{\pgfqpoint{0.000000in}{0.000000in}}%
\pgfpathlineto{\pgfqpoint{0.000000in}{-0.048611in}}%
\pgfusepath{stroke,fill}%
}%
\begin{pgfscope}%
\pgfsys@transformshift{4.671818in}{0.400000in}%
\pgfsys@useobject{currentmarker}{}%
\end{pgfscope}%
\end{pgfscope}%
\begin{pgfscope}%
\pgftext[x=4.671818in,y=0.302778in,,top]{\sffamily\fontsize{10.000000}{12.000000}\selectfont \(\displaystyle 4\)}%
\end{pgfscope}%
\begin{pgfscope}%
\pgfsetbuttcap%
\pgfsetroundjoin%
\definecolor{currentfill}{rgb}{0.000000,0.000000,0.000000}%
\pgfsetfillcolor{currentfill}%
\pgfsetlinewidth{0.803000pt}%
\definecolor{currentstroke}{rgb}{0.000000,0.000000,0.000000}%
\pgfsetstrokecolor{currentstroke}%
\pgfsetdash{}{0pt}%
\pgfsys@defobject{currentmarker}{\pgfqpoint{0.000000in}{-0.048611in}}{\pgfqpoint{0.000000in}{0.000000in}}{%
\pgfpathmoveto{\pgfqpoint{0.000000in}{0.000000in}}%
\pgfpathlineto{\pgfqpoint{0.000000in}{-0.048611in}}%
\pgfusepath{stroke,fill}%
}%
\begin{pgfscope}%
\pgfsys@transformshift{5.686364in}{0.400000in}%
\pgfsys@useobject{currentmarker}{}%
\end{pgfscope}%
\end{pgfscope}%
\begin{pgfscope}%
\pgftext[x=5.686364in,y=0.302778in,,top]{\sffamily\fontsize{10.000000}{12.000000}\selectfont \(\displaystyle 5\)}%
\end{pgfscope}%
\begin{pgfscope}%
\pgftext[x=3.150000in,y=0.168365in,,top]{\sffamily\fontsize{12.000000}{14.400000}\selectfont \(\displaystyle \tau\)}%
\end{pgfscope}%
\begin{pgfscope}%
\pgfsetbuttcap%
\pgfsetroundjoin%
\definecolor{currentfill}{rgb}{0.000000,0.000000,0.000000}%
\pgfsetfillcolor{currentfill}%
\pgfsetlinewidth{0.803000pt}%
\definecolor{currentstroke}{rgb}{0.000000,0.000000,0.000000}%
\pgfsetstrokecolor{currentstroke}%
\pgfsetdash{}{0pt}%
\pgfsys@defobject{currentmarker}{\pgfqpoint{-0.048611in}{0.000000in}}{\pgfqpoint{0.000000in}{0.000000in}}{%
\pgfpathmoveto{\pgfqpoint{0.000000in}{0.000000in}}%
\pgfpathlineto{\pgfqpoint{-0.048611in}{0.000000in}}%
\pgfusepath{stroke,fill}%
}%
\begin{pgfscope}%
\pgfsys@transformshift{0.360000in}{0.756000in}%
\pgfsys@useobject{currentmarker}{}%
\end{pgfscope}%
\end{pgfscope}%
\begin{pgfscope}%
\pgftext[x=-0.022717in,y=0.703238in,left,base]{\sffamily\fontsize{10.000000}{12.000000}\selectfont \(\displaystyle -0.4\)}%
\end{pgfscope}%
\begin{pgfscope}%
\pgfsetbuttcap%
\pgfsetroundjoin%
\definecolor{currentfill}{rgb}{0.000000,0.000000,0.000000}%
\pgfsetfillcolor{currentfill}%
\pgfsetlinewidth{0.803000pt}%
\definecolor{currentstroke}{rgb}{0.000000,0.000000,0.000000}%
\pgfsetstrokecolor{currentstroke}%
\pgfsetdash{}{0pt}%
\pgfsys@defobject{currentmarker}{\pgfqpoint{-0.048611in}{0.000000in}}{\pgfqpoint{0.000000in}{0.000000in}}{%
\pgfpathmoveto{\pgfqpoint{0.000000in}{0.000000in}}%
\pgfpathlineto{\pgfqpoint{-0.048611in}{0.000000in}}%
\pgfusepath{stroke,fill}%
}%
\begin{pgfscope}%
\pgfsys@transformshift{0.360000in}{1.468000in}%
\pgfsys@useobject{currentmarker}{}%
\end{pgfscope}%
\end{pgfscope}%
\begin{pgfscope}%
\pgftext[x=-0.022717in,y=1.415238in,left,base]{\sffamily\fontsize{10.000000}{12.000000}\selectfont \(\displaystyle -0.2\)}%
\end{pgfscope}%
\begin{pgfscope}%
\pgfsetbuttcap%
\pgfsetroundjoin%
\definecolor{currentfill}{rgb}{0.000000,0.000000,0.000000}%
\pgfsetfillcolor{currentfill}%
\pgfsetlinewidth{0.803000pt}%
\definecolor{currentstroke}{rgb}{0.000000,0.000000,0.000000}%
\pgfsetstrokecolor{currentstroke}%
\pgfsetdash{}{0pt}%
\pgfsys@defobject{currentmarker}{\pgfqpoint{-0.048611in}{0.000000in}}{\pgfqpoint{0.000000in}{0.000000in}}{%
\pgfpathmoveto{\pgfqpoint{0.000000in}{0.000000in}}%
\pgfpathlineto{\pgfqpoint{-0.048611in}{0.000000in}}%
\pgfusepath{stroke,fill}%
}%
\begin{pgfscope}%
\pgfsys@transformshift{0.360000in}{2.180000in}%
\pgfsys@useobject{currentmarker}{}%
\end{pgfscope}%
\end{pgfscope}%
\begin{pgfscope}%
\pgftext[x=0.085308in,y=2.127238in,left,base]{\sffamily\fontsize{10.000000}{12.000000}\selectfont \(\displaystyle 0.0\)}%
\end{pgfscope}%
\begin{pgfscope}%
\pgfsetbuttcap%
\pgfsetroundjoin%
\definecolor{currentfill}{rgb}{0.000000,0.000000,0.000000}%
\pgfsetfillcolor{currentfill}%
\pgfsetlinewidth{0.803000pt}%
\definecolor{currentstroke}{rgb}{0.000000,0.000000,0.000000}%
\pgfsetstrokecolor{currentstroke}%
\pgfsetdash{}{0pt}%
\pgfsys@defobject{currentmarker}{\pgfqpoint{-0.048611in}{0.000000in}}{\pgfqpoint{0.000000in}{0.000000in}}{%
\pgfpathmoveto{\pgfqpoint{0.000000in}{0.000000in}}%
\pgfpathlineto{\pgfqpoint{-0.048611in}{0.000000in}}%
\pgfusepath{stroke,fill}%
}%
\begin{pgfscope}%
\pgfsys@transformshift{0.360000in}{2.892000in}%
\pgfsys@useobject{currentmarker}{}%
\end{pgfscope}%
\end{pgfscope}%
\begin{pgfscope}%
\pgftext[x=0.085308in,y=2.839238in,left,base]{\sffamily\fontsize{10.000000}{12.000000}\selectfont \(\displaystyle 0.2\)}%
\end{pgfscope}%
\begin{pgfscope}%
\pgfsetbuttcap%
\pgfsetroundjoin%
\definecolor{currentfill}{rgb}{0.000000,0.000000,0.000000}%
\pgfsetfillcolor{currentfill}%
\pgfsetlinewidth{0.803000pt}%
\definecolor{currentstroke}{rgb}{0.000000,0.000000,0.000000}%
\pgfsetstrokecolor{currentstroke}%
\pgfsetdash{}{0pt}%
\pgfsys@defobject{currentmarker}{\pgfqpoint{-0.048611in}{0.000000in}}{\pgfqpoint{0.000000in}{0.000000in}}{%
\pgfpathmoveto{\pgfqpoint{0.000000in}{0.000000in}}%
\pgfpathlineto{\pgfqpoint{-0.048611in}{0.000000in}}%
\pgfusepath{stroke,fill}%
}%
\begin{pgfscope}%
\pgfsys@transformshift{0.360000in}{3.604000in}%
\pgfsys@useobject{currentmarker}{}%
\end{pgfscope}%
\end{pgfscope}%
\begin{pgfscope}%
\pgftext[x=0.085308in,y=3.551238in,left,base]{\sffamily\fontsize{10.000000}{12.000000}\selectfont \(\displaystyle 0.4\)}%
\end{pgfscope}%
\begin{pgfscope}%
\pgfpathrectangle{\pgfqpoint{0.360000in}{0.400000in}}{\pgfqpoint{5.580000in}{3.560000in}}%
\pgfusepath{clip}%
\pgfsetrectcap%
\pgfsetroundjoin%
\pgfsetlinewidth{1.505625pt}%
\definecolor{currentstroke}{rgb}{0.843137,0.188235,0.121569}%
\pgfsetstrokecolor{currentstroke}%
\pgfsetdash{}{0pt}%
\pgfpathmoveto{\pgfqpoint{0.613636in}{2.180000in}}%
\pgfpathlineto{\pgfqpoint{0.743056in}{2.276140in}}%
\pgfpathlineto{\pgfqpoint{0.791271in}{2.311574in}}%
\pgfpathlineto{\pgfqpoint{0.826797in}{2.337309in}}%
\pgfpathlineto{\pgfqpoint{0.854711in}{2.357168in}}%
\pgfpathlineto{\pgfqpoint{0.880088in}{2.374834in}}%
\pgfpathlineto{\pgfqpoint{0.902926in}{2.390320in}}%
\pgfpathlineto{\pgfqpoint{0.923228in}{2.403676in}}%
\pgfpathlineto{\pgfqpoint{0.940991in}{2.414981in}}%
\pgfpathlineto{\pgfqpoint{0.956217in}{2.424341in}}%
\pgfpathlineto{\pgfqpoint{0.971443in}{2.433355in}}%
\pgfpathlineto{\pgfqpoint{0.986668in}{2.441979in}}%
\pgfpathlineto{\pgfqpoint{0.999356in}{2.448834in}}%
\pgfpathlineto{\pgfqpoint{1.012045in}{2.455358in}}%
\pgfpathlineto{\pgfqpoint{1.024733in}{2.461519in}}%
\pgfpathlineto{\pgfqpoint{1.037421in}{2.467287in}}%
\pgfpathlineto{\pgfqpoint{1.047572in}{2.471595in}}%
\pgfpathlineto{\pgfqpoint{1.057722in}{2.475613in}}%
\pgfpathlineto{\pgfqpoint{1.067873in}{2.479321in}}%
\pgfpathlineto{\pgfqpoint{1.078023in}{2.482703in}}%
\pgfpathlineto{\pgfqpoint{1.088174in}{2.485737in}}%
\pgfpathlineto{\pgfqpoint{1.098324in}{2.488405in}}%
\pgfpathlineto{\pgfqpoint{1.108475in}{2.490687in}}%
\pgfpathlineto{\pgfqpoint{1.118625in}{2.492561in}}%
\pgfpathlineto{\pgfqpoint{1.126238in}{2.493687in}}%
\pgfpathlineto{\pgfqpoint{1.133851in}{2.494564in}}%
\pgfpathlineto{\pgfqpoint{1.141464in}{2.495181in}}%
\pgfpathlineto{\pgfqpoint{1.149077in}{2.495531in}}%
\pgfpathlineto{\pgfqpoint{1.156690in}{2.495604in}}%
\pgfpathlineto{\pgfqpoint{1.164303in}{2.495391in}}%
\pgfpathlineto{\pgfqpoint{1.171916in}{2.494883in}}%
\pgfpathlineto{\pgfqpoint{1.179528in}{2.494069in}}%
\pgfpathlineto{\pgfqpoint{1.187141in}{2.492942in}}%
\pgfpathlineto{\pgfqpoint{1.194754in}{2.491491in}}%
\pgfpathlineto{\pgfqpoint{1.202367in}{2.489708in}}%
\pgfpathlineto{\pgfqpoint{1.209980in}{2.487584in}}%
\pgfpathlineto{\pgfqpoint{1.217593in}{2.485108in}}%
\pgfpathlineto{\pgfqpoint{1.225206in}{2.482273in}}%
\pgfpathlineto{\pgfqpoint{1.232819in}{2.479070in}}%
\pgfpathlineto{\pgfqpoint{1.240432in}{2.475489in}}%
\pgfpathlineto{\pgfqpoint{1.248044in}{2.471522in}}%
\pgfpathlineto{\pgfqpoint{1.255657in}{2.467161in}}%
\pgfpathlineto{\pgfqpoint{1.263270in}{2.462398in}}%
\pgfpathlineto{\pgfqpoint{1.270883in}{2.457224in}}%
\pgfpathlineto{\pgfqpoint{1.278496in}{2.451633in}}%
\pgfpathlineto{\pgfqpoint{1.286109in}{2.445616in}}%
\pgfpathlineto{\pgfqpoint{1.293722in}{2.439167in}}%
\pgfpathlineto{\pgfqpoint{1.301335in}{2.432280in}}%
\pgfpathlineto{\pgfqpoint{1.308948in}{2.424947in}}%
\pgfpathlineto{\pgfqpoint{1.316561in}{2.417164in}}%
\pgfpathlineto{\pgfqpoint{1.324173in}{2.408924in}}%
\pgfpathlineto{\pgfqpoint{1.331786in}{2.400224in}}%
\pgfpathlineto{\pgfqpoint{1.339399in}{2.391059in}}%
\pgfpathlineto{\pgfqpoint{1.347012in}{2.381424in}}%
\pgfpathlineto{\pgfqpoint{1.354625in}{2.371318in}}%
\pgfpathlineto{\pgfqpoint{1.362238in}{2.360737in}}%
\pgfpathlineto{\pgfqpoint{1.369851in}{2.349680in}}%
\pgfpathlineto{\pgfqpoint{1.377464in}{2.338146in}}%
\pgfpathlineto{\pgfqpoint{1.385077in}{2.326134in}}%
\pgfpathlineto{\pgfqpoint{1.392690in}{2.313645in}}%
\pgfpathlineto{\pgfqpoint{1.400302in}{2.300679in}}%
\pgfpathlineto{\pgfqpoint{1.410453in}{2.282656in}}%
\pgfpathlineto{\pgfqpoint{1.420603in}{2.263798in}}%
\pgfpathlineto{\pgfqpoint{1.430754in}{2.244116in}}%
\pgfpathlineto{\pgfqpoint{1.440905in}{2.223624in}}%
\pgfpathlineto{\pgfqpoint{1.451055in}{2.202338in}}%
\pgfpathlineto{\pgfqpoint{1.461206in}{2.180280in}}%
\pgfpathlineto{\pgfqpoint{1.471356in}{2.157474in}}%
\pgfpathlineto{\pgfqpoint{1.481507in}{2.133947in}}%
\pgfpathlineto{\pgfqpoint{1.491657in}{2.109731in}}%
\pgfpathlineto{\pgfqpoint{1.504345in}{2.078546in}}%
\pgfpathlineto{\pgfqpoint{1.517034in}{2.046420in}}%
\pgfpathlineto{\pgfqpoint{1.529722in}{2.013441in}}%
\pgfpathlineto{\pgfqpoint{1.544947in}{1.972881in}}%
\pgfpathlineto{\pgfqpoint{1.560173in}{1.931429in}}%
\pgfpathlineto{\pgfqpoint{1.580474in}{1.875147in}}%
\pgfpathlineto{\pgfqpoint{1.643915in}{1.698223in}}%
\pgfpathlineto{\pgfqpoint{1.659141in}{1.657199in}}%
\pgfpathlineto{\pgfqpoint{1.671829in}{1.623950in}}%
\pgfpathlineto{\pgfqpoint{1.681980in}{1.598117in}}%
\pgfpathlineto{\pgfqpoint{1.692130in}{1.573085in}}%
\pgfpathlineto{\pgfqpoint{1.702281in}{1.548969in}}%
\pgfpathlineto{\pgfqpoint{1.709894in}{1.531553in}}%
\pgfpathlineto{\pgfqpoint{1.717506in}{1.514767in}}%
\pgfpathlineto{\pgfqpoint{1.725119in}{1.498661in}}%
\pgfpathlineto{\pgfqpoint{1.732732in}{1.483287in}}%
\pgfpathlineto{\pgfqpoint{1.740345in}{1.468693in}}%
\pgfpathlineto{\pgfqpoint{1.747958in}{1.454933in}}%
\pgfpathlineto{\pgfqpoint{1.755571in}{1.442055in}}%
\pgfpathlineto{\pgfqpoint{1.760646in}{1.433986in}}%
\pgfpathlineto{\pgfqpoint{1.765721in}{1.426347in}}%
\pgfpathlineto{\pgfqpoint{1.770797in}{1.419152in}}%
\pgfpathlineto{\pgfqpoint{1.775872in}{1.412417in}}%
\pgfpathlineto{\pgfqpoint{1.780947in}{1.406156in}}%
\pgfpathlineto{\pgfqpoint{1.786023in}{1.400384in}}%
\pgfpathlineto{\pgfqpoint{1.791098in}{1.395115in}}%
\pgfpathlineto{\pgfqpoint{1.796173in}{1.390363in}}%
\pgfpathlineto{\pgfqpoint{1.801248in}{1.386142in}}%
\pgfpathlineto{\pgfqpoint{1.806324in}{1.382466in}}%
\pgfpathlineto{\pgfqpoint{1.811399in}{1.379348in}}%
\pgfpathlineto{\pgfqpoint{1.816474in}{1.376801in}}%
\pgfpathlineto{\pgfqpoint{1.821549in}{1.374839in}}%
\pgfpathlineto{\pgfqpoint{1.826625in}{1.373474in}}%
\pgfpathlineto{\pgfqpoint{1.831700in}{1.372717in}}%
\pgfpathlineto{\pgfqpoint{1.836775in}{1.372582in}}%
\pgfpathlineto{\pgfqpoint{1.839313in}{1.372750in}}%
\pgfpathlineto{\pgfqpoint{1.841850in}{1.373078in}}%
\pgfpathlineto{\pgfqpoint{1.844388in}{1.373567in}}%
\pgfpathlineto{\pgfqpoint{1.849463in}{1.375033in}}%
\pgfpathlineto{\pgfqpoint{1.854539in}{1.377157in}}%
\pgfpathlineto{\pgfqpoint{1.859614in}{1.379951in}}%
\pgfpathlineto{\pgfqpoint{1.864689in}{1.383422in}}%
\pgfpathlineto{\pgfqpoint{1.869764in}{1.387581in}}%
\pgfpathlineto{\pgfqpoint{1.874840in}{1.392435in}}%
\pgfpathlineto{\pgfqpoint{1.879915in}{1.397991in}}%
\pgfpathlineto{\pgfqpoint{1.884990in}{1.404258in}}%
\pgfpathlineto{\pgfqpoint{1.890065in}{1.411241in}}%
\pgfpathlineto{\pgfqpoint{1.895141in}{1.418946in}}%
\pgfpathlineto{\pgfqpoint{1.900216in}{1.427377in}}%
\pgfpathlineto{\pgfqpoint{1.905291in}{1.436541in}}%
\pgfpathlineto{\pgfqpoint{1.910367in}{1.446439in}}%
\pgfpathlineto{\pgfqpoint{1.915442in}{1.457075in}}%
\pgfpathlineto{\pgfqpoint{1.920517in}{1.468450in}}%
\pgfpathlineto{\pgfqpoint{1.925592in}{1.480567in}}%
\pgfpathlineto{\pgfqpoint{1.930668in}{1.493425in}}%
\pgfpathlineto{\pgfqpoint{1.935743in}{1.507024in}}%
\pgfpathlineto{\pgfqpoint{1.940818in}{1.521363in}}%
\pgfpathlineto{\pgfqpoint{1.945893in}{1.536438in}}%
\pgfpathlineto{\pgfqpoint{1.950969in}{1.552248in}}%
\pgfpathlineto{\pgfqpoint{1.956044in}{1.568788in}}%
\pgfpathlineto{\pgfqpoint{1.963657in}{1.594955in}}%
\pgfpathlineto{\pgfqpoint{1.971270in}{1.622730in}}%
\pgfpathlineto{\pgfqpoint{1.978883in}{1.652089in}}%
\pgfpathlineto{\pgfqpoint{1.986496in}{1.682998in}}%
\pgfpathlineto{\pgfqpoint{1.994108in}{1.715421in}}%
\pgfpathlineto{\pgfqpoint{2.001721in}{1.749316in}}%
\pgfpathlineto{\pgfqpoint{2.009334in}{1.784634in}}%
\pgfpathlineto{\pgfqpoint{2.016947in}{1.821321in}}%
\pgfpathlineto{\pgfqpoint{2.024560in}{1.859319in}}%
\pgfpathlineto{\pgfqpoint{2.034711in}{1.911909in}}%
\pgfpathlineto{\pgfqpoint{2.044861in}{1.966543in}}%
\pgfpathlineto{\pgfqpoint{2.055012in}{2.023034in}}%
\pgfpathlineto{\pgfqpoint{2.065162in}{2.081178in}}%
\pgfpathlineto{\pgfqpoint{2.077850in}{2.155841in}}%
\pgfpathlineto{\pgfqpoint{2.093076in}{2.247717in}}%
\pgfpathlineto{\pgfqpoint{2.113377in}{2.372616in}}%
\pgfpathlineto{\pgfqpoint{2.143829in}{2.560317in}}%
\pgfpathlineto{\pgfqpoint{2.159055in}{2.651993in}}%
\pgfpathlineto{\pgfqpoint{2.171743in}{2.726169in}}%
\pgfpathlineto{\pgfqpoint{2.181893in}{2.783564in}}%
\pgfpathlineto{\pgfqpoint{2.192044in}{2.838846in}}%
\pgfpathlineto{\pgfqpoint{2.199657in}{2.878706in}}%
\pgfpathlineto{\pgfqpoint{2.207270in}{2.917026in}}%
\pgfpathlineto{\pgfqpoint{2.214882in}{2.953658in}}%
\pgfpathlineto{\pgfqpoint{2.222495in}{2.988454in}}%
\pgfpathlineto{\pgfqpoint{2.230108in}{3.021266in}}%
\pgfpathlineto{\pgfqpoint{2.235184in}{3.041969in}}%
\pgfpathlineto{\pgfqpoint{2.240259in}{3.061686in}}%
\pgfpathlineto{\pgfqpoint{2.245334in}{3.080374in}}%
\pgfpathlineto{\pgfqpoint{2.250409in}{3.097995in}}%
\pgfpathlineto{\pgfqpoint{2.255485in}{3.114509in}}%
\pgfpathlineto{\pgfqpoint{2.260560in}{3.129878in}}%
\pgfpathlineto{\pgfqpoint{2.265635in}{3.144063in}}%
\pgfpathlineto{\pgfqpoint{2.270710in}{3.157030in}}%
\pgfpathlineto{\pgfqpoint{2.275786in}{3.168744in}}%
\pgfpathlineto{\pgfqpoint{2.280861in}{3.179170in}}%
\pgfpathlineto{\pgfqpoint{2.285936in}{3.188277in}}%
\pgfpathlineto{\pgfqpoint{2.288474in}{3.192327in}}%
\pgfpathlineto{\pgfqpoint{2.291011in}{3.196035in}}%
\pgfpathlineto{\pgfqpoint{2.293549in}{3.199398in}}%
\pgfpathlineto{\pgfqpoint{2.296087in}{3.202413in}}%
\pgfpathlineto{\pgfqpoint{2.298624in}{3.205077in}}%
\pgfpathlineto{\pgfqpoint{2.301162in}{3.207386in}}%
\pgfpathlineto{\pgfqpoint{2.303700in}{3.209337in}}%
\pgfpathlineto{\pgfqpoint{2.306237in}{3.210927in}}%
\pgfpathlineto{\pgfqpoint{2.308775in}{3.212153in}}%
\pgfpathlineto{\pgfqpoint{2.311312in}{3.213013in}}%
\pgfpathlineto{\pgfqpoint{2.313850in}{3.213503in}}%
\pgfpathlineto{\pgfqpoint{2.316388in}{3.213621in}}%
\pgfpathlineto{\pgfqpoint{2.318925in}{3.213366in}}%
\pgfpathlineto{\pgfqpoint{2.321463in}{3.212733in}}%
\pgfpathlineto{\pgfqpoint{2.324001in}{3.211722in}}%
\pgfpathlineto{\pgfqpoint{2.326538in}{3.210331in}}%
\pgfpathlineto{\pgfqpoint{2.329076in}{3.208557in}}%
\pgfpathlineto{\pgfqpoint{2.331614in}{3.206398in}}%
\pgfpathlineto{\pgfqpoint{2.334151in}{3.203854in}}%
\pgfpathlineto{\pgfqpoint{2.336689in}{3.200923in}}%
\pgfpathlineto{\pgfqpoint{2.339226in}{3.197603in}}%
\pgfpathlineto{\pgfqpoint{2.341764in}{3.193894in}}%
\pgfpathlineto{\pgfqpoint{2.344302in}{3.189794in}}%
\pgfpathlineto{\pgfqpoint{2.346839in}{3.185302in}}%
\pgfpathlineto{\pgfqpoint{2.349377in}{3.180419in}}%
\pgfpathlineto{\pgfqpoint{2.351915in}{3.175143in}}%
\pgfpathlineto{\pgfqpoint{2.354452in}{3.169473in}}%
\pgfpathlineto{\pgfqpoint{2.359527in}{3.156956in}}%
\pgfpathlineto{\pgfqpoint{2.364603in}{3.142866in}}%
\pgfpathlineto{\pgfqpoint{2.369678in}{3.127207in}}%
\pgfpathlineto{\pgfqpoint{2.374753in}{3.109986in}}%
\pgfpathlineto{\pgfqpoint{2.379829in}{3.091210in}}%
\pgfpathlineto{\pgfqpoint{2.384904in}{3.070891in}}%
\pgfpathlineto{\pgfqpoint{2.389979in}{3.049045in}}%
\pgfpathlineto{\pgfqpoint{2.395054in}{3.025687in}}%
\pgfpathlineto{\pgfqpoint{2.400130in}{3.000839in}}%
\pgfpathlineto{\pgfqpoint{2.405205in}{2.974524in}}%
\pgfpathlineto{\pgfqpoint{2.410280in}{2.946767in}}%
\pgfpathlineto{\pgfqpoint{2.415355in}{2.917599in}}%
\pgfpathlineto{\pgfqpoint{2.420431in}{2.887051in}}%
\pgfpathlineto{\pgfqpoint{2.428044in}{2.838721in}}%
\pgfpathlineto{\pgfqpoint{2.435656in}{2.787498in}}%
\pgfpathlineto{\pgfqpoint{2.443269in}{2.733530in}}%
\pgfpathlineto{\pgfqpoint{2.450882in}{2.676980in}}%
\pgfpathlineto{\pgfqpoint{2.458495in}{2.618023in}}%
\pgfpathlineto{\pgfqpoint{2.466108in}{2.556852in}}%
\pgfpathlineto{\pgfqpoint{2.476259in}{2.472201in}}%
\pgfpathlineto{\pgfqpoint{2.486409in}{2.384504in}}%
\pgfpathlineto{\pgfqpoint{2.499097in}{2.271463in}}%
\pgfpathlineto{\pgfqpoint{2.514323in}{2.132374in}}%
\pgfpathlineto{\pgfqpoint{2.552388in}{1.782493in}}%
\pgfpathlineto{\pgfqpoint{2.562538in}{1.692194in}}%
\pgfpathlineto{\pgfqpoint{2.572689in}{1.604573in}}%
\pgfpathlineto{\pgfqpoint{2.580302in}{1.541065in}}%
\pgfpathlineto{\pgfqpoint{2.587914in}{1.479795in}}%
\pgfpathlineto{\pgfqpoint{2.595527in}{1.421076in}}%
\pgfpathlineto{\pgfqpoint{2.603140in}{1.365212in}}%
\pgfpathlineto{\pgfqpoint{2.610753in}{1.312502in}}%
\pgfpathlineto{\pgfqpoint{2.615828in}{1.279257in}}%
\pgfpathlineto{\pgfqpoint{2.620904in}{1.247628in}}%
\pgfpathlineto{\pgfqpoint{2.625979in}{1.217696in}}%
\pgfpathlineto{\pgfqpoint{2.631054in}{1.189539in}}%
\pgfpathlineto{\pgfqpoint{2.636129in}{1.163235in}}%
\pgfpathlineto{\pgfqpoint{2.641205in}{1.138857in}}%
\pgfpathlineto{\pgfqpoint{2.646280in}{1.116476in}}%
\pgfpathlineto{\pgfqpoint{2.651355in}{1.096159in}}%
\pgfpathlineto{\pgfqpoint{2.656430in}{1.077970in}}%
\pgfpathlineto{\pgfqpoint{2.658968in}{1.069693in}}%
\pgfpathlineto{\pgfqpoint{2.661506in}{1.061970in}}%
\pgfpathlineto{\pgfqpoint{2.664043in}{1.054808in}}%
\pgfpathlineto{\pgfqpoint{2.666581in}{1.048214in}}%
\pgfpathlineto{\pgfqpoint{2.669119in}{1.042194in}}%
\pgfpathlineto{\pgfqpoint{2.671656in}{1.036754in}}%
\pgfpathlineto{\pgfqpoint{2.674194in}{1.031901in}}%
\pgfpathlineto{\pgfqpoint{2.676732in}{1.027639in}}%
\pgfpathlineto{\pgfqpoint{2.679269in}{1.023975in}}%
\pgfpathlineto{\pgfqpoint{2.681807in}{1.020912in}}%
\pgfpathlineto{\pgfqpoint{2.684344in}{1.018456in}}%
\pgfpathlineto{\pgfqpoint{2.686882in}{1.016610in}}%
\pgfpathlineto{\pgfqpoint{2.689420in}{1.015380in}}%
\pgfpathlineto{\pgfqpoint{2.691957in}{1.014769in}}%
\pgfpathlineto{\pgfqpoint{2.694495in}{1.014779in}}%
\pgfpathlineto{\pgfqpoint{2.697033in}{1.015415in}}%
\pgfpathlineto{\pgfqpoint{2.699570in}{1.016679in}}%
\pgfpathlineto{\pgfqpoint{2.702108in}{1.018573in}}%
\pgfpathlineto{\pgfqpoint{2.704646in}{1.021099in}}%
\pgfpathlineto{\pgfqpoint{2.707183in}{1.024260in}}%
\pgfpathlineto{\pgfqpoint{2.709721in}{1.028056in}}%
\pgfpathlineto{\pgfqpoint{2.712258in}{1.032488in}}%
\pgfpathlineto{\pgfqpoint{2.714796in}{1.037558in}}%
\pgfpathlineto{\pgfqpoint{2.717334in}{1.043265in}}%
\pgfpathlineto{\pgfqpoint{2.719871in}{1.049609in}}%
\pgfpathlineto{\pgfqpoint{2.722409in}{1.056590in}}%
\pgfpathlineto{\pgfqpoint{2.724947in}{1.064208in}}%
\pgfpathlineto{\pgfqpoint{2.727484in}{1.072460in}}%
\pgfpathlineto{\pgfqpoint{2.730022in}{1.081345in}}%
\pgfpathlineto{\pgfqpoint{2.732559in}{1.090862in}}%
\pgfpathlineto{\pgfqpoint{2.737635in}{1.111780in}}%
\pgfpathlineto{\pgfqpoint{2.742710in}{1.135191in}}%
\pgfpathlineto{\pgfqpoint{2.747785in}{1.161064in}}%
\pgfpathlineto{\pgfqpoint{2.752861in}{1.189364in}}%
\pgfpathlineto{\pgfqpoint{2.757936in}{1.220049in}}%
\pgfpathlineto{\pgfqpoint{2.763011in}{1.253071in}}%
\pgfpathlineto{\pgfqpoint{2.768086in}{1.288374in}}%
\pgfpathlineto{\pgfqpoint{2.773162in}{1.325898in}}%
\pgfpathlineto{\pgfqpoint{2.778237in}{1.365575in}}%
\pgfpathlineto{\pgfqpoint{2.783312in}{1.407333in}}%
\pgfpathlineto{\pgfqpoint{2.788387in}{1.451091in}}%
\pgfpathlineto{\pgfqpoint{2.796000in}{1.520290in}}%
\pgfpathlineto{\pgfqpoint{2.803613in}{1.593484in}}%
\pgfpathlineto{\pgfqpoint{2.811226in}{1.670329in}}%
\pgfpathlineto{\pgfqpoint{2.818839in}{1.750456in}}%
\pgfpathlineto{\pgfqpoint{2.826452in}{1.833469in}}%
\pgfpathlineto{\pgfqpoint{2.836602in}{1.947912in}}%
\pgfpathlineto{\pgfqpoint{2.849291in}{2.095503in}}%
\pgfpathlineto{\pgfqpoint{2.869592in}{2.336879in}}%
\pgfpathlineto{\pgfqpoint{2.887355in}{2.546744in}}%
\pgfpathlineto{\pgfqpoint{2.897506in}{2.663374in}}%
\pgfpathlineto{\pgfqpoint{2.907656in}{2.776022in}}%
\pgfpathlineto{\pgfqpoint{2.915269in}{2.857136in}}%
\pgfpathlineto{\pgfqpoint{2.922882in}{2.934776in}}%
\pgfpathlineto{\pgfqpoint{2.930495in}{3.008429in}}%
\pgfpathlineto{\pgfqpoint{2.935570in}{3.055070in}}%
\pgfpathlineto{\pgfqpoint{2.940645in}{3.099574in}}%
\pgfpathlineto{\pgfqpoint{2.945721in}{3.141802in}}%
\pgfpathlineto{\pgfqpoint{2.950796in}{3.181620in}}%
\pgfpathlineto{\pgfqpoint{2.955871in}{3.218898in}}%
\pgfpathlineto{\pgfqpoint{2.960946in}{3.253512in}}%
\pgfpathlineto{\pgfqpoint{2.966022in}{3.285344in}}%
\pgfpathlineto{\pgfqpoint{2.971097in}{3.314282in}}%
\pgfpathlineto{\pgfqpoint{2.976172in}{3.340222in}}%
\pgfpathlineto{\pgfqpoint{2.978710in}{3.352036in}}%
\pgfpathlineto{\pgfqpoint{2.981247in}{3.363064in}}%
\pgfpathlineto{\pgfqpoint{2.983785in}{3.373296in}}%
\pgfpathlineto{\pgfqpoint{2.986323in}{3.382720in}}%
\pgfpathlineto{\pgfqpoint{2.988860in}{3.391327in}}%
\pgfpathlineto{\pgfqpoint{2.991398in}{3.399108in}}%
\pgfpathlineto{\pgfqpoint{2.993936in}{3.406052in}}%
\pgfpathlineto{\pgfqpoint{2.996473in}{3.412153in}}%
\pgfpathlineto{\pgfqpoint{2.999011in}{3.417401in}}%
\pgfpathlineto{\pgfqpoint{3.001549in}{3.421790in}}%
\pgfpathlineto{\pgfqpoint{3.004086in}{3.425314in}}%
\pgfpathlineto{\pgfqpoint{3.006624in}{3.427965in}}%
\pgfpathlineto{\pgfqpoint{3.009161in}{3.429739in}}%
\pgfpathlineto{\pgfqpoint{3.011699in}{3.430630in}}%
\pgfpathlineto{\pgfqpoint{3.014237in}{3.430635in}}%
\pgfpathlineto{\pgfqpoint{3.016774in}{3.429749in}}%
\pgfpathlineto{\pgfqpoint{3.019312in}{3.427970in}}%
\pgfpathlineto{\pgfqpoint{3.021850in}{3.425295in}}%
\pgfpathlineto{\pgfqpoint{3.024387in}{3.421723in}}%
\pgfpathlineto{\pgfqpoint{3.026925in}{3.417252in}}%
\pgfpathlineto{\pgfqpoint{3.029462in}{3.411883in}}%
\pgfpathlineto{\pgfqpoint{3.032000in}{3.405614in}}%
\pgfpathlineto{\pgfqpoint{3.034538in}{3.398448in}}%
\pgfpathlineto{\pgfqpoint{3.037075in}{3.390386in}}%
\pgfpathlineto{\pgfqpoint{3.039613in}{3.381429in}}%
\pgfpathlineto{\pgfqpoint{3.042151in}{3.371582in}}%
\pgfpathlineto{\pgfqpoint{3.044688in}{3.360847in}}%
\pgfpathlineto{\pgfqpoint{3.047226in}{3.349229in}}%
\pgfpathlineto{\pgfqpoint{3.049764in}{3.336733in}}%
\pgfpathlineto{\pgfqpoint{3.052301in}{3.323365in}}%
\pgfpathlineto{\pgfqpoint{3.057376in}{3.294038in}}%
\pgfpathlineto{\pgfqpoint{3.062452in}{3.261308in}}%
\pgfpathlineto{\pgfqpoint{3.067527in}{3.225245in}}%
\pgfpathlineto{\pgfqpoint{3.072602in}{3.185930in}}%
\pgfpathlineto{\pgfqpoint{3.077677in}{3.143458in}}%
\pgfpathlineto{\pgfqpoint{3.082753in}{3.097930in}}%
\pgfpathlineto{\pgfqpoint{3.087828in}{3.049462in}}%
\pgfpathlineto{\pgfqpoint{3.092903in}{2.998178in}}%
\pgfpathlineto{\pgfqpoint{3.097979in}{2.944213in}}%
\pgfpathlineto{\pgfqpoint{3.103054in}{2.887713in}}%
\pgfpathlineto{\pgfqpoint{3.110667in}{2.798548in}}%
\pgfpathlineto{\pgfqpoint{3.118280in}{2.704589in}}%
\pgfpathlineto{\pgfqpoint{3.125892in}{2.606442in}}%
\pgfpathlineto{\pgfqpoint{3.133505in}{2.504753in}}%
\pgfpathlineto{\pgfqpoint{3.143656in}{2.364843in}}%
\pgfpathlineto{\pgfqpoint{3.158882in}{2.149219in}}%
\pgfpathlineto{\pgfqpoint{3.181720in}{1.825021in}}%
\pgfpathlineto{\pgfqpoint{3.191871in}{1.685501in}}%
\pgfpathlineto{\pgfqpoint{3.199484in}{1.584372in}}%
\pgfpathlineto{\pgfqpoint{3.207097in}{1.487136in}}%
\pgfpathlineto{\pgfqpoint{3.214710in}{1.394567in}}%
\pgfpathlineto{\pgfqpoint{3.219785in}{1.335822in}}%
\pgfpathlineto{\pgfqpoint{3.224860in}{1.279704in}}%
\pgfpathlineto{\pgfqpoint{3.229935in}{1.226422in}}%
\pgfpathlineto{\pgfqpoint{3.235011in}{1.176179in}}%
\pgfpathlineto{\pgfqpoint{3.240086in}{1.129169in}}%
\pgfpathlineto{\pgfqpoint{3.245161in}{1.085577in}}%
\pgfpathlineto{\pgfqpoint{3.250236in}{1.045578in}}%
\pgfpathlineto{\pgfqpoint{3.255312in}{1.009337in}}%
\pgfpathlineto{\pgfqpoint{3.260387in}{0.977007in}}%
\pgfpathlineto{\pgfqpoint{3.262925in}{0.962353in}}%
\pgfpathlineto{\pgfqpoint{3.265462in}{0.948729in}}%
\pgfpathlineto{\pgfqpoint{3.268000in}{0.936150in}}%
\pgfpathlineto{\pgfqpoint{3.270538in}{0.924630in}}%
\pgfpathlineto{\pgfqpoint{3.273075in}{0.914185in}}%
\pgfpathlineto{\pgfqpoint{3.275613in}{0.904826in}}%
\pgfpathlineto{\pgfqpoint{3.278150in}{0.896567in}}%
\pgfpathlineto{\pgfqpoint{3.280688in}{0.889418in}}%
\pgfpathlineto{\pgfqpoint{3.283226in}{0.883389in}}%
\pgfpathlineto{\pgfqpoint{3.285763in}{0.878491in}}%
\pgfpathlineto{\pgfqpoint{3.288301in}{0.874731in}}%
\pgfpathlineto{\pgfqpoint{3.290839in}{0.872116in}}%
\pgfpathlineto{\pgfqpoint{3.293376in}{0.870653in}}%
\pgfpathlineto{\pgfqpoint{3.295914in}{0.870349in}}%
\pgfpathlineto{\pgfqpoint{3.298451in}{0.871205in}}%
\pgfpathlineto{\pgfqpoint{3.300989in}{0.873228in}}%
\pgfpathlineto{\pgfqpoint{3.303527in}{0.876417in}}%
\pgfpathlineto{\pgfqpoint{3.306064in}{0.880776in}}%
\pgfpathlineto{\pgfqpoint{3.308602in}{0.886304in}}%
\pgfpathlineto{\pgfqpoint{3.311140in}{0.893000in}}%
\pgfpathlineto{\pgfqpoint{3.313677in}{0.900862in}}%
\pgfpathlineto{\pgfqpoint{3.316215in}{0.909887in}}%
\pgfpathlineto{\pgfqpoint{3.318753in}{0.920072in}}%
\pgfpathlineto{\pgfqpoint{3.321290in}{0.931410in}}%
\pgfpathlineto{\pgfqpoint{3.323828in}{0.943896in}}%
\pgfpathlineto{\pgfqpoint{3.326365in}{0.957523in}}%
\pgfpathlineto{\pgfqpoint{3.328903in}{0.972282in}}%
\pgfpathlineto{\pgfqpoint{3.331441in}{0.988163in}}%
\pgfpathlineto{\pgfqpoint{3.333978in}{1.005155in}}%
\pgfpathlineto{\pgfqpoint{3.339054in}{1.042428in}}%
\pgfpathlineto{\pgfqpoint{3.344129in}{1.083992in}}%
\pgfpathlineto{\pgfqpoint{3.349204in}{1.129724in}}%
\pgfpathlineto{\pgfqpoint{3.354279in}{1.179483in}}%
\pgfpathlineto{\pgfqpoint{3.359355in}{1.233112in}}%
\pgfpathlineto{\pgfqpoint{3.364430in}{1.290435in}}%
\pgfpathlineto{\pgfqpoint{3.369505in}{1.351264in}}%
\pgfpathlineto{\pgfqpoint{3.374580in}{1.415393in}}%
\pgfpathlineto{\pgfqpoint{3.379656in}{1.482601in}}%
\pgfpathlineto{\pgfqpoint{3.387269in}{1.588672in}}%
\pgfpathlineto{\pgfqpoint{3.394882in}{1.700297in}}%
\pgfpathlineto{\pgfqpoint{3.402494in}{1.816567in}}%
\pgfpathlineto{\pgfqpoint{3.412645in}{1.977147in}}%
\pgfpathlineto{\pgfqpoint{3.427871in}{2.224977in}}%
\pgfpathlineto{\pgfqpoint{3.445634in}{2.514356in}}%
\pgfpathlineto{\pgfqpoint{3.455785in}{2.674712in}}%
\pgfpathlineto{\pgfqpoint{3.463398in}{2.790574in}}%
\pgfpathlineto{\pgfqpoint{3.471011in}{2.901449in}}%
\pgfpathlineto{\pgfqpoint{3.478623in}{3.006277in}}%
\pgfpathlineto{\pgfqpoint{3.483699in}{3.072298in}}%
\pgfpathlineto{\pgfqpoint{3.488774in}{3.134888in}}%
\pgfpathlineto{\pgfqpoint{3.493849in}{3.193767in}}%
\pgfpathlineto{\pgfqpoint{3.498924in}{3.248670in}}%
\pgfpathlineto{\pgfqpoint{3.504000in}{3.299344in}}%
\pgfpathlineto{\pgfqpoint{3.509075in}{3.345553in}}%
\pgfpathlineto{\pgfqpoint{3.514150in}{3.387074in}}%
\pgfpathlineto{\pgfqpoint{3.516688in}{3.406014in}}%
\pgfpathlineto{\pgfqpoint{3.519226in}{3.423706in}}%
\pgfpathlineto{\pgfqpoint{3.521763in}{3.440130in}}%
\pgfpathlineto{\pgfqpoint{3.524301in}{3.455264in}}%
\pgfpathlineto{\pgfqpoint{3.526838in}{3.469088in}}%
\pgfpathlineto{\pgfqpoint{3.529376in}{3.481582in}}%
\pgfpathlineto{\pgfqpoint{3.531914in}{3.492731in}}%
\pgfpathlineto{\pgfqpoint{3.534451in}{3.502517in}}%
\pgfpathlineto{\pgfqpoint{3.536989in}{3.510927in}}%
\pgfpathlineto{\pgfqpoint{3.539527in}{3.517946in}}%
\pgfpathlineto{\pgfqpoint{3.542064in}{3.523563in}}%
\pgfpathlineto{\pgfqpoint{3.544602in}{3.527768in}}%
\pgfpathlineto{\pgfqpoint{3.547139in}{3.530552in}}%
\pgfpathlineto{\pgfqpoint{3.549677in}{3.531906in}}%
\pgfpathlineto{\pgfqpoint{3.552215in}{3.531826in}}%
\pgfpathlineto{\pgfqpoint{3.554752in}{3.530307in}}%
\pgfpathlineto{\pgfqpoint{3.557290in}{3.527346in}}%
\pgfpathlineto{\pgfqpoint{3.559828in}{3.522941in}}%
\pgfpathlineto{\pgfqpoint{3.562365in}{3.517093in}}%
\pgfpathlineto{\pgfqpoint{3.564903in}{3.509804in}}%
\pgfpathlineto{\pgfqpoint{3.567441in}{3.501077in}}%
\pgfpathlineto{\pgfqpoint{3.569978in}{3.490916in}}%
\pgfpathlineto{\pgfqpoint{3.572516in}{3.479329in}}%
\pgfpathlineto{\pgfqpoint{3.575053in}{3.466324in}}%
\pgfpathlineto{\pgfqpoint{3.577591in}{3.451909in}}%
\pgfpathlineto{\pgfqpoint{3.580129in}{3.436097in}}%
\pgfpathlineto{\pgfqpoint{3.582666in}{3.418901in}}%
\pgfpathlineto{\pgfqpoint{3.585204in}{3.400334in}}%
\pgfpathlineto{\pgfqpoint{3.587742in}{3.380413in}}%
\pgfpathlineto{\pgfqpoint{3.592817in}{3.336581in}}%
\pgfpathlineto{\pgfqpoint{3.597892in}{3.287564in}}%
\pgfpathlineto{\pgfqpoint{3.602967in}{3.233546in}}%
\pgfpathlineto{\pgfqpoint{3.608043in}{3.174736in}}%
\pgfpathlineto{\pgfqpoint{3.613118in}{3.111364in}}%
\pgfpathlineto{\pgfqpoint{3.618193in}{3.043687in}}%
\pgfpathlineto{\pgfqpoint{3.623268in}{2.971979in}}%
\pgfpathlineto{\pgfqpoint{3.628344in}{2.896538in}}%
\pgfpathlineto{\pgfqpoint{3.635957in}{2.777074in}}%
\pgfpathlineto{\pgfqpoint{3.643570in}{2.651072in}}%
\pgfpathlineto{\pgfqpoint{3.651182in}{2.519763in}}%
\pgfpathlineto{\pgfqpoint{3.661333in}{2.338686in}}%
\pgfpathlineto{\pgfqpoint{3.696860in}{1.696766in}}%
\pgfpathlineto{\pgfqpoint{3.704473in}{1.567136in}}%
\pgfpathlineto{\pgfqpoint{3.712086in}{1.443498in}}%
\pgfpathlineto{\pgfqpoint{3.717161in}{1.365083in}}%
\pgfpathlineto{\pgfqpoint{3.722236in}{1.290337in}}%
\pgfpathlineto{\pgfqpoint{3.727311in}{1.219638in}}%
\pgfpathlineto{\pgfqpoint{3.732387in}{1.153349in}}%
\pgfpathlineto{\pgfqpoint{3.737462in}{1.091814in}}%
\pgfpathlineto{\pgfqpoint{3.742537in}{1.035359in}}%
\pgfpathlineto{\pgfqpoint{3.747612in}{0.984286in}}%
\pgfpathlineto{\pgfqpoint{3.752688in}{0.938875in}}%
\pgfpathlineto{\pgfqpoint{3.755225in}{0.918373in}}%
\pgfpathlineto{\pgfqpoint{3.757763in}{0.899379in}}%
\pgfpathlineto{\pgfqpoint{3.760301in}{0.881921in}}%
\pgfpathlineto{\pgfqpoint{3.762838in}{0.866026in}}%
\pgfpathlineto{\pgfqpoint{3.765376in}{0.851716in}}%
\pgfpathlineto{\pgfqpoint{3.767914in}{0.839014in}}%
\pgfpathlineto{\pgfqpoint{3.770451in}{0.827940in}}%
\pgfpathlineto{\pgfqpoint{3.772989in}{0.818513in}}%
\pgfpathlineto{\pgfqpoint{3.775526in}{0.810748in}}%
\pgfpathlineto{\pgfqpoint{3.778064in}{0.804661in}}%
\pgfpathlineto{\pgfqpoint{3.780602in}{0.800263in}}%
\pgfpathlineto{\pgfqpoint{3.783139in}{0.797565in}}%
\pgfpathlineto{\pgfqpoint{3.785677in}{0.796575in}}%
\pgfpathlineto{\pgfqpoint{3.788215in}{0.797299in}}%
\pgfpathlineto{\pgfqpoint{3.790752in}{0.799741in}}%
\pgfpathlineto{\pgfqpoint{3.793290in}{0.803903in}}%
\pgfpathlineto{\pgfqpoint{3.795827in}{0.809784in}}%
\pgfpathlineto{\pgfqpoint{3.798365in}{0.817382in}}%
\pgfpathlineto{\pgfqpoint{3.800903in}{0.826692in}}%
\pgfpathlineto{\pgfqpoint{3.803440in}{0.837708in}}%
\pgfpathlineto{\pgfqpoint{3.805978in}{0.850421in}}%
\pgfpathlineto{\pgfqpoint{3.808516in}{0.864818in}}%
\pgfpathlineto{\pgfqpoint{3.811053in}{0.880887in}}%
\pgfpathlineto{\pgfqpoint{3.813591in}{0.898613in}}%
\pgfpathlineto{\pgfqpoint{3.816129in}{0.917977in}}%
\pgfpathlineto{\pgfqpoint{3.818666in}{0.938959in}}%
\pgfpathlineto{\pgfqpoint{3.821204in}{0.961538in}}%
\pgfpathlineto{\pgfqpoint{3.823741in}{0.985689in}}%
\pgfpathlineto{\pgfqpoint{3.828817in}{1.038599in}}%
\pgfpathlineto{\pgfqpoint{3.833892in}{1.097455in}}%
\pgfpathlineto{\pgfqpoint{3.838967in}{1.161987in}}%
\pgfpathlineto{\pgfqpoint{3.844042in}{1.231895in}}%
\pgfpathlineto{\pgfqpoint{3.849118in}{1.306849in}}%
\pgfpathlineto{\pgfqpoint{3.854193in}{1.386489in}}%
\pgfpathlineto{\pgfqpoint{3.859268in}{1.470426in}}%
\pgfpathlineto{\pgfqpoint{3.866881in}{1.603481in}}%
\pgfpathlineto{\pgfqpoint{3.874494in}{1.743776in}}%
\pgfpathlineto{\pgfqpoint{3.882107in}{1.889713in}}%
\pgfpathlineto{\pgfqpoint{3.894795in}{2.140864in}}%
\pgfpathlineto{\pgfqpoint{3.915096in}{2.544922in}}%
\pgfpathlineto{\pgfqpoint{3.922709in}{2.691180in}}%
\pgfpathlineto{\pgfqpoint{3.930322in}{2.831735in}}%
\pgfpathlineto{\pgfqpoint{3.937935in}{2.964822in}}%
\pgfpathlineto{\pgfqpoint{3.943010in}{3.048556in}}%
\pgfpathlineto{\pgfqpoint{3.948085in}{3.127736in}}%
\pgfpathlineto{\pgfqpoint{3.953161in}{3.201902in}}%
\pgfpathlineto{\pgfqpoint{3.958236in}{3.270618in}}%
\pgfpathlineto{\pgfqpoint{3.963311in}{3.333477in}}%
\pgfpathlineto{\pgfqpoint{3.968386in}{3.390098in}}%
\pgfpathlineto{\pgfqpoint{3.970924in}{3.415959in}}%
\pgfpathlineto{\pgfqpoint{3.973462in}{3.440133in}}%
\pgfpathlineto{\pgfqpoint{3.975999in}{3.462582in}}%
\pgfpathlineto{\pgfqpoint{3.978537in}{3.483270in}}%
\pgfpathlineto{\pgfqpoint{3.981075in}{3.502163in}}%
\pgfpathlineto{\pgfqpoint{3.983612in}{3.519230in}}%
\pgfpathlineto{\pgfqpoint{3.986150in}{3.534442in}}%
\pgfpathlineto{\pgfqpoint{3.988688in}{3.547774in}}%
\pgfpathlineto{\pgfqpoint{3.991225in}{3.559201in}}%
\pgfpathlineto{\pgfqpoint{3.993763in}{3.568702in}}%
\pgfpathlineto{\pgfqpoint{3.996300in}{3.576260in}}%
\pgfpathlineto{\pgfqpoint{3.998838in}{3.581857in}}%
\pgfpathlineto{\pgfqpoint{4.001376in}{3.585482in}}%
\pgfpathlineto{\pgfqpoint{4.003913in}{3.587124in}}%
\pgfpathlineto{\pgfqpoint{4.006451in}{3.586775in}}%
\pgfpathlineto{\pgfqpoint{4.008989in}{3.584431in}}%
\pgfpathlineto{\pgfqpoint{4.011526in}{3.580090in}}%
\pgfpathlineto{\pgfqpoint{4.014064in}{3.573753in}}%
\pgfpathlineto{\pgfqpoint{4.016601in}{3.565423in}}%
\pgfpathlineto{\pgfqpoint{4.019139in}{3.555108in}}%
\pgfpathlineto{\pgfqpoint{4.021677in}{3.542817in}}%
\pgfpathlineto{\pgfqpoint{4.024214in}{3.528563in}}%
\pgfpathlineto{\pgfqpoint{4.026752in}{3.512361in}}%
\pgfpathlineto{\pgfqpoint{4.029290in}{3.494229in}}%
\pgfpathlineto{\pgfqpoint{4.031827in}{3.474189in}}%
\pgfpathlineto{\pgfqpoint{4.034365in}{3.452264in}}%
\pgfpathlineto{\pgfqpoint{4.036903in}{3.428482in}}%
\pgfpathlineto{\pgfqpoint{4.039440in}{3.402871in}}%
\pgfpathlineto{\pgfqpoint{4.041978in}{3.375465in}}%
\pgfpathlineto{\pgfqpoint{4.047053in}{3.315409in}}%
\pgfpathlineto{\pgfqpoint{4.052128in}{3.248628in}}%
\pgfpathlineto{\pgfqpoint{4.057204in}{3.175476in}}%
\pgfpathlineto{\pgfqpoint{4.062279in}{3.096350in}}%
\pgfpathlineto{\pgfqpoint{4.067354in}{3.011683in}}%
\pgfpathlineto{\pgfqpoint{4.072429in}{2.921947in}}%
\pgfpathlineto{\pgfqpoint{4.077505in}{2.827646in}}%
\pgfpathlineto{\pgfqpoint{4.085118in}{2.678814in}}%
\pgfpathlineto{\pgfqpoint{4.092730in}{2.522846in}}%
\pgfpathlineto{\pgfqpoint{4.102881in}{2.307333in}}%
\pgfpathlineto{\pgfqpoint{4.128257in}{1.762781in}}%
\pgfpathlineto{\pgfqpoint{4.135870in}{1.606944in}}%
\pgfpathlineto{\pgfqpoint{4.143483in}{1.458507in}}%
\pgfpathlineto{\pgfqpoint{4.148558in}{1.364706in}}%
\pgfpathlineto{\pgfqpoint{4.153634in}{1.275732in}}%
\pgfpathlineto{\pgfqpoint{4.158709in}{1.192160in}}%
\pgfpathlineto{\pgfqpoint{4.163784in}{1.114535in}}%
\pgfpathlineto{\pgfqpoint{4.168859in}{1.043369in}}%
\pgfpathlineto{\pgfqpoint{4.173935in}{0.979137in}}%
\pgfpathlineto{\pgfqpoint{4.176472in}{0.949759in}}%
\pgfpathlineto{\pgfqpoint{4.179010in}{0.922276in}}%
\pgfpathlineto{\pgfqpoint{4.181548in}{0.896733in}}%
\pgfpathlineto{\pgfqpoint{4.184085in}{0.873177in}}%
\pgfpathlineto{\pgfqpoint{4.186623in}{0.851649in}}%
\pgfpathlineto{\pgfqpoint{4.189160in}{0.832188in}}%
\pgfpathlineto{\pgfqpoint{4.191698in}{0.814830in}}%
\pgfpathlineto{\pgfqpoint{4.194236in}{0.799607in}}%
\pgfpathlineto{\pgfqpoint{4.196773in}{0.786549in}}%
\pgfpathlineto{\pgfqpoint{4.199311in}{0.775681in}}%
\pgfpathlineto{\pgfqpoint{4.201849in}{0.767026in}}%
\pgfpathlineto{\pgfqpoint{4.204386in}{0.760603in}}%
\pgfpathlineto{\pgfqpoint{4.206924in}{0.756427in}}%
\pgfpathlineto{\pgfqpoint{4.209462in}{0.754511in}}%
\pgfpathlineto{\pgfqpoint{4.211999in}{0.754864in}}%
\pgfpathlineto{\pgfqpoint{4.214537in}{0.757490in}}%
\pgfpathlineto{\pgfqpoint{4.217074in}{0.762390in}}%
\pgfpathlineto{\pgfqpoint{4.219612in}{0.769562in}}%
\pgfpathlineto{\pgfqpoint{4.222150in}{0.778999in}}%
\pgfpathlineto{\pgfqpoint{4.224687in}{0.790694in}}%
\pgfpathlineto{\pgfqpoint{4.227225in}{0.804631in}}%
\pgfpathlineto{\pgfqpoint{4.229763in}{0.820793in}}%
\pgfpathlineto{\pgfqpoint{4.232300in}{0.839161in}}%
\pgfpathlineto{\pgfqpoint{4.234838in}{0.859709in}}%
\pgfpathlineto{\pgfqpoint{4.237376in}{0.882409in}}%
\pgfpathlineto{\pgfqpoint{4.239913in}{0.907231in}}%
\pgfpathlineto{\pgfqpoint{4.242451in}{0.934138in}}%
\pgfpathlineto{\pgfqpoint{4.244988in}{0.963092in}}%
\pgfpathlineto{\pgfqpoint{4.247526in}{0.994051in}}%
\pgfpathlineto{\pgfqpoint{4.252601in}{1.061797in}}%
\pgfpathlineto{\pgfqpoint{4.257677in}{1.136970in}}%
\pgfpathlineto{\pgfqpoint{4.262752in}{1.219113in}}%
\pgfpathlineto{\pgfqpoint{4.267827in}{1.307719in}}%
\pgfpathlineto{\pgfqpoint{4.272902in}{1.402235in}}%
\pgfpathlineto{\pgfqpoint{4.277978in}{1.502064in}}%
\pgfpathlineto{\pgfqpoint{4.285591in}{1.660368in}}%
\pgfpathlineto{\pgfqpoint{4.293203in}{1.826891in}}%
\pgfpathlineto{\pgfqpoint{4.303354in}{2.057465in}}%
\pgfpathlineto{\pgfqpoint{4.326193in}{2.581861in}}%
\pgfpathlineto{\pgfqpoint{4.333806in}{2.748912in}}%
\pgfpathlineto{\pgfqpoint{4.341418in}{2.907707in}}%
\pgfpathlineto{\pgfqpoint{4.346494in}{3.007741in}}%
\pgfpathlineto{\pgfqpoint{4.351569in}{3.102280in}}%
\pgfpathlineto{\pgfqpoint{4.356644in}{3.190646in}}%
\pgfpathlineto{\pgfqpoint{4.361719in}{3.272201in}}%
\pgfpathlineto{\pgfqpoint{4.366795in}{3.346348in}}%
\pgfpathlineto{\pgfqpoint{4.371870in}{3.412539in}}%
\pgfpathlineto{\pgfqpoint{4.374408in}{3.442493in}}%
\pgfpathlineto{\pgfqpoint{4.376945in}{3.470276in}}%
\pgfpathlineto{\pgfqpoint{4.379483in}{3.495834in}}%
\pgfpathlineto{\pgfqpoint{4.382021in}{3.519117in}}%
\pgfpathlineto{\pgfqpoint{4.384558in}{3.540080in}}%
\pgfpathlineto{\pgfqpoint{4.387096in}{3.558680in}}%
\pgfpathlineto{\pgfqpoint{4.389633in}{3.574880in}}%
\pgfpathlineto{\pgfqpoint{4.392171in}{3.588646in}}%
\pgfpathlineto{\pgfqpoint{4.394709in}{3.599947in}}%
\pgfpathlineto{\pgfqpoint{4.397246in}{3.608759in}}%
\pgfpathlineto{\pgfqpoint{4.399784in}{3.615060in}}%
\pgfpathlineto{\pgfqpoint{4.402322in}{3.618833in}}%
\pgfpathlineto{\pgfqpoint{4.404859in}{3.620066in}}%
\pgfpathlineto{\pgfqpoint{4.407397in}{3.618751in}}%
\pgfpathlineto{\pgfqpoint{4.409935in}{3.614886in}}%
\pgfpathlineto{\pgfqpoint{4.412472in}{3.608470in}}%
\pgfpathlineto{\pgfqpoint{4.415010in}{3.599509in}}%
\pgfpathlineto{\pgfqpoint{4.417547in}{3.588015in}}%
\pgfpathlineto{\pgfqpoint{4.420085in}{3.574001in}}%
\pgfpathlineto{\pgfqpoint{4.422623in}{3.557487in}}%
\pgfpathlineto{\pgfqpoint{4.425160in}{3.538498in}}%
\pgfpathlineto{\pgfqpoint{4.427698in}{3.517060in}}%
\pgfpathlineto{\pgfqpoint{4.430236in}{3.493209in}}%
\pgfpathlineto{\pgfqpoint{4.432773in}{3.466980in}}%
\pgfpathlineto{\pgfqpoint{4.435311in}{3.438416in}}%
\pgfpathlineto{\pgfqpoint{4.437848in}{3.407563in}}%
\pgfpathlineto{\pgfqpoint{4.440386in}{3.374472in}}%
\pgfpathlineto{\pgfqpoint{4.445461in}{3.301798in}}%
\pgfpathlineto{\pgfqpoint{4.450537in}{3.220881in}}%
\pgfpathlineto{\pgfqpoint{4.455612in}{3.132273in}}%
\pgfpathlineto{\pgfqpoint{4.460687in}{3.036583in}}%
\pgfpathlineto{\pgfqpoint{4.465762in}{2.934480in}}%
\pgfpathlineto{\pgfqpoint{4.470838in}{2.826682in}}%
\pgfpathlineto{\pgfqpoint{4.478451in}{2.655994in}}%
\pgfpathlineto{\pgfqpoint{4.486063in}{2.476981in}}%
\pgfpathlineto{\pgfqpoint{4.498752in}{2.168119in}}%
\pgfpathlineto{\pgfqpoint{4.513977in}{1.797730in}}%
\pgfpathlineto{\pgfqpoint{4.521590in}{1.619926in}}%
\pgfpathlineto{\pgfqpoint{4.529203in}{1.451140in}}%
\pgfpathlineto{\pgfqpoint{4.534279in}{1.345078in}}%
\pgfpathlineto{\pgfqpoint{4.539354in}{1.245155in}}%
\pgfpathlineto{\pgfqpoint{4.544429in}{1.152158in}}%
\pgfpathlineto{\pgfqpoint{4.549504in}{1.066824in}}%
\pgfpathlineto{\pgfqpoint{4.554580in}{0.989837in}}%
\pgfpathlineto{\pgfqpoint{4.557117in}{0.954673in}}%
\pgfpathlineto{\pgfqpoint{4.559655in}{0.921825in}}%
\pgfpathlineto{\pgfqpoint{4.562192in}{0.891361in}}%
\pgfpathlineto{\pgfqpoint{4.564730in}{0.863346in}}%
\pgfpathlineto{\pgfqpoint{4.567268in}{0.837838in}}%
\pgfpathlineto{\pgfqpoint{4.569805in}{0.814891in}}%
\pgfpathlineto{\pgfqpoint{4.572343in}{0.794556in}}%
\pgfpathlineto{\pgfqpoint{4.574881in}{0.776878in}}%
\pgfpathlineto{\pgfqpoint{4.577418in}{0.761896in}}%
\pgfpathlineto{\pgfqpoint{4.579956in}{0.749644in}}%
\pgfpathlineto{\pgfqpoint{4.582494in}{0.740153in}}%
\pgfpathlineto{\pgfqpoint{4.585031in}{0.733447in}}%
\pgfpathlineto{\pgfqpoint{4.587569in}{0.729545in}}%
\pgfpathlineto{\pgfqpoint{4.590106in}{0.728459in}}%
\pgfpathlineto{\pgfqpoint{4.592644in}{0.730199in}}%
\pgfpathlineto{\pgfqpoint{4.595182in}{0.734767in}}%
\pgfpathlineto{\pgfqpoint{4.597719in}{0.742160in}}%
\pgfpathlineto{\pgfqpoint{4.600257in}{0.752370in}}%
\pgfpathlineto{\pgfqpoint{4.602795in}{0.765381in}}%
\pgfpathlineto{\pgfqpoint{4.605332in}{0.781176in}}%
\pgfpathlineto{\pgfqpoint{4.607870in}{0.799728in}}%
\pgfpathlineto{\pgfqpoint{4.610407in}{0.821006in}}%
\pgfpathlineto{\pgfqpoint{4.612945in}{0.844976in}}%
\pgfpathlineto{\pgfqpoint{4.615483in}{0.871595in}}%
\pgfpathlineto{\pgfqpoint{4.618020in}{0.900815in}}%
\pgfpathlineto{\pgfqpoint{4.620558in}{0.932586in}}%
\pgfpathlineto{\pgfqpoint{4.623096in}{0.966849in}}%
\pgfpathlineto{\pgfqpoint{4.625633in}{1.003542in}}%
\pgfpathlineto{\pgfqpoint{4.630709in}{1.083940in}}%
\pgfpathlineto{\pgfqpoint{4.635784in}{1.173181in}}%
\pgfpathlineto{\pgfqpoint{4.640859in}{1.270590in}}%
\pgfpathlineto{\pgfqpoint{4.645934in}{1.375422in}}%
\pgfpathlineto{\pgfqpoint{4.651010in}{1.486868in}}%
\pgfpathlineto{\pgfqpoint{4.656085in}{1.604059in}}%
\pgfpathlineto{\pgfqpoint{4.663698in}{1.788594in}}%
\pgfpathlineto{\pgfqpoint{4.673848in}{2.045804in}}%
\pgfpathlineto{\pgfqpoint{4.694149in}{2.567241in}}%
\pgfpathlineto{\pgfqpoint{4.701762in}{2.754202in}}%
\pgfpathlineto{\pgfqpoint{4.706838in}{2.873396in}}%
\pgfpathlineto{\pgfqpoint{4.711913in}{2.987060in}}%
\pgfpathlineto{\pgfqpoint{4.716988in}{3.094228in}}%
\pgfpathlineto{\pgfqpoint{4.722063in}{3.193987in}}%
\pgfpathlineto{\pgfqpoint{4.727139in}{3.285474in}}%
\pgfpathlineto{\pgfqpoint{4.732214in}{3.367895in}}%
\pgfpathlineto{\pgfqpoint{4.734751in}{3.405476in}}%
\pgfpathlineto{\pgfqpoint{4.737289in}{3.440523in}}%
\pgfpathlineto{\pgfqpoint{4.739827in}{3.472959in}}%
\pgfpathlineto{\pgfqpoint{4.742364in}{3.502710in}}%
\pgfpathlineto{\pgfqpoint{4.744902in}{3.529709in}}%
\pgfpathlineto{\pgfqpoint{4.747440in}{3.553892in}}%
\pgfpathlineto{\pgfqpoint{4.749977in}{3.575203in}}%
\pgfpathlineto{\pgfqpoint{4.752515in}{3.593592in}}%
\pgfpathlineto{\pgfqpoint{4.755053in}{3.609014in}}%
\pgfpathlineto{\pgfqpoint{4.757590in}{3.621430in}}%
\pgfpathlineto{\pgfqpoint{4.760128in}{3.630808in}}%
\pgfpathlineto{\pgfqpoint{4.762665in}{3.637123in}}%
\pgfpathlineto{\pgfqpoint{4.765203in}{3.640354in}}%
\pgfpathlineto{\pgfqpoint{4.767741in}{3.640488in}}%
\pgfpathlineto{\pgfqpoint{4.770278in}{3.637519in}}%
\pgfpathlineto{\pgfqpoint{4.772816in}{3.631448in}}%
\pgfpathlineto{\pgfqpoint{4.775354in}{3.622280in}}%
\pgfpathlineto{\pgfqpoint{4.777891in}{3.610029in}}%
\pgfpathlineto{\pgfqpoint{4.780429in}{3.594716in}}%
\pgfpathlineto{\pgfqpoint{4.782966in}{3.576367in}}%
\pgfpathlineto{\pgfqpoint{4.785504in}{3.555015in}}%
\pgfpathlineto{\pgfqpoint{4.788042in}{3.530700in}}%
\pgfpathlineto{\pgfqpoint{4.790579in}{3.503468in}}%
\pgfpathlineto{\pgfqpoint{4.793117in}{3.473373in}}%
\pgfpathlineto{\pgfqpoint{4.795655in}{3.440472in}}%
\pgfpathlineto{\pgfqpoint{4.798192in}{3.404832in}}%
\pgfpathlineto{\pgfqpoint{4.800730in}{3.366524in}}%
\pgfpathlineto{\pgfqpoint{4.805805in}{3.282220in}}%
\pgfpathlineto{\pgfqpoint{4.810880in}{3.188248in}}%
\pgfpathlineto{\pgfqpoint{4.815956in}{3.085387in}}%
\pgfpathlineto{\pgfqpoint{4.821031in}{2.974496in}}%
\pgfpathlineto{\pgfqpoint{4.826106in}{2.856511in}}%
\pgfpathlineto{\pgfqpoint{4.831181in}{2.732436in}}%
\pgfpathlineto{\pgfqpoint{4.838794in}{2.537245in}}%
\pgfpathlineto{\pgfqpoint{4.848945in}{2.266005in}}%
\pgfpathlineto{\pgfqpoint{4.866708in}{1.787934in}}%
\pgfpathlineto{\pgfqpoint{4.874321in}{1.592348in}}%
\pgfpathlineto{\pgfqpoint{4.879397in}{1.468039in}}%
\pgfpathlineto{\pgfqpoint{4.884472in}{1.349924in}}%
\pgfpathlineto{\pgfqpoint{4.889547in}{1.239089in}}%
\pgfpathlineto{\pgfqpoint{4.894622in}{1.136560in}}%
\pgfpathlineto{\pgfqpoint{4.899698in}{1.043295in}}%
\pgfpathlineto{\pgfqpoint{4.904773in}{0.960175in}}%
\pgfpathlineto{\pgfqpoint{4.907310in}{0.922670in}}%
\pgfpathlineto{\pgfqpoint{4.909848in}{0.887991in}}%
\pgfpathlineto{\pgfqpoint{4.912386in}{0.856223in}}%
\pgfpathlineto{\pgfqpoint{4.914923in}{0.827443in}}%
\pgfpathlineto{\pgfqpoint{4.917461in}{0.801723in}}%
\pgfpathlineto{\pgfqpoint{4.919999in}{0.779127in}}%
\pgfpathlineto{\pgfqpoint{4.922536in}{0.759713in}}%
\pgfpathlineto{\pgfqpoint{4.925074in}{0.743532in}}%
\pgfpathlineto{\pgfqpoint{4.927612in}{0.730626in}}%
\pgfpathlineto{\pgfqpoint{4.930149in}{0.721031in}}%
\pgfpathlineto{\pgfqpoint{4.932687in}{0.714777in}}%
\pgfpathlineto{\pgfqpoint{4.935224in}{0.711882in}}%
\pgfpathlineto{\pgfqpoint{4.937762in}{0.712362in}}%
\pgfpathlineto{\pgfqpoint{4.940300in}{0.716220in}}%
\pgfpathlineto{\pgfqpoint{4.942837in}{0.723454in}}%
\pgfpathlineto{\pgfqpoint{4.945375in}{0.734055in}}%
\pgfpathlineto{\pgfqpoint{4.947913in}{0.748004in}}%
\pgfpathlineto{\pgfqpoint{4.950450in}{0.765275in}}%
\pgfpathlineto{\pgfqpoint{4.952988in}{0.785834in}}%
\pgfpathlineto{\pgfqpoint{4.955525in}{0.809640in}}%
\pgfpathlineto{\pgfqpoint{4.958063in}{0.836644in}}%
\pgfpathlineto{\pgfqpoint{4.960601in}{0.866789in}}%
\pgfpathlineto{\pgfqpoint{4.963138in}{0.900010in}}%
\pgfpathlineto{\pgfqpoint{4.965676in}{0.936237in}}%
\pgfpathlineto{\pgfqpoint{4.968214in}{0.975390in}}%
\pgfpathlineto{\pgfqpoint{4.970751in}{1.017384in}}%
\pgfpathlineto{\pgfqpoint{4.975827in}{1.109514in}}%
\pgfpathlineto{\pgfqpoint{4.980902in}{1.211805in}}%
\pgfpathlineto{\pgfqpoint{4.985977in}{1.323334in}}%
\pgfpathlineto{\pgfqpoint{4.991052in}{1.443085in}}%
\pgfpathlineto{\pgfqpoint{4.996128in}{1.569959in}}%
\pgfpathlineto{\pgfqpoint{5.003741in}{1.771039in}}%
\pgfpathlineto{\pgfqpoint{5.011353in}{1.981272in}}%
\pgfpathlineto{\pgfqpoint{5.034192in}{2.621744in}}%
\pgfpathlineto{\pgfqpoint{5.041805in}{2.823133in}}%
\pgfpathlineto{\pgfqpoint{5.046880in}{2.950135in}}%
\pgfpathlineto{\pgfqpoint{5.051956in}{3.069862in}}%
\pgfpathlineto{\pgfqpoint{5.057031in}{3.181126in}}%
\pgfpathlineto{\pgfqpoint{5.062106in}{3.282813in}}%
\pgfpathlineto{\pgfqpoint{5.067181in}{3.373898in}}%
\pgfpathlineto{\pgfqpoint{5.069719in}{3.415170in}}%
\pgfpathlineto{\pgfqpoint{5.072257in}{3.453451in}}%
\pgfpathlineto{\pgfqpoint{5.074794in}{3.488642in}}%
\pgfpathlineto{\pgfqpoint{5.077332in}{3.520651in}}%
\pgfpathlineto{\pgfqpoint{5.079869in}{3.549394in}}%
\pgfpathlineto{\pgfqpoint{5.082407in}{3.574794in}}%
\pgfpathlineto{\pgfqpoint{5.084945in}{3.596782in}}%
\pgfpathlineto{\pgfqpoint{5.087482in}{3.615298in}}%
\pgfpathlineto{\pgfqpoint{5.090020in}{3.630291in}}%
\pgfpathlineto{\pgfqpoint{5.092558in}{3.641716in}}%
\pgfpathlineto{\pgfqpoint{5.095095in}{3.649538in}}%
\pgfpathlineto{\pgfqpoint{5.097633in}{3.653733in}}%
\pgfpathlineto{\pgfqpoint{5.100171in}{3.654283in}}%
\pgfpathlineto{\pgfqpoint{5.102708in}{3.651180in}}%
\pgfpathlineto{\pgfqpoint{5.105246in}{3.644425in}}%
\pgfpathlineto{\pgfqpoint{5.107783in}{3.634029in}}%
\pgfpathlineto{\pgfqpoint{5.110321in}{3.620010in}}%
\pgfpathlineto{\pgfqpoint{5.112859in}{3.602396in}}%
\pgfpathlineto{\pgfqpoint{5.115396in}{3.581226in}}%
\pgfpathlineto{\pgfqpoint{5.117934in}{3.556546in}}%
\pgfpathlineto{\pgfqpoint{5.120472in}{3.528410in}}%
\pgfpathlineto{\pgfqpoint{5.123009in}{3.496884in}}%
\pgfpathlineto{\pgfqpoint{5.125547in}{3.462040in}}%
\pgfpathlineto{\pgfqpoint{5.128084in}{3.423959in}}%
\pgfpathlineto{\pgfqpoint{5.130622in}{3.382731in}}%
\pgfpathlineto{\pgfqpoint{5.133160in}{3.338454in}}%
\pgfpathlineto{\pgfqpoint{5.138235in}{3.241185in}}%
\pgfpathlineto{\pgfqpoint{5.143310in}{3.133090in}}%
\pgfpathlineto{\pgfqpoint{5.148386in}{3.015222in}}%
\pgfpathlineto{\pgfqpoint{5.153461in}{2.888741in}}%
\pgfpathlineto{\pgfqpoint{5.158536in}{2.754899in}}%
\pgfpathlineto{\pgfqpoint{5.166149in}{2.543275in}}%
\pgfpathlineto{\pgfqpoint{5.176300in}{2.248353in}}%
\pgfpathlineto{\pgfqpoint{5.191525in}{1.803095in}}%
\pgfpathlineto{\pgfqpoint{5.199138in}{1.590349in}}%
\pgfpathlineto{\pgfqpoint{5.204213in}{1.455609in}}%
\pgfpathlineto{\pgfqpoint{5.209289in}{1.328206in}}%
\pgfpathlineto{\pgfqpoint{5.214364in}{1.209491in}}%
\pgfpathlineto{\pgfqpoint{5.219439in}{1.100733in}}%
\pgfpathlineto{\pgfqpoint{5.224515in}{1.003101in}}%
\pgfpathlineto{\pgfqpoint{5.227052in}{0.958794in}}%
\pgfpathlineto{\pgfqpoint{5.229590in}{0.917658in}}%
\pgfpathlineto{\pgfqpoint{5.232127in}{0.879806in}}%
\pgfpathlineto{\pgfqpoint{5.234665in}{0.845344in}}%
\pgfpathlineto{\pgfqpoint{5.237203in}{0.814367in}}%
\pgfpathlineto{\pgfqpoint{5.239740in}{0.786964in}}%
\pgfpathlineto{\pgfqpoint{5.242278in}{0.763213in}}%
\pgfpathlineto{\pgfqpoint{5.244816in}{0.743183in}}%
\pgfpathlineto{\pgfqpoint{5.247353in}{0.726934in}}%
\pgfpathlineto{\pgfqpoint{5.249891in}{0.714515in}}%
\pgfpathlineto{\pgfqpoint{5.252428in}{0.705965in}}%
\pgfpathlineto{\pgfqpoint{5.254966in}{0.701315in}}%
\pgfpathlineto{\pgfqpoint{5.257504in}{0.700583in}}%
\pgfpathlineto{\pgfqpoint{5.260041in}{0.703778in}}%
\pgfpathlineto{\pgfqpoint{5.262579in}{0.710898in}}%
\pgfpathlineto{\pgfqpoint{5.265117in}{0.721932in}}%
\pgfpathlineto{\pgfqpoint{5.267654in}{0.736856in}}%
\pgfpathlineto{\pgfqpoint{5.270192in}{0.755638in}}%
\pgfpathlineto{\pgfqpoint{5.272730in}{0.778234in}}%
\pgfpathlineto{\pgfqpoint{5.275267in}{0.804591in}}%
\pgfpathlineto{\pgfqpoint{5.277805in}{0.834644in}}%
\pgfpathlineto{\pgfqpoint{5.280342in}{0.868319in}}%
\pgfpathlineto{\pgfqpoint{5.282880in}{0.905532in}}%
\pgfpathlineto{\pgfqpoint{5.285418in}{0.946190in}}%
\pgfpathlineto{\pgfqpoint{5.287955in}{0.990189in}}%
\pgfpathlineto{\pgfqpoint{5.290493in}{1.037416in}}%
\pgfpathlineto{\pgfqpoint{5.295568in}{1.141061in}}%
\pgfpathlineto{\pgfqpoint{5.300644in}{1.256048in}}%
\pgfpathlineto{\pgfqpoint{5.305719in}{1.381173in}}%
\pgfpathlineto{\pgfqpoint{5.310794in}{1.515114in}}%
\pgfpathlineto{\pgfqpoint{5.315869in}{1.656447in}}%
\pgfpathlineto{\pgfqpoint{5.323482in}{1.878973in}}%
\pgfpathlineto{\pgfqpoint{5.336170in}{2.264430in}}%
\pgfpathlineto{\pgfqpoint{5.346321in}{2.570776in}}%
\pgfpathlineto{\pgfqpoint{5.353934in}{2.790474in}}%
\pgfpathlineto{\pgfqpoint{5.359009in}{2.929064in}}%
\pgfpathlineto{\pgfqpoint{5.364084in}{3.059525in}}%
\pgfpathlineto{\pgfqpoint{5.369160in}{3.180380in}}%
\pgfpathlineto{\pgfqpoint{5.374235in}{3.290250in}}%
\pgfpathlineto{\pgfqpoint{5.376772in}{3.340667in}}%
\pgfpathlineto{\pgfqpoint{5.379310in}{3.387873in}}%
\pgfpathlineto{\pgfqpoint{5.381848in}{3.431732in}}%
\pgfpathlineto{\pgfqpoint{5.384385in}{3.472114in}}%
\pgfpathlineto{\pgfqpoint{5.386923in}{3.508902in}}%
\pgfpathlineto{\pgfqpoint{5.389461in}{3.541985in}}%
\pgfpathlineto{\pgfqpoint{5.391998in}{3.571265in}}%
\pgfpathlineto{\pgfqpoint{5.394536in}{3.596653in}}%
\pgfpathlineto{\pgfqpoint{5.397074in}{3.618071in}}%
\pgfpathlineto{\pgfqpoint{5.399611in}{3.635451in}}%
\pgfpathlineto{\pgfqpoint{5.402149in}{3.648738in}}%
\pgfpathlineto{\pgfqpoint{5.404686in}{3.657888in}}%
\pgfpathlineto{\pgfqpoint{5.407224in}{3.662868in}}%
\pgfpathlineto{\pgfqpoint{5.409762in}{3.663656in}}%
\pgfpathlineto{\pgfqpoint{5.412299in}{3.660243in}}%
\pgfpathlineto{\pgfqpoint{5.414837in}{3.652633in}}%
\pgfpathlineto{\pgfqpoint{5.417375in}{3.640839in}}%
\pgfpathlineto{\pgfqpoint{5.419912in}{3.624888in}}%
\pgfpathlineto{\pgfqpoint{5.422450in}{3.604819in}}%
\pgfpathlineto{\pgfqpoint{5.424987in}{3.580682in}}%
\pgfpathlineto{\pgfqpoint{5.427525in}{3.552538in}}%
\pgfpathlineto{\pgfqpoint{5.430063in}{3.520462in}}%
\pgfpathlineto{\pgfqpoint{5.432600in}{3.484539in}}%
\pgfpathlineto{\pgfqpoint{5.435138in}{3.444864in}}%
\pgfpathlineto{\pgfqpoint{5.437676in}{3.401545in}}%
\pgfpathlineto{\pgfqpoint{5.440213in}{3.354701in}}%
\pgfpathlineto{\pgfqpoint{5.442751in}{3.304459in}}%
\pgfpathlineto{\pgfqpoint{5.447826in}{3.194348in}}%
\pgfpathlineto{\pgfqpoint{5.452901in}{3.072437in}}%
\pgfpathlineto{\pgfqpoint{5.457977in}{2.940091in}}%
\pgfpathlineto{\pgfqpoint{5.463052in}{2.798805in}}%
\pgfpathlineto{\pgfqpoint{5.468127in}{2.650185in}}%
\pgfpathlineto{\pgfqpoint{5.475740in}{2.417246in}}%
\pgfpathlineto{\pgfqpoint{5.498579in}{1.703759in}}%
\pgfpathlineto{\pgfqpoint{5.503654in}{1.554145in}}%
\pgfpathlineto{\pgfqpoint{5.508729in}{1.411717in}}%
\pgfpathlineto{\pgfqpoint{5.513805in}{1.278180in}}%
\pgfpathlineto{\pgfqpoint{5.518880in}{1.155141in}}%
\pgfpathlineto{\pgfqpoint{5.523955in}{1.044092in}}%
\pgfpathlineto{\pgfqpoint{5.526493in}{0.993493in}}%
\pgfpathlineto{\pgfqpoint{5.529030in}{0.946390in}}%
\pgfpathlineto{\pgfqpoint{5.531568in}{0.902927in}}%
\pgfpathlineto{\pgfqpoint{5.534106in}{0.863241in}}%
\pgfpathlineto{\pgfqpoint{5.536643in}{0.827455in}}%
\pgfpathlineto{\pgfqpoint{5.539181in}{0.795683in}}%
\pgfpathlineto{\pgfqpoint{5.541719in}{0.768027in}}%
\pgfpathlineto{\pgfqpoint{5.544256in}{0.744574in}}%
\pgfpathlineto{\pgfqpoint{5.546794in}{0.725404in}}%
\pgfpathlineto{\pgfqpoint{5.549331in}{0.710580in}}%
\pgfpathlineto{\pgfqpoint{5.551869in}{0.700153in}}%
\pgfpathlineto{\pgfqpoint{5.554407in}{0.694162in}}%
\pgfpathlineto{\pgfqpoint{5.556944in}{0.692632in}}%
\pgfpathlineto{\pgfqpoint{5.559482in}{0.695576in}}%
\pgfpathlineto{\pgfqpoint{5.562020in}{0.702990in}}%
\pgfpathlineto{\pgfqpoint{5.564557in}{0.714861in}}%
\pgfpathlineto{\pgfqpoint{5.567095in}{0.731160in}}%
\pgfpathlineto{\pgfqpoint{5.569633in}{0.751844in}}%
\pgfpathlineto{\pgfqpoint{5.572170in}{0.776858in}}%
\pgfpathlineto{\pgfqpoint{5.574708in}{0.806133in}}%
\pgfpathlineto{\pgfqpoint{5.577245in}{0.839588in}}%
\pgfpathlineto{\pgfqpoint{5.579783in}{0.877127in}}%
\pgfpathlineto{\pgfqpoint{5.582321in}{0.918644in}}%
\pgfpathlineto{\pgfqpoint{5.584858in}{0.964018in}}%
\pgfpathlineto{\pgfqpoint{5.587396in}{1.013117in}}%
\pgfpathlineto{\pgfqpoint{5.589934in}{1.065798in}}%
\pgfpathlineto{\pgfqpoint{5.595009in}{1.181276in}}%
\pgfpathlineto{\pgfqpoint{5.600084in}{1.309085in}}%
\pgfpathlineto{\pgfqpoint{5.605159in}{1.447699in}}%
\pgfpathlineto{\pgfqpoint{5.610235in}{1.595453in}}%
\pgfpathlineto{\pgfqpoint{5.617848in}{1.830285in}}%
\pgfpathlineto{\pgfqpoint{5.627998in}{2.157931in}}%
\pgfpathlineto{\pgfqpoint{5.640686in}{2.568152in}}%
\pgfpathlineto{\pgfqpoint{5.648299in}{2.802406in}}%
\pgfpathlineto{\pgfqpoint{5.653374in}{2.949496in}}%
\pgfpathlineto{\pgfqpoint{5.658450in}{3.087136in}}%
\pgfpathlineto{\pgfqpoint{5.663525in}{3.213572in}}%
\pgfpathlineto{\pgfqpoint{5.668600in}{3.327181in}}%
\pgfpathlineto{\pgfqpoint{5.671138in}{3.378709in}}%
\pgfpathlineto{\pgfqpoint{5.673675in}{3.426494in}}%
\pgfpathlineto{\pgfqpoint{5.676213in}{3.470376in}}%
\pgfpathlineto{\pgfqpoint{5.678751in}{3.510213in}}%
\pgfpathlineto{\pgfqpoint{5.681288in}{3.545872in}}%
\pgfpathlineto{\pgfqpoint{5.683826in}{3.577234in}}%
\pgfpathlineto{\pgfqpoint{5.686364in}{3.604192in}}%
\pgfpathlineto{\pgfqpoint{5.686364in}{3.604192in}}%
\pgfusepath{stroke}%
\end{pgfscope}%
\begin{pgfscope}%
\pgfpathrectangle{\pgfqpoint{0.360000in}{0.400000in}}{\pgfqpoint{5.580000in}{3.560000in}}%
\pgfusepath{clip}%
\pgfsetrectcap%
\pgfsetroundjoin%
\pgfsetlinewidth{1.505625pt}%
\definecolor{currentstroke}{rgb}{0.988235,0.552941,0.349020}%
\pgfsetstrokecolor{currentstroke}%
\pgfsetdash{}{0pt}%
\pgfpathmoveto{\pgfqpoint{0.613636in}{2.180000in}}%
\pgfpathlineto{\pgfqpoint{0.659314in}{2.180158in}}%
\pgfpathlineto{\pgfqpoint{0.684690in}{2.180594in}}%
\pgfpathlineto{\pgfqpoint{0.704991in}{2.181262in}}%
\pgfpathlineto{\pgfqpoint{0.722755in}{2.182150in}}%
\pgfpathlineto{\pgfqpoint{0.740518in}{2.183378in}}%
\pgfpathlineto{\pgfqpoint{0.755744in}{2.184744in}}%
\pgfpathlineto{\pgfqpoint{0.770970in}{2.186433in}}%
\pgfpathlineto{\pgfqpoint{0.786195in}{2.188480in}}%
\pgfpathlineto{\pgfqpoint{0.798884in}{2.190483in}}%
\pgfpathlineto{\pgfqpoint{0.811572in}{2.192775in}}%
\pgfpathlineto{\pgfqpoint{0.824260in}{2.195375in}}%
\pgfpathlineto{\pgfqpoint{0.836948in}{2.198301in}}%
\pgfpathlineto{\pgfqpoint{0.849636in}{2.201569in}}%
\pgfpathlineto{\pgfqpoint{0.862324in}{2.205195in}}%
\pgfpathlineto{\pgfqpoint{0.875013in}{2.209196in}}%
\pgfpathlineto{\pgfqpoint{0.887701in}{2.213585in}}%
\pgfpathlineto{\pgfqpoint{0.900389in}{2.218376in}}%
\pgfpathlineto{\pgfqpoint{0.913077in}{2.223580in}}%
\pgfpathlineto{\pgfqpoint{0.925765in}{2.229208in}}%
\pgfpathlineto{\pgfqpoint{0.938453in}{2.235270in}}%
\pgfpathlineto{\pgfqpoint{0.951141in}{2.241771in}}%
\pgfpathlineto{\pgfqpoint{0.963830in}{2.248718in}}%
\pgfpathlineto{\pgfqpoint{0.976518in}{2.256114in}}%
\pgfpathlineto{\pgfqpoint{0.989206in}{2.263957in}}%
\pgfpathlineto{\pgfqpoint{1.001894in}{2.272248in}}%
\pgfpathlineto{\pgfqpoint{1.014582in}{2.280979in}}%
\pgfpathlineto{\pgfqpoint{1.027270in}{2.290144in}}%
\pgfpathlineto{\pgfqpoint{1.039959in}{2.299731in}}%
\pgfpathlineto{\pgfqpoint{1.052647in}{2.309723in}}%
\pgfpathlineto{\pgfqpoint{1.065335in}{2.320103in}}%
\pgfpathlineto{\pgfqpoint{1.078023in}{2.330847in}}%
\pgfpathlineto{\pgfqpoint{1.093249in}{2.344181in}}%
\pgfpathlineto{\pgfqpoint{1.108475in}{2.357944in}}%
\pgfpathlineto{\pgfqpoint{1.126238in}{2.374451in}}%
\pgfpathlineto{\pgfqpoint{1.146539in}{2.393759in}}%
\pgfpathlineto{\pgfqpoint{1.212518in}{2.456953in}}%
\pgfpathlineto{\pgfqpoint{1.227743in}{2.470901in}}%
\pgfpathlineto{\pgfqpoint{1.240432in}{2.482116in}}%
\pgfpathlineto{\pgfqpoint{1.253120in}{2.492857in}}%
\pgfpathlineto{\pgfqpoint{1.263270in}{2.501042in}}%
\pgfpathlineto{\pgfqpoint{1.273421in}{2.508807in}}%
\pgfpathlineto{\pgfqpoint{1.283571in}{2.516097in}}%
\pgfpathlineto{\pgfqpoint{1.291184in}{2.521219in}}%
\pgfpathlineto{\pgfqpoint{1.298797in}{2.526017in}}%
\pgfpathlineto{\pgfqpoint{1.306410in}{2.530466in}}%
\pgfpathlineto{\pgfqpoint{1.314023in}{2.534543in}}%
\pgfpathlineto{\pgfqpoint{1.321636in}{2.538221in}}%
\pgfpathlineto{\pgfqpoint{1.329249in}{2.541475in}}%
\pgfpathlineto{\pgfqpoint{1.336862in}{2.544279in}}%
\pgfpathlineto{\pgfqpoint{1.344475in}{2.546608in}}%
\pgfpathlineto{\pgfqpoint{1.349550in}{2.547884in}}%
\pgfpathlineto{\pgfqpoint{1.354625in}{2.548930in}}%
\pgfpathlineto{\pgfqpoint{1.359700in}{2.549738in}}%
\pgfpathlineto{\pgfqpoint{1.364776in}{2.550300in}}%
\pgfpathlineto{\pgfqpoint{1.369851in}{2.550610in}}%
\pgfpathlineto{\pgfqpoint{1.374926in}{2.550660in}}%
\pgfpathlineto{\pgfqpoint{1.380001in}{2.550443in}}%
\pgfpathlineto{\pgfqpoint{1.385077in}{2.549951in}}%
\pgfpathlineto{\pgfqpoint{1.390152in}{2.549178in}}%
\pgfpathlineto{\pgfqpoint{1.395227in}{2.548116in}}%
\pgfpathlineto{\pgfqpoint{1.400302in}{2.546758in}}%
\pgfpathlineto{\pgfqpoint{1.405378in}{2.545099in}}%
\pgfpathlineto{\pgfqpoint{1.410453in}{2.543132in}}%
\pgfpathlineto{\pgfqpoint{1.415528in}{2.540849in}}%
\pgfpathlineto{\pgfqpoint{1.420603in}{2.538246in}}%
\pgfpathlineto{\pgfqpoint{1.425679in}{2.535317in}}%
\pgfpathlineto{\pgfqpoint{1.430754in}{2.532055in}}%
\pgfpathlineto{\pgfqpoint{1.435829in}{2.528456in}}%
\pgfpathlineto{\pgfqpoint{1.440905in}{2.524514in}}%
\pgfpathlineto{\pgfqpoint{1.445980in}{2.520225in}}%
\pgfpathlineto{\pgfqpoint{1.451055in}{2.515584in}}%
\pgfpathlineto{\pgfqpoint{1.456130in}{2.510586in}}%
\pgfpathlineto{\pgfqpoint{1.461206in}{2.505229in}}%
\pgfpathlineto{\pgfqpoint{1.466281in}{2.499509in}}%
\pgfpathlineto{\pgfqpoint{1.471356in}{2.493423in}}%
\pgfpathlineto{\pgfqpoint{1.476431in}{2.486967in}}%
\pgfpathlineto{\pgfqpoint{1.481507in}{2.480140in}}%
\pgfpathlineto{\pgfqpoint{1.486582in}{2.472941in}}%
\pgfpathlineto{\pgfqpoint{1.494195in}{2.461439in}}%
\pgfpathlineto{\pgfqpoint{1.501808in}{2.449092in}}%
\pgfpathlineto{\pgfqpoint{1.509421in}{2.435901in}}%
\pgfpathlineto{\pgfqpoint{1.517034in}{2.421868in}}%
\pgfpathlineto{\pgfqpoint{1.524646in}{2.406998in}}%
\pgfpathlineto{\pgfqpoint{1.532259in}{2.391301in}}%
\pgfpathlineto{\pgfqpoint{1.539872in}{2.374787in}}%
\pgfpathlineto{\pgfqpoint{1.547485in}{2.357472in}}%
\pgfpathlineto{\pgfqpoint{1.555098in}{2.339373in}}%
\pgfpathlineto{\pgfqpoint{1.562711in}{2.320512in}}%
\pgfpathlineto{\pgfqpoint{1.570324in}{2.300914in}}%
\pgfpathlineto{\pgfqpoint{1.577937in}{2.280608in}}%
\pgfpathlineto{\pgfqpoint{1.588087in}{2.252486in}}%
\pgfpathlineto{\pgfqpoint{1.598238in}{2.223250in}}%
\pgfpathlineto{\pgfqpoint{1.608388in}{2.192996in}}%
\pgfpathlineto{\pgfqpoint{1.618539in}{2.161838in}}%
\pgfpathlineto{\pgfqpoint{1.631227in}{2.121805in}}%
\pgfpathlineto{\pgfqpoint{1.646453in}{2.072531in}}%
\pgfpathlineto{\pgfqpoint{1.666754in}{2.005601in}}%
\pgfpathlineto{\pgfqpoint{1.694668in}{1.913672in}}%
\pgfpathlineto{\pgfqpoint{1.707356in}{1.872902in}}%
\pgfpathlineto{\pgfqpoint{1.717506in}{1.841139in}}%
\pgfpathlineto{\pgfqpoint{1.727657in}{1.810375in}}%
\pgfpathlineto{\pgfqpoint{1.737808in}{1.780837in}}%
\pgfpathlineto{\pgfqpoint{1.745420in}{1.759626in}}%
\pgfpathlineto{\pgfqpoint{1.753033in}{1.739335in}}%
\pgfpathlineto{\pgfqpoint{1.760646in}{1.720060in}}%
\pgfpathlineto{\pgfqpoint{1.768259in}{1.701901in}}%
\pgfpathlineto{\pgfqpoint{1.773334in}{1.690464in}}%
\pgfpathlineto{\pgfqpoint{1.778410in}{1.679595in}}%
\pgfpathlineto{\pgfqpoint{1.783485in}{1.669323in}}%
\pgfpathlineto{\pgfqpoint{1.788560in}{1.659676in}}%
\pgfpathlineto{\pgfqpoint{1.793635in}{1.650683in}}%
\pgfpathlineto{\pgfqpoint{1.798711in}{1.642370in}}%
\pgfpathlineto{\pgfqpoint{1.803786in}{1.634765in}}%
\pgfpathlineto{\pgfqpoint{1.808861in}{1.627893in}}%
\pgfpathlineto{\pgfqpoint{1.813937in}{1.621780in}}%
\pgfpathlineto{\pgfqpoint{1.819012in}{1.616451in}}%
\pgfpathlineto{\pgfqpoint{1.824087in}{1.611929in}}%
\pgfpathlineto{\pgfqpoint{1.826625in}{1.609979in}}%
\pgfpathlineto{\pgfqpoint{1.829162in}{1.608239in}}%
\pgfpathlineto{\pgfqpoint{1.831700in}{1.606712in}}%
\pgfpathlineto{\pgfqpoint{1.834238in}{1.605401in}}%
\pgfpathlineto{\pgfqpoint{1.836775in}{1.604308in}}%
\pgfpathlineto{\pgfqpoint{1.839313in}{1.603436in}}%
\pgfpathlineto{\pgfqpoint{1.841850in}{1.602788in}}%
\pgfpathlineto{\pgfqpoint{1.844388in}{1.602365in}}%
\pgfpathlineto{\pgfqpoint{1.846926in}{1.602170in}}%
\pgfpathlineto{\pgfqpoint{1.849463in}{1.602206in}}%
\pgfpathlineto{\pgfqpoint{1.852001in}{1.602473in}}%
\pgfpathlineto{\pgfqpoint{1.854539in}{1.602975in}}%
\pgfpathlineto{\pgfqpoint{1.857076in}{1.603713in}}%
\pgfpathlineto{\pgfqpoint{1.859614in}{1.604689in}}%
\pgfpathlineto{\pgfqpoint{1.862152in}{1.605905in}}%
\pgfpathlineto{\pgfqpoint{1.864689in}{1.607362in}}%
\pgfpathlineto{\pgfqpoint{1.867227in}{1.609062in}}%
\pgfpathlineto{\pgfqpoint{1.869764in}{1.611007in}}%
\pgfpathlineto{\pgfqpoint{1.872302in}{1.613197in}}%
\pgfpathlineto{\pgfqpoint{1.874840in}{1.615635in}}%
\pgfpathlineto{\pgfqpoint{1.877377in}{1.618321in}}%
\pgfpathlineto{\pgfqpoint{1.879915in}{1.621255in}}%
\pgfpathlineto{\pgfqpoint{1.884990in}{1.627876in}}%
\pgfpathlineto{\pgfqpoint{1.890065in}{1.635502in}}%
\pgfpathlineto{\pgfqpoint{1.895141in}{1.644138in}}%
\pgfpathlineto{\pgfqpoint{1.900216in}{1.653787in}}%
\pgfpathlineto{\pgfqpoint{1.905291in}{1.664447in}}%
\pgfpathlineto{\pgfqpoint{1.910367in}{1.676117in}}%
\pgfpathlineto{\pgfqpoint{1.915442in}{1.688792in}}%
\pgfpathlineto{\pgfqpoint{1.920517in}{1.702467in}}%
\pgfpathlineto{\pgfqpoint{1.925592in}{1.717131in}}%
\pgfpathlineto{\pgfqpoint{1.930668in}{1.732775in}}%
\pgfpathlineto{\pgfqpoint{1.935743in}{1.749385in}}%
\pgfpathlineto{\pgfqpoint{1.940818in}{1.766944in}}%
\pgfpathlineto{\pgfqpoint{1.945893in}{1.785436in}}%
\pgfpathlineto{\pgfqpoint{1.950969in}{1.804838in}}%
\pgfpathlineto{\pgfqpoint{1.956044in}{1.825129in}}%
\pgfpathlineto{\pgfqpoint{1.963657in}{1.857174in}}%
\pgfpathlineto{\pgfqpoint{1.971270in}{1.891061in}}%
\pgfpathlineto{\pgfqpoint{1.978883in}{1.926683in}}%
\pgfpathlineto{\pgfqpoint{1.986496in}{1.963917in}}%
\pgfpathlineto{\pgfqpoint{1.994108in}{2.002627in}}%
\pgfpathlineto{\pgfqpoint{2.004259in}{2.056285in}}%
\pgfpathlineto{\pgfqpoint{2.014409in}{2.111926in}}%
\pgfpathlineto{\pgfqpoint{2.027098in}{2.183630in}}%
\pgfpathlineto{\pgfqpoint{2.044861in}{2.286424in}}%
\pgfpathlineto{\pgfqpoint{2.067700in}{2.418744in}}%
\pgfpathlineto{\pgfqpoint{2.080388in}{2.490257in}}%
\pgfpathlineto{\pgfqpoint{2.090538in}{2.545484in}}%
\pgfpathlineto{\pgfqpoint{2.098151in}{2.585366in}}%
\pgfpathlineto{\pgfqpoint{2.105764in}{2.623659in}}%
\pgfpathlineto{\pgfqpoint{2.113377in}{2.660120in}}%
\pgfpathlineto{\pgfqpoint{2.120990in}{2.694506in}}%
\pgfpathlineto{\pgfqpoint{2.126065in}{2.716162in}}%
\pgfpathlineto{\pgfqpoint{2.131141in}{2.736722in}}%
\pgfpathlineto{\pgfqpoint{2.136216in}{2.756120in}}%
\pgfpathlineto{\pgfqpoint{2.141291in}{2.774291in}}%
\pgfpathlineto{\pgfqpoint{2.146366in}{2.791170in}}%
\pgfpathlineto{\pgfqpoint{2.151442in}{2.806697in}}%
\pgfpathlineto{\pgfqpoint{2.156517in}{2.820813in}}%
\pgfpathlineto{\pgfqpoint{2.161592in}{2.833460in}}%
\pgfpathlineto{\pgfqpoint{2.166667in}{2.844587in}}%
\pgfpathlineto{\pgfqpoint{2.169205in}{2.849563in}}%
\pgfpathlineto{\pgfqpoint{2.171743in}{2.854141in}}%
\pgfpathlineto{\pgfqpoint{2.174280in}{2.858314in}}%
\pgfpathlineto{\pgfqpoint{2.176818in}{2.862076in}}%
\pgfpathlineto{\pgfqpoint{2.179356in}{2.865422in}}%
\pgfpathlineto{\pgfqpoint{2.181893in}{2.868348in}}%
\pgfpathlineto{\pgfqpoint{2.184431in}{2.870848in}}%
\pgfpathlineto{\pgfqpoint{2.186968in}{2.872917in}}%
\pgfpathlineto{\pgfqpoint{2.189506in}{2.874551in}}%
\pgfpathlineto{\pgfqpoint{2.192044in}{2.875747in}}%
\pgfpathlineto{\pgfqpoint{2.194581in}{2.876500in}}%
\pgfpathlineto{\pgfqpoint{2.197119in}{2.876806in}}%
\pgfpathlineto{\pgfqpoint{2.199657in}{2.876662in}}%
\pgfpathlineto{\pgfqpoint{2.202194in}{2.876066in}}%
\pgfpathlineto{\pgfqpoint{2.204732in}{2.875014in}}%
\pgfpathlineto{\pgfqpoint{2.207270in}{2.873504in}}%
\pgfpathlineto{\pgfqpoint{2.209807in}{2.871534in}}%
\pgfpathlineto{\pgfqpoint{2.212345in}{2.869102in}}%
\pgfpathlineto{\pgfqpoint{2.214882in}{2.866207in}}%
\pgfpathlineto{\pgfqpoint{2.217420in}{2.862847in}}%
\pgfpathlineto{\pgfqpoint{2.219958in}{2.859021in}}%
\pgfpathlineto{\pgfqpoint{2.222495in}{2.854729in}}%
\pgfpathlineto{\pgfqpoint{2.225033in}{2.849971in}}%
\pgfpathlineto{\pgfqpoint{2.227571in}{2.844747in}}%
\pgfpathlineto{\pgfqpoint{2.230108in}{2.839056in}}%
\pgfpathlineto{\pgfqpoint{2.232646in}{2.832901in}}%
\pgfpathlineto{\pgfqpoint{2.235184in}{2.826281in}}%
\pgfpathlineto{\pgfqpoint{2.240259in}{2.811656in}}%
\pgfpathlineto{\pgfqpoint{2.245334in}{2.795198in}}%
\pgfpathlineto{\pgfqpoint{2.250409in}{2.776928in}}%
\pgfpathlineto{\pgfqpoint{2.255485in}{2.756874in}}%
\pgfpathlineto{\pgfqpoint{2.260560in}{2.735069in}}%
\pgfpathlineto{\pgfqpoint{2.265635in}{2.711552in}}%
\pgfpathlineto{\pgfqpoint{2.270710in}{2.686369in}}%
\pgfpathlineto{\pgfqpoint{2.275786in}{2.659571in}}%
\pgfpathlineto{\pgfqpoint{2.280861in}{2.631216in}}%
\pgfpathlineto{\pgfqpoint{2.285936in}{2.601365in}}%
\pgfpathlineto{\pgfqpoint{2.291011in}{2.570088in}}%
\pgfpathlineto{\pgfqpoint{2.298624in}{2.520662in}}%
\pgfpathlineto{\pgfqpoint{2.306237in}{2.468470in}}%
\pgfpathlineto{\pgfqpoint{2.313850in}{2.413815in}}%
\pgfpathlineto{\pgfqpoint{2.321463in}{2.357025in}}%
\pgfpathlineto{\pgfqpoint{2.331614in}{2.278595in}}%
\pgfpathlineto{\pgfqpoint{2.344302in}{2.177478in}}%
\pgfpathlineto{\pgfqpoint{2.377291in}{1.911893in}}%
\pgfpathlineto{\pgfqpoint{2.387441in}{1.833521in}}%
\pgfpathlineto{\pgfqpoint{2.395054in}{1.776968in}}%
\pgfpathlineto{\pgfqpoint{2.402667in}{1.722836in}}%
\pgfpathlineto{\pgfqpoint{2.410280in}{1.671578in}}%
\pgfpathlineto{\pgfqpoint{2.415355in}{1.639222in}}%
\pgfpathlineto{\pgfqpoint{2.420431in}{1.608467in}}%
\pgfpathlineto{\pgfqpoint{2.425506in}{1.579439in}}%
\pgfpathlineto{\pgfqpoint{2.430581in}{1.552258in}}%
\pgfpathlineto{\pgfqpoint{2.435656in}{1.527039in}}%
\pgfpathlineto{\pgfqpoint{2.440732in}{1.503894in}}%
\pgfpathlineto{\pgfqpoint{2.445807in}{1.482927in}}%
\pgfpathlineto{\pgfqpoint{2.450882in}{1.464238in}}%
\pgfpathlineto{\pgfqpoint{2.453420in}{1.455777in}}%
\pgfpathlineto{\pgfqpoint{2.455958in}{1.447920in}}%
\pgfpathlineto{\pgfqpoint{2.458495in}{1.440676in}}%
\pgfpathlineto{\pgfqpoint{2.461033in}{1.434057in}}%
\pgfpathlineto{\pgfqpoint{2.463570in}{1.428071in}}%
\pgfpathlineto{\pgfqpoint{2.466108in}{1.422728in}}%
\pgfpathlineto{\pgfqpoint{2.468646in}{1.418035in}}%
\pgfpathlineto{\pgfqpoint{2.471183in}{1.414002in}}%
\pgfpathlineto{\pgfqpoint{2.473721in}{1.410635in}}%
\pgfpathlineto{\pgfqpoint{2.476259in}{1.407942in}}%
\pgfpathlineto{\pgfqpoint{2.478796in}{1.405928in}}%
\pgfpathlineto{\pgfqpoint{2.481334in}{1.404599in}}%
\pgfpathlineto{\pgfqpoint{2.483871in}{1.403961in}}%
\pgfpathlineto{\pgfqpoint{2.486409in}{1.404018in}}%
\pgfpathlineto{\pgfqpoint{2.488947in}{1.404773in}}%
\pgfpathlineto{\pgfqpoint{2.491484in}{1.406230in}}%
\pgfpathlineto{\pgfqpoint{2.494022in}{1.408392in}}%
\pgfpathlineto{\pgfqpoint{2.496560in}{1.411261in}}%
\pgfpathlineto{\pgfqpoint{2.499097in}{1.414838in}}%
\pgfpathlineto{\pgfqpoint{2.501635in}{1.419123in}}%
\pgfpathlineto{\pgfqpoint{2.504173in}{1.424118in}}%
\pgfpathlineto{\pgfqpoint{2.506710in}{1.429820in}}%
\pgfpathlineto{\pgfqpoint{2.509248in}{1.436229in}}%
\pgfpathlineto{\pgfqpoint{2.511785in}{1.443342in}}%
\pgfpathlineto{\pgfqpoint{2.514323in}{1.451157in}}%
\pgfpathlineto{\pgfqpoint{2.516861in}{1.459671in}}%
\pgfpathlineto{\pgfqpoint{2.519398in}{1.468879in}}%
\pgfpathlineto{\pgfqpoint{2.521936in}{1.478776in}}%
\pgfpathlineto{\pgfqpoint{2.527011in}{1.500617in}}%
\pgfpathlineto{\pgfqpoint{2.532086in}{1.525138in}}%
\pgfpathlineto{\pgfqpoint{2.537162in}{1.552278in}}%
\pgfpathlineto{\pgfqpoint{2.542237in}{1.581959in}}%
\pgfpathlineto{\pgfqpoint{2.547312in}{1.614094in}}%
\pgfpathlineto{\pgfqpoint{2.552388in}{1.648586in}}%
\pgfpathlineto{\pgfqpoint{2.557463in}{1.685324in}}%
\pgfpathlineto{\pgfqpoint{2.562538in}{1.724190in}}%
\pgfpathlineto{\pgfqpoint{2.567613in}{1.765052in}}%
\pgfpathlineto{\pgfqpoint{2.575226in}{1.829777in}}%
\pgfpathlineto{\pgfqpoint{2.582839in}{1.898158in}}%
\pgfpathlineto{\pgfqpoint{2.590452in}{1.969634in}}%
\pgfpathlineto{\pgfqpoint{2.600603in}{2.068699in}}%
\pgfpathlineto{\pgfqpoint{2.613291in}{2.196400in}}%
\pgfpathlineto{\pgfqpoint{2.636129in}{2.427285in}}%
\pgfpathlineto{\pgfqpoint{2.646280in}{2.526087in}}%
\pgfpathlineto{\pgfqpoint{2.653893in}{2.597109in}}%
\pgfpathlineto{\pgfqpoint{2.661506in}{2.664675in}}%
\pgfpathlineto{\pgfqpoint{2.666581in}{2.707445in}}%
\pgfpathlineto{\pgfqpoint{2.671656in}{2.748156in}}%
\pgfpathlineto{\pgfqpoint{2.676732in}{2.786608in}}%
\pgfpathlineto{\pgfqpoint{2.681807in}{2.822608in}}%
\pgfpathlineto{\pgfqpoint{2.686882in}{2.855973in}}%
\pgfpathlineto{\pgfqpoint{2.691957in}{2.886527in}}%
\pgfpathlineto{\pgfqpoint{2.697033in}{2.914105in}}%
\pgfpathlineto{\pgfqpoint{2.702108in}{2.938554in}}%
\pgfpathlineto{\pgfqpoint{2.704646in}{2.949561in}}%
\pgfpathlineto{\pgfqpoint{2.707183in}{2.959733in}}%
\pgfpathlineto{\pgfqpoint{2.709721in}{2.969055in}}%
\pgfpathlineto{\pgfqpoint{2.712258in}{2.977512in}}%
\pgfpathlineto{\pgfqpoint{2.714796in}{2.985091in}}%
\pgfpathlineto{\pgfqpoint{2.717334in}{2.991778in}}%
\pgfpathlineto{\pgfqpoint{2.719871in}{2.997562in}}%
\pgfpathlineto{\pgfqpoint{2.722409in}{3.002430in}}%
\pgfpathlineto{\pgfqpoint{2.724947in}{3.006374in}}%
\pgfpathlineto{\pgfqpoint{2.727484in}{3.009385in}}%
\pgfpathlineto{\pgfqpoint{2.730022in}{3.011453in}}%
\pgfpathlineto{\pgfqpoint{2.732559in}{3.012573in}}%
\pgfpathlineto{\pgfqpoint{2.735097in}{3.012738in}}%
\pgfpathlineto{\pgfqpoint{2.737635in}{3.011943in}}%
\pgfpathlineto{\pgfqpoint{2.740172in}{3.010185in}}%
\pgfpathlineto{\pgfqpoint{2.742710in}{3.007462in}}%
\pgfpathlineto{\pgfqpoint{2.745248in}{3.003771in}}%
\pgfpathlineto{\pgfqpoint{2.747785in}{2.999112in}}%
\pgfpathlineto{\pgfqpoint{2.750323in}{2.993487in}}%
\pgfpathlineto{\pgfqpoint{2.752861in}{2.986897in}}%
\pgfpathlineto{\pgfqpoint{2.755398in}{2.979345in}}%
\pgfpathlineto{\pgfqpoint{2.757936in}{2.970836in}}%
\pgfpathlineto{\pgfqpoint{2.760473in}{2.961375in}}%
\pgfpathlineto{\pgfqpoint{2.763011in}{2.950970in}}%
\pgfpathlineto{\pgfqpoint{2.765549in}{2.939627in}}%
\pgfpathlineto{\pgfqpoint{2.768086in}{2.927356in}}%
\pgfpathlineto{\pgfqpoint{2.770624in}{2.914167in}}%
\pgfpathlineto{\pgfqpoint{2.775699in}{2.885084in}}%
\pgfpathlineto{\pgfqpoint{2.780774in}{2.852484in}}%
\pgfpathlineto{\pgfqpoint{2.785850in}{2.816491in}}%
\pgfpathlineto{\pgfqpoint{2.790925in}{2.777249in}}%
\pgfpathlineto{\pgfqpoint{2.796000in}{2.734921in}}%
\pgfpathlineto{\pgfqpoint{2.801076in}{2.689684in}}%
\pgfpathlineto{\pgfqpoint{2.806151in}{2.641737in}}%
\pgfpathlineto{\pgfqpoint{2.811226in}{2.591291in}}%
\pgfpathlineto{\pgfqpoint{2.818839in}{2.511441in}}%
\pgfpathlineto{\pgfqpoint{2.826452in}{2.427320in}}%
\pgfpathlineto{\pgfqpoint{2.834065in}{2.339825in}}%
\pgfpathlineto{\pgfqpoint{2.844215in}{2.219578in}}%
\pgfpathlineto{\pgfqpoint{2.869592in}{1.916147in}}%
\pgfpathlineto{\pgfqpoint{2.877205in}{1.828708in}}%
\pgfpathlineto{\pgfqpoint{2.884817in}{1.744840in}}%
\pgfpathlineto{\pgfqpoint{2.892430in}{1.665596in}}%
\pgfpathlineto{\pgfqpoint{2.897506in}{1.615839in}}%
\pgfpathlineto{\pgfqpoint{2.902581in}{1.568880in}}%
\pgfpathlineto{\pgfqpoint{2.907656in}{1.524997in}}%
\pgfpathlineto{\pgfqpoint{2.912731in}{1.484455in}}%
\pgfpathlineto{\pgfqpoint{2.917807in}{1.447503in}}%
\pgfpathlineto{\pgfqpoint{2.922882in}{1.414377in}}%
\pgfpathlineto{\pgfqpoint{2.925420in}{1.399316in}}%
\pgfpathlineto{\pgfqpoint{2.927957in}{1.385290in}}%
\pgfpathlineto{\pgfqpoint{2.930495in}{1.372324in}}%
\pgfpathlineto{\pgfqpoint{2.933032in}{1.360439in}}%
\pgfpathlineto{\pgfqpoint{2.935570in}{1.349657in}}%
\pgfpathlineto{\pgfqpoint{2.938108in}{1.339998in}}%
\pgfpathlineto{\pgfqpoint{2.940645in}{1.331479in}}%
\pgfpathlineto{\pgfqpoint{2.943183in}{1.324119in}}%
\pgfpathlineto{\pgfqpoint{2.945721in}{1.317931in}}%
\pgfpathlineto{\pgfqpoint{2.948258in}{1.312929in}}%
\pgfpathlineto{\pgfqpoint{2.950796in}{1.309126in}}%
\pgfpathlineto{\pgfqpoint{2.953333in}{1.306532in}}%
\pgfpathlineto{\pgfqpoint{2.955871in}{1.305156in}}%
\pgfpathlineto{\pgfqpoint{2.958409in}{1.305005in}}%
\pgfpathlineto{\pgfqpoint{2.960946in}{1.306085in}}%
\pgfpathlineto{\pgfqpoint{2.963484in}{1.308398in}}%
\pgfpathlineto{\pgfqpoint{2.966022in}{1.311948in}}%
\pgfpathlineto{\pgfqpoint{2.968559in}{1.316734in}}%
\pgfpathlineto{\pgfqpoint{2.971097in}{1.322755in}}%
\pgfpathlineto{\pgfqpoint{2.973635in}{1.330006in}}%
\pgfpathlineto{\pgfqpoint{2.976172in}{1.338484in}}%
\pgfpathlineto{\pgfqpoint{2.978710in}{1.348181in}}%
\pgfpathlineto{\pgfqpoint{2.981247in}{1.359088in}}%
\pgfpathlineto{\pgfqpoint{2.983785in}{1.371195in}}%
\pgfpathlineto{\pgfqpoint{2.986323in}{1.384489in}}%
\pgfpathlineto{\pgfqpoint{2.988860in}{1.398956in}}%
\pgfpathlineto{\pgfqpoint{2.991398in}{1.414579in}}%
\pgfpathlineto{\pgfqpoint{2.993936in}{1.431341in}}%
\pgfpathlineto{\pgfqpoint{2.999011in}{1.468202in}}%
\pgfpathlineto{\pgfqpoint{3.004086in}{1.509358in}}%
\pgfpathlineto{\pgfqpoint{3.009161in}{1.554605in}}%
\pgfpathlineto{\pgfqpoint{3.014237in}{1.603708in}}%
\pgfpathlineto{\pgfqpoint{3.019312in}{1.656408in}}%
\pgfpathlineto{\pgfqpoint{3.024387in}{1.712421in}}%
\pgfpathlineto{\pgfqpoint{3.029462in}{1.771439in}}%
\pgfpathlineto{\pgfqpoint{3.037075in}{1.864873in}}%
\pgfpathlineto{\pgfqpoint{3.044688in}{1.963125in}}%
\pgfpathlineto{\pgfqpoint{3.054839in}{2.099389in}}%
\pgfpathlineto{\pgfqpoint{3.082753in}{2.479477in}}%
\pgfpathlineto{\pgfqpoint{3.090366in}{2.577724in}}%
\pgfpathlineto{\pgfqpoint{3.097979in}{2.670911in}}%
\pgfpathlineto{\pgfqpoint{3.103054in}{2.729537in}}%
\pgfpathlineto{\pgfqpoint{3.108129in}{2.784898in}}%
\pgfpathlineto{\pgfqpoint{3.113204in}{2.836616in}}%
\pgfpathlineto{\pgfqpoint{3.118280in}{2.884329in}}%
\pgfpathlineto{\pgfqpoint{3.123355in}{2.927699in}}%
\pgfpathlineto{\pgfqpoint{3.128430in}{2.966411in}}%
\pgfpathlineto{\pgfqpoint{3.130968in}{2.983929in}}%
\pgfpathlineto{\pgfqpoint{3.133505in}{3.000177in}}%
\pgfpathlineto{\pgfqpoint{3.136043in}{3.015124in}}%
\pgfpathlineto{\pgfqpoint{3.138581in}{3.028740in}}%
\pgfpathlineto{\pgfqpoint{3.141118in}{3.040997in}}%
\pgfpathlineto{\pgfqpoint{3.143656in}{3.051870in}}%
\pgfpathlineto{\pgfqpoint{3.146194in}{3.061337in}}%
\pgfpathlineto{\pgfqpoint{3.148731in}{3.069375in}}%
\pgfpathlineto{\pgfqpoint{3.151269in}{3.075967in}}%
\pgfpathlineto{\pgfqpoint{3.153806in}{3.081096in}}%
\pgfpathlineto{\pgfqpoint{3.156344in}{3.084747in}}%
\pgfpathlineto{\pgfqpoint{3.158882in}{3.086910in}}%
\pgfpathlineto{\pgfqpoint{3.161419in}{3.087575in}}%
\pgfpathlineto{\pgfqpoint{3.163957in}{3.086734in}}%
\pgfpathlineto{\pgfqpoint{3.166495in}{3.084384in}}%
\pgfpathlineto{\pgfqpoint{3.169032in}{3.080523in}}%
\pgfpathlineto{\pgfqpoint{3.171570in}{3.075152in}}%
\pgfpathlineto{\pgfqpoint{3.174108in}{3.068273in}}%
\pgfpathlineto{\pgfqpoint{3.176645in}{3.059894in}}%
\pgfpathlineto{\pgfqpoint{3.179183in}{3.050021in}}%
\pgfpathlineto{\pgfqpoint{3.181720in}{3.038667in}}%
\pgfpathlineto{\pgfqpoint{3.184258in}{3.025845in}}%
\pgfpathlineto{\pgfqpoint{3.186796in}{3.011571in}}%
\pgfpathlineto{\pgfqpoint{3.189333in}{2.995863in}}%
\pgfpathlineto{\pgfqpoint{3.191871in}{2.978745in}}%
\pgfpathlineto{\pgfqpoint{3.194409in}{2.960238in}}%
\pgfpathlineto{\pgfqpoint{3.196946in}{2.940370in}}%
\pgfpathlineto{\pgfqpoint{3.202021in}{2.896668in}}%
\pgfpathlineto{\pgfqpoint{3.207097in}{2.847899in}}%
\pgfpathlineto{\pgfqpoint{3.212172in}{2.794363in}}%
\pgfpathlineto{\pgfqpoint{3.217247in}{2.736396in}}%
\pgfpathlineto{\pgfqpoint{3.222323in}{2.674367in}}%
\pgfpathlineto{\pgfqpoint{3.227398in}{2.608681in}}%
\pgfpathlineto{\pgfqpoint{3.232473in}{2.539770in}}%
\pgfpathlineto{\pgfqpoint{3.240086in}{2.431376in}}%
\pgfpathlineto{\pgfqpoint{3.247699in}{2.318431in}}%
\pgfpathlineto{\pgfqpoint{3.262925in}{2.086001in}}%
\pgfpathlineto{\pgfqpoint{3.273075in}{1.932178in}}%
\pgfpathlineto{\pgfqpoint{3.280688in}{1.820592in}}%
\pgfpathlineto{\pgfqpoint{3.288301in}{1.714329in}}%
\pgfpathlineto{\pgfqpoint{3.293376in}{1.647342in}}%
\pgfpathlineto{\pgfqpoint{3.298451in}{1.584045in}}%
\pgfpathlineto{\pgfqpoint{3.303527in}{1.524930in}}%
\pgfpathlineto{\pgfqpoint{3.308602in}{1.470468in}}%
\pgfpathlineto{\pgfqpoint{3.313677in}{1.421096in}}%
\pgfpathlineto{\pgfqpoint{3.318753in}{1.377220in}}%
\pgfpathlineto{\pgfqpoint{3.321290in}{1.357459in}}%
\pgfpathlineto{\pgfqpoint{3.323828in}{1.339208in}}%
\pgfpathlineto{\pgfqpoint{3.326365in}{1.322505in}}%
\pgfpathlineto{\pgfqpoint{3.328903in}{1.307387in}}%
\pgfpathlineto{\pgfqpoint{3.331441in}{1.293889in}}%
\pgfpathlineto{\pgfqpoint{3.333978in}{1.282042in}}%
\pgfpathlineto{\pgfqpoint{3.336516in}{1.271874in}}%
\pgfpathlineto{\pgfqpoint{3.339054in}{1.263410in}}%
\pgfpathlineto{\pgfqpoint{3.341591in}{1.256673in}}%
\pgfpathlineto{\pgfqpoint{3.344129in}{1.251680in}}%
\pgfpathlineto{\pgfqpoint{3.346667in}{1.248449in}}%
\pgfpathlineto{\pgfqpoint{3.349204in}{1.246990in}}%
\pgfpathlineto{\pgfqpoint{3.351742in}{1.247313in}}%
\pgfpathlineto{\pgfqpoint{3.354279in}{1.249424in}}%
\pgfpathlineto{\pgfqpoint{3.356817in}{1.253325in}}%
\pgfpathlineto{\pgfqpoint{3.359355in}{1.259014in}}%
\pgfpathlineto{\pgfqpoint{3.361892in}{1.266486in}}%
\pgfpathlineto{\pgfqpoint{3.364430in}{1.275733in}}%
\pgfpathlineto{\pgfqpoint{3.366968in}{1.286744in}}%
\pgfpathlineto{\pgfqpoint{3.369505in}{1.299503in}}%
\pgfpathlineto{\pgfqpoint{3.372043in}{1.313991in}}%
\pgfpathlineto{\pgfqpoint{3.374580in}{1.330186in}}%
\pgfpathlineto{\pgfqpoint{3.377118in}{1.348061in}}%
\pgfpathlineto{\pgfqpoint{3.379656in}{1.367589in}}%
\pgfpathlineto{\pgfqpoint{3.382193in}{1.388736in}}%
\pgfpathlineto{\pgfqpoint{3.384731in}{1.411466in}}%
\pgfpathlineto{\pgfqpoint{3.389806in}{1.461514in}}%
\pgfpathlineto{\pgfqpoint{3.394882in}{1.517381in}}%
\pgfpathlineto{\pgfqpoint{3.399957in}{1.578663in}}%
\pgfpathlineto{\pgfqpoint{3.405032in}{1.644909in}}%
\pgfpathlineto{\pgfqpoint{3.410107in}{1.715623in}}%
\pgfpathlineto{\pgfqpoint{3.415183in}{1.790269in}}%
\pgfpathlineto{\pgfqpoint{3.422795in}{1.908343in}}%
\pgfpathlineto{\pgfqpoint{3.430408in}{2.031891in}}%
\pgfpathlineto{\pgfqpoint{3.445634in}{2.286491in}}%
\pgfpathlineto{\pgfqpoint{3.455785in}{2.454306in}}%
\pgfpathlineto{\pgfqpoint{3.463398in}{2.575185in}}%
\pgfpathlineto{\pgfqpoint{3.468473in}{2.652110in}}%
\pgfpathlineto{\pgfqpoint{3.473548in}{2.725332in}}%
\pgfpathlineto{\pgfqpoint{3.478623in}{2.794211in}}%
\pgfpathlineto{\pgfqpoint{3.483699in}{2.858141in}}%
\pgfpathlineto{\pgfqpoint{3.488774in}{2.916550in}}%
\pgfpathlineto{\pgfqpoint{3.493849in}{2.968907in}}%
\pgfpathlineto{\pgfqpoint{3.496387in}{2.992663in}}%
\pgfpathlineto{\pgfqpoint{3.498924in}{3.014728in}}%
\pgfpathlineto{\pgfqpoint{3.501462in}{3.035050in}}%
\pgfpathlineto{\pgfqpoint{3.504000in}{3.053580in}}%
\pgfpathlineto{\pgfqpoint{3.506537in}{3.070272in}}%
\pgfpathlineto{\pgfqpoint{3.509075in}{3.085085in}}%
\pgfpathlineto{\pgfqpoint{3.511613in}{3.097981in}}%
\pgfpathlineto{\pgfqpoint{3.514150in}{3.108926in}}%
\pgfpathlineto{\pgfqpoint{3.516688in}{3.117891in}}%
\pgfpathlineto{\pgfqpoint{3.519226in}{3.124848in}}%
\pgfpathlineto{\pgfqpoint{3.521763in}{3.129778in}}%
\pgfpathlineto{\pgfqpoint{3.524301in}{3.132663in}}%
\pgfpathlineto{\pgfqpoint{3.526838in}{3.133490in}}%
\pgfpathlineto{\pgfqpoint{3.529376in}{3.132251in}}%
\pgfpathlineto{\pgfqpoint{3.531914in}{3.128942in}}%
\pgfpathlineto{\pgfqpoint{3.534451in}{3.123563in}}%
\pgfpathlineto{\pgfqpoint{3.536989in}{3.116120in}}%
\pgfpathlineto{\pgfqpoint{3.539527in}{3.106623in}}%
\pgfpathlineto{\pgfqpoint{3.542064in}{3.095085in}}%
\pgfpathlineto{\pgfqpoint{3.544602in}{3.081526in}}%
\pgfpathlineto{\pgfqpoint{3.547139in}{3.065969in}}%
\pgfpathlineto{\pgfqpoint{3.549677in}{3.048442in}}%
\pgfpathlineto{\pgfqpoint{3.552215in}{3.028978in}}%
\pgfpathlineto{\pgfqpoint{3.554752in}{3.007613in}}%
\pgfpathlineto{\pgfqpoint{3.557290in}{2.984389in}}%
\pgfpathlineto{\pgfqpoint{3.559828in}{2.959352in}}%
\pgfpathlineto{\pgfqpoint{3.562365in}{2.932551in}}%
\pgfpathlineto{\pgfqpoint{3.567441in}{2.873878in}}%
\pgfpathlineto{\pgfqpoint{3.572516in}{2.808854in}}%
\pgfpathlineto{\pgfqpoint{3.577591in}{2.738021in}}%
\pgfpathlineto{\pgfqpoint{3.582666in}{2.661979in}}%
\pgfpathlineto{\pgfqpoint{3.587742in}{2.581385in}}%
\pgfpathlineto{\pgfqpoint{3.595354in}{2.453500in}}%
\pgfpathlineto{\pgfqpoint{3.602967in}{2.319497in}}%
\pgfpathlineto{\pgfqpoint{3.630881in}{1.820436in}}%
\pgfpathlineto{\pgfqpoint{3.635957in}{1.735698in}}%
\pgfpathlineto{\pgfqpoint{3.641032in}{1.654878in}}%
\pgfpathlineto{\pgfqpoint{3.646107in}{1.578760in}}%
\pgfpathlineto{\pgfqpoint{3.651182in}{1.508087in}}%
\pgfpathlineto{\pgfqpoint{3.656258in}{1.443560in}}%
\pgfpathlineto{\pgfqpoint{3.661333in}{1.385827in}}%
\pgfpathlineto{\pgfqpoint{3.663871in}{1.359696in}}%
\pgfpathlineto{\pgfqpoint{3.666408in}{1.335479in}}%
\pgfpathlineto{\pgfqpoint{3.668946in}{1.313241in}}%
\pgfpathlineto{\pgfqpoint{3.671483in}{1.293039in}}%
\pgfpathlineto{\pgfqpoint{3.674021in}{1.274930in}}%
\pgfpathlineto{\pgfqpoint{3.676559in}{1.258962in}}%
\pgfpathlineto{\pgfqpoint{3.679096in}{1.245180in}}%
\pgfpathlineto{\pgfqpoint{3.681634in}{1.233624in}}%
\pgfpathlineto{\pgfqpoint{3.684172in}{1.224328in}}%
\pgfpathlineto{\pgfqpoint{3.686709in}{1.217321in}}%
\pgfpathlineto{\pgfqpoint{3.689247in}{1.212627in}}%
\pgfpathlineto{\pgfqpoint{3.691785in}{1.210265in}}%
\pgfpathlineto{\pgfqpoint{3.694322in}{1.210246in}}%
\pgfpathlineto{\pgfqpoint{3.696860in}{1.212579in}}%
\pgfpathlineto{\pgfqpoint{3.699397in}{1.217263in}}%
\pgfpathlineto{\pgfqpoint{3.701935in}{1.224295in}}%
\pgfpathlineto{\pgfqpoint{3.704473in}{1.233664in}}%
\pgfpathlineto{\pgfqpoint{3.707010in}{1.245355in}}%
\pgfpathlineto{\pgfqpoint{3.709548in}{1.259345in}}%
\pgfpathlineto{\pgfqpoint{3.712086in}{1.275608in}}%
\pgfpathlineto{\pgfqpoint{3.714623in}{1.294109in}}%
\pgfpathlineto{\pgfqpoint{3.717161in}{1.314810in}}%
\pgfpathlineto{\pgfqpoint{3.719698in}{1.337666in}}%
\pgfpathlineto{\pgfqpoint{3.722236in}{1.362627in}}%
\pgfpathlineto{\pgfqpoint{3.724774in}{1.389637in}}%
\pgfpathlineto{\pgfqpoint{3.727311in}{1.418637in}}%
\pgfpathlineto{\pgfqpoint{3.732387in}{1.482332in}}%
\pgfpathlineto{\pgfqpoint{3.737462in}{1.553123in}}%
\pgfpathlineto{\pgfqpoint{3.742537in}{1.630343in}}%
\pgfpathlineto{\pgfqpoint{3.747612in}{1.713258in}}%
\pgfpathlineto{\pgfqpoint{3.752688in}{1.801066in}}%
\pgfpathlineto{\pgfqpoint{3.760301in}{1.940064in}}%
\pgfpathlineto{\pgfqpoint{3.770451in}{2.134139in}}%
\pgfpathlineto{\pgfqpoint{3.785677in}{2.428045in}}%
\pgfpathlineto{\pgfqpoint{3.793290in}{2.568908in}}%
\pgfpathlineto{\pgfqpoint{3.798365in}{2.658286in}}%
\pgfpathlineto{\pgfqpoint{3.803440in}{2.742928in}}%
\pgfpathlineto{\pgfqpoint{3.808516in}{2.821927in}}%
\pgfpathlineto{\pgfqpoint{3.813591in}{2.894426in}}%
\pgfpathlineto{\pgfqpoint{3.818666in}{2.959628in}}%
\pgfpathlineto{\pgfqpoint{3.821204in}{2.989263in}}%
\pgfpathlineto{\pgfqpoint{3.823741in}{3.016808in}}%
\pgfpathlineto{\pgfqpoint{3.826279in}{3.042184in}}%
\pgfpathlineto{\pgfqpoint{3.828817in}{3.065317in}}%
\pgfpathlineto{\pgfqpoint{3.831354in}{3.086142in}}%
\pgfpathlineto{\pgfqpoint{3.833892in}{3.104596in}}%
\pgfpathlineto{\pgfqpoint{3.836430in}{3.120623in}}%
\pgfpathlineto{\pgfqpoint{3.838967in}{3.134175in}}%
\pgfpathlineto{\pgfqpoint{3.841505in}{3.145207in}}%
\pgfpathlineto{\pgfqpoint{3.844042in}{3.153685in}}%
\pgfpathlineto{\pgfqpoint{3.846580in}{3.159578in}}%
\pgfpathlineto{\pgfqpoint{3.849118in}{3.162863in}}%
\pgfpathlineto{\pgfqpoint{3.851655in}{3.163524in}}%
\pgfpathlineto{\pgfqpoint{3.854193in}{3.161553in}}%
\pgfpathlineto{\pgfqpoint{3.856731in}{3.156947in}}%
\pgfpathlineto{\pgfqpoint{3.859268in}{3.149711in}}%
\pgfpathlineto{\pgfqpoint{3.861806in}{3.139858in}}%
\pgfpathlineto{\pgfqpoint{3.864344in}{3.127407in}}%
\pgfpathlineto{\pgfqpoint{3.866881in}{3.112384in}}%
\pgfpathlineto{\pgfqpoint{3.869419in}{3.094823in}}%
\pgfpathlineto{\pgfqpoint{3.871956in}{3.074765in}}%
\pgfpathlineto{\pgfqpoint{3.874494in}{3.052257in}}%
\pgfpathlineto{\pgfqpoint{3.877032in}{3.027354in}}%
\pgfpathlineto{\pgfqpoint{3.879569in}{3.000118in}}%
\pgfpathlineto{\pgfqpoint{3.882107in}{2.970615in}}%
\pgfpathlineto{\pgfqpoint{3.884645in}{2.938920in}}%
\pgfpathlineto{\pgfqpoint{3.889720in}{2.869285in}}%
\pgfpathlineto{\pgfqpoint{3.894795in}{2.791930in}}%
\pgfpathlineto{\pgfqpoint{3.899870in}{2.707666in}}%
\pgfpathlineto{\pgfqpoint{3.904946in}{2.617386in}}%
\pgfpathlineto{\pgfqpoint{3.910021in}{2.522056in}}%
\pgfpathlineto{\pgfqpoint{3.917634in}{2.371868in}}%
\pgfpathlineto{\pgfqpoint{3.930322in}{2.111624in}}%
\pgfpathlineto{\pgfqpoint{3.940473in}{1.904933in}}%
\pgfpathlineto{\pgfqpoint{3.948085in}{1.756947in}}%
\pgfpathlineto{\pgfqpoint{3.953161in}{1.663783in}}%
\pgfpathlineto{\pgfqpoint{3.958236in}{1.576288in}}%
\pgfpathlineto{\pgfqpoint{3.963311in}{1.495493in}}%
\pgfpathlineto{\pgfqpoint{3.968386in}{1.422360in}}%
\pgfpathlineto{\pgfqpoint{3.970924in}{1.388946in}}%
\pgfpathlineto{\pgfqpoint{3.973462in}{1.357772in}}%
\pgfpathlineto{\pgfqpoint{3.975999in}{1.328932in}}%
\pgfpathlineto{\pgfqpoint{3.978537in}{1.302518in}}%
\pgfpathlineto{\pgfqpoint{3.981075in}{1.278611in}}%
\pgfpathlineto{\pgfqpoint{3.983612in}{1.257287in}}%
\pgfpathlineto{\pgfqpoint{3.986150in}{1.238615in}}%
\pgfpathlineto{\pgfqpoint{3.988688in}{1.222657in}}%
\pgfpathlineto{\pgfqpoint{3.991225in}{1.209466in}}%
\pgfpathlineto{\pgfqpoint{3.993763in}{1.199088in}}%
\pgfpathlineto{\pgfqpoint{3.996300in}{1.191560in}}%
\pgfpathlineto{\pgfqpoint{3.998838in}{1.186911in}}%
\pgfpathlineto{\pgfqpoint{4.001376in}{1.185164in}}%
\pgfpathlineto{\pgfqpoint{4.003913in}{1.186330in}}%
\pgfpathlineto{\pgfqpoint{4.006451in}{1.190414in}}%
\pgfpathlineto{\pgfqpoint{4.008989in}{1.197411in}}%
\pgfpathlineto{\pgfqpoint{4.011526in}{1.207308in}}%
\pgfpathlineto{\pgfqpoint{4.014064in}{1.220085in}}%
\pgfpathlineto{\pgfqpoint{4.016601in}{1.235709in}}%
\pgfpathlineto{\pgfqpoint{4.019139in}{1.254143in}}%
\pgfpathlineto{\pgfqpoint{4.021677in}{1.275338in}}%
\pgfpathlineto{\pgfqpoint{4.024214in}{1.299241in}}%
\pgfpathlineto{\pgfqpoint{4.026752in}{1.325785in}}%
\pgfpathlineto{\pgfqpoint{4.029290in}{1.354899in}}%
\pgfpathlineto{\pgfqpoint{4.031827in}{1.386503in}}%
\pgfpathlineto{\pgfqpoint{4.034365in}{1.420508in}}%
\pgfpathlineto{\pgfqpoint{4.039440in}{1.495336in}}%
\pgfpathlineto{\pgfqpoint{4.044515in}{1.578530in}}%
\pgfpathlineto{\pgfqpoint{4.049591in}{1.669133in}}%
\pgfpathlineto{\pgfqpoint{4.054666in}{1.766091in}}%
\pgfpathlineto{\pgfqpoint{4.059741in}{1.868261in}}%
\pgfpathlineto{\pgfqpoint{4.067354in}{2.028618in}}%
\pgfpathlineto{\pgfqpoint{4.090193in}{2.519542in}}%
\pgfpathlineto{\pgfqpoint{4.095268in}{2.622024in}}%
\pgfpathlineto{\pgfqpoint{4.100343in}{2.719251in}}%
\pgfpathlineto{\pgfqpoint{4.105419in}{2.809987in}}%
\pgfpathlineto{\pgfqpoint{4.110494in}{2.893070in}}%
\pgfpathlineto{\pgfqpoint{4.115569in}{2.967424in}}%
\pgfpathlineto{\pgfqpoint{4.118107in}{3.001017in}}%
\pgfpathlineto{\pgfqpoint{4.120644in}{3.032071in}}%
\pgfpathlineto{\pgfqpoint{4.123182in}{3.060482in}}%
\pgfpathlineto{\pgfqpoint{4.125720in}{3.086153in}}%
\pgfpathlineto{\pgfqpoint{4.128257in}{3.108997in}}%
\pgfpathlineto{\pgfqpoint{4.130795in}{3.128934in}}%
\pgfpathlineto{\pgfqpoint{4.133333in}{3.145895in}}%
\pgfpathlineto{\pgfqpoint{4.135870in}{3.159818in}}%
\pgfpathlineto{\pgfqpoint{4.138408in}{3.170651in}}%
\pgfpathlineto{\pgfqpoint{4.140945in}{3.178354in}}%
\pgfpathlineto{\pgfqpoint{4.143483in}{3.182892in}}%
\pgfpathlineto{\pgfqpoint{4.146021in}{3.184245in}}%
\pgfpathlineto{\pgfqpoint{4.148558in}{3.182399in}}%
\pgfpathlineto{\pgfqpoint{4.151096in}{3.177354in}}%
\pgfpathlineto{\pgfqpoint{4.153634in}{3.169117in}}%
\pgfpathlineto{\pgfqpoint{4.156171in}{3.157706in}}%
\pgfpathlineto{\pgfqpoint{4.158709in}{3.143151in}}%
\pgfpathlineto{\pgfqpoint{4.161247in}{3.125491in}}%
\pgfpathlineto{\pgfqpoint{4.163784in}{3.104774in}}%
\pgfpathlineto{\pgfqpoint{4.166322in}{3.081061in}}%
\pgfpathlineto{\pgfqpoint{4.168859in}{3.054419in}}%
\pgfpathlineto{\pgfqpoint{4.171397in}{3.024929in}}%
\pgfpathlineto{\pgfqpoint{4.173935in}{2.992678in}}%
\pgfpathlineto{\pgfqpoint{4.176472in}{2.957764in}}%
\pgfpathlineto{\pgfqpoint{4.179010in}{2.920294in}}%
\pgfpathlineto{\pgfqpoint{4.184085in}{2.838157in}}%
\pgfpathlineto{\pgfqpoint{4.189160in}{2.747288in}}%
\pgfpathlineto{\pgfqpoint{4.194236in}{2.648833in}}%
\pgfpathlineto{\pgfqpoint{4.199311in}{2.544044in}}%
\pgfpathlineto{\pgfqpoint{4.206924in}{2.377950in}}%
\pgfpathlineto{\pgfqpoint{4.219612in}{2.089503in}}%
\pgfpathlineto{\pgfqpoint{4.229763in}{1.861968in}}%
\pgfpathlineto{\pgfqpoint{4.234838in}{1.753480in}}%
\pgfpathlineto{\pgfqpoint{4.239913in}{1.650489in}}%
\pgfpathlineto{\pgfqpoint{4.244988in}{1.554405in}}%
\pgfpathlineto{\pgfqpoint{4.250064in}{1.466555in}}%
\pgfpathlineto{\pgfqpoint{4.255139in}{1.388166in}}%
\pgfpathlineto{\pgfqpoint{4.257677in}{1.352870in}}%
\pgfpathlineto{\pgfqpoint{4.260214in}{1.320343in}}%
\pgfpathlineto{\pgfqpoint{4.262752in}{1.290704in}}%
\pgfpathlineto{\pgfqpoint{4.265289in}{1.264059in}}%
\pgfpathlineto{\pgfqpoint{4.267827in}{1.240507in}}%
\pgfpathlineto{\pgfqpoint{4.270365in}{1.220135in}}%
\pgfpathlineto{\pgfqpoint{4.272902in}{1.203020in}}%
\pgfpathlineto{\pgfqpoint{4.275440in}{1.189229in}}%
\pgfpathlineto{\pgfqpoint{4.277978in}{1.178816in}}%
\pgfpathlineto{\pgfqpoint{4.280515in}{1.171825in}}%
\pgfpathlineto{\pgfqpoint{4.283053in}{1.168288in}}%
\pgfpathlineto{\pgfqpoint{4.285591in}{1.168226in}}%
\pgfpathlineto{\pgfqpoint{4.288128in}{1.171646in}}%
\pgfpathlineto{\pgfqpoint{4.290666in}{1.178546in}}%
\pgfpathlineto{\pgfqpoint{4.293203in}{1.188909in}}%
\pgfpathlineto{\pgfqpoint{4.295741in}{1.202709in}}%
\pgfpathlineto{\pgfqpoint{4.298279in}{1.219904in}}%
\pgfpathlineto{\pgfqpoint{4.300816in}{1.240443in}}%
\pgfpathlineto{\pgfqpoint{4.303354in}{1.264263in}}%
\pgfpathlineto{\pgfqpoint{4.305892in}{1.291288in}}%
\pgfpathlineto{\pgfqpoint{4.308429in}{1.321432in}}%
\pgfpathlineto{\pgfqpoint{4.310967in}{1.354597in}}%
\pgfpathlineto{\pgfqpoint{4.313504in}{1.390675in}}%
\pgfpathlineto{\pgfqpoint{4.316042in}{1.429545in}}%
\pgfpathlineto{\pgfqpoint{4.321117in}{1.515135in}}%
\pgfpathlineto{\pgfqpoint{4.326193in}{1.610214in}}%
\pgfpathlineto{\pgfqpoint{4.331268in}{1.713489in}}%
\pgfpathlineto{\pgfqpoint{4.336343in}{1.823539in}}%
\pgfpathlineto{\pgfqpoint{4.343956in}{1.997953in}}%
\pgfpathlineto{\pgfqpoint{4.366795in}{2.535885in}}%
\pgfpathlineto{\pgfqpoint{4.371870in}{2.647419in}}%
\pgfpathlineto{\pgfqpoint{4.376945in}{2.752418in}}%
\pgfpathlineto{\pgfqpoint{4.382021in}{2.849335in}}%
\pgfpathlineto{\pgfqpoint{4.387096in}{2.936728in}}%
\pgfpathlineto{\pgfqpoint{4.389633in}{2.976437in}}%
\pgfpathlineto{\pgfqpoint{4.392171in}{3.013282in}}%
\pgfpathlineto{\pgfqpoint{4.394709in}{3.047124in}}%
\pgfpathlineto{\pgfqpoint{4.397246in}{3.077833in}}%
\pgfpathlineto{\pgfqpoint{4.399784in}{3.105288in}}%
\pgfpathlineto{\pgfqpoint{4.402322in}{3.129381in}}%
\pgfpathlineto{\pgfqpoint{4.404859in}{3.150018in}}%
\pgfpathlineto{\pgfqpoint{4.407397in}{3.167114in}}%
\pgfpathlineto{\pgfqpoint{4.409935in}{3.180599in}}%
\pgfpathlineto{\pgfqpoint{4.412472in}{3.190414in}}%
\pgfpathlineto{\pgfqpoint{4.415010in}{3.196516in}}%
\pgfpathlineto{\pgfqpoint{4.417547in}{3.198873in}}%
\pgfpathlineto{\pgfqpoint{4.420085in}{3.197469in}}%
\pgfpathlineto{\pgfqpoint{4.422623in}{3.192300in}}%
\pgfpathlineto{\pgfqpoint{4.425160in}{3.183377in}}%
\pgfpathlineto{\pgfqpoint{4.427698in}{3.170725in}}%
\pgfpathlineto{\pgfqpoint{4.430236in}{3.154383in}}%
\pgfpathlineto{\pgfqpoint{4.432773in}{3.134402in}}%
\pgfpathlineto{\pgfqpoint{4.435311in}{3.110849in}}%
\pgfpathlineto{\pgfqpoint{4.437848in}{3.083805in}}%
\pgfpathlineto{\pgfqpoint{4.440386in}{3.053363in}}%
\pgfpathlineto{\pgfqpoint{4.442924in}{3.019628in}}%
\pgfpathlineto{\pgfqpoint{4.445461in}{2.982720in}}%
\pgfpathlineto{\pgfqpoint{4.447999in}{2.942771in}}%
\pgfpathlineto{\pgfqpoint{4.450537in}{2.899923in}}%
\pgfpathlineto{\pgfqpoint{4.455612in}{2.806161in}}%
\pgfpathlineto{\pgfqpoint{4.460687in}{2.702800in}}%
\pgfpathlineto{\pgfqpoint{4.465762in}{2.591361in}}%
\pgfpathlineto{\pgfqpoint{4.470838in}{2.473494in}}%
\pgfpathlineto{\pgfqpoint{4.478451in}{2.288525in}}%
\pgfpathlineto{\pgfqpoint{4.496214in}{1.851796in}}%
\pgfpathlineto{\pgfqpoint{4.501289in}{1.734324in}}%
\pgfpathlineto{\pgfqpoint{4.506365in}{1.623523in}}%
\pgfpathlineto{\pgfqpoint{4.511440in}{1.521137in}}%
\pgfpathlineto{\pgfqpoint{4.516515in}{1.428789in}}%
\pgfpathlineto{\pgfqpoint{4.519053in}{1.386848in}}%
\pgfpathlineto{\pgfqpoint{4.521590in}{1.347960in}}%
\pgfpathlineto{\pgfqpoint{4.524128in}{1.312281in}}%
\pgfpathlineto{\pgfqpoint{4.526666in}{1.279960in}}%
\pgfpathlineto{\pgfqpoint{4.529203in}{1.251128in}}%
\pgfpathlineto{\pgfqpoint{4.531741in}{1.225907in}}%
\pgfpathlineto{\pgfqpoint{4.534279in}{1.204403in}}%
\pgfpathlineto{\pgfqpoint{4.536816in}{1.186710in}}%
\pgfpathlineto{\pgfqpoint{4.539354in}{1.172903in}}%
\pgfpathlineto{\pgfqpoint{4.541891in}{1.163047in}}%
\pgfpathlineto{\pgfqpoint{4.544429in}{1.157188in}}%
\pgfpathlineto{\pgfqpoint{4.546967in}{1.155358in}}%
\pgfpathlineto{\pgfqpoint{4.549504in}{1.157574in}}%
\pgfpathlineto{\pgfqpoint{4.552042in}{1.163834in}}%
\pgfpathlineto{\pgfqpoint{4.554580in}{1.174124in}}%
\pgfpathlineto{\pgfqpoint{4.557117in}{1.188410in}}%
\pgfpathlineto{\pgfqpoint{4.559655in}{1.206644in}}%
\pgfpathlineto{\pgfqpoint{4.562192in}{1.228763in}}%
\pgfpathlineto{\pgfqpoint{4.564730in}{1.254686in}}%
\pgfpathlineto{\pgfqpoint{4.567268in}{1.284319in}}%
\pgfpathlineto{\pgfqpoint{4.569805in}{1.317550in}}%
\pgfpathlineto{\pgfqpoint{4.572343in}{1.354255in}}%
\pgfpathlineto{\pgfqpoint{4.574881in}{1.394295in}}%
\pgfpathlineto{\pgfqpoint{4.577418in}{1.437515in}}%
\pgfpathlineto{\pgfqpoint{4.582494in}{1.532819in}}%
\pgfpathlineto{\pgfqpoint{4.587569in}{1.638685in}}%
\pgfpathlineto{\pgfqpoint{4.592644in}{1.753452in}}%
\pgfpathlineto{\pgfqpoint{4.597719in}{1.875301in}}%
\pgfpathlineto{\pgfqpoint{4.605332in}{2.067073in}}%
\pgfpathlineto{\pgfqpoint{4.620558in}{2.457243in}}%
\pgfpathlineto{\pgfqpoint{4.625633in}{2.581467in}}%
\pgfpathlineto{\pgfqpoint{4.630709in}{2.699302in}}%
\pgfpathlineto{\pgfqpoint{4.635784in}{2.808780in}}%
\pgfpathlineto{\pgfqpoint{4.640859in}{2.908056in}}%
\pgfpathlineto{\pgfqpoint{4.643397in}{2.953335in}}%
\pgfpathlineto{\pgfqpoint{4.645934in}{2.995446in}}%
\pgfpathlineto{\pgfqpoint{4.648472in}{3.034207in}}%
\pgfpathlineto{\pgfqpoint{4.651010in}{3.069449in}}%
\pgfpathlineto{\pgfqpoint{4.653547in}{3.101020in}}%
\pgfpathlineto{\pgfqpoint{4.656085in}{3.128779in}}%
\pgfpathlineto{\pgfqpoint{4.658622in}{3.152603in}}%
\pgfpathlineto{\pgfqpoint{4.661160in}{3.172385in}}%
\pgfpathlineto{\pgfqpoint{4.663698in}{3.188032in}}%
\pgfpathlineto{\pgfqpoint{4.666235in}{3.199472in}}%
\pgfpathlineto{\pgfqpoint{4.668773in}{3.206647in}}%
\pgfpathlineto{\pgfqpoint{4.671311in}{3.209519in}}%
\pgfpathlineto{\pgfqpoint{4.673848in}{3.208066in}}%
\pgfpathlineto{\pgfqpoint{4.676386in}{3.202286in}}%
\pgfpathlineto{\pgfqpoint{4.678924in}{3.192194in}}%
\pgfpathlineto{\pgfqpoint{4.681461in}{3.177824in}}%
\pgfpathlineto{\pgfqpoint{4.683999in}{3.159228in}}%
\pgfpathlineto{\pgfqpoint{4.686536in}{3.136476in}}%
\pgfpathlineto{\pgfqpoint{4.689074in}{3.109655in}}%
\pgfpathlineto{\pgfqpoint{4.691612in}{3.078872in}}%
\pgfpathlineto{\pgfqpoint{4.694149in}{3.044249in}}%
\pgfpathlineto{\pgfqpoint{4.696687in}{3.005924in}}%
\pgfpathlineto{\pgfqpoint{4.699225in}{2.964055in}}%
\pgfpathlineto{\pgfqpoint{4.701762in}{2.918811in}}%
\pgfpathlineto{\pgfqpoint{4.706838in}{2.818961in}}%
\pgfpathlineto{\pgfqpoint{4.711913in}{2.708029in}}%
\pgfpathlineto{\pgfqpoint{4.716988in}{2.587872in}}%
\pgfpathlineto{\pgfqpoint{4.722063in}{2.460518in}}%
\pgfpathlineto{\pgfqpoint{4.729676in}{2.260758in}}%
\pgfpathlineto{\pgfqpoint{4.744902in}{1.858330in}}%
\pgfpathlineto{\pgfqpoint{4.749977in}{1.731886in}}%
\pgfpathlineto{\pgfqpoint{4.755053in}{1.613067in}}%
\pgfpathlineto{\pgfqpoint{4.760128in}{1.503984in}}%
\pgfpathlineto{\pgfqpoint{4.762665in}{1.453710in}}%
\pgfpathlineto{\pgfqpoint{4.765203in}{1.406589in}}%
\pgfpathlineto{\pgfqpoint{4.767741in}{1.362834in}}%
\pgfpathlineto{\pgfqpoint{4.770278in}{1.322644in}}%
\pgfpathlineto{\pgfqpoint{4.772816in}{1.286204in}}%
\pgfpathlineto{\pgfqpoint{4.775354in}{1.253682in}}%
\pgfpathlineto{\pgfqpoint{4.777891in}{1.225230in}}%
\pgfpathlineto{\pgfqpoint{4.780429in}{1.200981in}}%
\pgfpathlineto{\pgfqpoint{4.782966in}{1.181053in}}%
\pgfpathlineto{\pgfqpoint{4.785504in}{1.165542in}}%
\pgfpathlineto{\pgfqpoint{4.788042in}{1.154525in}}%
\pgfpathlineto{\pgfqpoint{4.790579in}{1.148061in}}%
\pgfpathlineto{\pgfqpoint{4.793117in}{1.146187in}}%
\pgfpathlineto{\pgfqpoint{4.795655in}{1.148921in}}%
\pgfpathlineto{\pgfqpoint{4.798192in}{1.156260in}}%
\pgfpathlineto{\pgfqpoint{4.800730in}{1.168180in}}%
\pgfpathlineto{\pgfqpoint{4.803268in}{1.184637in}}%
\pgfpathlineto{\pgfqpoint{4.805805in}{1.205566in}}%
\pgfpathlineto{\pgfqpoint{4.808343in}{1.230882in}}%
\pgfpathlineto{\pgfqpoint{4.810880in}{1.260480in}}%
\pgfpathlineto{\pgfqpoint{4.813418in}{1.294236in}}%
\pgfpathlineto{\pgfqpoint{4.815956in}{1.332004in}}%
\pgfpathlineto{\pgfqpoint{4.818493in}{1.373624in}}%
\pgfpathlineto{\pgfqpoint{4.821031in}{1.418914in}}%
\pgfpathlineto{\pgfqpoint{4.823569in}{1.467678in}}%
\pgfpathlineto{\pgfqpoint{4.828644in}{1.574755in}}%
\pgfpathlineto{\pgfqpoint{4.833719in}{1.692963in}}%
\pgfpathlineto{\pgfqpoint{4.838794in}{1.820198in}}%
\pgfpathlineto{\pgfqpoint{4.846407in}{2.022945in}}%
\pgfpathlineto{\pgfqpoint{4.864171in}{2.507988in}}%
\pgfpathlineto{\pgfqpoint{4.869246in}{2.638182in}}%
\pgfpathlineto{\pgfqpoint{4.874321in}{2.760105in}}%
\pgfpathlineto{\pgfqpoint{4.879397in}{2.871466in}}%
\pgfpathlineto{\pgfqpoint{4.881934in}{2.922519in}}%
\pgfpathlineto{\pgfqpoint{4.884472in}{2.970157in}}%
\pgfpathlineto{\pgfqpoint{4.887009in}{3.014151in}}%
\pgfpathlineto{\pgfqpoint{4.889547in}{3.054291in}}%
\pgfpathlineto{\pgfqpoint{4.892085in}{3.090380in}}%
\pgfpathlineto{\pgfqpoint{4.894622in}{3.122243in}}%
\pgfpathlineto{\pgfqpoint{4.897160in}{3.149721in}}%
\pgfpathlineto{\pgfqpoint{4.899698in}{3.172679in}}%
\pgfpathlineto{\pgfqpoint{4.902235in}{3.191000in}}%
\pgfpathlineto{\pgfqpoint{4.904773in}{3.204588in}}%
\pgfpathlineto{\pgfqpoint{4.907310in}{3.213370in}}%
\pgfpathlineto{\pgfqpoint{4.909848in}{3.217297in}}%
\pgfpathlineto{\pgfqpoint{4.912386in}{3.216340in}}%
\pgfpathlineto{\pgfqpoint{4.914923in}{3.210495in}}%
\pgfpathlineto{\pgfqpoint{4.917461in}{3.199779in}}%
\pgfpathlineto{\pgfqpoint{4.919999in}{3.184234in}}%
\pgfpathlineto{\pgfqpoint{4.922536in}{3.163925in}}%
\pgfpathlineto{\pgfqpoint{4.925074in}{3.138938in}}%
\pgfpathlineto{\pgfqpoint{4.927612in}{3.109384in}}%
\pgfpathlineto{\pgfqpoint{4.930149in}{3.075393in}}%
\pgfpathlineto{\pgfqpoint{4.932687in}{3.037120in}}%
\pgfpathlineto{\pgfqpoint{4.935224in}{2.994738in}}%
\pgfpathlineto{\pgfqpoint{4.937762in}{2.948441in}}%
\pgfpathlineto{\pgfqpoint{4.940300in}{2.898444in}}%
\pgfpathlineto{\pgfqpoint{4.945375in}{2.788292in}}%
\pgfpathlineto{\pgfqpoint{4.950450in}{2.666337in}}%
\pgfpathlineto{\pgfqpoint{4.955525in}{2.534877in}}%
\pgfpathlineto{\pgfqpoint{4.963138in}{2.325358in}}%
\pgfpathlineto{\pgfqpoint{4.980902in}{1.826435in}}%
\pgfpathlineto{\pgfqpoint{4.985977in}{1.693764in}}%
\pgfpathlineto{\pgfqpoint{4.991052in}{1.570386in}}%
\pgfpathlineto{\pgfqpoint{4.996128in}{1.458748in}}%
\pgfpathlineto{\pgfqpoint{4.998665in}{1.408038in}}%
\pgfpathlineto{\pgfqpoint{5.001203in}{1.361081in}}%
\pgfpathlineto{\pgfqpoint{5.003741in}{1.318114in}}%
\pgfpathlineto{\pgfqpoint{5.006278in}{1.279356in}}%
\pgfpathlineto{\pgfqpoint{5.008816in}{1.245005in}}%
\pgfpathlineto{\pgfqpoint{5.011353in}{1.215240in}}%
\pgfpathlineto{\pgfqpoint{5.013891in}{1.190216in}}%
\pgfpathlineto{\pgfqpoint{5.016429in}{1.170066in}}%
\pgfpathlineto{\pgfqpoint{5.018966in}{1.154897in}}%
\pgfpathlineto{\pgfqpoint{5.021504in}{1.144795in}}%
\pgfpathlineto{\pgfqpoint{5.024042in}{1.139820in}}%
\pgfpathlineto{\pgfqpoint{5.026579in}{1.140005in}}%
\pgfpathlineto{\pgfqpoint{5.029117in}{1.145359in}}%
\pgfpathlineto{\pgfqpoint{5.031654in}{1.155865in}}%
\pgfpathlineto{\pgfqpoint{5.034192in}{1.171480in}}%
\pgfpathlineto{\pgfqpoint{5.036730in}{1.192135in}}%
\pgfpathlineto{\pgfqpoint{5.039267in}{1.217738in}}%
\pgfpathlineto{\pgfqpoint{5.041805in}{1.248169in}}%
\pgfpathlineto{\pgfqpoint{5.044343in}{1.283284in}}%
\pgfpathlineto{\pgfqpoint{5.046880in}{1.322917in}}%
\pgfpathlineto{\pgfqpoint{5.049418in}{1.366876in}}%
\pgfpathlineto{\pgfqpoint{5.051956in}{1.414949in}}%
\pgfpathlineto{\pgfqpoint{5.054493in}{1.466902in}}%
\pgfpathlineto{\pgfqpoint{5.059568in}{1.581411in}}%
\pgfpathlineto{\pgfqpoint{5.064644in}{1.708146in}}%
\pgfpathlineto{\pgfqpoint{5.069719in}{1.844590in}}%
\pgfpathlineto{\pgfqpoint{5.077332in}{2.061448in}}%
\pgfpathlineto{\pgfqpoint{5.092558in}{2.502561in}}%
\pgfpathlineto{\pgfqpoint{5.097633in}{2.640776in}}%
\pgfpathlineto{\pgfqpoint{5.102708in}{2.769730in}}%
\pgfpathlineto{\pgfqpoint{5.107783in}{2.886741in}}%
\pgfpathlineto{\pgfqpoint{5.110321in}{2.939989in}}%
\pgfpathlineto{\pgfqpoint{5.112859in}{2.989352in}}%
\pgfpathlineto{\pgfqpoint{5.115396in}{3.034567in}}%
\pgfpathlineto{\pgfqpoint{5.117934in}{3.075394in}}%
\pgfpathlineto{\pgfqpoint{5.120472in}{3.111612in}}%
\pgfpathlineto{\pgfqpoint{5.123009in}{3.143025in}}%
\pgfpathlineto{\pgfqpoint{5.125547in}{3.169462in}}%
\pgfpathlineto{\pgfqpoint{5.128084in}{3.190776in}}%
\pgfpathlineto{\pgfqpoint{5.130622in}{3.206848in}}%
\pgfpathlineto{\pgfqpoint{5.133160in}{3.217583in}}%
\pgfpathlineto{\pgfqpoint{5.135697in}{3.222917in}}%
\pgfpathlineto{\pgfqpoint{5.138235in}{3.222812in}}%
\pgfpathlineto{\pgfqpoint{5.140773in}{3.217259in}}%
\pgfpathlineto{\pgfqpoint{5.143310in}{3.206276in}}%
\pgfpathlineto{\pgfqpoint{5.145848in}{3.189912in}}%
\pgfpathlineto{\pgfqpoint{5.148386in}{3.168243in}}%
\pgfpathlineto{\pgfqpoint{5.150923in}{3.141373in}}%
\pgfpathlineto{\pgfqpoint{5.153461in}{3.109434in}}%
\pgfpathlineto{\pgfqpoint{5.155998in}{3.072585in}}%
\pgfpathlineto{\pgfqpoint{5.158536in}{3.031010in}}%
\pgfpathlineto{\pgfqpoint{5.161074in}{2.984922in}}%
\pgfpathlineto{\pgfqpoint{5.163611in}{2.934555in}}%
\pgfpathlineto{\pgfqpoint{5.166149in}{2.880168in}}%
\pgfpathlineto{\pgfqpoint{5.171224in}{2.760476in}}%
\pgfpathlineto{\pgfqpoint{5.176300in}{2.628328in}}%
\pgfpathlineto{\pgfqpoint{5.181375in}{2.486486in}}%
\pgfpathlineto{\pgfqpoint{5.188988in}{2.262122in}}%
\pgfpathlineto{\pgfqpoint{5.201676in}{1.883878in}}%
\pgfpathlineto{\pgfqpoint{5.206751in}{1.740537in}}%
\pgfpathlineto{\pgfqpoint{5.211826in}{1.606461in}}%
\pgfpathlineto{\pgfqpoint{5.216902in}{1.484576in}}%
\pgfpathlineto{\pgfqpoint{5.219439in}{1.429053in}}%
\pgfpathlineto{\pgfqpoint{5.221977in}{1.377559in}}%
\pgfpathlineto{\pgfqpoint{5.224515in}{1.330380in}}%
\pgfpathlineto{\pgfqpoint{5.227052in}{1.287779in}}%
\pgfpathlineto{\pgfqpoint{5.229590in}{1.249997in}}%
\pgfpathlineto{\pgfqpoint{5.232127in}{1.217247in}}%
\pgfpathlineto{\pgfqpoint{5.234665in}{1.189716in}}%
\pgfpathlineto{\pgfqpoint{5.237203in}{1.167564in}}%
\pgfpathlineto{\pgfqpoint{5.239740in}{1.150921in}}%
\pgfpathlineto{\pgfqpoint{5.242278in}{1.139887in}}%
\pgfpathlineto{\pgfqpoint{5.244816in}{1.134533in}}%
\pgfpathlineto{\pgfqpoint{5.247353in}{1.134898in}}%
\pgfpathlineto{\pgfqpoint{5.249891in}{1.140989in}}%
\pgfpathlineto{\pgfqpoint{5.252428in}{1.152784in}}%
\pgfpathlineto{\pgfqpoint{5.254966in}{1.170227in}}%
\pgfpathlineto{\pgfqpoint{5.257504in}{1.193233in}}%
\pgfpathlineto{\pgfqpoint{5.260041in}{1.221684in}}%
\pgfpathlineto{\pgfqpoint{5.262579in}{1.255434in}}%
\pgfpathlineto{\pgfqpoint{5.265117in}{1.294305in}}%
\pgfpathlineto{\pgfqpoint{5.267654in}{1.338091in}}%
\pgfpathlineto{\pgfqpoint{5.270192in}{1.386560in}}%
\pgfpathlineto{\pgfqpoint{5.272730in}{1.439450in}}%
\pgfpathlineto{\pgfqpoint{5.275267in}{1.496476in}}%
\pgfpathlineto{\pgfqpoint{5.280342in}{1.621680in}}%
\pgfpathlineto{\pgfqpoint{5.285418in}{1.759443in}}%
\pgfpathlineto{\pgfqpoint{5.290493in}{1.906742in}}%
\pgfpathlineto{\pgfqpoint{5.298106in}{2.138408in}}%
\pgfpathlineto{\pgfqpoint{5.308256in}{2.449187in}}%
\pgfpathlineto{\pgfqpoint{5.313332in}{2.597533in}}%
\pgfpathlineto{\pgfqpoint{5.318407in}{2.736663in}}%
\pgfpathlineto{\pgfqpoint{5.323482in}{2.863406in}}%
\pgfpathlineto{\pgfqpoint{5.326020in}{2.921211in}}%
\pgfpathlineto{\pgfqpoint{5.328557in}{2.974852in}}%
\pgfpathlineto{\pgfqpoint{5.331095in}{3.024018in}}%
\pgfpathlineto{\pgfqpoint{5.333633in}{3.068421in}}%
\pgfpathlineto{\pgfqpoint{5.336170in}{3.107802in}}%
\pgfpathlineto{\pgfqpoint{5.338708in}{3.141927in}}%
\pgfpathlineto{\pgfqpoint{5.341246in}{3.170593in}}%
\pgfpathlineto{\pgfqpoint{5.343783in}{3.193627in}}%
\pgfpathlineto{\pgfqpoint{5.346321in}{3.210889in}}%
\pgfpathlineto{\pgfqpoint{5.348859in}{3.222269in}}%
\pgfpathlineto{\pgfqpoint{5.351396in}{3.227692in}}%
\pgfpathlineto{\pgfqpoint{5.353934in}{3.227117in}}%
\pgfpathlineto{\pgfqpoint{5.356471in}{3.220538in}}%
\pgfpathlineto{\pgfqpoint{5.359009in}{3.207982in}}%
\pgfpathlineto{\pgfqpoint{5.361547in}{3.189510in}}%
\pgfpathlineto{\pgfqpoint{5.364084in}{3.165220in}}%
\pgfpathlineto{\pgfqpoint{5.366622in}{3.135240in}}%
\pgfpathlineto{\pgfqpoint{5.369160in}{3.099735in}}%
\pgfpathlineto{\pgfqpoint{5.371697in}{3.058900in}}%
\pgfpathlineto{\pgfqpoint{5.374235in}{3.012962in}}%
\pgfpathlineto{\pgfqpoint{5.376772in}{2.962178in}}%
\pgfpathlineto{\pgfqpoint{5.379310in}{2.906834in}}%
\pgfpathlineto{\pgfqpoint{5.381848in}{2.847243in}}%
\pgfpathlineto{\pgfqpoint{5.386923in}{2.716698in}}%
\pgfpathlineto{\pgfqpoint{5.391998in}{2.573524in}}%
\pgfpathlineto{\pgfqpoint{5.397074in}{2.421012in}}%
\pgfpathlineto{\pgfqpoint{5.407224in}{2.102238in}}%
\pgfpathlineto{\pgfqpoint{5.414837in}{1.865779in}}%
\pgfpathlineto{\pgfqpoint{5.419912in}{1.716364in}}%
\pgfpathlineto{\pgfqpoint{5.424987in}{1.577685in}}%
\pgfpathlineto{\pgfqpoint{5.430063in}{1.453051in}}%
\pgfpathlineto{\pgfqpoint{5.432600in}{1.396952in}}%
\pgfpathlineto{\pgfqpoint{5.435138in}{1.345459in}}%
\pgfpathlineto{\pgfqpoint{5.437676in}{1.298885in}}%
\pgfpathlineto{\pgfqpoint{5.440213in}{1.257515in}}%
\pgfpathlineto{\pgfqpoint{5.442751in}{1.221604in}}%
\pgfpathlineto{\pgfqpoint{5.445289in}{1.191374in}}%
\pgfpathlineto{\pgfqpoint{5.447826in}{1.167015in}}%
\pgfpathlineto{\pgfqpoint{5.450364in}{1.148682in}}%
\pgfpathlineto{\pgfqpoint{5.452901in}{1.136494in}}%
\pgfpathlineto{\pgfqpoint{5.455439in}{1.130535in}}%
\pgfpathlineto{\pgfqpoint{5.457977in}{1.130851in}}%
\pgfpathlineto{\pgfqpoint{5.460514in}{1.137449in}}%
\pgfpathlineto{\pgfqpoint{5.463052in}{1.150301in}}%
\pgfpathlineto{\pgfqpoint{5.465590in}{1.169341in}}%
\pgfpathlineto{\pgfqpoint{5.468127in}{1.194463in}}%
\pgfpathlineto{\pgfqpoint{5.470665in}{1.225527in}}%
\pgfpathlineto{\pgfqpoint{5.473203in}{1.262355in}}%
\pgfpathlineto{\pgfqpoint{5.475740in}{1.304736in}}%
\pgfpathlineto{\pgfqpoint{5.478278in}{1.352422in}}%
\pgfpathlineto{\pgfqpoint{5.480815in}{1.405134in}}%
\pgfpathlineto{\pgfqpoint{5.483353in}{1.462562in}}%
\pgfpathlineto{\pgfqpoint{5.485891in}{1.524367in}}%
\pgfpathlineto{\pgfqpoint{5.490966in}{1.659609in}}%
\pgfpathlineto{\pgfqpoint{5.496041in}{1.807634in}}%
\pgfpathlineto{\pgfqpoint{5.501116in}{1.964884in}}%
\pgfpathlineto{\pgfqpoint{5.521418in}{2.608431in}}%
\pgfpathlineto{\pgfqpoint{5.526493in}{2.753202in}}%
\pgfpathlineto{\pgfqpoint{5.531568in}{2.884042in}}%
\pgfpathlineto{\pgfqpoint{5.534106in}{2.943199in}}%
\pgfpathlineto{\pgfqpoint{5.536643in}{2.997677in}}%
\pgfpathlineto{\pgfqpoint{5.539181in}{3.047130in}}%
\pgfpathlineto{\pgfqpoint{5.541719in}{3.091244in}}%
\pgfpathlineto{\pgfqpoint{5.544256in}{3.129737in}}%
\pgfpathlineto{\pgfqpoint{5.546794in}{3.162362in}}%
\pgfpathlineto{\pgfqpoint{5.549331in}{3.188905in}}%
\pgfpathlineto{\pgfqpoint{5.551869in}{3.209192in}}%
\pgfpathlineto{\pgfqpoint{5.554407in}{3.223087in}}%
\pgfpathlineto{\pgfqpoint{5.556944in}{3.230493in}}%
\pgfpathlineto{\pgfqpoint{5.559482in}{3.231352in}}%
\pgfpathlineto{\pgfqpoint{5.562020in}{3.225650in}}%
\pgfpathlineto{\pgfqpoint{5.564557in}{3.213411in}}%
\pgfpathlineto{\pgfqpoint{5.567095in}{3.194702in}}%
\pgfpathlineto{\pgfqpoint{5.569633in}{3.169628in}}%
\pgfpathlineto{\pgfqpoint{5.572170in}{3.138337in}}%
\pgfpathlineto{\pgfqpoint{5.574708in}{3.101015in}}%
\pgfpathlineto{\pgfqpoint{5.577245in}{3.057887in}}%
\pgfpathlineto{\pgfqpoint{5.579783in}{3.009215in}}%
\pgfpathlineto{\pgfqpoint{5.582321in}{2.955296in}}%
\pgfpathlineto{\pgfqpoint{5.584858in}{2.896461in}}%
\pgfpathlineto{\pgfqpoint{5.587396in}{2.833074in}}%
\pgfpathlineto{\pgfqpoint{5.592471in}{2.694242in}}%
\pgfpathlineto{\pgfqpoint{5.597547in}{2.542255in}}%
\pgfpathlineto{\pgfqpoint{5.602622in}{2.380919in}}%
\pgfpathlineto{\pgfqpoint{5.620385in}{1.802420in}}%
\pgfpathlineto{\pgfqpoint{5.625460in}{1.650592in}}%
\pgfpathlineto{\pgfqpoint{5.630536in}{1.512145in}}%
\pgfpathlineto{\pgfqpoint{5.633073in}{1.449077in}}%
\pgfpathlineto{\pgfqpoint{5.635611in}{1.390673in}}%
\pgfpathlineto{\pgfqpoint{5.638149in}{1.337314in}}%
\pgfpathlineto{\pgfqpoint{5.640686in}{1.289353in}}%
\pgfpathlineto{\pgfqpoint{5.643224in}{1.247108in}}%
\pgfpathlineto{\pgfqpoint{5.645762in}{1.210860in}}%
\pgfpathlineto{\pgfqpoint{5.648299in}{1.180853in}}%
\pgfpathlineto{\pgfqpoint{5.650837in}{1.157292in}}%
\pgfpathlineto{\pgfqpoint{5.653374in}{1.140338in}}%
\pgfpathlineto{\pgfqpoint{5.655912in}{1.130112in}}%
\pgfpathlineto{\pgfqpoint{5.658450in}{1.126692in}}%
\pgfpathlineto{\pgfqpoint{5.660987in}{1.130110in}}%
\pgfpathlineto{\pgfqpoint{5.663525in}{1.140354in}}%
\pgfpathlineto{\pgfqpoint{5.666063in}{1.157370in}}%
\pgfpathlineto{\pgfqpoint{5.668600in}{1.181056in}}%
\pgfpathlineto{\pgfqpoint{5.671138in}{1.211269in}}%
\pgfpathlineto{\pgfqpoint{5.673675in}{1.247821in}}%
\pgfpathlineto{\pgfqpoint{5.676213in}{1.290484in}}%
\pgfpathlineto{\pgfqpoint{5.678751in}{1.338989in}}%
\pgfpathlineto{\pgfqpoint{5.681288in}{1.393027in}}%
\pgfpathlineto{\pgfqpoint{5.683826in}{1.452252in}}%
\pgfpathlineto{\pgfqpoint{5.686364in}{1.516284in}}%
\pgfpathlineto{\pgfqpoint{5.686364in}{1.516284in}}%
\pgfusepath{stroke}%
\end{pgfscope}%
\begin{pgfscope}%
\pgfsetrectcap%
\pgfsetmiterjoin%
\pgfsetlinewidth{0.803000pt}%
\definecolor{currentstroke}{rgb}{0.000000,0.000000,0.000000}%
\pgfsetstrokecolor{currentstroke}%
\pgfsetdash{}{0pt}%
\pgfpathmoveto{\pgfqpoint{0.360000in}{0.400000in}}%
\pgfpathlineto{\pgfqpoint{0.360000in}{3.960000in}}%
\pgfusepath{stroke}%
\end{pgfscope}%
\begin{pgfscope}%
\pgfsetrectcap%
\pgfsetmiterjoin%
\pgfsetlinewidth{0.803000pt}%
\definecolor{currentstroke}{rgb}{0.000000,0.000000,0.000000}%
\pgfsetstrokecolor{currentstroke}%
\pgfsetdash{}{0pt}%
\pgfpathmoveto{\pgfqpoint{5.940000in}{0.400000in}}%
\pgfpathlineto{\pgfqpoint{5.940000in}{3.960000in}}%
\pgfusepath{stroke}%
\end{pgfscope}%
\begin{pgfscope}%
\pgfsetrectcap%
\pgfsetmiterjoin%
\pgfsetlinewidth{0.803000pt}%
\definecolor{currentstroke}{rgb}{0.000000,0.000000,0.000000}%
\pgfsetstrokecolor{currentstroke}%
\pgfsetdash{}{0pt}%
\pgfpathmoveto{\pgfqpoint{0.360000in}{0.400000in}}%
\pgfpathlineto{\pgfqpoint{5.940000in}{0.400000in}}%
\pgfusepath{stroke}%
\end{pgfscope}%
\begin{pgfscope}%
\pgfsetrectcap%
\pgfsetmiterjoin%
\pgfsetlinewidth{0.803000pt}%
\definecolor{currentstroke}{rgb}{0.000000,0.000000,0.000000}%
\pgfsetstrokecolor{currentstroke}%
\pgfsetdash{}{0pt}%
\pgfpathmoveto{\pgfqpoint{0.360000in}{3.960000in}}%
\pgfpathlineto{\pgfqpoint{5.940000in}{3.960000in}}%
\pgfusepath{stroke}%
\end{pgfscope}%
\begin{pgfscope}%
\pgfsetbuttcap%
\pgfsetmiterjoin%
\definecolor{currentfill}{rgb}{1.000000,1.000000,1.000000}%
\pgfsetfillcolor{currentfill}%
\pgfsetfillopacity{0.300000}%
\pgfsetlinewidth{1.003750pt}%
\definecolor{currentstroke}{rgb}{0.800000,0.800000,0.800000}%
\pgfsetstrokecolor{currentstroke}%
\pgfsetstrokeopacity{0.300000}%
\pgfsetdash{}{0pt}%
\pgfpathmoveto{\pgfqpoint{0.457222in}{3.369722in}}%
\pgfpathlineto{\pgfqpoint{1.475668in}{3.369722in}}%
\pgfpathquadraticcurveto{\pgfqpoint{1.503445in}{3.369722in}}{\pgfqpoint{1.503445in}{3.397500in}}%
\pgfpathlineto{\pgfqpoint{1.503445in}{3.862778in}}%
\pgfpathquadraticcurveto{\pgfqpoint{1.503445in}{3.890556in}}{\pgfqpoint{1.475668in}{3.890556in}}%
\pgfpathlineto{\pgfqpoint{0.457222in}{3.890556in}}%
\pgfpathquadraticcurveto{\pgfqpoint{0.429444in}{3.890556in}}{\pgfqpoint{0.429444in}{3.862778in}}%
\pgfpathlineto{\pgfqpoint{0.429444in}{3.397500in}}%
\pgfpathquadraticcurveto{\pgfqpoint{0.429444in}{3.369722in}}{\pgfqpoint{0.457222in}{3.369722in}}%
\pgfpathclose%
\pgfusepath{stroke,fill}%
\end{pgfscope}%
\begin{pgfscope}%
\pgfsetrectcap%
\pgfsetroundjoin%
\pgfsetlinewidth{1.505625pt}%
\definecolor{currentstroke}{rgb}{0.843137,0.188235,0.121569}%
\pgfsetstrokecolor{currentstroke}%
\pgfsetdash{}{0pt}%
\pgfpathmoveto{\pgfqpoint{0.485000in}{3.748194in}}%
\pgfpathlineto{\pgfqpoint{0.679444in}{3.748194in}}%
\pgfusepath{stroke}%
\end{pgfscope}%
\begin{pgfscope}%
\pgftext[x=0.790556in,y=3.699583in,left,base]{\sffamily\fontsize{10.000000}{12.000000}\selectfont \(\displaystyle \widetilde{B}\,(\tau)-B_{0}\)}%
\end{pgfscope}%
\begin{pgfscope}%
\pgfsetrectcap%
\pgfsetroundjoin%
\pgfsetlinewidth{1.505625pt}%
\definecolor{currentstroke}{rgb}{0.988235,0.552941,0.349020}%
\pgfsetstrokecolor{currentstroke}%
\pgfsetdash{}{0pt}%
\pgfpathmoveto{\pgfqpoint{0.485000in}{3.508611in}}%
\pgfpathlineto{\pgfqpoint{0.679444in}{3.508611in}}%
\pgfusepath{stroke}%
\end{pgfscope}%
\begin{pgfscope}%
\pgftext[x=0.790556in,y=3.460000in,left,base]{\sffamily\fontsize{10.000000}{12.000000}\selectfont \(\displaystyle \widetilde{C}\,(\tau)-C_{0}\)}%
\end{pgfscope}%
\end{pgfpicture}%
\makeatother%
\endgroup%
}
    \caption[Aviici is love, Aviici is life]{Time evolution of nonstationary ABC coefficients}
    \label{fig:u0_dom_errs}
\end{figure}



\section{Flow systems defined by gridded velocity data}
\label{sec:flow_systems_defined_by_gridded_velocity_data}

\subsection{Oceanic currents in the Førde fjord}
\label{ssub:oceanic_currents_in_the_forde_fjord}

In 2016, the mining company Nordic Mining ASA received permission from the
Norwegian Ministry of Climate and Environment to extract rutile from the
Engebø mountain in Naustdal, Norway \parencite{garvik2017gruvekonflikten,%
haugan2015sjodeponi}. Furthermore, the company were authorized to dump
the mining waste into the nearby Førde fjord; a legislation which has been
debated fiercely, and heavily protested against, ever since the original
application was submitted in 2008. Early estimates suggest that, when operating
at full scale, the mining operation will result in yearly oceanic mine tailings
deposits in excess of five million tonnes \parencite{garvik2017gruvekonflikten}.

Several centres of technical expertise --- such as the Norwegian Institute of
Marine Research --- have publically advised against depositing mine tailings
into the fjord, emphasizing the potentially severe negative consequences
for the aquaculture \parencite{haugan2015sjodeponi}. Not only is the surrounding
area a significant spawning ground for cod, there is always a possibility of
of particles being transported by the water currents such that they
contaminate the outer edges of the fjord, or even into the ocean. Accordingly,
the use of LCSs in order to predict possible flow patterns for contaminants
resulting from the deposit of mine tailings would be of great environmental
interest.

To this end, gridded three-dimensional velocity data, modelling oceanic
currents in the (depths of the) Førde Fjord, was made available by SINTEF
Fisheries and Aquaculture. The data set considered here contains velocity data
for the time period between June 1 2013, 00:00 and June 2 2013, 23:40, sampled
in intervals of 20 minutes. For our simulations, the time interval of interest
was the 18 hour time window between 00:00 and 18:00 on June 1 2013 ---
practically ensuring the encapsulation of at least one tidal cycle.

Seeing as some of the most harmful contaminants found within the mine tailings
are heavy metals, we concentrated our analysis on the depths of the fjord.
Accordingly, we looked for LCSs in a region of water which was neither
particularly close to the oceanic surface, nor hit the coastline when
advected for the 18 hour duration of the time interval of interest. This
limited our research to a
$500\,\si{\meter}\times500\,\si{\meter}\times250\,\si{\meter}$ region, with
depths ranging from $50\,\si{\meter}$ to $300\,\si{\meter}$ below the
surface. A bird's-eye view of the region is shown in white in
\cref{fig:currentmap}, which also contains a map view of the geographical
region.


\begin{figure}[htpb]
    \centering
    \resizebox{0.9\textwidth}{!}{
        \importpgf{figures/mpl-figs}{currentmap-hires.pgf}
    }
    \caption[Stereographical map projection of the Førde fjord and its
    surroundings]
    {Stereographical map projection of the Førde fjord and its surroundings.
        The local fjord depths are indicated by a varying background color. A
        subset of the gridded velocity vectors indicates the macroscopic trends
        of the oceanic currents at a depth of $50\,\si{\meter}$ below the
        surface. Outlined in white is a bird's-eye view of the main region of
        interest; namely, a region of water which was neither particularly
        close to the oceanic surface, nor struck the coastline when advected
        during the 18 hour time interval of interest.}
    \label{fig:currentmap}
\end{figure}



\subsection{Interpolating gridded velocity data}
\label{sub:interpolating_gridded_velocity_data}








In order to identify LCSs in three-dimensional flow by means of geodesic level
set approximations, a system which has been studied extensively in the
literature was chosen. The system is a simple example of a fluid flow which can
exhibit chaotic behaviour \parencite[p.204]{frisch1995turbulence}.
