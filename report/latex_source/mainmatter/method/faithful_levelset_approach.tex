\section[Legacy approach to computing new mesh points]
{Legacy approach to computing new mesh points}
\label{sec:legacy_approach_to_computing_new_mesh_points}

As tentatively suggested in \cref{sub:parametrizing_the_innermost_level_set},
each of the points in the first level set
$\mathcal{M}_{1} = \{\mathcal{M}_{1,j}\}_{j=1}^{n}$ is used to compute a point
in the ensuing level set $\mathcal{M}_{2}$. This notion extends to all of the
subsequent level sets; namely, the points in level set $\mathcal{M}_{i+1}$
are computed from the points in the prior level set $\mathcal{M}_{i}$.
For reasons of brevity in the discussion to follow, we denote the points
$\{\mathcal{M}_{i,j}\}$ and $\{\mathcal{M}_{i+1,j}\}$ as \emph{ancestor} and
\emph{descendant points}, respectively. Furthermore, the set of mesh points
which can be traced backwards to a single, common ancestor, is referred to as
a \emph{point strand}. The considerations to follow rely on each mesh point
$\mathcal{M}_{i,j}$ inheriting its tangential vector from its direct ancestor;
that is, $\vct{t}_{i,j}:=\vct{t}_{i-1,j}$. The treatment of the special cases
of this inheritance-based approach will be described in greater detail in
\cref{sub:maintaining_mesh_point_density}.

From the mesh point $\mathcal{M}_{i,j}$, we wish to place a new mesh point
$\mathcal{M}_{i+1,j}$ at an intersection of the manifold $\mathcal{M}$ and the
half-plane $\mathcal{H}_{i,j}$ located a distance $\Delta_{i}$ from
$\mathcal{M}_{i,j}$. The aforementioned half-plane is defined by the coordinate
$\vct{x}_{i,j}$, the unit normal $\vct{t}_{i,j}$ and the guidance vector
$\vct{\rho}_{i,j}$ (cf.\ \cref{eq:innermost_prevvec,eq:general_prevvec}).
Note that this intersection may occur anywhere on the half-circle within
$\mathcal{H}_{i,j}$ of radius $\Delta_{i}$, centered at $\vct{x}_{i,j}$. The
search for a new mesh point is conducted by defining an aim point
$\vct{x}_{\text{aim}}$ within $\mathcal{H}_{i,j}$, computed by performing a
single, classical, \nth{4}-order Runge-Kutta step (cf.\ \cref{tab:butcherrk4})
of length $\Delta_{i}$ in the vector field locally defined as
\begin{equation}
    \label{eq:faithful_intermediary_aimpoint_vectorfield}
    \vct{\psi}(\vct{x}) = %
    \frac{\vct{\xi}_{3}(\vct{x})\times\vct{t}_{i,j}}%
    {\norm{\vct{\xi}_{3}(\vct{x})\times\vct{t}_{i,j}}},
\end{equation}
starting at $\vct{x}_{i,j}$. Moreover, all of the vectors of the intermediary
Runge-Kutta evaluations of $\vct{\psi}(\vct{x})$ were corrected, if necessary,
by continuous comparison with $\vct{\rho}_{i,j}$ and sign-reversion if an
intermediary vector was directed radially inwards. Finally, the computed aim
point was projected into the half-plane $\mathcal{H}_{i,j}$ as follows:
\begin{equation}
    \label{eq:aimpoint_planeprojection}
    \vct{x}_{\text{aim}} := \vct{x}_{\text{aim}} - \inp[]{\vct{t}_{i,j}}%
    {\vct{x}_{\text{aim}}-\vct{x}}\vct{t}_{i,j}.
\end{equation}

The idea is then to look for a new position within $\mathcal{H}_{i,j}$, in the
vicinity of $\vct{x}_{\text{aim}}$, at a distance $\Delta_{i}$ from
$\vct{x}_{i,j}$, by moving within the constraints of the manifold. Motivated by
\cref{rmk:invariance_lcs} --- on the invariance of the time-$t_{0}$ image of
repelling LCSs under perturbations in the local $\vct{\xi}_{1}$- and
$\vct{\xi}_{2}$-direction fields --- this involves computing trajectories which
everywhere l2ay within the plane spanned by the local $\vct{\xi}_{1}$- and
$\vct{\xi}_{2}$-vectors. This was done by defining a local, normalized
direction field as
\begin{equation}
    \label{eq:faithful_local_directionfield}
    \vct{\mu}(\vct{x},\vct{x}_{\text{aim}}) = %
    \frac{\vct{x}_{\text{aim}}-\vct{x} - \inp[]{\vct{\xi}_{3}(\vct{x})}%
    {\vct{x}_{\text{aim}}-\vct{x}}\vct{\xi}_{3}(\vct{x})}%
    {\norm{\vct{x}_{\text{aim}}-\vct{x} - \inp[]{\vct{\xi}_{3}(\vct{x})}%
    {\vct{x}_{\text{aim}}-\vct{x}}\vct{\xi}_{3}(\vct{x})}},
\end{equation}
that is, the normalized projection of the vector separating $\vct{x}$ and
$\vct{x}_{\text{aim}}$ into the plane orthogonal to the local $\vct{\xi}_{3}$
vector, in accordance with LCS existence criterion~\eqref{eq:lcs_condition_c}.
A visual representation of this direction field is given in
\cref{fig:aim_procedure}. The choice of initial conditions for computing
trajectories within the manifold, with a view to expanding it, is the topic
of (the immediately forthcoming)
\cref{sub:selecting_initial_conditions_from_which_to_compute_new_mesh_points}.

\documentclass[crop]{standalone}
\usepackage{tikz}
\usepackage[]{tikz-3dplot}
\usepackage{pgfplots}
\usepackage[]{amsmath}
\usepackage[]{libertine}
\usepackage[libertine]{newtxmath}
\usepackage[]{bm}
\usepackage[]{physics}
% Macros for greek letters in roman style, in math mode
\DeclareRobustCommand{\mathup}[1]{%
\begingroup\changegreek\mathrm{#1}\endgroup}
\DeclareRobustCommand{\mathbfup}[1]{%
\begingroup\changegreekbf\mathbf{#1}\endgroup}

\makeatletter
\def\changegreek{\@for\next:={%
        alpha,beta,gamma,delta,epsilon,zeta,eta,theta,iota,kappa,lambda,mu,nu,%
        xi,pi,rho,sigma,tau,upsilon,phi,chi,psi,omega,varepsilon,varpi,%
    varrho,varsigma,varphi}%
\do{\expandafter\let\csname\next\expandafter\endcsname\csname\next up\endcsname}}
\def\changegreekbf{\@for\next:={%
        alpha,beta,gamma,delta,epsilon,zeta,eta,theta,iota,kappa,lambda,mu,nu,%
        xi,pi,rho,sigma,tau,upsilon,phi,chi,psi,omega,varepsilon,varpi,%
    varrho,varsigma,varphi}%
    \do{\expandafter\def\csname\next\expandafter\endcsname\expandafter{%
\expandafter\bm\expandafter{\csname\next up\endcsname}}}}
\makeatother

% Define vectors in bold roman font
\newcommand{\vct}[1]{\ensuremath{\mathbfup{#1}}}

% Define unit vectors in bold roman font, with hats
\newcommand{\uvct}[1]{\ensuremath{\mathbfup{\hat{#1}}}}

% Define matrices in bold italic font
\newcommand{\mtrx}[1]{\ensuremath{\bm{#1}}}

\usetikzlibrary{%
    angles,%
    arrows.meta,%
    backgrounds,%
    calc,%
    decorations,%
    fit,%
    hobby,%
    patterns,%
    positioning,%
    quotes
}
\makeatletter
\tikzset{
  fitting node/.style={
    inner sep=0pt,
    fill=none,
    draw=none,
    reset transform,
    fit={(\pgf@pathminx,\pgf@pathminy) (\pgf@pathmaxx,\pgf@pathmaxy)}
  },
  reset transform/.code={\pgftransformreset}
}
\makeatother
% A simple empty decoration, that is used to ignore the last bit of the path
\pgfdeclaredecoration{ignore}{final}
{
\state{final}{}
}

% Declare the actual decoration.
\pgfdeclaremetadecoration{middle}{initial}{
    \state{initial}[
        width={0pt},
        next state=middle
    ]
    {\decoration{moveto}}

    \state{middle}[
        width={\pgfdecorationsegmentlength*\pgfmetadecoratedpathlength},
        next state=final
    ]
    {\decoration{curveto}}

    \state{final}
    {\decoration{ignore}}
}


% Create a key for easy access to the decoration
\tikzset{middle segment/.style={decoration={middle},decorate, segment length=#1}}

\def\getangle(#1)(#2)#3{%
  \begingroup%
    \pgftransformreset%
    \pgfmathanglebetweenpoints{\pgfpointanchor{#1}{center}}{\pgfpointanchor{#2}{center}}%
    \expandafter\xdef\csname angle#3\endcsname{\pgfmathresult}%
  \endgroup%
}


\begin{document}


\tdplotsetmaincoords{60}{20}

\begin{tikzpicture}[tdplot_main_coords]
    % Macro for the unit vector scale (i.e., the length of the unit vectors)
    \pgfmathsetmacro{\usclx}{2.15};
    \pgfmathsetmacro{\uscly}{2.35};
    \pgfmathsetmacro{\usclz}{1.55};

    % Macro for the axes parallel to the unit vectors
    \pgfmathsetmacro{\asclx}{7}
    \pgfmathsetmacro{\ascly}{7}
    \pgfmathsetmacro{\asclz}{3}

    % Macro for the vector separating origin point and aim point
    \pgfmathsetmacro{\x}{0.78*\asclx}
    \pgfmathsetmacro{\y}{0.64*\ascly}
    \pgfmathsetmacro{\z}{0.78*\asclz}

    % Set coordinates for origin point
    \coordinate (r) at (0,0,0);

    % Set coordinates for aiming point
    \coordinate (ra) at ($(r) + (\x,\y,\z)$);

    % Set coordinates for end points of unit vectors
    \coordinate (xi1) at ($(r) + (\usclx,0,0)$);
    \coordinate (xi2) at ($(r) + (0,\uscly,0)$);
    \coordinate (xi3) at ($(r) + (0,0,\usclz)$);

    % Set coordinates of orthogonal projection of aiming vector
    \coordinate (rort) at ($(r) + (0,0,\z)$);

    % Set coordinates of parallel projection of aiming vector
    \coordinate (rpar) at ($(r) + (\x,\y,0)$);

    % Shade plane spanned by xi1 and xi2
    \draw[fill=gray!15,draw opacity = 0] (r) -- ($(r)+(\asclx,0,0)$) -- ($(r)+(\asclx,\ascly,0)$) -- ($(r)+(0,\ascly,0)$) -- cycle;

    % Draw axes parallel to unit vectors
    \draw[color=gray!80] (r) -- ($(r) + (\asclx,0,0)$); % xi1
    \draw[color=gray!80] (r) -- ($(r) + (0,\ascly,0)$); % xi2
    \draw[color=gray!80] (r) -- ($(r) + (0,0,\asclz)$); % xi3

    % Draw unit vectors
	\getangle(r)(xi1)b;
    \draw[->,very thick,color=black!90] (r) -- (xi1) node[midway,below,rotate=\angleb]{$\vct{\xi}_{1}(\vct{r})$};
	\getangle(r)(xi2)b;
    \draw[->,very thick,color=black!90] (r) -- (xi2) node[above right=-0.25cm and -0.25cm,rotate=\angleb] {$\vct{\xi}_{2}(\vct{r})$};
    \draw[->,very thick,color=black!90] (r) -- (xi3) node[midway,left]{$\vct{\xi}_{3}(\vct{r})$};

    % Draw guide lines to orthogonal component of aim vector
    \draw[dashed,stroke=black!65] (ra) -- (rort);

    % Draw guide lines to parallel component of aim vector
    \draw[dashed,stroke=black!65] (ra) -- (rpar);


    % Draw parallel component of aim vector
	\getangle(r)(rpar)b;
    \draw[thick, stroke=black!90,dashed] (r) -- (rpar);
    \draw[->,stroke=black!90,thick,middle segment = 0.3] (r) -- (rpar) node[midway,below left,rotate=\angleb]{$\vct{f}(\vct{r},\vct{r}_{\mathrm{aim}})$};

    % Draw aim point
    \draw[fill=gray!10,stroke=black!90] (ra) circle(2.5pt) node[right]{$\vct{r}_{\mathrm{aim}}$};
    % Draw nonmodified aim vector
    \draw[->,stroke=black!90,thick,densely dashed,middle segment = 1] (r) -- (ra);

    % Draw origin point
    \draw[fill=gray!10,stroke=black!90,fill opacity=1] (r) circle(2.5pt) node[below left = 1pt and 1pt] {$\vct{r}$};


\end{tikzpicture}
\end{document}


\subsection{Selecting initial conditions from which to compute new mesh points}
\label{sub:selecting_initial_conditions_from_which_to_compute_new_mesh_points}

As suggested by \textcite{krauskopf2005survey}, the starting point for all
trajectories intended to reach $\vct{x}_{\text{aim}}$ could be chosen as any
point not equal to $\vct{x}_{i,j}$ along the parametrized curve
$\mathcal{C}_{i}$. However, trajectories starting out at points along
$\mathcal{C}_{i}$ which are far removed from $\vct{x}_{i,j}$ are likely to
require a long integration path, which would result in an increase in the
accumulated numerical round-off error. Subject to such errors, these
trajectories might not even get close to $\vct{x}_{\text{aim}}$ with
a reasonable computational resource consumption. Accordingly, we limited the
potential number of trajectories to compute by only considering a subset of the
interpolation curve $\mathcal{C}_{i}$ as initial conditions, as follows:
\begin{equation}
    \label{eq:faithful_initialcondition_interval}
    \vct{x}_{\text{init}} = \mathcal{C}_{i}(\breve{s}),\quad %
    \breve{s}\in\big\{[s_{j}-\varsigma,s_{j})\cup(s_{j},s_{j}+\varsigma]\big\},\quad%
    0<\varsigma\leq\frac{1}{2},
\end{equation}
where the inherent periodicity of the (quasi-)arclength parametrization of
$\mathcal{C}_{i}$ is implicitly applied. Here, $\varsigma$ was set to $0.1$,
ensuring that $20$ \% of all possible initial conditions along $\mathcal{C}_{i}$
were considered.

For computing trajectories whose initial conditions are given by
\cref{eq:faithful_initialcondition_interval} and direction fields given by
\cref{eq:faithful_local_directionfield}, the Dormand-Prince 8(7) adaptive
ODE solver (cf.\ \cref{tab:butcherdopri87,%
sub:the_implementation_of_dynamic_runge_kutta_step_size}) was the method of
choice. In order to ensure that any trajectory did not overstep the half-plane
$\mathcal{H}_{i,j}$ in passing, the step length was continuously limited from
above by $\norm{\vct{x}_{\text{aim}}-\vct{x}}$. Moreover, in order to avoid
spending unreasonable computational resources on trajectories which for
practical purposes never would result in acceptable, new mesh points, the
total allowed integration arclength was limited by a scalar multiple of
the initial separation $\norm{\vct{x}_{\text{init}}-\vct{x}_{\text{aim}}}$.
In particular, this limitation meant that trajectories which ended in
stable orbits around $\vct{x}_{\text{aim}}$ were not allowed to keep going
indefinitely.

If any trajectory terminated in a point $\vct{x}_{\text{fin}}$
which lied within the half-plane $\mathcal{H}_{i,j}$ at a distance
$\Delta_{i}$ from $\vct{x}_{i,j}$, then $\vct{x}_{\text{fin}}$ was used as
coordinates for a new mesh point $\mathcal{M}_{i+1,j}$. Numerically, these
checks were implemented by means of tolerance parameters, as comparing
floating-point numbers for equivalence is prone to numerical round-off error.
More precisely, a point $\vct{x}$ was said to lie within $\mathcal{H}_{i,j}$
provided that
\begin{equation}
    \label{eq:plane_tolerance}
    \vct{\eta}: = \frac{\vct{x}-\vct{x}_{i,j}}{\norm{\vct{x}-\vct{x}_{i,j}}};%
    \quad \abs{\inp[]{\vct{\eta}}{\vct{t}_{i,j}}} < \gamma_{\mathcal{H}},
\end{equation}
whereas
\begin{equation}
    \label{eq:dist_tolerance}
    \abs{\frac{\norm{\vct{x}-\vct{x}_{i,j}}}{\Delta_{i}}-1} < \gamma_{\Delta}
\end{equation}
sufficed for it to be flagged as lying a distance $\Delta_{i}$ from
$\vct{x}_{i,j}$, with $\gamma_{\mathcal{H}}$ and $\gamma_{\Delta}$ small
numbers. When a trajectory first intersected $\mathcal{H}_{i,j}$, the
integration was stopped abrubtly, leaving its endpoint $\vct{x}_{\text{fin}}$
as its suggested coordinates for the new mesh point. As briefly mentioned in
\cref{sub:parametrizing_the_innermost_level_set}, the unit tangent vectors
$\vct{t}_{i,j}$ were generally inherited --- the treatment of special cases
will be explained in greater detail in
\cref{sub:maintaining_mesh_point_density}. \Cref{fig:faithful_point_generation}
depicts a few characteristic trajectory patterns which commonly occured when
searching for new mesh points in the fashion discussed in the above.

\documentclass[crop]{standalone}
\usepackage{tikz}
\usepackage[]{tikz-3dplot}
\usepackage{pgfplots}
\pgfplotsset{compat=1.15}
\usepackage[]{amsmath}
\usepackage[]{libertine}
\usepackage[libertine]{newtxmath}
\usepackage[]{bm}
\usepackage[]{physics}
% Macros for greek letters in roman style, in math mode
\DeclareRobustCommand{\mathup}[1]{%
\begingroup\ensuremath\changegreek\mathrm{#1}\endgroup}
\DeclareRobustCommand{\mathbfup}[1]{%
\begingroup\ensuremath\changegreek\bm{\mathrm{#1}}\endgroup}


\makeatletter
\def\changegreek{\@for\next:={%
        alpha,beta,gamma,delta,epsilon,zeta,eta,theta,iota,kappa,lambda,mu,nu,%
        xi,pi,rho,sigma,tau,upsilon,phi,chi,psi,omega,varepsilon,varpi,%
    varrho,varsigma,varphi}%
\do{\expandafter\let\csname\next\expandafter\endcsname\csname\next up\endcsname}}
\makeatother

% Define vectors in bold, roman, lowercase font
\newcommand{\vct}[1]{\ensuremath{\mathbfup{\MakeLowercase{#1}}}}

% Define unit vectors in bold, roman, lowercase font, with hats
\newcommand{\uvct}[1]{\ensuremath{\mathbfup{\hat{\MakeLowercase{#1}}}}}

% Define matrices in bold, roman, uppercase font
\newcommand{\mtrx}[1]{\ensuremath{\mathbfup{\MakeUppercase{#1}}}}
\usetikzlibrary{%
    angles,%
    arrows.meta,%
    backgrounds,%
    calc,%
    decorations,%
    fit,%
    hobby,%
    patterns,%
    positioning,%
    quotes
}


\begin{document}

\tdplotsetmaincoords{70}{120}

\begin{tikzpicture}[tdplot_main_coords]
    \pgfmathsetmacro{\radius}{5}
    \pgfmathsetmacro{\size}{2.5}
%    % Place the set of initial points
    \foreach [count = \i] \ang in {205,240,260,280,165}%
    {%
        \coordinate (\i) at ( {\radius*cos(\ang-90)}, {\radius*sin(\ang-90)},{0} );
    }

    % Place coordinates giving the circle of positions at which one aims
    \coordinate (tp) at ({\radius*cos(150)},{\radius*sin(150)},1);
    \coordinate (bt) at ({\radius*cos(150)},{\radius*sin(150)},-1);


    % Define coordinates of half-plane $\mathcal{H}_{i,j}$
    \coordinate (lu) at ($(2)!2!(tp)$);
    \coordinate (ld) at ($(2)!2!(bt)$);
    \coordinate (ru) at ($(lu)+7*({-0.866},{0.5},0)$);
    \coordinate (rd) at ($(ld)+7*({-0.866},{0.5},0)$);
    % Shade half-plane
    \draw[fill=gray!30,draw opacity=0] (lu) -- (ld) -- (rd) -- (ru) -- cycle;
    % Draw interpolation curve
    \draw[stroke=black!80,thin,dotted] (70:\radius) arc (70:200:\radius);
    % Draw $\vct{rho}_{i,j}$ at $x_{i,j}$
    \path[draw,stroke=black!65,->,thick,dashdotted ] (2) to ($(2) + 5*({-0.866},{0.5},0)$) coordinate (rhoend);
    \node[below left = 7.5pt and 35pt of rhoend,rotate=12.5] {$\vct{\rho}_{i,j}$};
    \node[above left = 0pt and 0pt of rd] {$\mathcal{H}_{i,j}$};

    % Define coordinates of pivot for attaching $\mathcal{C}_{i}$ label
    \coordinate (mrk) at ({\radius*cos(195)},{\radius*sin(195)},0);
    % Attach label
    \coordinate [above = 10pt of mrk] (lbl);
    \node at (lbl) {$\mathcal{C}_{i}$};
    \draw[stroke=black!80] ($(mrk)!0.1!(lbl)$) to[in= 270,out=20] ($(mrk)!0.55!(lbl)$);

    % Draw points in initial level set
    \foreach \i in {1,2,3,4,5}%
    {%
        \draw[fill=gray!10,stroke=black!65] (\i) circle (\size pt);
    }
    % Draw semicircle of radius $\Delta_{i}$, centered at $x_{i,j}$
    \pic[densely dashed,draw,stroke=grey!80,angle radius=39] {angle = bt--2--tp};



    % Label ancestor point
    \coordinate[below left = 0pt and 30pt of 2] (plbl);
    \node at (plbl) {$\mathcal{M}_{i,j}$};
    \draw[stroke=black!80] ($(plbl)!0.4!(2)$) to[out = 20, in = 160]  ($(plbl)!0.85!(2)$);


    % Define aim point
    \coordinate (aim) at ($(2) + 1.2*({-0.866},{0.5},0) + (0,0,0.75)$);

    % Define support point in semicircle, to indicate radius
    \coordinate (supp) at ($(2) + 1.57*({-0.866},{0.5},0)$);
    % Draw semicircle to indicate angle offset of aim point
    \pic[draw,stroke=gray!80,angle eccentricity = 1.7,angle radius=10] {angle = supp--2--aim};
    % Add text node to explain the angle
    \node[below right = 10pt and -2.5pt of 2] (mrk) {$\alpha_{i,j}$};
    % Add anchor for line from node to within angle
    \coordinate (inang) at ($(2) + 0.25*({-0.866},{0.5},0) + (0,0,0.25)$);
    \path[draw,stroke=black!80,thin] ($(mrk)!0.1!(inang)$) to [out = 60, in = -180+60] ($(mrk)!0.8!(inang)$);
    % Draw vector from start point to aim point
    \draw[stroke=black!65,thin,dashed] (2) -- (aim);
    % Redraw start point
    \draw[fill=gray!19,stroke=black!65] (2) circle (\size pt);
    % Draw aim point
    \draw[stroke=black!65,fill=gray!10] (aim) circle (\size pt);
    \node[above right = 0pt and 0pt of aim]  {$\vct{x}_{\text{aim}}$};
    % Suggest initial radius
    \draw[decorate,decoration={brace,amplitude=5pt,raise=0pt},yshift=0pt] ($(2)!1.46!(bt)$) -- (2)   node [midway,left=2pt] {\footnotesize{$\Delta_{i}$}};

    % Draw some trajectories, the last of which manages to find aim point
    \coordinate (p1i) at ($({\radius*cos(180)},{\radius*sin(180)},0)$);
    \coordinate (p1f) at ($(2) + 2.8*({-0.866},{0.5},0) + (0,0,0.75)$);
    \coordinate (p2i) at ($({\radius*cos(160)},{\radius*sin(160)},0)$);
    \coordinate (p2f) at ($(2) + 1.1*({-0.866},{0.5},0) + 1.7*(0,0,0.75)$);
    \coordinate (p3i) at ($({\radius*cos(140)},{\radius*sin(140)},0)$);
    \coordinate (p3f) at ($(2) + 0.7*({-0.866},{0.5},0) + (0,0,0.75)$);
    \coordinate (p4i) at ($({\radius*cos(125)},{\radius*sin(125)},0)$);
    \path[draw,stroke=black!80,->] (p1i) to[out = 30, in = 150]  (p1f);
    \path[draw,stroke=black!80,->] (p2i) to[out = 30, in = 140]  (p2f);
    \path[draw,stroke=black!80,->] (p3i) to[out = 40, in = 30]  (p3f);
    \path[draw,stroke=black!80,->] (p4i) to[out = 40, in = -70]  (aim);

\end{tikzpicture}
\end{document}



\subsection{Choosing new trajectory start points by an algorithm with memory}
\label{sub:choosing_new_trajectory_start_points_by_an_algorithm_with_memory}

Our method of choosing parameter values $\breve{s}$ corresponding to points
along $\mathcal{C}_{i}$ (cf.\ \cref{eq:faithful_initialcondition_interval}),
from which to compute trajectories of the direction field given by
\cref{eq:faithful_local_directionfield} --- with the intention of computing new
mesh points --- rests on the assumption that
\begin{equation}
    \label{eq:faithful_endpoint_separation}
    \Delta(\breve{s}) := \norm{\vct{x}_{\text{fin}}(\breve{s})-\vct{x}_{i,j}},%
    \quad \vct{x}_{\text{fin}}\in\mathcal{H}_{i,j}
\end{equation}
is a continuous function of $\breve{s}$. To this end, we keep track of
why each computed trajectory is terminated. In particular, we first note
whether or not each trajectory, corresponding to a start point
$\vct{x}(\breve{s})$, ends up at some point
$\vct{x}_{\text{fin}}\in\mathcal{H}_{i,j}$. If this is the case, we also note
whether the corresponding separation $\Delta(\breve{s})$ (defined in
\cref{eq:faithful_endpoint_separation}) was an over- or
undershoot with regards to the desired separation $\Delta_{i}$.

Based on the premises outlined above, we then make use of the intermediate
value theorem; specifically, if we have
\begin{subequations}
    \begin{equation}
        \label{eq:faithful_IVT_part_one}
        \Delta(\breve{s}_{1}) < \Delta_{i},\quad \Delta(\breve{s}_{2}) >
        \Delta_{i}
    \end{equation}
    for $\breve{s}_{1} < \breve{s}_{2}$, then the intermediate value theorem
    implies that there must exist an $\breve{s}$, such that
    \begin{equation}
        \label{eq:faithful_IVT_part_two}
        \Delta(\breve{s}) = \Delta_{i}, \quad \breve{s}_{1} < \breve{s} %
        < \breve{s}_{2},
    \end{equation}
\end{subequations}
under the assumption that $\Delta(\breve{s})$ is a continuous function. In
order to optimize our use of computational resources, we thus endeavor to take
large steps when moving along $\mathcal{C}_{i}$ whenever the computed
intersections with $\mathcal{H}_{i,j}$ are far from fulfilling
$\Delta(\breve{s}) = \Delta_{i}$. However, when a subinterval of
$\mathcal{C}_{i}$ is identified, within which the intermediate value theorem
suggests that a trajectory may fulfill our requirements, we decrease the
(quasi-)arclength increment $\delta\breve{s}$ in order to increase our odds of
finding said trajectory. While the purpose of $\delta\breve{s}_{\min}$ was to
manage resource requirements, $\delta\breve{s}_{\max}$ was used in order to
avoid bypassing subsets of $\mathcal{C}_{i}$ from which two or more
trajectories satisfy $\Delta(\breve{s}) = \Delta_{i}$. Overstepping a region
containing an even number of such intersections could render it undetectable
using our algorithm (see \cref{fig:s_update_flowchart}), as no change in
trajectory termination status need be detected.

The feedback received by tracking why each trajectory is terminated, allows
us to dynamically select new trajectory start points along $\mathcal{C}_{i}$.
We do so by increasing the (quasi-)arclength increment $\delta\breve{s}$ as
long as there is no change in trajectory termination status, and, conversely,
backtracking and reducing $\delta\breve{s}$ when a status change is detected.
This process is shown schematically in \cref{fig:s_update_flowchart}. As we
assume asymptotic behaviour close to any regions in which $\Delta(\breve{s})$
is not defined (that is, regions where no trajectories reach the half-plane
$\mathcal{H}_{i,j}$, cf.\ \cref{eq:faithful_endpoint_separation}), the
adjustment of $\delta\breve{s}$ is treated in the same fashion there.

\begin{figure}[htb]
    \centering
    \resizebox{0.9\linewidth}{!}%
    {\includestandalone{figures/tikz-figs/faithful_approach_s_update}}
    \caption[Flowchart illustrating the algorithm for iteratively choosing new
    trajectory start points based on the termination status of the preceding
    trajectories, using the legacy approach.]
    {Flowchart illustrating the algorithm for iteratively choosing new
        trajectory start points based on the termination status of the
        preceding trajectories, using the legacy approach. All possible
        trajectory start points are contained within a subset of
        $\mathcal{C}_{i}$, per \cref{eq:faithful_initialcondition_interval}.
        Whether or not a given trajectory intersected with the half-plane
        $\mathcal{H}_{i,j}$, and satisfied $\Delta(\breve{s})=\Delta_{i}$, was
        determined using
        \cref{eq:dist_tolerance,eq:plane_tolerance}.
    }
    \label{fig:s_update_flowchart}
\end{figure}



%\afterpage{\clearpage}
\clearpage

\input{mainmatter/method/levelset_method_take_one/handling_failures_to_compute%
_points}
