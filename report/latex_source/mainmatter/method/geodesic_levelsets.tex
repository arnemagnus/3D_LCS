\section[Computing repelling LCSs for 3D flow systems using the method of geodesic level sets]
{Computing repelling LCSs for 3D flow systems using \\\phantom{3.4} the method of geodesic level sets}
\label{sec:_computing_repelling_lcss_in_three_spatial_dimensions_using_the_method_of_geodesic_level_sets_}

Repelling LCSs in three spatial dimensions are quite challenging to compute.
Straightforward numerical integration of the flow is insufficient ---
in three dimensions, the existence criterion~\eqref{eq:lcs_condition_c} provides
another degree of freedom, as, everywhere within such an LCS, one is
allowed to move `freely' within a plane which is orthogonal to
$\vct{\xi}_{3}(\vct{x})$. Dedicated algorithms are needed. Here, we consider
a (variation of) the method of geodesic level sets for computing repelling
LCSs as invariant manifolds of the $\vct{\xi}_{1}$ and $\vct{\xi}_{2}$ direction
fields (cf.~\cref{rmk:invariance_lcs}), as presented by
\textcite{krauskopf2005survey}. The short presentation to follow in the
next paragraph will be explained further in depth in the subsequent
subsections.

The method is based on the concept of developing an unstable manifold from a
local neighborhood of an initial condition $\vct{x}_{0}$ (how these initial
conditions were chosen will be described in \textbf{\emph{SETT INN REF}}).
In particular, a small, closed curve $\mathcal{C}_{\delta}$ consisting
of points which all lie within the tangent plane defined by the coordinate
$\vct{x}_{0}$ and the unit normal $\vct{\xi}_{3}(\vct{x}_{0})$, separated from
$\vct{x}_{0}$ by a distance $\delta$ is assumed to consist of points which all
lie within the same manifold as $\vct{x}_{0}$. The idea is then to compute
the next geodesic circle in a local, dynamic coordinate system, defined by
hyperplanes which are orthogonal to the most recently computed geodesic circle.
A set of accuracy parameters governs the number of next points by which
the next geodesic circle is approximated, by solving a set of boundary value
problems. During the computation, the interpolation error stays bounded by
the density of mesh points, so that the overall quality of the mesh is
preserved.


\subsection{Identifying suitable initial conditions for developing LCSs}
\label{sub:identifying_suitable_initial_conditions_for_developing_lcss}



