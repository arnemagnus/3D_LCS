\section[Computing repelling LCSs for 3D flow systems using the method of geodesic level sets]
{Computing repelling LCSs for 3D flow systems using \\\phantom{3.4} the method of geodesic level sets}
\label{sec:_computing_repelling_lcss_in_three_spatial_dimensions_using_the_method_of_geodesic_level_sets_}

Repelling LCSs in three spatial dimensions are quite challenging to compute.
Straightforward numerical integration of the flow is insufficient ---
in three dimensions, the existence criterion~\eqref{eq:lcs_condition_c} provides
another degree of freedom, as, everywhere within such an LCS, one is
allowed to move `freely' within a plane which is orthogonal to
$\vct{\xi}_{3}(\vct{x})$. Dedicated algorithms are needed. Here, we consider
a (variation of) the method of geodesic level sets for computing repelling
LCSs as invariant manifolds of the $\vct{\xi}_{1}$ and $\vct{\xi}_{2}$ direction
fields (cf.~\cref{rmk:invariance_lcs}), as presented by
\textcite{krauskopf2005survey}. The short presentation to follow in the
next paragraph will be explained further in depth in the subsequent
subsections.

The method is based on the concept of developing an unstable manifold from a
local neighborhood of an initial condition $\vct{x}_{0}$ (how these initial
conditions were chosen will be described in
\cref{sub:identifying_suitable_initial_conditions_for_developing_lcss}).
In particular, a small, closed curve $\mathcal{C}_{1}$ consisting
of points which all lie within the tangent plane defined by the coordinate
$\vct{x}_{0}$ and the unit normal $\vct{\xi}_{3}(\vct{x}_{0})$, separated from
$\vct{x}_{0}$ by a distance $\delta$, is assumed to consist of points which all
lie within the same manifold as $\vct{x}_{0}$. The idea is then to compute
the next geodesic circle in a local, dynamic coordinate system, defined by
hyperplanes which are orthogonal to the most recently computed geodesic circle.
A set of accuracy parameters governs the number of next points by which
the next geodesic circle is approximated, in solving a set of boundary value
problems. During the computation, the interpolation error stays bounded by
the density of mesh points, so that the overall quality of the mesh is
preserved.

\begin{table}[htpb]
    \centering
    \caption[Parameter choices for selecting LCS initial conditions]
    {Parameter choices for selecting LCS initial conditions. $\varepsilon$ was
    made to be one order of magnitude smaller than the grid spacing, and
    $\nu$ was selected to be a common divisor of the number of grid points
    along each Cartesian abscissa, cf.\ \cref{tab:gridparams}. Note that
    $\varepsilon$ for the fjord model data is given in units of metre. Observe
    how the reduced set of initial conditions contains orders of magnitude
    fewer points than the total number of grid points which satisfy the LCS
    existence criteria~\eqref{eq:lcs_condition_a},~\eqref{eq:lcs_condition_b}
    and~\eqref{eq:lcs_condition_d}.}
    \label{tab:initialconditionparams}
    \begin{tabular}{ccc}
        \toprule
        & Analytical ABC flow & Fjord model data\\
        \midrule
        $\varepsilon$ & $5\cdot10^{.-3}$ & $10^{-1}$\\
        $\nu$ & $8$ & $5$ \\[3pt]
        \makecell[c]{\# initial conditions\\without $\nu$-filtering} & \makecell[c]{\numprint{340951} (steady) \\ \numprint{361461} (unsteady)} & \numprint{209945}\\[9pt]
        \makecell[c]{\# initial conditions\\with $\nu$-filtering} & \makecell[c]{\numprint{618} (steady)\\\numprint{676} (unsteady)} & \numprint{1631}\\
        \bottomrule
    \end{tabular}
\end{table}

\subsection{Parametrizing the innermost level set}
\label{sub:pametrizing_the_innermost_level_set}

For an initial condition $\vct{x}_{0}$ identified by means of the method
outlined in
\cref{sub:identifying_suitable_initial_conditions_for_developing_lcss}, the
corresponding LCS must locally be tangent to the plane with unit normal given
by $\vct{\xi}_{3}(\vct{x}_{0})$, as a consequence of LCS existence criterion~%
\eqref{eq:lcs_condition_c}. Accordingly, the first geodesic level set is
approximated by a set of $n$ points $\{\mathcal{M}_{1,j}\}_{j=1}^{n}$, which all
lie within the aforementioned tangent plane, separated from $\vct{x}_{0}$ by a
distance $\delta$; all of which assumed to lie within the same manifold. The
parameter $\delta$ was chosen small compared to the grid spacing, in order to
limit the inherent errors of this linearization.

An interpolation curve $\mathcal{C}_{1}$ was then made, with a view to
representing the innermost level set in a smoother fashion. In particular,
this interpolation curve was designed as a parametric spline. To this end,
the points $\{\mathcal{M}_{1,j}\}_{j=1}^{n}$ were first ordered in clockwise
or counterclockwise --- merely a matter of perspective --- fashion. Then,
each mesh point was assigned an independent variable $s$ based upon the
cumulative interpoint distance along the ordered list of points, starting
at $j=1$; estimated by means of the Euclidean norm, then normalized by dividing
$s$ through by the total interpoint distance around the entire initial level
set.

In an intermediary step, the coordinates of the mesh point
$\mathcal{M}_{1,1}$ was appended to a list containing the ordered coordinates
of the mesh points $\{\mathcal{M}_{1,j}\}_{j=1}^{n}$, such that, aside from
$\mathcal{M}_{1,1}$, which corresponded to both $s=0$ and $s=1$, there was a
one-to-one correspondence between the $s$-values and the coordinates of the
mesh points. Then, considering each of the lists containing the Cartesian
coordinates of the mesh points as univariate functions of the pseudo-arclength
parameter $s$, separate cubic B-splines were made for each set of coordinates,
making use of the \texttt{bspline\_1d} extension type from the Bspline-Fortran
library \parencite{williams2018bspline}, which was ported to Python as outlined
in \cref{sub:interpolating_gridded_velocity_data}. The constructed innermost
level set and its associated interpolation curve is illustrated in
\cref{fig:innermost_levelset}.



\begin{figure}[htpb]
    \centering
    \resizebox{0.9\linewidth}{!}{\includestandalone{figures/tikz-figs/initial_levelset_and_interpolation}}
    \caption[The construction of the innermost geodesic level set]
    {The construction of the innermost geodesic level set. An initial
        condition $\vct{x}_{0}$ is found by means of the method outlined in
        \cref{sub:identifying_suitable_initial_conditions_for_developing_lcss}.
        A set of $n$ meshpoints $\{\mathcal{M}_{1,j}\}_{j=1}^{n}$ is then evenly
        distributed within the plane defined by the point $\vct{x}_{0}$ and the
        unit normal $\vct{\xi}_{3}(\vct{x}_{0})$, which is shaded. Each mesh
        point is separated from $\vct{x}_{0}$ by a small distance
        $\delta_{\text{init}}$ (dashed). Using a normalized pseudo-arclength
        parameter $s$, the mesh point coordinates are interpolated using cubic
        B-splines, forming the smooth curve $\mathcal{C}_{1}$ (dotted).}
    \label{fig:innermost_levelset}
\end{figure}


