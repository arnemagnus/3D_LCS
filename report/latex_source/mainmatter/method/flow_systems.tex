%\section{The considered flow systems}
%\label{sec:the_considered_flow_systems}

\section{Flow systems defined by analytical velocity fields}
\label{sec:flow_systems_defined_by_analytical_velocity_fields}

\subsection{Steady Arnold-Beltrami-Childress flow}
\label{sub:steady_arnold_beltrami_childress_flow}

The Arnold-Beltrami-Childress (henceforth abbreviated to ABC) flow is a
three-dimensional, incompressible velocity field which solves the Euler
equations exactly. It is a simple example of a fluid flow which can exhibit
chaotic behaviour \parencite[p.204]{frisch1995turbulence}. In terms of the
Cartesian coordinate vector $\vct{x}=(x,y,z)$, the system can be expressed
mathematically as
\begin{equation}
    \label{eq:abc_flow}
    \dot{\vct{x}} = \vct{v}(t,\vct{x}) = %
    \begin{pmatrix}
        A\sin(z) + C\cos(y)\\
        B\sin(x) + A\cos(z)\\
        C\sin(y) + B\cos(x)
    \end{pmatrix},
\end{equation}
where $A$, $B$ and $C$ are parameters which dictate the nature of the flow
pattern. The inherent periodicity with regards to the Cartesian axes naturally
leads to a domain of interest $\mathcal{U} = [0,2\pi]^{3}$, with
periodic boundary conditions imposed in $x$, $y$ and $z$.

Here, the parameter values
\begin{equation}
    \label{eq:abc_params_stationary}
    A = \sqrt{3},\quad B = \sqrt{2},\quad C = 1
\end{equation}
were used, as has been common in litterature (e.g.\ by
\textcite{oettinger2016autonomous}), as these values are known to result in
chaotic tracer trajectories \parencite{zhao1993chaotic}. The time domain of
interest for this system was $\mathcal{I}=[0,5]$.

\subsection{Unsteady Arnold-Beltrami-Childress flow}%
\label{sub:unsteady_arnold_beltrami_childress_flow}

Inspired by~\textcite{oettinger2016autonomous}, a temporally aperiodic
modification of the ABC flow (\cref{eq:abc_flow}) was made by the replacements
\begin{equation}
    \label{eq:abc_params_nonstationary}
    \begin{gathered}
    B\to{}\widetilde{B}(t) = B + B\cdot{}k_{0}\tanh(k_{1}t)\cos({({k_{2}t})}^{2}),\\
    C\to{}\widetilde{C}(t) = C + C\cdot{}k_{0}\tanh(k_{1}t)\sin({({k_{3}t})}^{2}),
    \end{gathered}
\end{equation}
with $A$, $B$ and $C$ given by \cref{eq:abc_params_stationary}, where the
parameters values
\begin{equation}
    \label{eq:abc_params_nonstationary_frequencies}
    k_{0}=0.3,\quad k_{1}=0.5,\quad k_{2}=1.5,\quad k_{3}=1.8,
\end{equation}
were used. The fundamental idea of this modification is to further enhance
the chaotic nature of the resulting flow patterns. Similarly modified
ABC flow has previously been at the centre of other three-dimensional transport
barrier investigations --- including hyperbolic LCSs --- such as the work of
\textcite{blazevski2014hyperbolic}; albeit with quite different methods of
computing said LCSs than the one considered here. Like for its stationary
sibling, the time interval of interest for this system was $\mathcal{I}=[0,5]$.
The time dependence of the $\widetilde{B}$ and  $\widetilde{C}$ coefficients is
illustrated in \cref{fig:abc_timedep_coeff}.

\begin{figure}[htpb]
    \centering
    \resizebox{0.9\textwidth}{!}{
        \importpgf{figures/mpl-figs}{dep-bc-coeff.pgf}
    }
    \caption[Time dependence of the coefficient functions for unsteady ABC flow]
    {Time dependence of the coefficient functions for unsteady ABC flow,
        defined in
        \cref{eq:abc_params_stationary,eq:abc_params_nonstationary,%
        eq:abc_params_nonstationary_frequencies}.
    }
    \label{fig:abc_timedep_coeff}
\end{figure}



\section{Flow systems defined by gridded velocity data}
\label{sec:flow_systems_defined_by_gridded_velocity_data}

\subsection{Oceanic currents in the Førde fjord}
\label{ssub:oceanic_currents_in_the_forde_fjord}

In 2016, the mining company Nordic Mining ASA received permission from the
Norwegian Ministry of Climate and Environment to extract rutile from the
Engebø mountain in Naustdal, Norway \parencite{garvik2017gruvekonflikten,%
haugan2015sjodeponi}. Furthermore, the company were authorized to dump
the mining waste into the nearby Førde fjord; a legislation which has been
debated fiercely, and heavily protested against, ever since the original
application was submitted in 2008. Early estimates suggest that, when operating
at full scale, the mining operation will result in yearly oceanic mine tailings
deposits in excess of five million tonnes \parencite{garvik2017gruvekonflikten}.

Several centres of technical expertise --- such as the Norwegian Institute of
Marine Research --- have publically advised against depositing mine tailings
into the fjord, emphasizing the potentially severe negative consequences
for the aquaculture \parencite{haugan2015sjodeponi}. Not only is the surrounding
area a significant spawning ground for cod, there is always a possibility of
of particles being transported by the water currents such that they
contaminate the outer edges of the fjord, or even into the ocean. Accordingly,
the use of LCSs in order to predict possible flow patterns for contaminants
resulting from the deposit of mine tailings would be of great environmental
interest.

To this end, gridded three-dimensional velocity data, modelling oceanic
currents in the (depths of the) Førde Fjord, was made available by SINTEF
Fisheries and Aquaculture. The data set considered here contains velocity data
for the time period between June 1 2013, 00:00 and June 2 2013, 23:40, sampled
in intervals of 20 minutes. For our simulations, the time interval of interest
was the 12 hour time window between 00:00 and 12:00 on June 1 2013 ---
practically ensuring the encapsulation of a tidal cycle.

Seeing as some of the most harmful contaminants found within the mine tailings
are heavy metals, we concentrated our analysis on the depths of the fjord.
Accordingly, we looked for LCSs in a region of water which was neither
particularly close to the oceanic surface, nor hit the coastline when
advected for the 12 hour duration of the time interval of interest. This
limited our research to a
$500\,\si{\meter}\times500\,\si{\meter}\times250\,\si{\meter}$ region, with
depths ranging from $50\,\si{\meter}$ to $300\,\si{\meter}$ below the
surface. A bird's-eye view of the region is shown in white in
\cref{fig:currentmap}, which also contains a map view of the geographical
region.

\begin{figure}[htpb]
    \centering
    \resizebox{0.9\textwidth}{!}{
        \importpgf{figures/mpl-figs}{currentmap-hires.pgf}
    }
    \caption[Stereographical map projection of the Førde fjord and its
    surroundings]
    {Stereographical map projection of the Førde fjord and its surroundings.
        The local fjord depths are indicated by a varying background color. A
        subset of the gridded velocity vectors indicates the macroscopic trends
        of the oceanic currents at a depth of $50\,\si{\meter}$ below the
        surface. Outlined in white is a bird's-eye view of the main region of
        interest; namely, a region of water which was neither particularly
        close to the oceanic surface, nor struck the coastline when transported
        by the currents for the duration of the 18 hour time interval of
    interest.}
    \label{fig:currentmap}
\end{figure}



\subsection{Interpolating gridded velocity data}
\label{sub:interpolating_gridded_velocity_data}

In order to describe transport phenomena in the Førde fjord, interpolating
the discretely sampled velocity field becomes necessary. Based upon the
considerations presented in \cref{sub:spline_interpolation_of_discrete_data},
we elected to do so by means of cubic B-splines in time and space. Thus,
each of the velocity field's three components was considered to be a
quadvariate function of time and the three spatial coordinates.

Several multidimensional B-spline interpolation libraries are publically
available under open access licensing. For this project, we elected to use
the Bspline-Fortran library, partly motivated by its extensive documentation
\parencite{williams2018bspline}. In particular, we made use of the
\texttt{bspline\_4d} derived type, which we, along with a subset of its
type-bound procedures, made available in C making use of the Fortran standard
interoperability with C-languages, that is, the \texttt{iso\_c\_binding} module
--- which is shipped with most modern Fortran compilers. From there, we wrote a
thin wrapper class in C++, which was exposed to Python via Cython.

The choice of Python as our main programming language was made partly due to
its relatively low development costs, in addition to its beneficial properties
as a `glue language' --- as examplified by the above, and the ease of
parallelization by means of e.g. MPI (the use of which will be outlined in
greater detail in \cref{sub:advecting_a_set_of_tracers} and
\emph{\textbf{SETT INN REF TIL AVSN OM BEREGNING AV MANGFOLDIGHETER}}).
Moreover, by utilizing Fortran's reference-based subroutine call structure, in
addition to pointers at C-level (typed memoryviews in Cython), we were able to
minimize memory duplication. This could otherwise have been a significant
bottleneck.

