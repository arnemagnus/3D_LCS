\section{Computed LCSs in the Førde fjord}
\label{sec:computed_lcss_in_the_forde_fjord}

Just like for the ABC flow, we used the $\mathcal{U}_{0}$ domain --- i.e.,
the grid points satisfying the LCS existence criteria
\eqref{eq:lcs_condition_a},~\eqref{eq:lcs_condition_b} and~
\eqref{eq:lcs_condition_d} (the implementations of which are described in
\cref{sub:identifying_suitable_initial_conditions_for_developing_lcss}) ---
obtained for the transport system governed by the oceanic currents in the
Førde fjord as a first approximation to where repelling LCSs can reasonably
be expected to be found. Four different views of the $\mathcal{U}_{0}$ domain
are shown in \cref{fig:fjord_abd}. Compared to the corresponding domains
for the two variants of the ABC flow (as shown in
\cref{fig:steady_abd,fig:unsteady_abd}), the flow in the Førde fjord appears
a lot more chaotic, with fewer discernible macroscopic trends.
\Cref{fig:fjord_abd_y,fig:fjord_abd_x} do, however, indicate that the
points in the oceanic $\mathcal{U}_{0}$ domain are loosely organized
in horizontal layers.

\begin{figure}[htpb]
    \centering
    \begin{subfigure}[b]{0.475\textwidth}
        \centering
        %% Creator: Matplotlib, PGF backend
%%
%% To include the figure in your LaTeX document, write
%%   \input{<filename>.pgf}
%%
%% Make sure the required packages are loaded in your preamble
%%   \usepackage{pgf}
%%
%% Figures using additional raster images can only be included by \input if
%% they are in the same directory as the main LaTeX file. For loading figures
%% from other directories you can use the `import` package
%%   \usepackage{import}
%% and then include the figures with
%%   \import{<path to file>}{<filename>.pgf}
%%
%% Matplotlib used the following preamble
%%   \usepackage{fontspec}
%%   \setmainfont{DejaVu Serif}
%%   \setsansfont{DejaVu Sans}
%%   \setmonofont{DejaVu Sans Mono}
%%
\begingroup%
\makeatletter%
\begin{pgfpicture}%
\pgfpathrectangle{\pgfpointorigin}{\pgfqpoint{2.660000in}{1.740000in}}%
\pgfusepath{use as bounding box, clip}%
\begin{pgfscope}%
\pgfsetbuttcap%
\pgfsetmiterjoin%
\definecolor{currentfill}{rgb}{1.000000,1.000000,1.000000}%
\pgfsetfillcolor{currentfill}%
\pgfsetlinewidth{0.000000pt}%
\definecolor{currentstroke}{rgb}{1.000000,1.000000,1.000000}%
\pgfsetstrokecolor{currentstroke}%
\pgfsetdash{}{0pt}%
\pgfpathmoveto{\pgfqpoint{0.000000in}{0.000000in}}%
\pgfpathlineto{\pgfqpoint{2.660000in}{0.000000in}}%
\pgfpathlineto{\pgfqpoint{2.660000in}{1.740000in}}%
\pgfpathlineto{\pgfqpoint{0.000000in}{1.740000in}}%
\pgfpathclose%
\pgfusepath{fill}%
\end{pgfscope}%
\begin{pgfscope}%
\pgfsetbuttcap%
\pgfsetmiterjoin%
\definecolor{currentfill}{rgb}{1.000000,1.000000,1.000000}%
\pgfsetfillcolor{currentfill}%
\pgfsetlinewidth{0.000000pt}%
\definecolor{currentstroke}{rgb}{0.000000,0.000000,0.000000}%
\pgfsetstrokecolor{currentstroke}%
\pgfsetstrokeopacity{0.000000}%
\pgfsetdash{}{0pt}%
\pgfpathmoveto{\pgfqpoint{-0.798000in}{-0.174000in}}%
\pgfpathlineto{\pgfqpoint{3.059000in}{-0.174000in}}%
\pgfpathlineto{\pgfqpoint{3.059000in}{2.088000in}}%
\pgfpathlineto{\pgfqpoint{-0.798000in}{2.088000in}}%
\pgfpathclose%
\pgfusepath{fill}%
\end{pgfscope}%
\begin{pgfscope}%
\pgfsetbuttcap%
\pgfsetmiterjoin%
\pgfsetlinewidth{0.000000pt}%
\definecolor{currentstroke}{rgb}{1.000000,1.000000,1.000000}%
\pgfsetstrokecolor{currentstroke}%
\pgfsetstrokeopacity{0.000000}%
\pgfsetdash{}{0pt}%
\pgfpathmoveto{\pgfqpoint{0.148693in}{0.360411in}}%
\pgfpathlineto{\pgfqpoint{0.152259in}{1.570383in}}%
\pgfpathlineto{\pgfqpoint{0.039322in}{1.681066in}}%
\pgfpathlineto{\pgfqpoint{0.034929in}{0.338385in}}%
\pgfusepath{}%
\end{pgfscope}%
\begin{pgfscope}%
\pgfsetbuttcap%
\pgfsetmiterjoin%
\pgfsetlinewidth{0.000000pt}%
\definecolor{currentstroke}{rgb}{1.000000,1.000000,1.000000}%
\pgfsetstrokecolor{currentstroke}%
\pgfsetstrokeopacity{0.000000}%
\pgfsetdash{}{0pt}%
\pgfpathmoveto{\pgfqpoint{0.152259in}{1.570383in}}%
\pgfpathlineto{\pgfqpoint{2.212984in}{1.570383in}}%
\pgfpathlineto{\pgfqpoint{2.325921in}{1.681066in}}%
\pgfpathlineto{\pgfqpoint{0.039322in}{1.681066in}}%
\pgfusepath{}%
\end{pgfscope}%
\begin{pgfscope}%
\pgfsetbuttcap%
\pgfsetmiterjoin%
\pgfsetlinewidth{0.000000pt}%
\definecolor{currentstroke}{rgb}{1.000000,1.000000,1.000000}%
\pgfsetstrokecolor{currentstroke}%
\pgfsetstrokeopacity{0.000000}%
\pgfsetdash{}{0pt}%
\pgfpathmoveto{\pgfqpoint{0.148693in}{0.360411in}}%
\pgfpathlineto{\pgfqpoint{2.216550in}{0.360411in}}%
\pgfpathlineto{\pgfqpoint{2.212984in}{1.570383in}}%
\pgfpathlineto{\pgfqpoint{0.152259in}{1.570383in}}%
\pgfusepath{}%
\end{pgfscope}%
\begin{pgfscope}%
\pgfsetrectcap%
\pgfsetroundjoin%
\pgfsetlinewidth{0.803000pt}%
\definecolor{currentstroke}{rgb}{0.000000,0.000000,0.000000}%
\pgfsetstrokecolor{currentstroke}%
\pgfsetdash{}{0pt}%
\pgfpathmoveto{\pgfqpoint{0.148693in}{0.360411in}}%
\pgfpathlineto{\pgfqpoint{2.216550in}{0.360411in}}%
\pgfusepath{stroke}%
\end{pgfscope}%
\begin{pgfscope}%
\pgftext[x=1.182622in,y=0.075589in,,]{\sffamily\fontsize{10.000000}{12.000000}\selectfont \(\displaystyle x\)}%
\end{pgfscope}%
\begin{pgfscope}%
\pgfsetbuttcap%
\pgfsetroundjoin%
\pgfsetlinewidth{0.803000pt}%
\definecolor{currentstroke}{rgb}{0.690196,0.690196,0.690196}%
\pgfsetstrokecolor{currentstroke}%
\pgfsetdash{}{0pt}%
\pgfpathmoveto{\pgfqpoint{0.452548in}{0.360411in}}%
\pgfpathlineto{\pgfqpoint{0.455067in}{1.570383in}}%
\pgfpathlineto{\pgfqpoint{0.375320in}{1.681066in}}%
\pgfusepath{stroke}%
\end{pgfscope}%
\begin{pgfscope}%
\pgfsetbuttcap%
\pgfsetroundjoin%
\pgfsetlinewidth{0.803000pt}%
\definecolor{currentstroke}{rgb}{0.690196,0.690196,0.690196}%
\pgfsetstrokecolor{currentstroke}%
\pgfsetdash{}{0pt}%
\pgfpathmoveto{\pgfqpoint{1.178064in}{0.360411in}}%
\pgfpathlineto{\pgfqpoint{1.178080in}{1.570383in}}%
\pgfpathlineto{\pgfqpoint{1.177582in}{1.681066in}}%
\pgfusepath{stroke}%
\end{pgfscope}%
\begin{pgfscope}%
\pgfsetbuttcap%
\pgfsetroundjoin%
\pgfsetlinewidth{0.803000pt}%
\definecolor{currentstroke}{rgb}{0.690196,0.690196,0.690196}%
\pgfsetstrokecolor{currentstroke}%
\pgfsetdash{}{0pt}%
\pgfpathmoveto{\pgfqpoint{1.903580in}{0.360411in}}%
\pgfpathlineto{\pgfqpoint{1.901093in}{1.570383in}}%
\pgfpathlineto{\pgfqpoint{1.979845in}{1.681066in}}%
\pgfusepath{stroke}%
\end{pgfscope}%
\begin{pgfscope}%
\pgfsetrectcap%
\pgfsetroundjoin%
\pgfsetlinewidth{0.803000pt}%
\definecolor{currentstroke}{rgb}{0.000000,0.000000,0.000000}%
\pgfsetstrokecolor{currentstroke}%
\pgfsetdash{}{0pt}%
\pgfpathmoveto{\pgfqpoint{0.452568in}{0.370124in}}%
\pgfpathlineto{\pgfqpoint{0.452508in}{0.340983in}}%
\pgfusepath{stroke}%
\end{pgfscope}%
\begin{pgfscope}%
\pgftext[x=0.457386in,y=0.277685in,,top]{\sffamily\fontsize{10.000000}{12.000000}\selectfont \(\displaystyle 50\)}%
\end{pgfscope}%
\begin{pgfscope}%
\pgfsetrectcap%
\pgfsetroundjoin%
\pgfsetlinewidth{0.803000pt}%
\definecolor{currentstroke}{rgb}{0.000000,0.000000,0.000000}%
\pgfsetstrokecolor{currentstroke}%
\pgfsetdash{}{0pt}%
\pgfpathmoveto{\pgfqpoint{1.178064in}{0.370124in}}%
\pgfpathlineto{\pgfqpoint{1.178064in}{0.340983in}}%
\pgfusepath{stroke}%
\end{pgfscope}%
\begin{pgfscope}%
\pgftext[x=1.178095in,y=0.277685in,,top]{\sffamily\fontsize{10.000000}{12.000000}\selectfont \(\displaystyle 250\)}%
\end{pgfscope}%
\begin{pgfscope}%
\pgfsetrectcap%
\pgfsetroundjoin%
\pgfsetlinewidth{0.803000pt}%
\definecolor{currentstroke}{rgb}{0.000000,0.000000,0.000000}%
\pgfsetstrokecolor{currentstroke}%
\pgfsetdash{}{0pt}%
\pgfpathmoveto{\pgfqpoint{1.903561in}{0.370124in}}%
\pgfpathlineto{\pgfqpoint{1.903620in}{0.340983in}}%
\pgfusepath{stroke}%
\end{pgfscope}%
\begin{pgfscope}%
\pgftext[x=1.898803in,y=0.277685in,,top]{\sffamily\fontsize{10.000000}{12.000000}\selectfont \(\displaystyle 450\)}%
\end{pgfscope}%
\begin{pgfscope}%
\pgfsetrectcap%
\pgfsetroundjoin%
\pgfsetlinewidth{0.803000pt}%
\definecolor{currentstroke}{rgb}{0.000000,0.000000,0.000000}%
\pgfsetstrokecolor{currentstroke}%
\pgfsetdash{}{0pt}%
\pgfpathmoveto{\pgfqpoint{2.212984in}{1.570383in}}%
\pgfpathlineto{\pgfqpoint{2.216550in}{0.360411in}}%
\pgfusepath{stroke}%
\end{pgfscope}%
\begin{pgfscope}%
\pgftext[x=2.600463in,y=0.958548in,,]{\sffamily\fontsize{10.000000}{12.000000}\selectfont \(\displaystyle y\)}%
\end{pgfscope}%
\begin{pgfscope}%
\pgfsetbuttcap%
\pgfsetroundjoin%
\pgfsetlinewidth{0.803000pt}%
\definecolor{currentstroke}{rgb}{0.690196,0.690196,0.690196}%
\pgfsetstrokecolor{currentstroke}%
\pgfsetdash{}{0pt}%
\pgfpathmoveto{\pgfqpoint{0.035577in}{0.536328in}}%
\pgfpathlineto{\pgfqpoint{0.149218in}{0.538732in}}%
\pgfpathlineto{\pgfqpoint{2.216025in}{0.538732in}}%
\pgfusepath{stroke}%
\end{pgfscope}%
\begin{pgfscope}%
\pgfsetbuttcap%
\pgfsetroundjoin%
\pgfsetlinewidth{0.803000pt}%
\definecolor{currentstroke}{rgb}{0.690196,0.690196,0.690196}%
\pgfsetstrokecolor{currentstroke}%
\pgfsetdash{}{0pt}%
\pgfpathmoveto{\pgfqpoint{0.037120in}{1.008054in}}%
\pgfpathlineto{\pgfqpoint{0.150471in}{0.963776in}}%
\pgfpathlineto{\pgfqpoint{2.214772in}{0.963776in}}%
\pgfusepath{stroke}%
\end{pgfscope}%
\begin{pgfscope}%
\pgfsetbuttcap%
\pgfsetroundjoin%
\pgfsetlinewidth{0.803000pt}%
\definecolor{currentstroke}{rgb}{0.690196,0.690196,0.690196}%
\pgfsetstrokecolor{currentstroke}%
\pgfsetdash{}{0pt}%
\pgfpathmoveto{\pgfqpoint{0.038659in}{1.478512in}}%
\pgfpathlineto{\pgfqpoint{0.151721in}{1.387790in}}%
\pgfpathlineto{\pgfqpoint{2.213522in}{1.387790in}}%
\pgfusepath{stroke}%
\end{pgfscope}%
\begin{pgfscope}%
\pgfsetrectcap%
\pgfsetroundjoin%
\pgfsetlinewidth{0.803000pt}%
\definecolor{currentstroke}{rgb}{0.000000,0.000000,0.000000}%
\pgfsetstrokecolor{currentstroke}%
\pgfsetdash{}{0pt}%
\pgfpathmoveto{\pgfqpoint{2.199490in}{0.538732in}}%
\pgfpathlineto{\pgfqpoint{2.249094in}{0.538732in}}%
\pgfusepath{stroke}%
\end{pgfscope}%
\begin{pgfscope}%
\pgftext[x=2.385779in,y=0.538919in,,top]{\sffamily\fontsize{10.000000}{12.000000}\selectfont \(\displaystyle 50\)}%
\end{pgfscope}%
\begin{pgfscope}%
\pgfsetrectcap%
\pgfsetroundjoin%
\pgfsetlinewidth{0.803000pt}%
\definecolor{currentstroke}{rgb}{0.000000,0.000000,0.000000}%
\pgfsetstrokecolor{currentstroke}%
\pgfsetdash{}{0pt}%
\pgfpathmoveto{\pgfqpoint{2.198258in}{0.963776in}}%
\pgfpathlineto{\pgfqpoint{2.247801in}{0.963776in}}%
\pgfusepath{stroke}%
\end{pgfscope}%
\begin{pgfscope}%
\pgftext[x=2.384332in,y=0.960328in,,top]{\sffamily\fontsize{10.000000}{12.000000}\selectfont \(\displaystyle 250\)}%
\end{pgfscope}%
\begin{pgfscope}%
\pgfsetrectcap%
\pgfsetroundjoin%
\pgfsetlinewidth{0.803000pt}%
\definecolor{currentstroke}{rgb}{0.000000,0.000000,0.000000}%
\pgfsetstrokecolor{currentstroke}%
\pgfsetdash{}{0pt}%
\pgfpathmoveto{\pgfqpoint{2.197028in}{1.387790in}}%
\pgfpathlineto{\pgfqpoint{2.246511in}{1.387790in}}%
\pgfusepath{stroke}%
\end{pgfscope}%
\begin{pgfscope}%
\pgftext[x=2.382890in,y=1.380726in,,top]{\sffamily\fontsize{10.000000}{12.000000}\selectfont \(\displaystyle 450\)}%
\end{pgfscope}%
\begin{pgfscope}%
\pgfsetrectcap%
\pgfsetroundjoin%
\pgfsetlinewidth{0.803000pt}%
\definecolor{currentstroke}{rgb}{0.000000,0.000000,0.000000}%
\pgfsetstrokecolor{currentstroke}%
\pgfsetdash{}{0pt}%
\pgfpathmoveto{\pgfqpoint{2.212984in}{1.570383in}}%
\pgfpathlineto{\pgfqpoint{2.325921in}{1.681066in}}%
\pgfusepath{stroke}%
\end{pgfscope}%
\begin{pgfscope}%
\pgftext[x=2.407937in,y=1.705736in,,]{\sffamily\fontsize{10.000000}{12.000000}\selectfont \(\displaystyle z\)}%
\end{pgfscope}%
\begin{pgfscope}%
\pgfsetbuttcap%
\pgfsetroundjoin%
\pgfsetlinewidth{0.803000pt}%
\definecolor{currentstroke}{rgb}{0.690196,0.690196,0.690196}%
\pgfsetstrokecolor{currentstroke}%
\pgfsetdash{}{0pt}%
\pgfpathmoveto{\pgfqpoint{2.218585in}{1.575872in}}%
\pgfpathlineto{\pgfqpoint{0.146659in}{1.575872in}}%
\pgfpathlineto{\pgfqpoint{0.143053in}{0.359319in}}%
\pgfusepath{stroke}%
\end{pgfscope}%
\begin{pgfscope}%
\pgfsetbuttcap%
\pgfsetroundjoin%
\pgfsetlinewidth{0.803000pt}%
\definecolor{currentstroke}{rgb}{0.690196,0.690196,0.690196}%
\pgfsetstrokecolor{currentstroke}%
\pgfsetdash{}{0pt}%
\pgfpathmoveto{\pgfqpoint{2.237050in}{1.593969in}}%
\pgfpathlineto{\pgfqpoint{0.128193in}{1.593969in}}%
\pgfpathlineto{\pgfqpoint{0.124457in}{0.355719in}}%
\pgfusepath{stroke}%
\end{pgfscope}%
\begin{pgfscope}%
\pgfsetbuttcap%
\pgfsetroundjoin%
\pgfsetlinewidth{0.803000pt}%
\definecolor{currentstroke}{rgb}{0.690196,0.690196,0.690196}%
\pgfsetstrokecolor{currentstroke}%
\pgfsetdash{}{0pt}%
\pgfpathmoveto{\pgfqpoint{2.256186in}{1.612723in}}%
\pgfpathlineto{\pgfqpoint{0.109057in}{1.612723in}}%
\pgfpathlineto{\pgfqpoint{0.105184in}{0.351987in}}%
\pgfusepath{stroke}%
\end{pgfscope}%
\begin{pgfscope}%
\pgfsetbuttcap%
\pgfsetroundjoin%
\pgfsetlinewidth{0.803000pt}%
\definecolor{currentstroke}{rgb}{0.690196,0.690196,0.690196}%
\pgfsetstrokecolor{currentstroke}%
\pgfsetdash{}{0pt}%
\pgfpathmoveto{\pgfqpoint{2.276030in}{1.632171in}}%
\pgfpathlineto{\pgfqpoint{0.089213in}{1.632171in}}%
\pgfpathlineto{\pgfqpoint{0.085196in}{0.348117in}}%
\pgfusepath{stroke}%
\end{pgfscope}%
\begin{pgfscope}%
\pgfsetbuttcap%
\pgfsetroundjoin%
\pgfsetlinewidth{0.803000pt}%
\definecolor{currentstroke}{rgb}{0.690196,0.690196,0.690196}%
\pgfsetstrokecolor{currentstroke}%
\pgfsetdash{}{0pt}%
\pgfpathmoveto{\pgfqpoint{2.296621in}{1.652350in}}%
\pgfpathlineto{\pgfqpoint{0.068623in}{1.652350in}}%
\pgfpathlineto{\pgfqpoint{0.064452in}{0.344101in}}%
\pgfusepath{stroke}%
\end{pgfscope}%
\begin{pgfscope}%
\pgfsetbuttcap%
\pgfsetroundjoin%
\pgfsetlinewidth{0.803000pt}%
\definecolor{currentstroke}{rgb}{0.690196,0.690196,0.690196}%
\pgfsetstrokecolor{currentstroke}%
\pgfsetdash{}{0pt}%
\pgfpathmoveto{\pgfqpoint{2.318002in}{1.673305in}}%
\pgfpathlineto{\pgfqpoint{0.047242in}{1.673305in}}%
\pgfpathlineto{\pgfqpoint{0.042909in}{0.339930in}}%
\pgfusepath{stroke}%
\end{pgfscope}%
\begin{pgfscope}%
\pgfsetrectcap%
\pgfsetroundjoin%
\pgfsetlinewidth{0.803000pt}%
\definecolor{currentstroke}{rgb}{0.000000,0.000000,0.000000}%
\pgfsetstrokecolor{currentstroke}%
\pgfsetdash{}{0pt}%
\pgfpathmoveto{\pgfqpoint{2.202009in}{1.575872in}}%
\pgfpathlineto{\pgfqpoint{2.251736in}{1.575872in}}%
\pgfusepath{stroke}%
\end{pgfscope}%
\begin{pgfscope}%
\pgfsetrectcap%
\pgfsetroundjoin%
\pgfsetlinewidth{0.803000pt}%
\definecolor{currentstroke}{rgb}{0.000000,0.000000,0.000000}%
\pgfsetstrokecolor{currentstroke}%
\pgfsetdash{}{0pt}%
\pgfpathmoveto{\pgfqpoint{2.220180in}{1.593969in}}%
\pgfpathlineto{\pgfqpoint{2.270792in}{1.593969in}}%
\pgfusepath{stroke}%
\end{pgfscope}%
\begin{pgfscope}%
\pgfsetrectcap%
\pgfsetroundjoin%
\pgfsetlinewidth{0.803000pt}%
\definecolor{currentstroke}{rgb}{0.000000,0.000000,0.000000}%
\pgfsetstrokecolor{currentstroke}%
\pgfsetdash{}{0pt}%
\pgfpathmoveto{\pgfqpoint{2.239009in}{1.612723in}}%
\pgfpathlineto{\pgfqpoint{2.290541in}{1.612723in}}%
\pgfusepath{stroke}%
\end{pgfscope}%
\begin{pgfscope}%
\pgfsetrectcap%
\pgfsetroundjoin%
\pgfsetlinewidth{0.803000pt}%
\definecolor{currentstroke}{rgb}{0.000000,0.000000,0.000000}%
\pgfsetstrokecolor{currentstroke}%
\pgfsetdash{}{0pt}%
\pgfpathmoveto{\pgfqpoint{2.258535in}{1.632171in}}%
\pgfpathlineto{\pgfqpoint{2.311019in}{1.632171in}}%
\pgfusepath{stroke}%
\end{pgfscope}%
\begin{pgfscope}%
\pgfsetrectcap%
\pgfsetroundjoin%
\pgfsetlinewidth{0.803000pt}%
\definecolor{currentstroke}{rgb}{0.000000,0.000000,0.000000}%
\pgfsetstrokecolor{currentstroke}%
\pgfsetdash{}{0pt}%
\pgfpathmoveto{\pgfqpoint{2.278797in}{1.652350in}}%
\pgfpathlineto{\pgfqpoint{2.332269in}{1.652350in}}%
\pgfusepath{stroke}%
\end{pgfscope}%
\begin{pgfscope}%
\pgfsetrectcap%
\pgfsetroundjoin%
\pgfsetlinewidth{0.803000pt}%
\definecolor{currentstroke}{rgb}{0.000000,0.000000,0.000000}%
\pgfsetstrokecolor{currentstroke}%
\pgfsetdash{}{0pt}%
\pgfpathmoveto{\pgfqpoint{2.299836in}{1.673305in}}%
\pgfpathlineto{\pgfqpoint{2.354334in}{1.673305in}}%
\pgfusepath{stroke}%
\end{pgfscope}%
\begin{pgfscope}%
\pgfsys@transformshift{0.174286in}{0.411429in}%
\pgftext[left,bottom]{\pgfimage[interpolate=true,width=2.014286in,height=1.191429in]{fjord-abd-domain-view1-small-img0.png}}%
\end{pgfscope}%
\end{pgfpicture}%
\makeatother%
\endgroup%

        \caption[]{{\small View along the negative $z$-axis}}
        \label{fig:u0_dom_err_bs32}
    \end{subfigure}
    \begin{subfigure}[b]{0.475\textwidth}
        \centering
        %% Creator: Matplotlib, PGF backend
%%
%% To include the figure in your LaTeX document, write
%%   \input{<filename>.pgf}
%%
%% Make sure the required packages are loaded in your preamble
%%   \usepackage{pgf}
%%
%% Figures using additional raster images can only be included by \input if
%% they are in the same directory as the main LaTeX file. For loading figures
%% from other directories you can use the `import` package
%%   \usepackage{import}
%% and then include the figures with
%%   \import{<path to file>}{<filename>.pgf}
%%
%% Matplotlib used the following preamble
%%   \usepackage{fontspec}
%%   \setmainfont{DejaVu Serif}
%%   \setsansfont{DejaVu Sans}
%%   \setmonofont{DejaVu Sans Mono}
%%
\begingroup%
\makeatletter%
\begin{pgfpicture}%
\pgfpathrectangle{\pgfpointorigin}{\pgfqpoint{2.660000in}{1.740000in}}%
\pgfusepath{use as bounding box, clip}%
\begin{pgfscope}%
\pgfsetbuttcap%
\pgfsetmiterjoin%
\definecolor{currentfill}{rgb}{1.000000,1.000000,1.000000}%
\pgfsetfillcolor{currentfill}%
\pgfsetlinewidth{0.000000pt}%
\definecolor{currentstroke}{rgb}{1.000000,1.000000,1.000000}%
\pgfsetstrokecolor{currentstroke}%
\pgfsetdash{}{0pt}%
\pgfpathmoveto{\pgfqpoint{0.000000in}{0.000000in}}%
\pgfpathlineto{\pgfqpoint{2.660000in}{0.000000in}}%
\pgfpathlineto{\pgfqpoint{2.660000in}{1.740000in}}%
\pgfpathlineto{\pgfqpoint{0.000000in}{1.740000in}}%
\pgfpathclose%
\pgfusepath{fill}%
\end{pgfscope}%
\begin{pgfscope}%
\pgfsetbuttcap%
\pgfsetmiterjoin%
\definecolor{currentfill}{rgb}{1.000000,1.000000,1.000000}%
\pgfsetfillcolor{currentfill}%
\pgfsetlinewidth{0.000000pt}%
\definecolor{currentstroke}{rgb}{0.000000,0.000000,0.000000}%
\pgfsetstrokecolor{currentstroke}%
\pgfsetstrokeopacity{0.000000}%
\pgfsetdash{}{0pt}%
\pgfpathmoveto{\pgfqpoint{-0.798000in}{-0.174000in}}%
\pgfpathlineto{\pgfqpoint{3.005800in}{-0.174000in}}%
\pgfpathlineto{\pgfqpoint{3.005800in}{2.175000in}}%
\pgfpathlineto{\pgfqpoint{-0.798000in}{2.175000in}}%
\pgfpathclose%
\pgfusepath{fill}%
\end{pgfscope}%
\begin{pgfscope}%
\pgfsetbuttcap%
\pgfsetmiterjoin%
\pgfsetlinewidth{0.000000pt}%
\definecolor{currentstroke}{rgb}{1.000000,1.000000,1.000000}%
\pgfsetstrokecolor{currentstroke}%
\pgfsetstrokeopacity{0.000000}%
\pgfsetdash{}{0pt}%
\pgfpathmoveto{\pgfqpoint{0.025573in}{0.334590in}}%
\pgfpathlineto{\pgfqpoint{0.137427in}{0.403664in}}%
\pgfpathlineto{\pgfqpoint{0.137427in}{1.660822in}}%
\pgfpathlineto{\pgfqpoint{0.025573in}{1.729897in}}%
\pgfusepath{}%
\end{pgfscope}%
\begin{pgfscope}%
\pgfsetbuttcap%
\pgfsetmiterjoin%
\pgfsetlinewidth{0.000000pt}%
\definecolor{currentstroke}{rgb}{1.000000,1.000000,1.000000}%
\pgfsetstrokecolor{currentstroke}%
\pgfsetstrokeopacity{0.000000}%
\pgfsetdash{}{0pt}%
\pgfpathmoveto{\pgfqpoint{0.137427in}{0.403664in}}%
\pgfpathlineto{\pgfqpoint{2.173178in}{0.403664in}}%
\pgfpathlineto{\pgfqpoint{2.173178in}{1.660822in}}%
\pgfpathlineto{\pgfqpoint{0.137427in}{1.660822in}}%
\pgfusepath{}%
\end{pgfscope}%
\begin{pgfscope}%
\pgfsetbuttcap%
\pgfsetmiterjoin%
\pgfsetlinewidth{0.000000pt}%
\definecolor{currentstroke}{rgb}{1.000000,1.000000,1.000000}%
\pgfsetstrokecolor{currentstroke}%
\pgfsetstrokeopacity{0.000000}%
\pgfsetdash{}{0pt}%
\pgfpathmoveto{\pgfqpoint{0.025573in}{0.334590in}}%
\pgfpathlineto{\pgfqpoint{2.285032in}{0.334590in}}%
\pgfpathlineto{\pgfqpoint{2.173178in}{0.403664in}}%
\pgfpathlineto{\pgfqpoint{0.137427in}{0.403664in}}%
\pgfusepath{}%
\end{pgfscope}%
\begin{pgfscope}%
\pgfsetrectcap%
\pgfsetroundjoin%
\pgfsetlinewidth{0.803000pt}%
\definecolor{currentstroke}{rgb}{0.000000,0.000000,0.000000}%
\pgfsetstrokecolor{currentstroke}%
\pgfsetdash{}{0pt}%
\pgfpathmoveto{\pgfqpoint{0.025573in}{0.334590in}}%
\pgfpathlineto{\pgfqpoint{2.285032in}{0.334590in}}%
\pgfusepath{stroke}%
\end{pgfscope}%
\begin{pgfscope}%
\pgftext[x=1.155303in,y=0.061176in,,]{\sffamily\fontsize{10.000000}{12.000000}\selectfont \(\displaystyle x\)}%
\end{pgfscope}%
\begin{pgfscope}%
\pgfsetbuttcap%
\pgfsetroundjoin%
\pgfsetlinewidth{0.803000pt}%
\definecolor{currentstroke}{rgb}{0.690196,0.690196,0.690196}%
\pgfsetstrokecolor{currentstroke}%
\pgfsetdash{}{0pt}%
\pgfpathmoveto{\pgfqpoint{0.357583in}{0.334590in}}%
\pgfpathlineto{\pgfqpoint{0.436565in}{0.403664in}}%
\pgfpathlineto{\pgfqpoint{0.436565in}{1.660822in}}%
\pgfusepath{stroke}%
\end{pgfscope}%
\begin{pgfscope}%
\pgfsetbuttcap%
\pgfsetroundjoin%
\pgfsetlinewidth{0.803000pt}%
\definecolor{currentstroke}{rgb}{0.690196,0.690196,0.690196}%
\pgfsetstrokecolor{currentstroke}%
\pgfsetdash{}{0pt}%
\pgfpathmoveto{\pgfqpoint{1.150323in}{0.334590in}}%
\pgfpathlineto{\pgfqpoint{1.150816in}{0.403664in}}%
\pgfpathlineto{\pgfqpoint{1.150816in}{1.660822in}}%
\pgfusepath{stroke}%
\end{pgfscope}%
\begin{pgfscope}%
\pgfsetbuttcap%
\pgfsetroundjoin%
\pgfsetlinewidth{0.803000pt}%
\definecolor{currentstroke}{rgb}{0.690196,0.690196,0.690196}%
\pgfsetstrokecolor{currentstroke}%
\pgfsetdash{}{0pt}%
\pgfpathmoveto{\pgfqpoint{1.943064in}{0.334590in}}%
\pgfpathlineto{\pgfqpoint{1.865067in}{0.403664in}}%
\pgfpathlineto{\pgfqpoint{1.865067in}{1.660822in}}%
\pgfusepath{stroke}%
\end{pgfscope}%
\begin{pgfscope}%
\pgfsetrectcap%
\pgfsetroundjoin%
\pgfsetlinewidth{0.803000pt}%
\definecolor{currentstroke}{rgb}{0.000000,0.000000,0.000000}%
\pgfsetstrokecolor{currentstroke}%
\pgfsetdash{}{0pt}%
\pgfpathmoveto{\pgfqpoint{0.358283in}{0.335202in}}%
\pgfpathlineto{\pgfqpoint{0.356178in}{0.333361in}}%
\pgfusepath{stroke}%
\end{pgfscope}%
\begin{pgfscope}%
\pgftext[x=0.353765in,y=0.270488in,,top]{\sffamily\fontsize{10.000000}{12.000000}\selectfont \(\displaystyle 50\)}%
\end{pgfscope}%
\begin{pgfscope}%
\pgfsetrectcap%
\pgfsetroundjoin%
\pgfsetlinewidth{0.803000pt}%
\definecolor{currentstroke}{rgb}{0.000000,0.000000,0.000000}%
\pgfsetstrokecolor{currentstroke}%
\pgfsetdash{}{0pt}%
\pgfpathmoveto{\pgfqpoint{1.150328in}{0.335202in}}%
\pgfpathlineto{\pgfqpoint{1.150314in}{0.333361in}}%
\pgfusepath{stroke}%
\end{pgfscope}%
\begin{pgfscope}%
\pgftext[x=1.150299in,y=0.270488in,,top]{\sffamily\fontsize{10.000000}{12.000000}\selectfont \(\displaystyle 250\)}%
\end{pgfscope}%
\begin{pgfscope}%
\pgfsetrectcap%
\pgfsetroundjoin%
\pgfsetlinewidth{0.803000pt}%
\definecolor{currentstroke}{rgb}{0.000000,0.000000,0.000000}%
\pgfsetstrokecolor{currentstroke}%
\pgfsetdash{}{0pt}%
\pgfpathmoveto{\pgfqpoint{1.942372in}{0.335202in}}%
\pgfpathlineto{\pgfqpoint{1.944451in}{0.333361in}}%
\pgfusepath{stroke}%
\end{pgfscope}%
\begin{pgfscope}%
\pgftext[x=1.946833in,y=0.270488in,,top]{\sffamily\fontsize{10.000000}{12.000000}\selectfont \(\displaystyle 450\)}%
\end{pgfscope}%
\begin{pgfscope}%
\pgfsetrectcap%
\pgfsetroundjoin%
\pgfsetlinewidth{0.803000pt}%
\definecolor{currentstroke}{rgb}{0.000000,0.000000,0.000000}%
\pgfsetstrokecolor{currentstroke}%
\pgfsetdash{}{0pt}%
\pgfpathmoveto{\pgfqpoint{2.173178in}{0.403664in}}%
\pgfpathlineto{\pgfqpoint{2.285032in}{0.334590in}}%
\pgfusepath{stroke}%
\end{pgfscope}%
\begin{pgfscope}%
\pgftext[x=2.393280in,y=0.267743in,,]{\sffamily\fontsize{10.000000}{12.000000}\selectfont \(\displaystyle y\)}%
\end{pgfscope}%
\begin{pgfscope}%
\pgfsetbuttcap%
\pgfsetroundjoin%
\pgfsetlinewidth{0.803000pt}%
\definecolor{currentstroke}{rgb}{0.690196,0.690196,0.690196}%
\pgfsetstrokecolor{currentstroke}%
\pgfsetdash{}{0pt}%
\pgfpathmoveto{\pgfqpoint{0.043525in}{1.718811in}}%
\pgfpathlineto{\pgfqpoint{0.043525in}{0.345676in}}%
\pgfpathlineto{\pgfqpoint{2.267080in}{0.345676in}}%
\pgfusepath{stroke}%
\end{pgfscope}%
\begin{pgfscope}%
\pgfsetbuttcap%
\pgfsetroundjoin%
\pgfsetlinewidth{0.803000pt}%
\definecolor{currentstroke}{rgb}{0.690196,0.690196,0.690196}%
\pgfsetstrokecolor{currentstroke}%
\pgfsetdash{}{0pt}%
\pgfpathmoveto{\pgfqpoint{0.064225in}{1.706028in}}%
\pgfpathlineto{\pgfqpoint{0.064225in}{0.358458in}}%
\pgfpathlineto{\pgfqpoint{2.246381in}{0.358458in}}%
\pgfusepath{stroke}%
\end{pgfscope}%
\begin{pgfscope}%
\pgfsetbuttcap%
\pgfsetroundjoin%
\pgfsetlinewidth{0.803000pt}%
\definecolor{currentstroke}{rgb}{0.690196,0.690196,0.690196}%
\pgfsetstrokecolor{currentstroke}%
\pgfsetdash{}{0pt}%
\pgfpathmoveto{\pgfqpoint{0.084167in}{1.693713in}}%
\pgfpathlineto{\pgfqpoint{0.084167in}{0.370774in}}%
\pgfpathlineto{\pgfqpoint{2.226438in}{0.370774in}}%
\pgfusepath{stroke}%
\end{pgfscope}%
\begin{pgfscope}%
\pgfsetbuttcap%
\pgfsetroundjoin%
\pgfsetlinewidth{0.803000pt}%
\definecolor{currentstroke}{rgb}{0.690196,0.690196,0.690196}%
\pgfsetstrokecolor{currentstroke}%
\pgfsetdash{}{0pt}%
\pgfpathmoveto{\pgfqpoint{0.103394in}{1.681839in}}%
\pgfpathlineto{\pgfqpoint{0.103394in}{0.382647in}}%
\pgfpathlineto{\pgfqpoint{2.207212in}{0.382647in}}%
\pgfusepath{stroke}%
\end{pgfscope}%
\begin{pgfscope}%
\pgfsetbuttcap%
\pgfsetroundjoin%
\pgfsetlinewidth{0.803000pt}%
\definecolor{currentstroke}{rgb}{0.690196,0.690196,0.690196}%
\pgfsetstrokecolor{currentstroke}%
\pgfsetdash{}{0pt}%
\pgfpathmoveto{\pgfqpoint{0.121942in}{1.670385in}}%
\pgfpathlineto{\pgfqpoint{0.121942in}{0.394102in}}%
\pgfpathlineto{\pgfqpoint{2.188663in}{0.394102in}}%
\pgfusepath{stroke}%
\end{pgfscope}%
\begin{pgfscope}%
\pgfsetrectcap%
\pgfsetroundjoin%
\pgfsetlinewidth{0.803000pt}%
\definecolor{currentstroke}{rgb}{0.000000,0.000000,0.000000}%
\pgfsetstrokecolor{currentstroke}%
\pgfsetdash{}{0pt}%
\pgfpathmoveto{\pgfqpoint{2.249292in}{0.345676in}}%
\pgfpathlineto{\pgfqpoint{2.302657in}{0.345676in}}%
\pgfusepath{stroke}%
\end{pgfscope}%
\begin{pgfscope}%
\pgfsetrectcap%
\pgfsetroundjoin%
\pgfsetlinewidth{0.803000pt}%
\definecolor{currentstroke}{rgb}{0.000000,0.000000,0.000000}%
\pgfsetstrokecolor{currentstroke}%
\pgfsetdash{}{0pt}%
\pgfpathmoveto{\pgfqpoint{2.228924in}{0.358458in}}%
\pgfpathlineto{\pgfqpoint{2.281295in}{0.358458in}}%
\pgfusepath{stroke}%
\end{pgfscope}%
\begin{pgfscope}%
\pgfsetrectcap%
\pgfsetroundjoin%
\pgfsetlinewidth{0.803000pt}%
\definecolor{currentstroke}{rgb}{0.000000,0.000000,0.000000}%
\pgfsetstrokecolor{currentstroke}%
\pgfsetdash{}{0pt}%
\pgfpathmoveto{\pgfqpoint{2.209300in}{0.370774in}}%
\pgfpathlineto{\pgfqpoint{2.260715in}{0.370774in}}%
\pgfusepath{stroke}%
\end{pgfscope}%
\begin{pgfscope}%
\pgfsetrectcap%
\pgfsetroundjoin%
\pgfsetlinewidth{0.803000pt}%
\definecolor{currentstroke}{rgb}{0.000000,0.000000,0.000000}%
\pgfsetstrokecolor{currentstroke}%
\pgfsetdash{}{0pt}%
\pgfpathmoveto{\pgfqpoint{2.190381in}{0.382647in}}%
\pgfpathlineto{\pgfqpoint{2.240873in}{0.382647in}}%
\pgfusepath{stroke}%
\end{pgfscope}%
\begin{pgfscope}%
\pgfsetrectcap%
\pgfsetroundjoin%
\pgfsetlinewidth{0.803000pt}%
\definecolor{currentstroke}{rgb}{0.000000,0.000000,0.000000}%
\pgfsetstrokecolor{currentstroke}%
\pgfsetdash{}{0pt}%
\pgfpathmoveto{\pgfqpoint{2.172129in}{0.394102in}}%
\pgfpathlineto{\pgfqpoint{2.221731in}{0.394102in}}%
\pgfusepath{stroke}%
\end{pgfscope}%
\begin{pgfscope}%
\pgfsetrectcap%
\pgfsetroundjoin%
\pgfsetlinewidth{0.803000pt}%
\definecolor{currentstroke}{rgb}{0.000000,0.000000,0.000000}%
\pgfsetstrokecolor{currentstroke}%
\pgfsetdash{}{0pt}%
\pgfpathmoveto{\pgfqpoint{2.173178in}{0.403664in}}%
\pgfpathlineto{\pgfqpoint{2.173178in}{1.660822in}}%
\pgfusepath{stroke}%
\end{pgfscope}%
\begin{pgfscope}%
\pgftext[x=2.567596in,y=1.032243in,,]{\sffamily\fontsize{10.000000}{12.000000}\selectfont \(\displaystyle z\)}%
\end{pgfscope}%
\begin{pgfscope}%
\pgfsetbuttcap%
\pgfsetroundjoin%
\pgfsetlinewidth{0.803000pt}%
\definecolor{currentstroke}{rgb}{0.690196,0.690196,0.690196}%
\pgfsetstrokecolor{currentstroke}%
\pgfsetdash{}{0pt}%
\pgfpathmoveto{\pgfqpoint{2.173178in}{0.694140in}}%
\pgfpathlineto{\pgfqpoint{0.137427in}{0.694140in}}%
\pgfpathlineto{\pgfqpoint{0.025573in}{0.656986in}}%
\pgfusepath{stroke}%
\end{pgfscope}%
\begin{pgfscope}%
\pgfsetbuttcap%
\pgfsetroundjoin%
\pgfsetlinewidth{0.803000pt}%
\definecolor{currentstroke}{rgb}{0.690196,0.690196,0.690196}%
\pgfsetstrokecolor{currentstroke}%
\pgfsetdash{}{0pt}%
\pgfpathmoveto{\pgfqpoint{2.173178in}{0.915810in}}%
\pgfpathlineto{\pgfqpoint{0.137427in}{0.915810in}}%
\pgfpathlineto{\pgfqpoint{0.025573in}{0.903016in}}%
\pgfusepath{stroke}%
\end{pgfscope}%
\begin{pgfscope}%
\pgfsetbuttcap%
\pgfsetroundjoin%
\pgfsetlinewidth{0.803000pt}%
\definecolor{currentstroke}{rgb}{0.690196,0.690196,0.690196}%
\pgfsetstrokecolor{currentstroke}%
\pgfsetdash{}{0pt}%
\pgfpathmoveto{\pgfqpoint{2.173178in}{1.137481in}}%
\pgfpathlineto{\pgfqpoint{0.137427in}{1.137481in}}%
\pgfpathlineto{\pgfqpoint{0.025573in}{1.149045in}}%
\pgfusepath{stroke}%
\end{pgfscope}%
\begin{pgfscope}%
\pgfsetbuttcap%
\pgfsetroundjoin%
\pgfsetlinewidth{0.803000pt}%
\definecolor{currentstroke}{rgb}{0.690196,0.690196,0.690196}%
\pgfsetstrokecolor{currentstroke}%
\pgfsetdash{}{0pt}%
\pgfpathmoveto{\pgfqpoint{2.173178in}{1.359151in}}%
\pgfpathlineto{\pgfqpoint{0.137427in}{1.359151in}}%
\pgfpathlineto{\pgfqpoint{0.025573in}{1.395075in}}%
\pgfusepath{stroke}%
\end{pgfscope}%
\begin{pgfscope}%
\pgfsetrectcap%
\pgfsetroundjoin%
\pgfsetlinewidth{0.803000pt}%
\definecolor{currentstroke}{rgb}{0.000000,0.000000,0.000000}%
\pgfsetstrokecolor{currentstroke}%
\pgfsetdash{}{0pt}%
\pgfpathmoveto{\pgfqpoint{2.156892in}{0.694140in}}%
\pgfpathlineto{\pgfqpoint{2.205750in}{0.694140in}}%
\pgfusepath{stroke}%
\end{pgfscope}%
\begin{pgfscope}%
\pgftext[x=2.355951in,y=0.697302in,,top]{\sffamily\fontsize{10.000000}{12.000000}\selectfont \(\displaystyle 100\)}%
\end{pgfscope}%
\begin{pgfscope}%
\pgfsetrectcap%
\pgfsetroundjoin%
\pgfsetlinewidth{0.803000pt}%
\definecolor{currentstroke}{rgb}{0.000000,0.000000,0.000000}%
\pgfsetstrokecolor{currentstroke}%
\pgfsetdash{}{0pt}%
\pgfpathmoveto{\pgfqpoint{2.156892in}{0.915810in}}%
\pgfpathlineto{\pgfqpoint{2.205750in}{0.915810in}}%
\pgfusepath{stroke}%
\end{pgfscope}%
\begin{pgfscope}%
\pgftext[x=2.355951in,y=0.916899in,,top]{\sffamily\fontsize{10.000000}{12.000000}\selectfont \(\displaystyle 150\)}%
\end{pgfscope}%
\begin{pgfscope}%
\pgfsetrectcap%
\pgfsetroundjoin%
\pgfsetlinewidth{0.803000pt}%
\definecolor{currentstroke}{rgb}{0.000000,0.000000,0.000000}%
\pgfsetstrokecolor{currentstroke}%
\pgfsetdash{}{0pt}%
\pgfpathmoveto{\pgfqpoint{2.156892in}{1.137481in}}%
\pgfpathlineto{\pgfqpoint{2.205750in}{1.137481in}}%
\pgfusepath{stroke}%
\end{pgfscope}%
\begin{pgfscope}%
\pgftext[x=2.355951in,y=1.136496in,,top]{\sffamily\fontsize{10.000000}{12.000000}\selectfont \(\displaystyle 200\)}%
\end{pgfscope}%
\begin{pgfscope}%
\pgfsetrectcap%
\pgfsetroundjoin%
\pgfsetlinewidth{0.803000pt}%
\definecolor{currentstroke}{rgb}{0.000000,0.000000,0.000000}%
\pgfsetstrokecolor{currentstroke}%
\pgfsetdash{}{0pt}%
\pgfpathmoveto{\pgfqpoint{2.156892in}{1.359151in}}%
\pgfpathlineto{\pgfqpoint{2.205750in}{1.359151in}}%
\pgfusepath{stroke}%
\end{pgfscope}%
\begin{pgfscope}%
\pgftext[x=2.355951in,y=1.356094in,,top]{\sffamily\fontsize{10.000000}{12.000000}\selectfont \(\displaystyle 250\)}%
\end{pgfscope}%
\begin{pgfscope}%
\pgfsys@transformshift{0.161429in}{0.412857in}%
\pgftext[left,bottom]{\pgfimage[interpolate=true,width=1.987143in,height=1.237143in]{fjord-abd-domain-view2-small-img0.png}}%
\end{pgfscope}%
\end{pgfpicture}%
\makeatother%
\endgroup%

        \caption[]{{\small View along the positive $y$-axis}}
        \label{fig:u0_dom_err_bs54}
    \end{subfigure}

    \begin{subfigure}[b]{0.475\textwidth}
        \centering
        %% Creator: Matplotlib, PGF backend
%%
%% To include the figure in your LaTeX document, write
%%   \input{<filename>.pgf}
%%
%% Make sure the required packages are loaded in your preamble
%%   \usepackage{pgf}
%%
%% Figures using additional raster images can only be included by \input if
%% they are in the same directory as the main LaTeX file. For loading figures
%% from other directories you can use the `import` package
%%   \usepackage{import}
%% and then include the figures with
%%   \import{<path to file>}{<filename>.pgf}
%%
%% Matplotlib used the following preamble
%%   \usepackage{fontspec}
%%   \setmainfont{DejaVu Serif}
%%   \setsansfont{DejaVu Sans}
%%   \setmonofont{DejaVu Sans Mono}
%%
\begingroup%
\makeatletter%
\begin{pgfpicture}%
\pgfpathrectangle{\pgfpointorigin}{\pgfqpoint{2.660000in}{1.740000in}}%
\pgfusepath{use as bounding box, clip}%
\begin{pgfscope}%
\pgfsetbuttcap%
\pgfsetmiterjoin%
\definecolor{currentfill}{rgb}{1.000000,1.000000,1.000000}%
\pgfsetfillcolor{currentfill}%
\pgfsetlinewidth{0.000000pt}%
\definecolor{currentstroke}{rgb}{1.000000,1.000000,1.000000}%
\pgfsetstrokecolor{currentstroke}%
\pgfsetdash{}{0pt}%
\pgfpathmoveto{\pgfqpoint{0.000000in}{0.000000in}}%
\pgfpathlineto{\pgfqpoint{2.660000in}{0.000000in}}%
\pgfpathlineto{\pgfqpoint{2.660000in}{1.740000in}}%
\pgfpathlineto{\pgfqpoint{0.000000in}{1.740000in}}%
\pgfpathclose%
\pgfusepath{fill}%
\end{pgfscope}%
\begin{pgfscope}%
\pgfsetbuttcap%
\pgfsetmiterjoin%
\definecolor{currentfill}{rgb}{1.000000,1.000000,1.000000}%
\pgfsetfillcolor{currentfill}%
\pgfsetlinewidth{0.000000pt}%
\definecolor{currentstroke}{rgb}{0.000000,0.000000,0.000000}%
\pgfsetstrokecolor{currentstroke}%
\pgfsetstrokeopacity{0.000000}%
\pgfsetdash{}{0pt}%
\pgfpathmoveto{\pgfqpoint{-0.585200in}{-0.087000in}}%
\pgfpathlineto{\pgfqpoint{2.926000in}{-0.087000in}}%
\pgfpathlineto{\pgfqpoint{2.926000in}{2.175000in}}%
\pgfpathlineto{\pgfqpoint{-0.585200in}{2.175000in}}%
\pgfpathclose%
\pgfusepath{fill}%
\end{pgfscope}%
\begin{pgfscope}%
\pgfsetbuttcap%
\pgfsetmiterjoin%
\pgfsetlinewidth{0.000000pt}%
\definecolor{currentstroke}{rgb}{1.000000,1.000000,1.000000}%
\pgfsetstrokecolor{currentstroke}%
\pgfsetstrokeopacity{0.000000}%
\pgfsetdash{}{0pt}%
\pgfpathmoveto{\pgfqpoint{2.155834in}{0.491752in}}%
\pgfpathlineto{\pgfqpoint{0.279864in}{0.491752in}}%
\pgfpathlineto{\pgfqpoint{0.276617in}{1.701724in}}%
\pgfpathlineto{\pgfqpoint{2.159080in}{1.701724in}}%
\pgfusepath{}%
\end{pgfscope}%
\begin{pgfscope}%
\pgfsetbuttcap%
\pgfsetmiterjoin%
\pgfsetlinewidth{0.000000pt}%
\definecolor{currentstroke}{rgb}{1.000000,1.000000,1.000000}%
\pgfsetstrokecolor{currentstroke}%
\pgfsetstrokeopacity{0.000000}%
\pgfsetdash{}{0pt}%
\pgfpathmoveto{\pgfqpoint{2.258645in}{0.381069in}}%
\pgfpathlineto{\pgfqpoint{2.155834in}{0.491752in}}%
\pgfpathlineto{\pgfqpoint{2.159080in}{1.701724in}}%
\pgfpathlineto{\pgfqpoint{2.262645in}{1.723750in}}%
\pgfusepath{}%
\end{pgfscope}%
\begin{pgfscope}%
\pgfsetbuttcap%
\pgfsetmiterjoin%
\pgfsetlinewidth{0.000000pt}%
\definecolor{currentstroke}{rgb}{1.000000,1.000000,1.000000}%
\pgfsetstrokecolor{currentstroke}%
\pgfsetstrokeopacity{0.000000}%
\pgfsetdash{}{0pt}%
\pgfpathmoveto{\pgfqpoint{2.258645in}{0.381069in}}%
\pgfpathlineto{\pgfqpoint{2.155834in}{0.491752in}}%
\pgfpathlineto{\pgfqpoint{0.279864in}{0.491752in}}%
\pgfpathlineto{\pgfqpoint{0.177052in}{0.381069in}}%
\pgfusepath{}%
\end{pgfscope}%
\begin{pgfscope}%
\pgfsetrectcap%
\pgfsetroundjoin%
\pgfsetlinewidth{0.803000pt}%
\definecolor{currentstroke}{rgb}{0.000000,0.000000,0.000000}%
\pgfsetstrokecolor{currentstroke}%
\pgfsetdash{}{0pt}%
\pgfpathmoveto{\pgfqpoint{0.279864in}{0.491752in}}%
\pgfpathlineto{\pgfqpoint{0.177052in}{0.381069in}}%
\pgfusepath{stroke}%
\end{pgfscope}%
\begin{pgfscope}%
\pgftext[x=0.085586in,y=0.345581in,,]{\sffamily\fontsize{10.000000}{12.000000}\selectfont \(\displaystyle x\)}%
\end{pgfscope}%
\begin{pgfscope}%
\pgfsetbuttcap%
\pgfsetroundjoin%
\pgfsetlinewidth{0.803000pt}%
\definecolor{currentstroke}{rgb}{0.690196,0.690196,0.690196}%
\pgfsetstrokecolor{currentstroke}%
\pgfsetdash{}{0pt}%
\pgfpathmoveto{\pgfqpoint{0.193549in}{0.398830in}}%
\pgfpathlineto{\pgfqpoint{2.242148in}{0.398830in}}%
\pgfpathlineto{\pgfqpoint{2.246021in}{1.720215in}}%
\pgfusepath{stroke}%
\end{pgfscope}%
\begin{pgfscope}%
\pgfsetbuttcap%
\pgfsetroundjoin%
\pgfsetlinewidth{0.803000pt}%
\definecolor{currentstroke}{rgb}{0.690196,0.690196,0.690196}%
\pgfsetstrokecolor{currentstroke}%
\pgfsetdash{}{0pt}%
\pgfpathmoveto{\pgfqpoint{0.212573in}{0.419310in}}%
\pgfpathlineto{\pgfqpoint{2.223124in}{0.419310in}}%
\pgfpathlineto{\pgfqpoint{2.226855in}{1.716138in}}%
\pgfusepath{stroke}%
\end{pgfscope}%
\begin{pgfscope}%
\pgfsetbuttcap%
\pgfsetroundjoin%
\pgfsetlinewidth{0.803000pt}%
\definecolor{currentstroke}{rgb}{0.690196,0.690196,0.690196}%
\pgfsetstrokecolor{currentstroke}%
\pgfsetdash{}{0pt}%
\pgfpathmoveto{\pgfqpoint{0.230903in}{0.439043in}}%
\pgfpathlineto{\pgfqpoint{2.204795in}{0.439043in}}%
\pgfpathlineto{\pgfqpoint{2.208390in}{1.712211in}}%
\pgfusepath{stroke}%
\end{pgfscope}%
\begin{pgfscope}%
\pgfsetbuttcap%
\pgfsetroundjoin%
\pgfsetlinewidth{0.803000pt}%
\definecolor{currentstroke}{rgb}{0.690196,0.690196,0.690196}%
\pgfsetstrokecolor{currentstroke}%
\pgfsetdash{}{0pt}%
\pgfpathmoveto{\pgfqpoint{0.248576in}{0.458069in}}%
\pgfpathlineto{\pgfqpoint{2.187121in}{0.458069in}}%
\pgfpathlineto{\pgfqpoint{2.190589in}{1.708425in}}%
\pgfusepath{stroke}%
\end{pgfscope}%
\begin{pgfscope}%
\pgfsetbuttcap%
\pgfsetroundjoin%
\pgfsetlinewidth{0.803000pt}%
\definecolor{currentstroke}{rgb}{0.690196,0.690196,0.690196}%
\pgfsetstrokecolor{currentstroke}%
\pgfsetdash{}{0pt}%
\pgfpathmoveto{\pgfqpoint{0.265627in}{0.476426in}}%
\pgfpathlineto{\pgfqpoint{2.170070in}{0.476426in}}%
\pgfpathlineto{\pgfqpoint{2.173416in}{1.704773in}}%
\pgfusepath{stroke}%
\end{pgfscope}%
\begin{pgfscope}%
\pgfsetrectcap%
\pgfsetroundjoin%
\pgfsetlinewidth{0.803000pt}%
\definecolor{currentstroke}{rgb}{0.000000,0.000000,0.000000}%
\pgfsetstrokecolor{currentstroke}%
\pgfsetdash{}{0pt}%
\pgfpathmoveto{\pgfqpoint{0.209938in}{0.398830in}}%
\pgfpathlineto{\pgfqpoint{0.160772in}{0.398830in}}%
\pgfusepath{stroke}%
\end{pgfscope}%
\begin{pgfscope}%
\pgfsetrectcap%
\pgfsetroundjoin%
\pgfsetlinewidth{0.803000pt}%
\definecolor{currentstroke}{rgb}{0.000000,0.000000,0.000000}%
\pgfsetstrokecolor{currentstroke}%
\pgfsetdash{}{0pt}%
\pgfpathmoveto{\pgfqpoint{0.228657in}{0.419310in}}%
\pgfpathlineto{\pgfqpoint{0.180404in}{0.419310in}}%
\pgfusepath{stroke}%
\end{pgfscope}%
\begin{pgfscope}%
\pgfsetrectcap%
\pgfsetroundjoin%
\pgfsetlinewidth{0.803000pt}%
\definecolor{currentstroke}{rgb}{0.000000,0.000000,0.000000}%
\pgfsetstrokecolor{currentstroke}%
\pgfsetdash{}{0pt}%
\pgfpathmoveto{\pgfqpoint{0.246694in}{0.439043in}}%
\pgfpathlineto{\pgfqpoint{0.199320in}{0.439043in}}%
\pgfusepath{stroke}%
\end{pgfscope}%
\begin{pgfscope}%
\pgfsetrectcap%
\pgfsetroundjoin%
\pgfsetlinewidth{0.803000pt}%
\definecolor{currentstroke}{rgb}{0.000000,0.000000,0.000000}%
\pgfsetstrokecolor{currentstroke}%
\pgfsetdash{}{0pt}%
\pgfpathmoveto{\pgfqpoint{0.264084in}{0.458069in}}%
\pgfpathlineto{\pgfqpoint{0.217559in}{0.458069in}}%
\pgfusepath{stroke}%
\end{pgfscope}%
\begin{pgfscope}%
\pgfsetrectcap%
\pgfsetroundjoin%
\pgfsetlinewidth{0.803000pt}%
\definecolor{currentstroke}{rgb}{0.000000,0.000000,0.000000}%
\pgfsetstrokecolor{currentstroke}%
\pgfsetdash{}{0pt}%
\pgfpathmoveto{\pgfqpoint{0.280863in}{0.476426in}}%
\pgfpathlineto{\pgfqpoint{0.235156in}{0.476426in}}%
\pgfusepath{stroke}%
\end{pgfscope}%
\begin{pgfscope}%
\pgfsetrectcap%
\pgfsetroundjoin%
\pgfsetlinewidth{0.803000pt}%
\definecolor{currentstroke}{rgb}{0.000000,0.000000,0.000000}%
\pgfsetstrokecolor{currentstroke}%
\pgfsetdash{}{0pt}%
\pgfpathmoveto{\pgfqpoint{2.258645in}{0.381069in}}%
\pgfpathlineto{\pgfqpoint{0.177052in}{0.381069in}}%
\pgfusepath{stroke}%
\end{pgfscope}%
\begin{pgfscope}%
\pgftext[x=1.217849in,y=0.119893in,,]{\sffamily\fontsize{10.000000}{12.000000}\selectfont \(\displaystyle y\)}%
\end{pgfscope}%
\begin{pgfscope}%
\pgfsetbuttcap%
\pgfsetroundjoin%
\pgfsetlinewidth{0.803000pt}%
\definecolor{currentstroke}{rgb}{0.690196,0.690196,0.690196}%
\pgfsetstrokecolor{currentstroke}%
\pgfsetdash{}{0pt}%
\pgfpathmoveto{\pgfqpoint{1.882467in}{1.701724in}}%
\pgfpathlineto{\pgfqpoint{1.880175in}{0.491752in}}%
\pgfpathlineto{\pgfqpoint{1.952772in}{0.381069in}}%
\pgfusepath{stroke}%
\end{pgfscope}%
\begin{pgfscope}%
\pgfsetbuttcap%
\pgfsetroundjoin%
\pgfsetlinewidth{0.803000pt}%
\definecolor{currentstroke}{rgb}{0.690196,0.690196,0.690196}%
\pgfsetstrokecolor{currentstroke}%
\pgfsetdash{}{0pt}%
\pgfpathmoveto{\pgfqpoint{1.221997in}{1.701724in}}%
\pgfpathlineto{\pgfqpoint{1.221983in}{0.491752in}}%
\pgfpathlineto{\pgfqpoint{1.222436in}{0.381069in}}%
\pgfusepath{stroke}%
\end{pgfscope}%
\begin{pgfscope}%
\pgfsetbuttcap%
\pgfsetroundjoin%
\pgfsetlinewidth{0.803000pt}%
\definecolor{currentstroke}{rgb}{0.690196,0.690196,0.690196}%
\pgfsetstrokecolor{currentstroke}%
\pgfsetdash{}{0pt}%
\pgfpathmoveto{\pgfqpoint{0.561527in}{1.701724in}}%
\pgfpathlineto{\pgfqpoint{0.563791in}{0.491752in}}%
\pgfpathlineto{\pgfqpoint{0.492101in}{0.381069in}}%
\pgfusepath{stroke}%
\end{pgfscope}%
\begin{pgfscope}%
\pgfsetrectcap%
\pgfsetroundjoin%
\pgfsetlinewidth{0.803000pt}%
\definecolor{currentstroke}{rgb}{0.000000,0.000000,0.000000}%
\pgfsetstrokecolor{currentstroke}%
\pgfsetdash{}{0pt}%
\pgfpathmoveto{\pgfqpoint{1.952128in}{0.382051in}}%
\pgfpathlineto{\pgfqpoint{1.954063in}{0.379101in}}%
\pgfusepath{stroke}%
\end{pgfscope}%
\begin{pgfscope}%
\pgftext[x=1.954933in,y=0.340477in,,top]{\sffamily\fontsize{10.000000}{12.000000}\selectfont \(\displaystyle 50\)}%
\end{pgfscope}%
\begin{pgfscope}%
\pgfsetrectcap%
\pgfsetroundjoin%
\pgfsetlinewidth{0.803000pt}%
\definecolor{currentstroke}{rgb}{0.000000,0.000000,0.000000}%
\pgfsetstrokecolor{currentstroke}%
\pgfsetdash{}{0pt}%
\pgfpathmoveto{\pgfqpoint{1.222432in}{0.382051in}}%
\pgfpathlineto{\pgfqpoint{1.222444in}{0.379101in}}%
\pgfusepath{stroke}%
\end{pgfscope}%
\begin{pgfscope}%
\pgftext[x=1.222450in,y=0.340477in,,top]{\sffamily\fontsize{10.000000}{12.000000}\selectfont \(\displaystyle 250\)}%
\end{pgfscope}%
\begin{pgfscope}%
\pgfsetrectcap%
\pgfsetroundjoin%
\pgfsetlinewidth{0.803000pt}%
\definecolor{currentstroke}{rgb}{0.000000,0.000000,0.000000}%
\pgfsetstrokecolor{currentstroke}%
\pgfsetdash{}{0pt}%
\pgfpathmoveto{\pgfqpoint{0.492737in}{0.382051in}}%
\pgfpathlineto{\pgfqpoint{0.490826in}{0.379101in}}%
\pgfusepath{stroke}%
\end{pgfscope}%
\begin{pgfscope}%
\pgftext[x=0.489967in,y=0.340477in,,top]{\sffamily\fontsize{10.000000}{12.000000}\selectfont \(\displaystyle 450\)}%
\end{pgfscope}%
\begin{pgfscope}%
\pgfsetrectcap%
\pgfsetroundjoin%
\pgfsetlinewidth{0.803000pt}%
\definecolor{currentstroke}{rgb}{0.000000,0.000000,0.000000}%
\pgfsetstrokecolor{currentstroke}%
\pgfsetdash{}{0pt}%
\pgfpathmoveto{\pgfqpoint{2.258645in}{0.381069in}}%
\pgfpathlineto{\pgfqpoint{2.262645in}{1.723750in}}%
\pgfusepath{stroke}%
\end{pgfscope}%
\begin{pgfscope}%
\pgftext[x=2.612226in,y=1.043218in,,]{\sffamily\fontsize{10.000000}{12.000000}\selectfont \(\displaystyle z\)}%
\end{pgfscope}%
\begin{pgfscope}%
\pgfsetbuttcap%
\pgfsetroundjoin%
\pgfsetlinewidth{0.803000pt}%
\definecolor{currentstroke}{rgb}{0.690196,0.690196,0.690196}%
\pgfsetstrokecolor{currentstroke}%
\pgfsetdash{}{0pt}%
\pgfpathmoveto{\pgfqpoint{2.259567in}{0.690392in}}%
\pgfpathlineto{\pgfqpoint{2.156582in}{0.770583in}}%
\pgfpathlineto{\pgfqpoint{0.279116in}{0.770583in}}%
\pgfusepath{stroke}%
\end{pgfscope}%
\begin{pgfscope}%
\pgfsetbuttcap%
\pgfsetroundjoin%
\pgfsetlinewidth{0.803000pt}%
\definecolor{currentstroke}{rgb}{0.690196,0.690196,0.690196}%
\pgfsetstrokecolor{currentstroke}%
\pgfsetdash{}{0pt}%
\pgfpathmoveto{\pgfqpoint{2.260271in}{0.926813in}}%
\pgfpathlineto{\pgfqpoint{2.157154in}{0.983666in}}%
\pgfpathlineto{\pgfqpoint{0.278544in}{0.983666in}}%
\pgfusepath{stroke}%
\end{pgfscope}%
\begin{pgfscope}%
\pgfsetbuttcap%
\pgfsetroundjoin%
\pgfsetlinewidth{0.803000pt}%
\definecolor{currentstroke}{rgb}{0.690196,0.690196,0.690196}%
\pgfsetstrokecolor{currentstroke}%
\pgfsetdash{}{0pt}%
\pgfpathmoveto{\pgfqpoint{2.260976in}{1.163555in}}%
\pgfpathlineto{\pgfqpoint{2.157726in}{1.197009in}}%
\pgfpathlineto{\pgfqpoint{0.277971in}{1.197009in}}%
\pgfusepath{stroke}%
\end{pgfscope}%
\begin{pgfscope}%
\pgfsetbuttcap%
\pgfsetroundjoin%
\pgfsetlinewidth{0.803000pt}%
\definecolor{currentstroke}{rgb}{0.690196,0.690196,0.690196}%
\pgfsetstrokecolor{currentstroke}%
\pgfsetdash{}{0pt}%
\pgfpathmoveto{\pgfqpoint{2.261682in}{1.400616in}}%
\pgfpathlineto{\pgfqpoint{2.158299in}{1.410613in}}%
\pgfpathlineto{\pgfqpoint{0.277398in}{1.410613in}}%
\pgfusepath{stroke}%
\end{pgfscope}%
\begin{pgfscope}%
\pgfsetrectcap%
\pgfsetroundjoin%
\pgfsetlinewidth{0.803000pt}%
\definecolor{currentstroke}{rgb}{0.000000,0.000000,0.000000}%
\pgfsetstrokecolor{currentstroke}%
\pgfsetdash{}{0pt}%
\pgfpathmoveto{\pgfqpoint{2.258653in}{0.691103in}}%
\pgfpathlineto{\pgfqpoint{2.261399in}{0.688965in}}%
\pgfusepath{stroke}%
\end{pgfscope}%
\begin{pgfscope}%
\pgftext[x=2.423316in,y=0.683761in,,top]{\sffamily\fontsize{10.000000}{12.000000}\selectfont \(\displaystyle 100\)}%
\end{pgfscope}%
\begin{pgfscope}%
\pgfsetrectcap%
\pgfsetroundjoin%
\pgfsetlinewidth{0.803000pt}%
\definecolor{currentstroke}{rgb}{0.000000,0.000000,0.000000}%
\pgfsetstrokecolor{currentstroke}%
\pgfsetdash{}{0pt}%
\pgfpathmoveto{\pgfqpoint{2.259356in}{0.927318in}}%
\pgfpathlineto{\pgfqpoint{2.262105in}{0.925802in}}%
\pgfusepath{stroke}%
\end{pgfscope}%
\begin{pgfscope}%
\pgftext[x=2.424138in,y=0.922112in,,top]{\sffamily\fontsize{10.000000}{12.000000}\selectfont \(\displaystyle 150\)}%
\end{pgfscope}%
\begin{pgfscope}%
\pgfsetrectcap%
\pgfsetroundjoin%
\pgfsetlinewidth{0.803000pt}%
\definecolor{currentstroke}{rgb}{0.000000,0.000000,0.000000}%
\pgfsetstrokecolor{currentstroke}%
\pgfsetdash{}{0pt}%
\pgfpathmoveto{\pgfqpoint{2.260060in}{1.163851in}}%
\pgfpathlineto{\pgfqpoint{2.262813in}{1.162960in}}%
\pgfusepath{stroke}%
\end{pgfscope}%
\begin{pgfscope}%
\pgftext[x=2.424960in,y=1.160788in,,top]{\sffamily\fontsize{10.000000}{12.000000}\selectfont \(\displaystyle 200\)}%
\end{pgfscope}%
\begin{pgfscope}%
\pgfsetrectcap%
\pgfsetroundjoin%
\pgfsetlinewidth{0.803000pt}%
\definecolor{currentstroke}{rgb}{0.000000,0.000000,0.000000}%
\pgfsetstrokecolor{currentstroke}%
\pgfsetdash{}{0pt}%
\pgfpathmoveto{\pgfqpoint{2.260765in}{1.400705in}}%
\pgfpathlineto{\pgfqpoint{2.263521in}{1.400439in}}%
\pgfusepath{stroke}%
\end{pgfscope}%
\begin{pgfscope}%
\pgftext[x=2.425784in,y=1.399790in,,top]{\sffamily\fontsize{10.000000}{12.000000}\selectfont \(\displaystyle 250\)}%
\end{pgfscope}%
\begin{pgfscope}%
\pgfsys@transformshift{0.298571in}{0.461429in}%
\pgftext[left,bottom]{\pgfimage[interpolate=true,width=1.838571in,height=1.190000in]{fjord-abd-domain-view3-small-img0.png}}%
\end{pgfscope}%
\end{pgfpicture}%
\makeatother%
\endgroup%

        \caption[]{{\small View along the positive $x$-axis}}
        \label{fig:u0_dom_err_dp54}
    \end{subfigure}
    \begin{subfigure}[b]{0.475\textwidth}
        \centering
        %% Creator: Matplotlib, PGF backend
%%
%% To include the figure in your LaTeX document, write
%%   \input{<filename>.pgf}
%%
%% Make sure the required packages are loaded in your preamble
%%   \usepackage{pgf}
%%
%% Figures using additional raster images can only be included by \input if
%% they are in the same directory as the main LaTeX file. For loading figures
%% from other directories you can use the `import` package
%%   \usepackage{import}
%% and then include the figures with
%%   \import{<path to file>}{<filename>.pgf}
%%
%% Matplotlib used the following preamble
%%   \usepackage{fontspec}
%%   \setmainfont{DejaVu Serif}
%%   \setsansfont{DejaVu Sans}
%%   \setmonofont{DejaVu Sans Mono}
%%
\begingroup%
\makeatletter%
\begin{pgfpicture}%
\pgfpathrectangle{\pgfpointorigin}{\pgfqpoint{2.660000in}{1.740000in}}%
\pgfusepath{use as bounding box, clip}%
\begin{pgfscope}%
\pgfsetbuttcap%
\pgfsetmiterjoin%
\definecolor{currentfill}{rgb}{1.000000,1.000000,1.000000}%
\pgfsetfillcolor{currentfill}%
\pgfsetlinewidth{0.000000pt}%
\definecolor{currentstroke}{rgb}{1.000000,1.000000,1.000000}%
\pgfsetstrokecolor{currentstroke}%
\pgfsetdash{}{0pt}%
\pgfpathmoveto{\pgfqpoint{0.000000in}{0.000000in}}%
\pgfpathlineto{\pgfqpoint{2.660000in}{0.000000in}}%
\pgfpathlineto{\pgfqpoint{2.660000in}{1.740000in}}%
\pgfpathlineto{\pgfqpoint{0.000000in}{1.740000in}}%
\pgfpathclose%
\pgfusepath{fill}%
\end{pgfscope}%
\begin{pgfscope}%
\pgfsetbuttcap%
\pgfsetmiterjoin%
\definecolor{currentfill}{rgb}{1.000000,1.000000,1.000000}%
\pgfsetfillcolor{currentfill}%
\pgfsetlinewidth{0.000000pt}%
\definecolor{currentstroke}{rgb}{0.000000,0.000000,0.000000}%
\pgfsetstrokecolor{currentstroke}%
\pgfsetstrokeopacity{0.000000}%
\pgfsetdash{}{0pt}%
\pgfpathmoveto{\pgfqpoint{0.106400in}{0.087000in}}%
\pgfpathlineto{\pgfqpoint{2.793000in}{0.087000in}}%
\pgfpathlineto{\pgfqpoint{2.793000in}{1.809600in}}%
\pgfpathlineto{\pgfqpoint{0.106400in}{1.809600in}}%
\pgfpathclose%
\pgfusepath{fill}%
\end{pgfscope}%
\begin{pgfscope}%
\pgfsetbuttcap%
\pgfsetmiterjoin%
\pgfsetlinewidth{0.000000pt}%
\definecolor{currentstroke}{rgb}{1.000000,1.000000,1.000000}%
\pgfsetstrokecolor{currentstroke}%
\pgfsetstrokeopacity{0.000000}%
\pgfsetdash{}{0pt}%
\pgfpathmoveto{\pgfqpoint{1.486005in}{1.102024in}}%
\pgfpathlineto{\pgfqpoint{2.517669in}{0.640836in}}%
\pgfpathlineto{\pgfqpoint{2.596564in}{1.327614in}}%
\pgfpathlineto{\pgfqpoint{1.486005in}{1.787028in}}%
\pgfusepath{}%
\end{pgfscope}%
\begin{pgfscope}%
\pgfsetbuttcap%
\pgfsetmiterjoin%
\pgfsetlinewidth{0.000000pt}%
\definecolor{currentstroke}{rgb}{1.000000,1.000000,1.000000}%
\pgfsetstrokecolor{currentstroke}%
\pgfsetstrokeopacity{0.000000}%
\pgfsetdash{}{0pt}%
\pgfpathmoveto{\pgfqpoint{1.486005in}{1.102024in}}%
\pgfpathlineto{\pgfqpoint{0.454342in}{0.640836in}}%
\pgfpathlineto{\pgfqpoint{0.375447in}{1.327614in}}%
\pgfpathlineto{\pgfqpoint{1.486005in}{1.787028in}}%
\pgfusepath{}%
\end{pgfscope}%
\begin{pgfscope}%
\pgfsetbuttcap%
\pgfsetmiterjoin%
\pgfsetlinewidth{0.000000pt}%
\definecolor{currentstroke}{rgb}{1.000000,1.000000,1.000000}%
\pgfsetstrokecolor{currentstroke}%
\pgfsetstrokeopacity{0.000000}%
\pgfsetdash{}{0pt}%
\pgfpathmoveto{\pgfqpoint{1.486005in}{1.102024in}}%
\pgfpathlineto{\pgfqpoint{0.454342in}{0.640836in}}%
\pgfpathlineto{\pgfqpoint{1.486005in}{0.130864in}}%
\pgfpathlineto{\pgfqpoint{2.517669in}{0.640836in}}%
\pgfusepath{}%
\end{pgfscope}%
\begin{pgfscope}%
\pgfsetrectcap%
\pgfsetroundjoin%
\pgfsetlinewidth{0.803000pt}%
\definecolor{currentstroke}{rgb}{0.000000,0.000000,0.000000}%
\pgfsetstrokecolor{currentstroke}%
\pgfsetdash{}{0pt}%
\pgfpathmoveto{\pgfqpoint{2.517669in}{0.640836in}}%
\pgfpathlineto{\pgfqpoint{1.486005in}{0.130864in}}%
\pgfusepath{stroke}%
\end{pgfscope}%
\begin{pgfscope}%
\pgftext[x=2.242021in,y=0.144069in,,]{\sffamily\fontsize{10.000000}{12.000000}\selectfont \(\displaystyle x\)}%
\end{pgfscope}%
\begin{pgfscope}%
\pgfsetbuttcap%
\pgfsetroundjoin%
\pgfsetlinewidth{0.803000pt}%
\definecolor{currentstroke}{rgb}{0.690196,0.690196,0.690196}%
\pgfsetstrokecolor{currentstroke}%
\pgfsetdash{}{0pt}%
\pgfpathmoveto{\pgfqpoint{2.372619in}{0.569135in}}%
\pgfpathlineto{\pgfqpoint{1.340640in}{1.037040in}}%
\pgfpathlineto{\pgfqpoint{1.330013in}{1.722497in}}%
\pgfusepath{stroke}%
\end{pgfscope}%
\begin{pgfscope}%
\pgfsetbuttcap%
\pgfsetroundjoin%
\pgfsetlinewidth{0.803000pt}%
\definecolor{currentstroke}{rgb}{0.690196,0.690196,0.690196}%
\pgfsetstrokecolor{currentstroke}%
\pgfsetdash{}{0pt}%
\pgfpathmoveto{\pgfqpoint{2.017399in}{0.393543in}}%
\pgfpathlineto{\pgfqpoint{0.985084in}{0.878095in}}%
\pgfpathlineto{\pgfqpoint{0.947790in}{1.564380in}}%
\pgfusepath{stroke}%
\end{pgfscope}%
\begin{pgfscope}%
\pgfsetbuttcap%
\pgfsetroundjoin%
\pgfsetlinewidth{0.803000pt}%
\definecolor{currentstroke}{rgb}{0.690196,0.690196,0.690196}%
\pgfsetstrokecolor{currentstroke}%
\pgfsetdash{}{0pt}%
\pgfpathmoveto{\pgfqpoint{1.649100in}{0.211485in}}%
\pgfpathlineto{\pgfqpoint{0.617090in}{0.713590in}}%
\pgfpathlineto{\pgfqpoint{0.551181in}{1.400311in}}%
\pgfusepath{stroke}%
\end{pgfscope}%
\begin{pgfscope}%
\pgfsetrectcap%
\pgfsetroundjoin%
\pgfsetlinewidth{0.803000pt}%
\definecolor{currentstroke}{rgb}{0.000000,0.000000,0.000000}%
\pgfsetstrokecolor{currentstroke}%
\pgfsetdash{}{0pt}%
\pgfpathmoveto{\pgfqpoint{2.363949in}{0.573066in}}%
\pgfpathlineto{\pgfqpoint{2.389980in}{0.561263in}}%
\pgfusepath{stroke}%
\end{pgfscope}%
\begin{pgfscope}%
\pgftext[x=2.421956in,y=0.514556in,,top]{\sffamily\fontsize{10.000000}{12.000000}\selectfont \(\displaystyle 50\)}%
\end{pgfscope}%
\begin{pgfscope}%
\pgfsetrectcap%
\pgfsetroundjoin%
\pgfsetlinewidth{0.803000pt}%
\definecolor{currentstroke}{rgb}{0.000000,0.000000,0.000000}%
\pgfsetstrokecolor{currentstroke}%
\pgfsetdash{}{0pt}%
\pgfpathmoveto{\pgfqpoint{2.008718in}{0.397617in}}%
\pgfpathlineto{\pgfqpoint{2.034781in}{0.385384in}}%
\pgfusepath{stroke}%
\end{pgfscope}%
\begin{pgfscope}%
\pgftext[x=2.067994in,y=0.338159in,,top]{\sffamily\fontsize{10.000000}{12.000000}\selectfont \(\displaystyle 250\)}%
\end{pgfscope}%
\begin{pgfscope}%
\pgfsetrectcap%
\pgfsetroundjoin%
\pgfsetlinewidth{0.803000pt}%
\definecolor{currentstroke}{rgb}{0.000000,0.000000,0.000000}%
\pgfsetstrokecolor{currentstroke}%
\pgfsetdash{}{0pt}%
\pgfpathmoveto{\pgfqpoint{1.640415in}{0.215711in}}%
\pgfpathlineto{\pgfqpoint{1.666494in}{0.203023in}}%
\pgfusepath{stroke}%
\end{pgfscope}%
\begin{pgfscope}%
\pgftext[x=1.701013in,y=0.155275in,,top]{\sffamily\fontsize{10.000000}{12.000000}\selectfont \(\displaystyle 450\)}%
\end{pgfscope}%
\begin{pgfscope}%
\pgfsetrectcap%
\pgfsetroundjoin%
\pgfsetlinewidth{0.803000pt}%
\definecolor{currentstroke}{rgb}{0.000000,0.000000,0.000000}%
\pgfsetstrokecolor{currentstroke}%
\pgfsetdash{}{0pt}%
\pgfpathmoveto{\pgfqpoint{0.454342in}{0.640836in}}%
\pgfpathlineto{\pgfqpoint{1.486005in}{0.130864in}}%
\pgfusepath{stroke}%
\end{pgfscope}%
\begin{pgfscope}%
\pgftext[x=0.729989in,y=0.144069in,,]{\sffamily\fontsize{10.000000}{12.000000}\selectfont \(\displaystyle y\)}%
\end{pgfscope}%
\begin{pgfscope}%
\pgfsetbuttcap%
\pgfsetroundjoin%
\pgfsetlinewidth{0.803000pt}%
\definecolor{currentstroke}{rgb}{0.690196,0.690196,0.690196}%
\pgfsetstrokecolor{currentstroke}%
\pgfsetdash{}{0pt}%
\pgfpathmoveto{\pgfqpoint{1.641997in}{1.722497in}}%
\pgfpathlineto{\pgfqpoint{1.631371in}{1.037040in}}%
\pgfpathlineto{\pgfqpoint{0.599392in}{0.569135in}}%
\pgfusepath{stroke}%
\end{pgfscope}%
\begin{pgfscope}%
\pgfsetbuttcap%
\pgfsetroundjoin%
\pgfsetlinewidth{0.803000pt}%
\definecolor{currentstroke}{rgb}{0.690196,0.690196,0.690196}%
\pgfsetstrokecolor{currentstroke}%
\pgfsetdash{}{0pt}%
\pgfpathmoveto{\pgfqpoint{2.024221in}{1.564380in}}%
\pgfpathlineto{\pgfqpoint{1.986927in}{0.878095in}}%
\pgfpathlineto{\pgfqpoint{0.954612in}{0.393543in}}%
\pgfusepath{stroke}%
\end{pgfscope}%
\begin{pgfscope}%
\pgfsetbuttcap%
\pgfsetroundjoin%
\pgfsetlinewidth{0.803000pt}%
\definecolor{currentstroke}{rgb}{0.690196,0.690196,0.690196}%
\pgfsetstrokecolor{currentstroke}%
\pgfsetdash{}{0pt}%
\pgfpathmoveto{\pgfqpoint{2.420830in}{1.400311in}}%
\pgfpathlineto{\pgfqpoint{2.354921in}{0.713590in}}%
\pgfpathlineto{\pgfqpoint{1.322910in}{0.211485in}}%
\pgfusepath{stroke}%
\end{pgfscope}%
\begin{pgfscope}%
\pgfsetrectcap%
\pgfsetroundjoin%
\pgfsetlinewidth{0.803000pt}%
\definecolor{currentstroke}{rgb}{0.000000,0.000000,0.000000}%
\pgfsetstrokecolor{currentstroke}%
\pgfsetdash{}{0pt}%
\pgfpathmoveto{\pgfqpoint{0.608062in}{0.573066in}}%
\pgfpathlineto{\pgfqpoint{0.582031in}{0.561263in}}%
\pgfusepath{stroke}%
\end{pgfscope}%
\begin{pgfscope}%
\pgftext[x=0.550055in,y=0.514556in,,top]{\sffamily\fontsize{10.000000}{12.000000}\selectfont \(\displaystyle 50\)}%
\end{pgfscope}%
\begin{pgfscope}%
\pgfsetrectcap%
\pgfsetroundjoin%
\pgfsetlinewidth{0.803000pt}%
\definecolor{currentstroke}{rgb}{0.000000,0.000000,0.000000}%
\pgfsetstrokecolor{currentstroke}%
\pgfsetdash{}{0pt}%
\pgfpathmoveto{\pgfqpoint{0.963292in}{0.397617in}}%
\pgfpathlineto{\pgfqpoint{0.937230in}{0.385384in}}%
\pgfusepath{stroke}%
\end{pgfscope}%
\begin{pgfscope}%
\pgftext[x=0.904017in,y=0.338159in,,top]{\sffamily\fontsize{10.000000}{12.000000}\selectfont \(\displaystyle 250\)}%
\end{pgfscope}%
\begin{pgfscope}%
\pgfsetrectcap%
\pgfsetroundjoin%
\pgfsetlinewidth{0.803000pt}%
\definecolor{currentstroke}{rgb}{0.000000,0.000000,0.000000}%
\pgfsetstrokecolor{currentstroke}%
\pgfsetdash{}{0pt}%
\pgfpathmoveto{\pgfqpoint{1.331596in}{0.215711in}}%
\pgfpathlineto{\pgfqpoint{1.305517in}{0.203023in}}%
\pgfusepath{stroke}%
\end{pgfscope}%
\begin{pgfscope}%
\pgftext[x=1.270998in,y=0.155275in,,top]{\sffamily\fontsize{10.000000}{12.000000}\selectfont \(\displaystyle 450\)}%
\end{pgfscope}%
\begin{pgfscope}%
\pgfsetrectcap%
\pgfsetroundjoin%
\pgfsetlinewidth{0.803000pt}%
\definecolor{currentstroke}{rgb}{0.000000,0.000000,0.000000}%
\pgfsetstrokecolor{currentstroke}%
\pgfsetdash{}{0pt}%
\pgfpathmoveto{\pgfqpoint{0.454342in}{0.640836in}}%
\pgfpathlineto{\pgfqpoint{0.375447in}{1.327614in}}%
\pgfusepath{stroke}%
\end{pgfscope}%
\begin{pgfscope}%
\pgfsetbuttcap%
\pgfsetroundjoin%
\pgfsetlinewidth{0.803000pt}%
\definecolor{currentstroke}{rgb}{0.690196,0.690196,0.690196}%
\pgfsetstrokecolor{currentstroke}%
\pgfsetdash{}{0pt}%
\pgfpathmoveto{\pgfqpoint{0.437125in}{0.790708in}}%
\pgfpathlineto{\pgfqpoint{1.486005in}{1.251936in}}%
\pgfpathlineto{\pgfqpoint{2.534886in}{0.790708in}}%
\pgfusepath{stroke}%
\end{pgfscope}%
\begin{pgfscope}%
\pgfsetbuttcap%
\pgfsetroundjoin%
\pgfsetlinewidth{0.803000pt}%
\definecolor{currentstroke}{rgb}{0.690196,0.690196,0.690196}%
\pgfsetstrokecolor{currentstroke}%
\pgfsetdash{}{0pt}%
\pgfpathmoveto{\pgfqpoint{0.423595in}{0.908489in}}%
\pgfpathlineto{\pgfqpoint{1.486005in}{1.369580in}}%
\pgfpathlineto{\pgfqpoint{2.548416in}{0.908489in}}%
\pgfusepath{stroke}%
\end{pgfscope}%
\begin{pgfscope}%
\pgfsetbuttcap%
\pgfsetroundjoin%
\pgfsetlinewidth{0.803000pt}%
\definecolor{currentstroke}{rgb}{0.690196,0.690196,0.690196}%
\pgfsetstrokecolor{currentstroke}%
\pgfsetdash{}{0pt}%
\pgfpathmoveto{\pgfqpoint{0.409711in}{1.029347in}}%
\pgfpathlineto{\pgfqpoint{1.486005in}{1.490146in}}%
\pgfpathlineto{\pgfqpoint{2.562300in}{1.029347in}}%
\pgfusepath{stroke}%
\end{pgfscope}%
\begin{pgfscope}%
\pgfsetbuttcap%
\pgfsetroundjoin%
\pgfsetlinewidth{0.803000pt}%
\definecolor{currentstroke}{rgb}{0.690196,0.690196,0.690196}%
\pgfsetstrokecolor{currentstroke}%
\pgfsetdash{}{0pt}%
\pgfpathmoveto{\pgfqpoint{0.395459in}{1.153406in}}%
\pgfpathlineto{\pgfqpoint{1.486005in}{1.613744in}}%
\pgfpathlineto{\pgfqpoint{2.576551in}{1.153406in}}%
\pgfusepath{stroke}%
\end{pgfscope}%
\begin{pgfscope}%
\pgfsetrectcap%
\pgfsetroundjoin%
\pgfsetlinewidth{0.803000pt}%
\definecolor{currentstroke}{rgb}{0.000000,0.000000,0.000000}%
\pgfsetstrokecolor{currentstroke}%
\pgfsetdash{}{0pt}%
\pgfpathmoveto{\pgfqpoint{0.445941in}{0.794585in}}%
\pgfpathlineto{\pgfqpoint{0.419471in}{0.782945in}}%
\pgfusepath{stroke}%
\end{pgfscope}%
\begin{pgfscope}%
\pgftext[x=0.284879in,y=0.790708in,,top]{\sffamily\fontsize{10.000000}{12.000000}\selectfont \(\displaystyle 100\)}%
\end{pgfscope}%
\begin{pgfscope}%
\pgfsetrectcap%
\pgfsetroundjoin%
\pgfsetlinewidth{0.803000pt}%
\definecolor{currentstroke}{rgb}{0.000000,0.000000,0.000000}%
\pgfsetstrokecolor{currentstroke}%
\pgfsetdash{}{0pt}%
\pgfpathmoveto{\pgfqpoint{0.432530in}{0.912366in}}%
\pgfpathlineto{\pgfqpoint{0.405702in}{0.900723in}}%
\pgfusepath{stroke}%
\end{pgfscope}%
\begin{pgfscope}%
\pgftext[x=0.269385in,y=0.908489in,,top]{\sffamily\fontsize{10.000000}{12.000000}\selectfont \(\displaystyle 150\)}%
\end{pgfscope}%
\begin{pgfscope}%
\pgfsetrectcap%
\pgfsetroundjoin%
\pgfsetlinewidth{0.803000pt}%
\definecolor{currentstroke}{rgb}{0.000000,0.000000,0.000000}%
\pgfsetstrokecolor{currentstroke}%
\pgfsetdash{}{0pt}%
\pgfpathmoveto{\pgfqpoint{0.418769in}{1.033225in}}%
\pgfpathlineto{\pgfqpoint{0.391572in}{1.021581in}}%
\pgfusepath{stroke}%
\end{pgfscope}%
\begin{pgfscope}%
\pgftext[x=0.253486in,y=1.029347in,,top]{\sffamily\fontsize{10.000000}{12.000000}\selectfont \(\displaystyle 200\)}%
\end{pgfscope}%
\begin{pgfscope}%
\pgfsetrectcap%
\pgfsetroundjoin%
\pgfsetlinewidth{0.803000pt}%
\definecolor{currentstroke}{rgb}{0.000000,0.000000,0.000000}%
\pgfsetstrokecolor{currentstroke}%
\pgfsetdash{}{0pt}%
\pgfpathmoveto{\pgfqpoint{0.404643in}{1.157283in}}%
\pgfpathlineto{\pgfqpoint{0.377068in}{1.145643in}}%
\pgfusepath{stroke}%
\end{pgfscope}%
\begin{pgfscope}%
\pgftext[x=0.237165in,y=1.153406in,,top]{\sffamily\fontsize{10.000000}{12.000000}\selectfont \(\displaystyle 250\)}%
\end{pgfscope}%
\begin{pgfscope}%
\pgfsys@transformshift{0.012857in}{0.241429in}%
\pgftext[left,bottom]{\pgfimage[interpolate=true,width=2.444286in,height=1.434286in]{fjord-abd-domain-view4-small-img0.png}}%
\end{pgfscope}%
\end{pgfpicture}%
\makeatother%
\endgroup%

        \caption[]{{\small Approximately isometric view}}
        \label{fig:u0_dom_err_dp87}
    \end{subfigure}
    \caption[Aviici is love, Aviici is life]{ABD domain obtained for gridded data. The axes have dimension meters. z = 0 corresponds to the sea level, increasing downwards. $\epsilon=1$.}
    \label{fig:u0_dom_errs}
\end{figure}



Using the filtering parameters provided in \cref{tab:gridparams}, we identified
initial conditions for the development of manifolds as a subset of the
$\mathcal{U}_{0}$ domain; yielding a total of \numprint{1631} points.
Then we computed manifolds and extracted LCSs using the parameters
in \cref{tab:initialconditionparams,tab:fjord_manifold_params} and the method
outlined in \cref{sec:preliminaries_for_computing_repelling_lcss_in_3d_flow%
    _by_means_of_geodesic_level_sets,%
    sec:revised_approach_to_computing_new_mesh_points,%
    sec:managing_mesh_accuracy,%
    sec:continuously_reconstructing_three_dimensional_manifold_surfaces_from_%
    point_meshes,%
    sec:macroscale_stopping_criteria_for_the_expansion_of_computed_manifolds,%
    sec:identifying_lcss_as_subsets_of_computed_manifolds}. This resulted in
a total of \numprint{110} LCS surfaces. As \cref{fig:fjord_lcss} indicates,
these largely appear to be organized in a sequence of horizontal layers.
Moreover, all of the LCSs are sufficiently large to warrant treating them
as individual entities. Thus, in contrast to the LCSs obtained in (either
version of) the ABC flow (see \cref{sec:computed_lcss_in_the_abc_flow}), we
elected to assign each LCS a numerical value
\begin{equation}
    \label{eq:fjord_lcs_colorscaling}
    \mathcal{Q}_{i} = %
    \frac{\log{{\big({\mkern2mu}\overline{\lambda}_{3}\big)}_{i}}}%
    {\max\limits_{i}%
    \big\{\log{{\big({\mkern2mu}\overline{\lambda}_{3}\big)}_{i}}\big\}},
\end{equation}
where ${\big({\mkern2mu}\overline{\lambda}_{3}}\big)_{i}$, the repulsion
average of LCS surface $i$, is defined in \cref{eq:lcs_lm3_weight}. We then
used the (unit normalized) set of numbers $\big\{\mathcal{Q}_{i}\big\}$ to
select a color for the corresponding LCSs, drawn from a perceptually uniform
colormap.

\begin{table}[htpb]
    \centering
    \caption[Parameter values used to compute manifolds, and subsequently
    repelling LCSs, in the Førde fjord]
    {
        Parameter values used to compute manifolds, and subsequently repelling
        LCSs, in the Førde fjord (see
        \cref{sec:flow_systems_defined_by_gridded_velocity_data}). Note that,
        as the domain of interest is scaled in units of metre, so too are
        $\delta_{\text{init}}$, $\Delta_{\min}$, $\Delta_{\max}$ and
        $\Delta_{1}$; whereas $\mathcal{W}_{\min}$ is given in square metre.
}
    \label{tab:fjord_manifold_params}
    \begin{tabular}{ccc}
        \toprule
        Parameter & Value & Description\\
        \midrule
        $\delta_{\text{init}}$ & $10^{-1}$ %
        & \makecell{Separation of innermost geodesic level set \\
        from the manifold epicentre, cf. \cref{fig:innermost_levelset}}%
        \\[9pt]
        %
        $\Delta_{\min}$, $\Delta_{\max}$
        & $2$, $8$ %
        & \makecell{Boundaries for interpoint \\separations (details
        found in \cref{sub:maintaining_mesh_point_density})}%
        \\[9pt]
        %
        $\Delta_{1}$ %
        & $2\Delta_{\min}$ %
        & \makecell{Interset distance used to compute the second \\ geodesic
        level set (see \cref{sec:revised_approach_to_computing_new_mesh_points,%
        sub:a_curvature_based_approach_to_determining_interset_separations})}%
        \\[9pt]
        %
        $\gamma_{\Delta}$ %
        & $5\cdot10^{-3}$ %
        & \makecell{Tolerance for the separation of a mesh point\\ from
        its ancestor (per \cref{eq:revised_dist_tol})}
        \\[9pt]
        $\gamma_{\text{arc}}$ %
        & 5 %
        & \makecell{Sets an upper limit to trajectory lengths as \\
        $\gamma_{\text{arc}}\Delta_{i}$ (briefly mentioned in
        \cref{sub:computing_pseudoradial_trajectories_directly})}
        \\[9pt]
        %
        $\gamma_{\circlearrowleft}$ %
        & $7\cdot10^{-1}$
        & \makecell{Sets an upper limit to the extent of loop-like\\
        segments of any level set (see
        \cref{sub:limiting_the_accumulation_of_numerical_noise})}
        \\[9pt]
        %
        \makecell[c]{$\alpha_{\uparrow}$\\ $\alpha_{\downarrow}$ \\[1.5pt]%
        ${(\delta\alpha)}_{\uparrow}$, ${(\delta\alpha)}_{\downarrow}$} &
        \makecell[c]{$8.7\cdot10^{-2}\si{\radian}$ \phantom{2}$(5\si{\degree})$\\ %
            $4.4\cdot10^{-1}\si{\radian}$ $(25\si{\degree})$\\[1.5pt]%
        $2\delta_{\max}\alpha_{\uparrow}$, $2\delta_{\min}\alpha_{\downarrow}$}%
        & \makecell[c]{Used in a curvature-based approach to adjust\\
        interset distances (outlined in
        \cref{sub:a_curvature_based_approach_to_determining_interset_separations})}
        \\[18pt]
        %
        $\gamma_{\square}$ &
        $1.2$ &
        \makecell[c]{Relaxation parameter for extracting LCSs from\\ the
            computed manifolds (see
        \cref{sec:identifying_lcss_as_subsets_of_computed_manifolds})}
        \\[9pt]
        %
        $\mathcal{W}_{\text{min}}$ &
        \numprint{20000} &
        \makecell[c]{Filters away the smallest LCSs measured in\\
        (pseudo-)surface area (see
        \cref{sec:identifying_lcss_as_subsets_of_computed_manifolds})}
        \\
        \bottomrule
    \end{tabular}
\end{table}



\begin{figure}[htpb]
    \centering
    \resizebox{0.5\textwidth}{!}{\importpgf{figures/mpl-figs}{fjord-lcs-colorbar.pgf}}
    \vspace{5.0pt}

    \hspace*{\fill}
    \begin{subfigure}[b]{0.425\textwidth}
        \centering
        \resizebox{0.9\linewidth}{!}{\importpgf{figures/mpl-figs}{fjord-lcss-view1-small.pgf}}
        \caption[]{{\small Top-down view along the $z$-axis.}}
        \label{fig:fjord_lcss_z}
    \end{subfigure}\hfill%
    \begin{subfigure}[b]{0.425\textwidth}
        \centering
        \resizebox{0.9\linewidth}{!}{\importpgf{figures/mpl-figs}{fjord-lcss-view2-small.pgf}}
        \caption[]{{\small View along the positive $y$-axis.}}
        \label{fig:fjord_lcss_y}
    \end{subfigure}
    \hspace*{\fill}

    \hspace*{\fill}
    \begin{subfigure}[b]{0.425\textwidth}
        \centering
        \resizebox{0.9\linewidth}{!}{\importpgf{figures/mpl-figs}{fjord-lcss-view3-small.pgf}}
        \caption[]{{\small View along the negative $x$-axis.}}
        \label{fig:fjord_lcss_x}
    \end{subfigure}\hfill%
    \begin{subfigure}[b]{0.425\textwidth}
        \centering
        \resizebox{0.9\linewidth}{!}{\importpgf{figures/mpl-figs}{fjord-lcss-view4-small.pgf}}
        \caption[]{{\small Approximately isometric view.}}
        \label{fig:fjord_lcss_isometric}
    \end{subfigure}%
    \hspace*{\fill}
    \caption[Four views of the repelling LCSs obtained for transport in the
    Førde fjord]
    {
        Four views of the repelling LCSs obtained for transport in the Førde
        fjord, over a time interval of \numprint{12} hours (see
        \cref{sec:flow_systems_defined_by_gridded_velocity_data}). A total
        of \numprint{110} distinct surfaces are shown, which are colored
        according to their relative repulsion averages (per
        \cref{eq:fjord_lcs_colorscaling}) using the perceptually uniform
        colormap shown at the top. We provide four different viewing angles
        (the same as the ones used in \cref{fig:fjord_abd}, which shows
        the computed $\mathcal{U}_{0}$ domain), chosen in order to convey
        the three-dimensional structures in as great detail as possible.
}
    \label{fig:fjord_lcss}
\end{figure}



When comparing the computed LCSs (\cref{fig:fjord_lcss}) to the corresponding
$\mathcal{U}_{0}$ domain (\cref{fig:fjord_abd}), the organization of
the LCSs in horizontal layers seems reasonable. This is particularly apparent
from inspecting \cref{fig:fjord_abd_y,fig:fjord_lcss_y}, and
\cref{fig:fjord_abd_x,fig:fjord_lcss_x}. This indicates that the oceanic
flow undergoes the most stretching in the vertical direction. Moreover, the
repulsion appears to be largely uniform within any given depth layer.

Similarly to our treatment of an LCS surface in the steady ABC flow in
\cref{sub:verifying_that_the_computed_lcss_are_in_fact_repelling}, we placed
two blobs of particles on either side of the most strongly repulsive LCS
present in the fjord subdomain --- perhaps most easily spotted in the middle
of \cref{fig:fjord_lcss_y,fig:fjord_lcss_x} --- and allowed the oceanic
currents to transport the particles as well as the computed LCS for the
entirety of our \numprint{12} hour time interval of interest. The initial and
final states are shown in \cref{fig:blobtest-fjord}. Like for the ABC flow
case, the triangulated structure of the LCS breaks down, yet the two blobs of
particles diverge from eachother in the transition from
\cref{fig:blobtest-fjord-pre} to \cref{fig:blobtest-fjord-post}. The particles
belonging to either blob remain close together, and none of them appear to
move across the LCS mesh point. This suggests that, like the the ones computed
for the ABC flow, the LCSs obtained for flow in the Førde fjord act as
repelling material surfaces, and thereby transport barriers.

\begin{figure}[htpb]
    \centering
    \hspace*{\fill}
    \begin{subfigure}[b]{0.425\textwidth}
        \centering
        \resizebox{0.9\linewidth}{!}{\importpgf{figures/mpl-figs}{blobtest-fjord-pre-small.pgf}}
        \caption[]{{\small Initial state, June 1 2013, 00:00}}
        \label{fig:blobtest-fjord-pre}
    \end{subfigure}\hfill%
    \begin{subfigure}[b]{0.425\textwidth}
        \centering
        \resizebox{0.9\linewidth}{!}{\importpgf{figures/mpl-figs}{blobtest-fjord-post-small.pgf}}
        \caption[]{{\small Final state, June 1 2013, 12:00}}
        \label{fig:blobtest-fjord-post}
    \end{subfigure}
    \hspace*{\fill}
    \caption[Advection of tracers in order to verify the repelling nature of
    the computed LCSs in the Førde fjord]
    {%
        Advection of tracers in order to verify the repelling nature of the
        computed LCSs in the Førde fjord. At midnight June 1, 2013, two blobs
        of initial conditions are placed at opposite sides of the most strongly
        repulsive LCS identified for flow in the Førde fjord (see
        \cref{fig:fjord_lcss}) as shown in (\subref*{fig:blobtest-pre}). The
        two blobs, and the mesh points in the parametrization of the LCS, are
        then advected in the model data for the oceanic currents briefly
        described in \cref{sub:oceanic_currents_in_the_forde_fjord}, using the
        Dormand-Prince 8(7) ODE solver, for the 12 hour time interval of
        interest, where the final state is shown in
        (\subref*{fig:blobtest-post}). Note that, although the LCS
        triangulation breaks down under the advection, the two blobs of
        particles remain on fairly compact, and remain on the opposite sides of
        the LCS, never crossing the LCS surface. This indicates that the local
        centre of strongest repulsion remains located \emph{inbetween} the
        two blobs of particles throughout --- which
        is the exact behaviour we expect for a repelling LCS.
    }
    \label{fig:blobtest-fjord}
\end{figure}




