\section[Verifying our method of extracting LCSs from the computed manifolds]
{Verifying our method of extracting LCSs from the \\\phantom{4.2} computed
manifolds}
\label{sec:verifying_our_method_of_extracting_repelling_lcss_from_the_computed%
_manifolds}

In \cref{sub:an_analytical_lcs_test_case}, we present an analytical flow field
defined to exhibit a single repelling LCS, and how we may use the accompanying
$\lambda_{3}$ field (i.e.,\ the largest Cauchy-Green strain eigenvalues) to
determine its location. We then show that, using the method outlined in
\cref{cha:method}, we reproduce it exactly.
\Cref{sub:verifying_that_the_computed_lcss_are_in_fact_repelling} contains a
description of how we verified that the computed LCSs are, in fact, repelling.

\subsection{An analytical test case}
\label{sub:an_analytical_lcs_test_case}

As a verification test case for our way of extracting repelling LCSs from the
computed manifolds (which is described in detail in
\cref{sec:identifying_lcss_as_subsets_of_computed_manifolds}), we defined
the purely radial velocity field
\begin{equation}
    \label{eq:spherical_velocity_field}
    \dot{\vct{x}}(t) = \frac{\vct{x}}{\norm{\vct{x}}}%
    \sin\big(\pi(\norm{\vct{x}}-r)\big),
\end{equation}
which is changes from being directed radially inwards, to pointing radially
outwards, on the sphere $\norm{\vct{x}}=r$ (on which the velocity field is
zero). We then computed strain eigenvalues and -vectors over the time interval
$\mathcal{I}=[0,1]$ for an equidistant grid of $200\times200\times200$ tracers
in the domain $\mathcal{U}=[-2,2]^{3}$, as outlined in
\cref{sub:advecting_a_set_of_tracers,%
sec:computing_cauchy_green_strain_eigenvalues_and_vectors}, for $r=1$.
Moreover, we computed the arithmetic average of $\lambda_{3}(\vct{x}_{0})$
across all solid angles in order to approximate it as a function of radius
alone. The dependence of $\lambda_{3}$ as a function of radius is shown in
\cref{fig:spherical_lm3}, which reveals a sharp repulsion peak at
$\norm{\vct{x}}=r=1$. This agrees well with the underlying velocity field
(\cref{eq:spherical_velocity_field}); in particular, in passing through
$\norm{\vct{x}}=r$, the flow direction changes from radially inwards to
radially outwards (or vice versa). Accordingly, we expect to find a single
repelling LCS, forming a unit sphere.

\begin{figure}[htpb]
    \centering
    \resizebox{0.9\textwidth}{!}{%% Creator: Matplotlib, PGF backend
%%
%% To include the figure in your LaTeX document, write
%%   \input{<filename>.pgf}
%%
%% Make sure the required packages are loaded in your preamble
%%   \usepackage{pgf}
%%
%% Figures using additional raster images can only be included by \input if
%% they are in the same directory as the main LaTeX file. For loading figures
%% from other directories you can use the `import` package
%%   \usepackage{import}
%% and then include the figures with
%%   \import{<path to file>}{<filename>.pgf}
%%
%% Matplotlib used the following preamble
%%   \usepackage{fontspec}
%%   \setmainfont{DejaVu Serif}
%%   \setsansfont{DejaVu Sans}
%%   \setmonofont{DejaVu Sans Mono}
%%
\begingroup%
\makeatletter%
\begin{pgfpicture}%
\pgfpathrectangle{\pgfpointorigin}{\pgfqpoint{6.000000in}{4.000000in}}%
\pgfusepath{use as bounding box, clip}%
\begin{pgfscope}%
\pgfsetbuttcap%
\pgfsetmiterjoin%
\definecolor{currentfill}{rgb}{1.000000,1.000000,1.000000}%
\pgfsetfillcolor{currentfill}%
\pgfsetlinewidth{0.000000pt}%
\definecolor{currentstroke}{rgb}{1.000000,1.000000,1.000000}%
\pgfsetstrokecolor{currentstroke}%
\pgfsetdash{}{0pt}%
\pgfpathmoveto{\pgfqpoint{0.000000in}{0.000000in}}%
\pgfpathlineto{\pgfqpoint{6.000000in}{0.000000in}}%
\pgfpathlineto{\pgfqpoint{6.000000in}{4.000000in}}%
\pgfpathlineto{\pgfqpoint{0.000000in}{4.000000in}}%
\pgfpathclose%
\pgfusepath{fill}%
\end{pgfscope}%
\begin{pgfscope}%
\pgfsetbuttcap%
\pgfsetmiterjoin%
\pgfsetlinewidth{0.000000pt}%
\definecolor{currentstroke}{rgb}{0.000000,0.000000,0.000000}%
\pgfsetstrokecolor{currentstroke}%
\pgfsetstrokeopacity{0.000000}%
\pgfsetdash{}{0pt}%
\pgfpathmoveto{\pgfqpoint{0.540000in}{0.440000in}}%
\pgfpathlineto{\pgfqpoint{5.940000in}{0.440000in}}%
\pgfpathlineto{\pgfqpoint{5.940000in}{3.960000in}}%
\pgfpathlineto{\pgfqpoint{0.540000in}{3.960000in}}%
\pgfpathclose%
\pgfusepath{}%
\end{pgfscope}%
\begin{pgfscope}%
\pgfsetbuttcap%
\pgfsetroundjoin%
\definecolor{currentfill}{rgb}{0.000000,0.000000,0.000000}%
\pgfsetfillcolor{currentfill}%
\pgfsetlinewidth{0.803000pt}%
\definecolor{currentstroke}{rgb}{0.000000,0.000000,0.000000}%
\pgfsetstrokecolor{currentstroke}%
\pgfsetdash{}{0pt}%
\pgfsys@defobject{currentmarker}{\pgfqpoint{0.000000in}{-0.048611in}}{\pgfqpoint{0.000000in}{0.000000in}}{%
\pgfpathmoveto{\pgfqpoint{0.000000in}{0.000000in}}%
\pgfpathlineto{\pgfqpoint{0.000000in}{-0.048611in}}%
\pgfusepath{stroke,fill}%
}%
\begin{pgfscope}%
\pgfsys@transformshift{0.760661in}{0.440000in}%
\pgfsys@useobject{currentmarker}{}%
\end{pgfscope}%
\end{pgfscope}%
\begin{pgfscope}%
\pgftext[x=0.760661in,y=0.342778in,,top]{\sffamily\fontsize{10.000000}{12.000000}\selectfont \(\displaystyle 0.0\)}%
\end{pgfscope}%
\begin{pgfscope}%
\pgfsetbuttcap%
\pgfsetroundjoin%
\definecolor{currentfill}{rgb}{0.000000,0.000000,0.000000}%
\pgfsetfillcolor{currentfill}%
\pgfsetlinewidth{0.803000pt}%
\definecolor{currentstroke}{rgb}{0.000000,0.000000,0.000000}%
\pgfsetstrokecolor{currentstroke}%
\pgfsetdash{}{0pt}%
\pgfsys@defobject{currentmarker}{\pgfqpoint{0.000000in}{-0.048611in}}{\pgfqpoint{0.000000in}{0.000000in}}{%
\pgfpathmoveto{\pgfqpoint{0.000000in}{0.000000in}}%
\pgfpathlineto{\pgfqpoint{0.000000in}{-0.048611in}}%
\pgfusepath{stroke,fill}%
}%
\begin{pgfscope}%
\pgfsys@transformshift{1.947569in}{0.440000in}%
\pgfsys@useobject{currentmarker}{}%
\end{pgfscope}%
\end{pgfscope}%
\begin{pgfscope}%
\pgftext[x=1.947569in,y=0.342778in,,top]{\sffamily\fontsize{10.000000}{12.000000}\selectfont \(\displaystyle 0.5\)}%
\end{pgfscope}%
\begin{pgfscope}%
\pgfsetbuttcap%
\pgfsetroundjoin%
\definecolor{currentfill}{rgb}{0.000000,0.000000,0.000000}%
\pgfsetfillcolor{currentfill}%
\pgfsetlinewidth{0.803000pt}%
\definecolor{currentstroke}{rgb}{0.000000,0.000000,0.000000}%
\pgfsetstrokecolor{currentstroke}%
\pgfsetdash{}{0pt}%
\pgfsys@defobject{currentmarker}{\pgfqpoint{0.000000in}{-0.048611in}}{\pgfqpoint{0.000000in}{0.000000in}}{%
\pgfpathmoveto{\pgfqpoint{0.000000in}{0.000000in}}%
\pgfpathlineto{\pgfqpoint{0.000000in}{-0.048611in}}%
\pgfusepath{stroke,fill}%
}%
\begin{pgfscope}%
\pgfsys@transformshift{3.134477in}{0.440000in}%
\pgfsys@useobject{currentmarker}{}%
\end{pgfscope}%
\end{pgfscope}%
\begin{pgfscope}%
\pgftext[x=3.134477in,y=0.342778in,,top]{\sffamily\fontsize{10.000000}{12.000000}\selectfont \(\displaystyle 1.0\)}%
\end{pgfscope}%
\begin{pgfscope}%
\pgfsetbuttcap%
\pgfsetroundjoin%
\definecolor{currentfill}{rgb}{0.000000,0.000000,0.000000}%
\pgfsetfillcolor{currentfill}%
\pgfsetlinewidth{0.803000pt}%
\definecolor{currentstroke}{rgb}{0.000000,0.000000,0.000000}%
\pgfsetstrokecolor{currentstroke}%
\pgfsetdash{}{0pt}%
\pgfsys@defobject{currentmarker}{\pgfqpoint{0.000000in}{-0.048611in}}{\pgfqpoint{0.000000in}{0.000000in}}{%
\pgfpathmoveto{\pgfqpoint{0.000000in}{0.000000in}}%
\pgfpathlineto{\pgfqpoint{0.000000in}{-0.048611in}}%
\pgfusepath{stroke,fill}%
}%
\begin{pgfscope}%
\pgfsys@transformshift{4.321385in}{0.440000in}%
\pgfsys@useobject{currentmarker}{}%
\end{pgfscope}%
\end{pgfscope}%
\begin{pgfscope}%
\pgftext[x=4.321385in,y=0.342778in,,top]{\sffamily\fontsize{10.000000}{12.000000}\selectfont \(\displaystyle 1.5\)}%
\end{pgfscope}%
\begin{pgfscope}%
\pgfsetbuttcap%
\pgfsetroundjoin%
\definecolor{currentfill}{rgb}{0.000000,0.000000,0.000000}%
\pgfsetfillcolor{currentfill}%
\pgfsetlinewidth{0.803000pt}%
\definecolor{currentstroke}{rgb}{0.000000,0.000000,0.000000}%
\pgfsetstrokecolor{currentstroke}%
\pgfsetdash{}{0pt}%
\pgfsys@defobject{currentmarker}{\pgfqpoint{0.000000in}{-0.048611in}}{\pgfqpoint{0.000000in}{0.000000in}}{%
\pgfpathmoveto{\pgfqpoint{0.000000in}{0.000000in}}%
\pgfpathlineto{\pgfqpoint{0.000000in}{-0.048611in}}%
\pgfusepath{stroke,fill}%
}%
\begin{pgfscope}%
\pgfsys@transformshift{5.508294in}{0.440000in}%
\pgfsys@useobject{currentmarker}{}%
\end{pgfscope}%
\end{pgfscope}%
\begin{pgfscope}%
\pgftext[x=5.508294in,y=0.342778in,,top]{\sffamily\fontsize{10.000000}{12.000000}\selectfont \(\displaystyle 2.0\)}%
\end{pgfscope}%
\begin{pgfscope}%
\pgftext[x=3.240000in,y=0.166698in,,top]{\sffamily\fontsize{12.000000}{14.400000}\selectfont \(\displaystyle ||\mathbf{x}||\)}%
\end{pgfscope}%
\begin{pgfscope}%
\pgfsetbuttcap%
\pgfsetroundjoin%
\definecolor{currentfill}{rgb}{0.000000,0.000000,0.000000}%
\pgfsetfillcolor{currentfill}%
\pgfsetlinewidth{0.803000pt}%
\definecolor{currentstroke}{rgb}{0.000000,0.000000,0.000000}%
\pgfsetstrokecolor{currentstroke}%
\pgfsetdash{}{0pt}%
\pgfsys@defobject{currentmarker}{\pgfqpoint{-0.048611in}{0.000000in}}{\pgfqpoint{0.000000in}{0.000000in}}{%
\pgfpathmoveto{\pgfqpoint{0.000000in}{0.000000in}}%
\pgfpathlineto{\pgfqpoint{-0.048611in}{0.000000in}}%
\pgfusepath{stroke,fill}%
}%
\begin{pgfscope}%
\pgfsys@transformshift{0.540000in}{0.599989in}%
\pgfsys@useobject{currentmarker}{}%
\end{pgfscope}%
\end{pgfscope}%
\begin{pgfscope}%
\pgftext[x=0.373333in,y=0.547227in,left,base]{\sffamily\fontsize{10.000000}{12.000000}\selectfont \(\displaystyle 0\)}%
\end{pgfscope}%
\begin{pgfscope}%
\pgfsetbuttcap%
\pgfsetroundjoin%
\definecolor{currentfill}{rgb}{0.000000,0.000000,0.000000}%
\pgfsetfillcolor{currentfill}%
\pgfsetlinewidth{0.803000pt}%
\definecolor{currentstroke}{rgb}{0.000000,0.000000,0.000000}%
\pgfsetstrokecolor{currentstroke}%
\pgfsetdash{}{0pt}%
\pgfsys@defobject{currentmarker}{\pgfqpoint{-0.048611in}{0.000000in}}{\pgfqpoint{0.000000in}{0.000000in}}{%
\pgfpathmoveto{\pgfqpoint{0.000000in}{0.000000in}}%
\pgfpathlineto{\pgfqpoint{-0.048611in}{0.000000in}}%
\pgfusepath{stroke,fill}%
}%
\begin{pgfscope}%
\pgfsys@transformshift{0.540000in}{1.197574in}%
\pgfsys@useobject{currentmarker}{}%
\end{pgfscope}%
\end{pgfscope}%
\begin{pgfscope}%
\pgftext[x=0.234444in,y=1.144813in,left,base]{\sffamily\fontsize{10.000000}{12.000000}\selectfont \(\displaystyle 100\)}%
\end{pgfscope}%
\begin{pgfscope}%
\pgfsetbuttcap%
\pgfsetroundjoin%
\definecolor{currentfill}{rgb}{0.000000,0.000000,0.000000}%
\pgfsetfillcolor{currentfill}%
\pgfsetlinewidth{0.803000pt}%
\definecolor{currentstroke}{rgb}{0.000000,0.000000,0.000000}%
\pgfsetstrokecolor{currentstroke}%
\pgfsetdash{}{0pt}%
\pgfsys@defobject{currentmarker}{\pgfqpoint{-0.048611in}{0.000000in}}{\pgfqpoint{0.000000in}{0.000000in}}{%
\pgfpathmoveto{\pgfqpoint{0.000000in}{0.000000in}}%
\pgfpathlineto{\pgfqpoint{-0.048611in}{0.000000in}}%
\pgfusepath{stroke,fill}%
}%
\begin{pgfscope}%
\pgfsys@transformshift{0.540000in}{1.795160in}%
\pgfsys@useobject{currentmarker}{}%
\end{pgfscope}%
\end{pgfscope}%
\begin{pgfscope}%
\pgftext[x=0.234444in,y=1.742399in,left,base]{\sffamily\fontsize{10.000000}{12.000000}\selectfont \(\displaystyle 200\)}%
\end{pgfscope}%
\begin{pgfscope}%
\pgfsetbuttcap%
\pgfsetroundjoin%
\definecolor{currentfill}{rgb}{0.000000,0.000000,0.000000}%
\pgfsetfillcolor{currentfill}%
\pgfsetlinewidth{0.803000pt}%
\definecolor{currentstroke}{rgb}{0.000000,0.000000,0.000000}%
\pgfsetstrokecolor{currentstroke}%
\pgfsetdash{}{0pt}%
\pgfsys@defobject{currentmarker}{\pgfqpoint{-0.048611in}{0.000000in}}{\pgfqpoint{0.000000in}{0.000000in}}{%
\pgfpathmoveto{\pgfqpoint{0.000000in}{0.000000in}}%
\pgfpathlineto{\pgfqpoint{-0.048611in}{0.000000in}}%
\pgfusepath{stroke,fill}%
}%
\begin{pgfscope}%
\pgfsys@transformshift{0.540000in}{2.392746in}%
\pgfsys@useobject{currentmarker}{}%
\end{pgfscope}%
\end{pgfscope}%
\begin{pgfscope}%
\pgftext[x=0.234444in,y=2.339984in,left,base]{\sffamily\fontsize{10.000000}{12.000000}\selectfont \(\displaystyle 300\)}%
\end{pgfscope}%
\begin{pgfscope}%
\pgfsetbuttcap%
\pgfsetroundjoin%
\definecolor{currentfill}{rgb}{0.000000,0.000000,0.000000}%
\pgfsetfillcolor{currentfill}%
\pgfsetlinewidth{0.803000pt}%
\definecolor{currentstroke}{rgb}{0.000000,0.000000,0.000000}%
\pgfsetstrokecolor{currentstroke}%
\pgfsetdash{}{0pt}%
\pgfsys@defobject{currentmarker}{\pgfqpoint{-0.048611in}{0.000000in}}{\pgfqpoint{0.000000in}{0.000000in}}{%
\pgfpathmoveto{\pgfqpoint{0.000000in}{0.000000in}}%
\pgfpathlineto{\pgfqpoint{-0.048611in}{0.000000in}}%
\pgfusepath{stroke,fill}%
}%
\begin{pgfscope}%
\pgfsys@transformshift{0.540000in}{2.990331in}%
\pgfsys@useobject{currentmarker}{}%
\end{pgfscope}%
\end{pgfscope}%
\begin{pgfscope}%
\pgftext[x=0.234444in,y=2.937570in,left,base]{\sffamily\fontsize{10.000000}{12.000000}\selectfont \(\displaystyle 400\)}%
\end{pgfscope}%
\begin{pgfscope}%
\pgfsetbuttcap%
\pgfsetroundjoin%
\definecolor{currentfill}{rgb}{0.000000,0.000000,0.000000}%
\pgfsetfillcolor{currentfill}%
\pgfsetlinewidth{0.803000pt}%
\definecolor{currentstroke}{rgb}{0.000000,0.000000,0.000000}%
\pgfsetstrokecolor{currentstroke}%
\pgfsetdash{}{0pt}%
\pgfsys@defobject{currentmarker}{\pgfqpoint{-0.048611in}{0.000000in}}{\pgfqpoint{0.000000in}{0.000000in}}{%
\pgfpathmoveto{\pgfqpoint{0.000000in}{0.000000in}}%
\pgfpathlineto{\pgfqpoint{-0.048611in}{0.000000in}}%
\pgfusepath{stroke,fill}%
}%
\begin{pgfscope}%
\pgfsys@transformshift{0.540000in}{3.587917in}%
\pgfsys@useobject{currentmarker}{}%
\end{pgfscope}%
\end{pgfscope}%
\begin{pgfscope}%
\pgftext[x=0.234444in,y=3.535155in,left,base]{\sffamily\fontsize{10.000000}{12.000000}\selectfont \(\displaystyle 500\)}%
\end{pgfscope}%
\begin{pgfscope}%
\pgftext[x=0.178888in,y=2.200000in,,bottom,rotate=90.000000]{\sffamily\fontsize{12.000000}{14.400000}\selectfont \(\displaystyle \lambda_{3}\)}%
\end{pgfscope}%
\begin{pgfscope}%
\pgfpathrectangle{\pgfqpoint{0.540000in}{0.440000in}}{\pgfqpoint{5.400000in}{3.520000in}}%
\pgfusepath{clip}%
\pgfsetbuttcap%
\pgfsetroundjoin%
\pgfsetlinewidth{1.505625pt}%
\definecolor{currentstroke}{rgb}{0.996078,0.941176,0.850980}%
\pgfsetstrokecolor{currentstroke}%
\pgfsetdash{{1.500000pt}{2.475000pt}}{0.000000pt}%
\pgfpathmoveto{\pgfqpoint{3.134477in}{0.440000in}}%
\pgfpathlineto{\pgfqpoint{3.134477in}{3.960000in}}%
\pgfusepath{stroke}%
\end{pgfscope}%
\begin{pgfscope}%
\pgfpathrectangle{\pgfqpoint{0.540000in}{0.440000in}}{\pgfqpoint{5.400000in}{3.520000in}}%
\pgfusepath{clip}%
\pgfsetrectcap%
\pgfsetroundjoin%
\pgfsetlinewidth{1.505625pt}%
\definecolor{currentstroke}{rgb}{0.843137,0.188235,0.121569}%
\pgfsetstrokecolor{currentstroke}%
\pgfsetdash{}{0pt}%
\pgfpathmoveto{\pgfqpoint{0.785455in}{0.600000in}}%
\pgfpathlineto{\pgfqpoint{2.335708in}{0.600160in}}%
\pgfpathlineto{\pgfqpoint{2.335708in}{0.600160in}}%
\pgfpathlineto{\pgfqpoint{2.531205in}{0.600468in}}%
\pgfpathlineto{\pgfqpoint{2.628082in}{0.600914in}}%
\pgfpathlineto{\pgfqpoint{2.628082in}{0.600914in}}%
\pgfpathlineto{\pgfqpoint{2.688975in}{0.601493in}}%
\pgfpathlineto{\pgfqpoint{2.688975in}{0.601493in}}%
\pgfpathlineto{\pgfqpoint{2.732268in}{0.602208in}}%
\pgfpathlineto{\pgfqpoint{2.732268in}{0.602208in}}%
\pgfpathlineto{\pgfqpoint{2.764840in}{0.603048in}}%
\pgfpathlineto{\pgfqpoint{2.764840in}{0.603048in}}%
\pgfpathlineto{\pgfqpoint{2.790845in}{0.604023in}}%
\pgfpathlineto{\pgfqpoint{2.790845in}{0.604023in}}%
\pgfpathlineto{\pgfqpoint{2.812130in}{0.605129in}}%
\pgfpathlineto{\pgfqpoint{2.812130in}{0.605129in}}%
\pgfpathlineto{\pgfqpoint{2.830030in}{0.606366in}}%
\pgfpathlineto{\pgfqpoint{2.830030in}{0.606366in}}%
\pgfpathlineto{\pgfqpoint{2.845420in}{0.607741in}}%
\pgfpathlineto{\pgfqpoint{2.845420in}{0.607741in}}%
\pgfpathlineto{\pgfqpoint{2.858744in}{0.609241in}}%
\pgfpathlineto{\pgfqpoint{2.858744in}{0.609241in}}%
\pgfpathlineto{\pgfqpoint{2.870431in}{0.610866in}}%
\pgfpathlineto{\pgfqpoint{2.870431in}{0.610866in}}%
\pgfpathlineto{\pgfqpoint{2.880895in}{0.612632in}}%
\pgfpathlineto{\pgfqpoint{2.880895in}{0.612632in}}%
\pgfpathlineto{\pgfqpoint{2.890152in}{0.614501in}}%
\pgfpathlineto{\pgfqpoint{2.890152in}{0.614501in}}%
\pgfpathlineto{\pgfqpoint{2.898603in}{0.616514in}}%
\pgfpathlineto{\pgfqpoint{2.898603in}{0.616514in}}%
\pgfpathlineto{\pgfqpoint{2.906638in}{0.618758in}}%
\pgfpathlineto{\pgfqpoint{2.906638in}{0.618758in}}%
\pgfpathlineto{\pgfqpoint{2.913883in}{0.621113in}}%
\pgfpathlineto{\pgfqpoint{2.920724in}{0.623679in}}%
\pgfpathlineto{\pgfqpoint{2.920724in}{0.623679in}}%
\pgfpathlineto{\pgfqpoint{2.926786in}{0.626282in}}%
\pgfpathlineto{\pgfqpoint{2.932454in}{0.629041in}}%
\pgfpathlineto{\pgfqpoint{2.938108in}{0.632156in}}%
\pgfpathlineto{\pgfqpoint{2.938108in}{0.632156in}}%
\pgfpathlineto{\pgfqpoint{2.943371in}{0.635431in}}%
\pgfpathlineto{\pgfqpoint{2.943371in}{0.635431in}}%
\pgfpathlineto{\pgfqpoint{2.948247in}{0.638838in}}%
\pgfpathlineto{\pgfqpoint{2.948247in}{0.638838in}}%
\pgfpathlineto{\pgfqpoint{2.953113in}{0.642645in}}%
\pgfpathlineto{\pgfqpoint{2.953113in}{0.642645in}}%
\pgfpathlineto{\pgfqpoint{2.957594in}{0.646562in}}%
\pgfpathlineto{\pgfqpoint{2.962066in}{0.650917in}}%
\pgfpathlineto{\pgfqpoint{2.966158in}{0.655343in}}%
\pgfpathlineto{\pgfqpoint{2.966158in}{0.655343in}}%
\pgfpathlineto{\pgfqpoint{2.970242in}{0.660237in}}%
\pgfpathlineto{\pgfqpoint{2.974319in}{0.665655in}}%
\pgfpathlineto{\pgfqpoint{2.974319in}{0.665655in}}%
\pgfpathlineto{\pgfqpoint{2.978018in}{0.671090in}}%
\pgfpathlineto{\pgfqpoint{2.981712in}{0.677069in}}%
\pgfpathlineto{\pgfqpoint{2.985399in}{0.683653in}}%
\pgfpathlineto{\pgfqpoint{2.989080in}{0.690915in}}%
\pgfpathlineto{\pgfqpoint{2.992755in}{0.698933in}}%
\pgfpathlineto{\pgfqpoint{2.996057in}{0.706872in}}%
\pgfpathlineto{\pgfqpoint{2.999354in}{0.715572in}}%
\pgfpathlineto{\pgfqpoint{2.999354in}{0.715572in}}%
\pgfpathlineto{\pgfqpoint{3.002647in}{0.725116in}}%
\pgfpathlineto{\pgfqpoint{3.005935in}{0.735597in}}%
\pgfpathlineto{\pgfqpoint{3.005935in}{0.735597in}}%
\pgfpathlineto{\pgfqpoint{3.009218in}{0.747119in}}%
\pgfpathlineto{\pgfqpoint{3.009218in}{0.747119in}}%
\pgfpathlineto{\pgfqpoint{3.012496in}{0.759799in}}%
\pgfpathlineto{\pgfqpoint{3.012496in}{0.759799in}}%
\pgfpathlineto{\pgfqpoint{3.015769in}{0.773768in}}%
\pgfpathlineto{\pgfqpoint{3.019038in}{0.789172in}}%
\pgfpathlineto{\pgfqpoint{3.022302in}{0.806173in}}%
\pgfpathlineto{\pgfqpoint{3.025561in}{0.824956in}}%
\pgfpathlineto{\pgfqpoint{3.025561in}{0.824956in}}%
\pgfpathlineto{\pgfqpoint{3.028816in}{0.845725in}}%
\pgfpathlineto{\pgfqpoint{3.032066in}{0.868706in}}%
\pgfpathlineto{\pgfqpoint{3.032066in}{0.868706in}}%
\pgfpathlineto{\pgfqpoint{3.035311in}{0.894155in}}%
\pgfpathlineto{\pgfqpoint{3.038552in}{0.922354in}}%
\pgfpathlineto{\pgfqpoint{3.038552in}{0.922354in}}%
\pgfpathlineto{\pgfqpoint{3.041788in}{0.953614in}}%
\pgfpathlineto{\pgfqpoint{3.041788in}{0.953613in}}%
\pgfpathlineto{\pgfqpoint{3.045019in}{0.988281in}}%
\pgfpathlineto{\pgfqpoint{3.045019in}{0.988281in}}%
\pgfpathlineto{\pgfqpoint{3.048246in}{1.026730in}}%
\pgfpathlineto{\pgfqpoint{3.048246in}{1.026730in}}%
\pgfpathlineto{\pgfqpoint{3.051468in}{1.069378in}}%
\pgfpathlineto{\pgfqpoint{3.051468in}{1.069378in}}%
\pgfpathlineto{\pgfqpoint{3.054686in}{1.116672in}}%
\pgfpathlineto{\pgfqpoint{3.054686in}{1.116672in}}%
\pgfpathlineto{\pgfqpoint{3.057899in}{1.169093in}}%
\pgfpathlineto{\pgfqpoint{3.057899in}{1.169093in}}%
\pgfpathlineto{\pgfqpoint{3.061108in}{1.227151in}}%
\pgfpathlineto{\pgfqpoint{3.064313in}{1.291381in}}%
\pgfpathlineto{\pgfqpoint{3.064313in}{1.291381in}}%
\pgfpathlineto{\pgfqpoint{3.067512in}{1.362330in}}%
\pgfpathlineto{\pgfqpoint{3.067512in}{1.362330in}}%
\pgfpathlineto{\pgfqpoint{3.070708in}{1.440545in}}%
\pgfpathlineto{\pgfqpoint{3.073899in}{1.526551in}}%
\pgfpathlineto{\pgfqpoint{3.077086in}{1.620828in}}%
\pgfpathlineto{\pgfqpoint{3.080268in}{1.723773in}}%
\pgfpathlineto{\pgfqpoint{3.080268in}{1.723773in}}%
\pgfpathlineto{\pgfqpoint{3.083799in}{1.848649in}}%
\pgfpathlineto{\pgfqpoint{3.083799in}{1.848649in}}%
\pgfpathlineto{\pgfqpoint{3.087324in}{1.984673in}}%
\pgfpathlineto{\pgfqpoint{3.087324in}{1.984673in}}%
\pgfpathlineto{\pgfqpoint{3.091196in}{2.146852in}}%
\pgfpathlineto{\pgfqpoint{3.091196in}{2.146852in}}%
\pgfpathlineto{\pgfqpoint{3.095412in}{2.337548in}}%
\pgfpathlineto{\pgfqpoint{3.095412in}{2.337548in}}%
\pgfpathlineto{\pgfqpoint{3.100322in}{2.574714in}}%
\pgfpathlineto{\pgfqpoint{3.115685in}{3.330097in}}%
\pgfpathlineto{\pgfqpoint{3.115685in}{3.330097in}}%
\pgfpathlineto{\pgfqpoint{3.118815in}{3.461977in}}%
\pgfpathlineto{\pgfqpoint{3.118815in}{3.461977in}}%
\pgfpathlineto{\pgfqpoint{3.121594in}{3.565177in}}%
\pgfpathlineto{\pgfqpoint{3.121594in}{3.565176in}}%
\pgfpathlineto{\pgfqpoint{3.123676in}{3.632170in}}%
\pgfpathlineto{\pgfqpoint{3.123676in}{3.632170in}}%
\pgfpathlineto{\pgfqpoint{3.125756in}{3.689062in}}%
\pgfpathlineto{\pgfqpoint{3.127488in}{3.728084in}}%
\pgfpathlineto{\pgfqpoint{3.127488in}{3.728084in}}%
\pgfpathlineto{\pgfqpoint{3.128873in}{3.753481in}}%
\pgfpathlineto{\pgfqpoint{3.130257in}{3.773499in}}%
\pgfpathlineto{\pgfqpoint{3.130257in}{3.773499in}}%
\pgfpathlineto{\pgfqpoint{3.131295in}{3.784890in}}%
\pgfpathlineto{\pgfqpoint{3.131295in}{3.784890in}}%
\pgfpathlineto{\pgfqpoint{3.132332in}{3.793124in}}%
\pgfpathlineto{\pgfqpoint{3.133023in}{3.796841in}}%
\pgfpathlineto{\pgfqpoint{3.133714in}{3.799134in}}%
\pgfpathlineto{\pgfqpoint{3.133714in}{3.799134in}}%
\pgfpathlineto{\pgfqpoint{3.134059in}{3.799746in}}%
\pgfpathlineto{\pgfqpoint{3.134404in}{3.800000in}}%
\pgfpathlineto{\pgfqpoint{3.134749in}{3.799897in}}%
\pgfpathlineto{\pgfqpoint{3.135095in}{3.799437in}}%
\pgfpathlineto{\pgfqpoint{3.135095in}{3.799437in}}%
\pgfpathlineto{\pgfqpoint{3.135785in}{3.797448in}}%
\pgfpathlineto{\pgfqpoint{3.135785in}{3.797449in}}%
\pgfpathlineto{\pgfqpoint{3.136475in}{3.794039in}}%
\pgfpathlineto{\pgfqpoint{3.136475in}{3.794039in}}%
\pgfpathlineto{\pgfqpoint{3.137165in}{3.789217in}}%
\pgfpathlineto{\pgfqpoint{3.137165in}{3.789217in}}%
\pgfpathlineto{\pgfqpoint{3.138199in}{3.779360in}}%
\pgfpathlineto{\pgfqpoint{3.138199in}{3.779360in}}%
\pgfpathlineto{\pgfqpoint{3.139233in}{3.766397in}}%
\pgfpathlineto{\pgfqpoint{3.139233in}{3.766397in}}%
\pgfpathlineto{\pgfqpoint{3.140611in}{3.744390in}}%
\pgfpathlineto{\pgfqpoint{3.140611in}{3.744390in}}%
\pgfpathlineto{\pgfqpoint{3.141988in}{3.717153in}}%
\pgfpathlineto{\pgfqpoint{3.143709in}{3.676088in}}%
\pgfpathlineto{\pgfqpoint{3.143709in}{3.676088in}}%
\pgfpathlineto{\pgfqpoint{3.145428in}{3.627692in}}%
\pgfpathlineto{\pgfqpoint{3.145428in}{3.627692in}}%
\pgfpathlineto{\pgfqpoint{3.147489in}{3.560740in}}%
\pgfpathlineto{\pgfqpoint{3.149891in}{3.471764in}}%
\pgfpathlineto{\pgfqpoint{3.152634in}{3.358016in}}%
\pgfpathlineto{\pgfqpoint{3.152634in}{3.358016in}}%
\pgfpathlineto{\pgfqpoint{3.156058in}{3.201915in}}%
\pgfpathlineto{\pgfqpoint{3.160844in}{2.967155in}}%
\pgfpathlineto{\pgfqpoint{3.160844in}{2.967156in}}%
\pgfpathlineto{\pgfqpoint{3.173106in}{2.358054in}}%
\pgfpathlineto{\pgfqpoint{3.173106in}{2.358054in}}%
\pgfpathlineto{\pgfqpoint{3.177518in}{2.157350in}}%
\pgfpathlineto{\pgfqpoint{3.177518in}{2.157350in}}%
\pgfpathlineto{\pgfqpoint{3.181585in}{1.986529in}}%
\pgfpathlineto{\pgfqpoint{3.181585in}{1.986530in}}%
\pgfpathlineto{\pgfqpoint{3.185306in}{1.843124in}}%
\pgfpathlineto{\pgfqpoint{3.185306in}{1.843124in}}%
\pgfpathlineto{\pgfqpoint{3.189021in}{1.712528in}}%
\pgfpathlineto{\pgfqpoint{3.192394in}{1.604697in}}%
\pgfpathlineto{\pgfqpoint{3.192394in}{1.604697in}}%
\pgfpathlineto{\pgfqpoint{3.195762in}{1.506799in}}%
\pgfpathlineto{\pgfqpoint{3.195762in}{1.506799in}}%
\pgfpathlineto{\pgfqpoint{3.199126in}{1.418262in}}%
\pgfpathlineto{\pgfqpoint{3.202485in}{1.338433in}}%
\pgfpathlineto{\pgfqpoint{3.202485in}{1.338433in}}%
\pgfpathlineto{\pgfqpoint{3.205839in}{1.266623in}}%
\pgfpathlineto{\pgfqpoint{3.205839in}{1.266623in}}%
\pgfpathlineto{\pgfqpoint{3.209189in}{1.202136in}}%
\pgfpathlineto{\pgfqpoint{3.209189in}{1.202136in}}%
\pgfpathlineto{\pgfqpoint{3.212534in}{1.144295in}}%
\pgfpathlineto{\pgfqpoint{3.212534in}{1.144295in}}%
\pgfpathlineto{\pgfqpoint{3.215874in}{1.092453in}}%
\pgfpathlineto{\pgfqpoint{3.215874in}{1.092453in}}%
\pgfpathlineto{\pgfqpoint{3.219210in}{1.046005in}}%
\pgfpathlineto{\pgfqpoint{3.219210in}{1.046006in}}%
\pgfpathlineto{\pgfqpoint{3.222542in}{1.004393in}}%
\pgfpathlineto{\pgfqpoint{3.222542in}{1.004393in}}%
\pgfpathlineto{\pgfqpoint{3.225869in}{0.967106in}}%
\pgfpathlineto{\pgfqpoint{3.225869in}{0.967108in}}%
\pgfpathlineto{\pgfqpoint{3.229191in}{0.933678in}}%
\pgfpathlineto{\pgfqpoint{3.229191in}{0.933678in}}%
\pgfpathlineto{\pgfqpoint{3.232509in}{0.903694in}}%
\pgfpathlineto{\pgfqpoint{3.235823in}{0.876777in}}%
\pgfpathlineto{\pgfqpoint{3.235823in}{0.876777in}}%
\pgfpathlineto{\pgfqpoint{3.239132in}{0.852593in}}%
\pgfpathlineto{\pgfqpoint{3.239132in}{0.852593in}}%
\pgfpathlineto{\pgfqpoint{3.242437in}{0.830842in}}%
\pgfpathlineto{\pgfqpoint{3.242437in}{0.830842in}}%
\pgfpathlineto{\pgfqpoint{3.245737in}{0.811258in}}%
\pgfpathlineto{\pgfqpoint{3.249033in}{0.793606in}}%
\pgfpathlineto{\pgfqpoint{3.249033in}{0.793606in}}%
\pgfpathlineto{\pgfqpoint{3.252325in}{0.777676in}}%
\pgfpathlineto{\pgfqpoint{3.255612in}{0.763281in}}%
\pgfpathlineto{\pgfqpoint{3.255612in}{0.763281in}}%
\pgfpathlineto{\pgfqpoint{3.258895in}{0.750259in}}%
\pgfpathlineto{\pgfqpoint{3.262174in}{0.738461in}}%
\pgfpathlineto{\pgfqpoint{3.262174in}{0.738461in}}%
\pgfpathlineto{\pgfqpoint{3.265448in}{0.727760in}}%
\pgfpathlineto{\pgfqpoint{3.265448in}{0.727760in}}%
\pgfpathlineto{\pgfqpoint{3.269045in}{0.717119in}}%
\pgfpathlineto{\pgfqpoint{3.269045in}{0.717119in}}%
\pgfpathlineto{\pgfqpoint{3.272636in}{0.707532in}}%
\pgfpathlineto{\pgfqpoint{3.272636in}{0.707532in}}%
\pgfpathlineto{\pgfqpoint{3.276223in}{0.698881in}}%
\pgfpathlineto{\pgfqpoint{3.276223in}{0.698881in}}%
\pgfpathlineto{\pgfqpoint{3.279804in}{0.691061in}}%
\pgfpathlineto{\pgfqpoint{3.283381in}{0.683983in}}%
\pgfpathlineto{\pgfqpoint{3.283381in}{0.683983in}}%
\pgfpathlineto{\pgfqpoint{3.286952in}{0.677565in}}%
\pgfpathlineto{\pgfqpoint{3.286952in}{0.677565in}}%
\pgfpathlineto{\pgfqpoint{3.290842in}{0.671235in}}%
\pgfpathlineto{\pgfqpoint{3.290842in}{0.671235in}}%
\pgfpathlineto{\pgfqpoint{3.294727in}{0.665528in}}%
\pgfpathlineto{\pgfqpoint{3.294727in}{0.665528in}}%
\pgfpathlineto{\pgfqpoint{3.298605in}{0.660373in}}%
\pgfpathlineto{\pgfqpoint{3.298605in}{0.660373in}}%
\pgfpathlineto{\pgfqpoint{3.302800in}{0.655340in}}%
\pgfpathlineto{\pgfqpoint{3.302800in}{0.655340in}}%
\pgfpathlineto{\pgfqpoint{3.306988in}{0.650815in}}%
\pgfpathlineto{\pgfqpoint{3.306988in}{0.650815in}}%
\pgfpathlineto{\pgfqpoint{3.311490in}{0.646443in}}%
\pgfpathlineto{\pgfqpoint{3.311490in}{0.646443in}}%
\pgfpathlineto{\pgfqpoint{3.315984in}{0.642527in}}%
\pgfpathlineto{\pgfqpoint{3.315984in}{0.642527in}}%
\pgfpathlineto{\pgfqpoint{3.320791in}{0.638776in}}%
\pgfpathlineto{\pgfqpoint{3.325589in}{0.635428in}}%
\pgfpathlineto{\pgfqpoint{3.325589in}{0.635428in}}%
\pgfpathlineto{\pgfqpoint{3.330696in}{0.632244in}}%
\pgfpathlineto{\pgfqpoint{3.336112in}{0.629243in}}%
\pgfpathlineto{\pgfqpoint{3.336112in}{0.629243in}}%
\pgfpathlineto{\pgfqpoint{3.341834in}{0.626436in}}%
\pgfpathlineto{\pgfqpoint{3.347860in}{0.623829in}}%
\pgfpathlineto{\pgfqpoint{3.347860in}{0.623829in}}%
\pgfpathlineto{\pgfqpoint{3.354505in}{0.621312in}}%
\pgfpathlineto{\pgfqpoint{3.354505in}{0.621312in}}%
\pgfpathlineto{\pgfqpoint{3.361447in}{0.619023in}}%
\pgfpathlineto{\pgfqpoint{3.361447in}{0.619023in}}%
\pgfpathlineto{\pgfqpoint{3.368999in}{0.616867in}}%
\pgfpathlineto{\pgfqpoint{3.368999in}{0.616867in}}%
\pgfpathlineto{\pgfqpoint{3.370570in}{0.616536in}}%
\pgfpathlineto{\pgfqpoint{3.370570in}{0.616536in}}%
\pgfpathlineto{\pgfqpoint{3.434786in}{0.616375in}}%
\pgfpathlineto{\pgfqpoint{3.534678in}{0.615776in}}%
\pgfpathlineto{\pgfqpoint{3.534678in}{0.615776in}}%
\pgfpathlineto{\pgfqpoint{3.788671in}{0.613820in}}%
\pgfpathlineto{\pgfqpoint{3.788671in}{0.613820in}}%
\pgfpathlineto{\pgfqpoint{4.071824in}{0.611806in}}%
\pgfpathlineto{\pgfqpoint{4.071824in}{0.611806in}}%
\pgfpathlineto{\pgfqpoint{4.341768in}{0.610213in}}%
\pgfpathlineto{\pgfqpoint{4.341768in}{0.610213in}}%
\pgfpathlineto{\pgfqpoint{4.642892in}{0.608768in}}%
\pgfpathlineto{\pgfqpoint{4.979983in}{0.607479in}}%
\pgfpathlineto{\pgfqpoint{4.979983in}{0.607479in}}%
\pgfpathlineto{\pgfqpoint{5.370609in}{0.606311in}}%
\pgfpathlineto{\pgfqpoint{5.694545in}{0.605541in}}%
\pgfpathlineto{\pgfqpoint{5.694545in}{0.605541in}}%
\pgfusepath{stroke}%
\end{pgfscope}%
\begin{pgfscope}%
\pgfsetrectcap%
\pgfsetmiterjoin%
\pgfsetlinewidth{0.803000pt}%
\definecolor{currentstroke}{rgb}{0.000000,0.000000,0.000000}%
\pgfsetstrokecolor{currentstroke}%
\pgfsetdash{}{0pt}%
\pgfpathmoveto{\pgfqpoint{0.540000in}{0.440000in}}%
\pgfpathlineto{\pgfqpoint{0.540000in}{3.960000in}}%
\pgfusepath{stroke}%
\end{pgfscope}%
\begin{pgfscope}%
\pgfsetrectcap%
\pgfsetmiterjoin%
\pgfsetlinewidth{0.803000pt}%
\definecolor{currentstroke}{rgb}{0.000000,0.000000,0.000000}%
\pgfsetstrokecolor{currentstroke}%
\pgfsetdash{}{0pt}%
\pgfpathmoveto{\pgfqpoint{5.940000in}{0.440000in}}%
\pgfpathlineto{\pgfqpoint{5.940000in}{3.960000in}}%
\pgfusepath{stroke}%
\end{pgfscope}%
\begin{pgfscope}%
\pgfsetrectcap%
\pgfsetmiterjoin%
\pgfsetlinewidth{0.803000pt}%
\definecolor{currentstroke}{rgb}{0.000000,0.000000,0.000000}%
\pgfsetstrokecolor{currentstroke}%
\pgfsetdash{}{0pt}%
\pgfpathmoveto{\pgfqpoint{0.540000in}{0.440000in}}%
\pgfpathlineto{\pgfqpoint{5.940000in}{0.440000in}}%
\pgfusepath{stroke}%
\end{pgfscope}%
\begin{pgfscope}%
\pgfsetrectcap%
\pgfsetmiterjoin%
\pgfsetlinewidth{0.803000pt}%
\definecolor{currentstroke}{rgb}{0.000000,0.000000,0.000000}%
\pgfsetstrokecolor{currentstroke}%
\pgfsetdash{}{0pt}%
\pgfpathmoveto{\pgfqpoint{0.540000in}{3.960000in}}%
\pgfpathlineto{\pgfqpoint{5.940000in}{3.960000in}}%
\pgfusepath{stroke}%
\end{pgfscope}%
\end{pgfpicture}%
\makeatother%
\endgroup%
}
    \caption[Aviici is love, Aviici is life]{$\lambda_{3}$ as a function of radius, for the spherical flow. Displayed here is the arithmetic average across all solid angles.}
    \label{fig:u0_dom_errs}
\end{figure}



Using $\epsilon=5\cdot10^{-3}$, a filtering frequency $\nu=20$ (see
\cref{sub:identifying_suitable_initial_conditions_for_developing_lcss,%
tab:initialconditionparams}), and otherwise the same parameters as given
in \cref{tab:abc_manifold_params}, we developed manifolds from the set
of $291$ grid points in the reduced $\mathcal{U}_{0}$ domain by the method
described in
\cref{sec:preliminaries_for_computing_repelling_lcss_in_3d_flow_by_means_of%
    _geodesic_level_sets,sec:revised_approach_to_computing_new_mesh_points,%
    sec:managing_mesh_accuracy,%
    sec:continuously_reconstructing_three_dimensional_manifold_surfaces_from%
    _point_meshes,%
    sec:macroscale_stopping_criteria_for_the_expansion_of_computed_manifolds}.
Then, by means of the method outlined in
\cref{sec:identifying_lcss_as_subsets_of_computed_manifolds}, we extracted
repelling LCSs from the computed manifolds, using a tolerance parameter
$\gamma_{\square}=1.2$, keeping only the LCSs with (pseudo-)surface area
greater than or equal to $\mathcal{W}_{\min}=1$. The result was a total of
three identical (to numerical precision) spherical LCSs of radius $1$. These
are shown in \cref{fig:spherical_lcs}. Seeing as the expected LCS was
successfully reproduced without false positives, we concluded that our LCS
extraction routine functions as intended.


\begin{figure}[htpb]
    \centering
    \resizebox{0.9\textwidth}{!}{
        \importpgf{figures/mpl-figs}{spherical-lcs.pgf}
    }
    \caption[The single repelling LCS present in the purely radial velocity
    field]
    {The single repelling LCS present in the purely radial velocity field given
        by \cref{eq:spherical_velocity_field} with $r=1$. In computing
        the underlying manifolds, we used the filtering parameter $\nu=20$
        (cf.\
        \cref{sub:identifying_suitable_initial_conditions_for_developing_lcss,%
        tab:initialconditionparams})
        and otherwise the same parameter values as given in
        \cref{tab:abc_manifold_params}. To extract the LCSs, we used
        the tolerance parameter $\gamma_{\square}=1.2$, keeping only
        the LCSs whose (pseudo-)surface area greater than or equal to
        $\mathcal{W}_{\min}=1$ (see
        \cref{sec:identifying_lcss_as_subsets_of_computed_manifolds}).
        This resulted in 18 identical (to numerical precision) copies of a
        single, spherical, repelling LCS of unit radius --- which is exactly as
        expected, given the sharp repulsion maximum at $\norm{\vct{x}}=1$, as
        shown in \cref{fig:spherical_lm3}.
    }
    \label{fig:spherical_lcs}
\end{figure}



\subsection{Verifying that the computed LCSs are in fact repelling}
\label{sub:verifying_that_the_computed_lcss_are_in_fact_repelling}

Per \cref{def:normal_repellence,def:repelling_lcs}, we expect lumps of
particles which start out at opposite sides of a repelling LCS to quickly
diverge under the advection of the underlying flow system. Moreover, courtesy
of being material surfaces, no particle may ever cross a repelling LCS. In
order to verify the impenetrable and repelling nature of the computed LCSs, we
used a setup of two blobs of initial conditions, situated at opposite sides of
an identified LCS surface in the steady ABC flow (more on which to follow in
\cref{sec:computed_lcss_in_the_abc_flow}). We then advected the particle blobs
and the points in the parametrization of the LCS for five units of time,
in the velocity field given by \cref{eq:abc_flow,eq:abc_params_stationary},
using the Dormand-Prince 8(7) ODE solver in similar fashion to how we
`advected' the flow map Jacobian field in the first place
(see \cref{sec:computing_the_flow_map_and_its_directional_derivatives}).
The initial and final states are shown in \cref{fig:blobtest}.

\begin{figure}[htpb]
\centering
    \begin{subfigure}[b]{0.475\textwidth}
        \centering
        \importpgf{figures/mpl-figs}{blobtest-pre-small.pgf}
        \caption[]{{\small $t=0$}}
        \label{fig:u0_dom_err_dp54}
    \end{subfigure}
    \begin{subfigure}[b]{0.475\textwidth}
        \centering
        \importpgf{figures/mpl-figs}{blobtest-post-small.pgf}
        \caption[]{{\small $t=5$}}
        \label{fig:u0_dom_err_dp87}
    \end{subfigure}
\caption[Veni, vidi, Aviici]{Blob test for verifying expected LCS behaviour}
    \label{fig:u0_dom_errs}
\end{figure}



\Cref{fig:blobtest} shows that, while the triangulated structure of the LCS
breaks down, the two blobs of particles have been far removed from eachother
in the transition from \cref{fig:blobtest-pre} to \cref{fig:blobtest-post}.
Furthermore, particles belonging to a single blob remain reasonably compact,
indicating that the local repulsion centre was situated between the two
blobs throughout --- i.e., along the LCS. One possible explanation
for the relatively large amount of stretching of the mesh points in the
parametrization of the LCS under the aforementioned advection, could be
that (several of) the mesh points being slightly perturbed away from
the \emph{actual} LCS surface, due to round-off errors (which are practically
unavoidable, as LCSs are of infinitesimal width per existence criterion%
~\eqref{eq:lcs_condition_b}). Finally, the fact that none of the particles
belonging to either of the two blobs ever appear to move across the
LCS mesh points, supports the notion that the computed LCSs do in fact act as
repelling material surfaces, and thus barriers to transport.
