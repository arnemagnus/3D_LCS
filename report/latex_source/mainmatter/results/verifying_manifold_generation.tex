\section{Verifying our method of generating manifolds}
\label{sec:verifying_our_method_of_generating_manifolds}

To start things off, \cref{sub:an_analytical_manifold_test_case}
outlines a series of geodesic level set approximations of an analytically known
three-dimensional surface. From these, we extract key insight regarding which
mesh point configurations might be reasonable for general application.
In \cref{sub:sample_manifolds_computed_from_the_steady_abc_flow}, we
illustrate how a sample manifold from the steady ABC flow (see
\cref{sec:flow_systems_defined_by_analytical_velocity_fields}) depends on the
mesh point parameters; which we then used to guide our parameter choices for
the generation of manifolds in the general case. Lastly, we illustrate that the
computed manifolds are in fact invariant for trajectories everywhere tangent to
(arbitrary linear combinations of) the $\vct{\xi}_{1}$- and
$\vct{\xi}_{2}$-direction fields (see \cref{rmk:invariance_lcs}) --- and,
consequently, everywhere orthogonal to the $\vct{\xi}_{3}$-field, in compliance
with existence criterion~\eqref{eq:lcs_condition_c}.

\subsection{An analytical test case}
\label{sub:an_analytical_manifold_test_case}
As a verification test case for (our variant of) the method of geodesic level
sets to compute three-dimensional manifolds, we sought to reproduce a
three-dimensional surface defined by
\begin{subequations}
    \label{eq:sinusoidal_definition}
    \begin{equation}
        \label{eq:sinusoidal_surface_function}
        z = g(x,y) = A\sin(\omega_{x}x)\sin(\omega_{y}y) + z_{0},
    \end{equation}
    which can be expressed as the zeros of the scalar function
    \begin{equation}
        \label{eq:sinusoidal_implicit_surface_function}
        f(\vct{x}) = g(x,y) - z,
    \end{equation}
    where $\vct{x}$ denotes the Cartesian coordinate vector $(x,y,z)$.
\end{subequations}
In order to do so, we computed its unit normal vector field as
\begin{equation}
    \label{eq:sinusoidal_normal_vector_field}
    \vct{n}(\vct{x}) = \frac{\grad{f(\vct{x})}}{\norm{\grad{f(\vct{x})}}},
\end{equation}
which we then substituted for the $\vct{\xi}_{3}$-direction field
in \cref{eq:revised_direction_field}. Here, we chose the parameters
\begin{equation}
    \label{eq:sinusoidal_surface_params}
    A=1,\quad \omega_{x}=\omega_{y}=2, \quad z_{0} = \pi
\end{equation}
and used a single initial position $\vct{x}_{0} = (\pi,\pi,\pi)$ from which
to develop the surface, by subsequently adding mesh points organized in level
sets, as outlined in \cref{sec:revised_approach_to_computing_new_mesh_points,%
    sec:managing_mesh_accuracy,%
    sec:continuously_reconstructing_three_dimensional_manifold_surfaces_from%
    _point_meshes,%
    sec:macroscale_stopping_criteria_for_the_expansion_of_computed_manifolds}.

Specifically, we computed a total of seven surface approximations, using
the parameter values provided in \cref{tab:sinusoidal_manifold_params} with
different values of $\Delta_{\min}$, thereby using different mesh point
densities. Four of the resulting manifolds are shown in
\cref{fig:sinusoidal_manifolds}. Although all of the presented manifolds
successfully encapsulate the macroscale behaviour of the underlying surface
(given by \cref{eq:sinusoidal_definition,eq:sinusoidal_surface_params}),
increasing the mesh point density clearly facilitates more accurate
approximations. In particular, some of the visual discrepancies of
\cref{fig:sinusoidal_manifolds} can be attributed to the linear interpolation
inherent to our triangulation scheme (see
\cref{sec:continuously_reconstructing_three_dimensional_manifold_surfaces_%
from_point_meshes}).

\begin{table}[htpb]
    \centering
    \caption[Parameter values used to approximate an analytically known
    three-dimensional surface by the method of geodesic level sets]
    {Parameter values used to approximate an analytically known
        three-dimensional surface (see
        \cref{eq:sinusoidal_definition,eq:sinusoidal_surface_params}) by
        the method of geodesic level sets. Note that the interset distances
        $\Delta_{i}$ were dynamically altered, as described in
        \cref{sub:a_curvature_based_approach_to_determining_interset%
        _separations}. Accordingly, only the \emph{first} interset distance,
        $\Delta_{1}$, was set explicitly. Moreover, while we varied the
        overall mesh point density by altering $\Delta_{\min}$ and
        $\Delta_{\max}$, we kept the ratio between them constant.
    }
    \label{tab:sinusoidal_manifold_params}
    \begin{tabular}{ccc}
        \toprule
        Parameter & Value & Description\\
        \midrule
        $\delta_{\text{init}}$ & $10^{-3}$ %
        & \makecell{Separation of innermost geodesic level set \\
        from the manifold epicentre, cf. \cref{fig:innermost_levelset}}%
        \\[9pt]
        %
        $\Delta_{\min}$, $\Delta_{\max}$
        & $\dfrac{\Delta_{\max}}{\Delta_{\min}} = 4$ %
        & \makecell{(Variable) boundaries for interpoint \\separations (details
        found in \cref{sub:maintaining_mesh_point_density})}%
        \\[9pt]
        %
        $\Delta_{1}$ %
        & $2\Delta_{\min}$ %
        & \makecell{Interset distance used to compute the second \\ geodesic
        level set (see
        \cref{sec:revised_approach_to_computing_new_mesh_points,%
        sub:a_curvature_based_approach_to_determining_interset_separations})}%
        \\[9pt]
        %
        $\gamma_{\|}$ %
        & $10^{-4}$ %
        & \makecell{Tolerance for detecting regions in which
        $\vct{\xi}_{3}$\\ is (anti-)parallel to $\vct{t}_{i,j}$
        (see \cref{eq:revised_xi3_tan_parallel})}
        \\[9pt]
        %
        $\gamma_{\Delta}$ %
        & $5\cdot10^{-3}$ %
        & \makecell{Tolerance for the separation of a mesh point\\ from
        its ancestor (per \cref{eq:revised_dist_tol})}
        \\[9pt]
        $\gamma_{\text{arc}}$ %
        & 5 %
        & \makecell{Sets an upper limit to trajectory lengths as \\
        $\gamma_{\text{arc}}\Delta_{i}$ (briefly mentioned in
        \cref{sub:computing_pseudoradial_trajectories_directly})}
        \\[9pt]
        %
        $\gamma_{\circlearrowleft}$ %
        & $7\cdot10^{-1}$
        & \makecell{Sets an upper limit to the extent of loop-like\\
        segments of any level set (see
        \cref{sub:limiting_the_accumulation_of_numerical_noise})}
        \\[9pt]
        %
        \makecell[c]{$\alpha_{\uparrow}$\\ $\alpha_{\downarrow}$ \\[1.5pt]%
        ${(\delta\alpha)}_{\uparrow}$, ${(\delta\alpha)}_{\downarrow}$} &
        \makecell[c]{$8.7\cdot10^{-2}\;\si{\radian}$ %
            \phantom{2}$(5\si{\degree})$\\ %
            $4.4\cdot10^{-1}\;\si{\radian}$ $(25\si{\degree})$\\[1.5pt]%
        $2\delta_{\max}\alpha_{\uparrow}$, %
        $2\delta_{\min}\alpha_{\downarrow}$}%
        & \makecell[c]{Used in a curvature-based approach to adjust\\
        interset distances (outlined in
        \cref{sub:a_curvature_based_approach_to_determining_interset%
        _separations})}
        \\[18pt]
        \bottomrule
    \end{tabular}
\end{table}



\begin{figure}[htpb]
    \centering
    \begin{subfigure}[b]{0.475\textwidth}
        \centering
        \importpgf{figures/mpl-figs}{sinusoidal-minsep=0.2.pgf}
        \caption[]{{\small min\_sep = 0.2}}
        \label{fig:u0_dom_err_bs32}
    \end{subfigure}
    \begin{subfigure}[b]{0.475\textwidth}
        \centering
        \importpgf{figures/mpl-figs}{sinusoidal-minsep=0.12.pgf}
        \caption[]{{\small min\_sep = 0.12}}
        \label{fig:u0_dom_err_bs54}
    \end{subfigure}

    \begin{subfigure}[b]{0.475\textwidth}
        \centering
        \importpgf{figures/mpl-figs}{sinusoidal-minsep=0.04.pgf}
        \caption[]{{\small min\_sep = 0.04}}
        \label{fig:u0_dom_err_dp54}
    \end{subfigure}
    \begin{subfigure}[b]{0.475\textwidth}
        \centering
        \importpgf{figures/mpl-figs}{sinusoidal-minsep=0.01.pgf}
        \caption[]{{\small min\_sep = 0.01}}
        \label{fig:u0_dom_err_dp87}
    \end{subfigure}
    \caption[Veni, vidi, Aviici]{Manifolds generated from analytically determined vector field.}
    \label{fig:u0_dom_errs}
\end{figure}



In order to obtain a quantitative measure of how an increase in the mesh
point density impacts the accuracy of the overall approximation, we
computed the root mean square (hereafter abbreviated to RMS) error of each mesh
point as
\begin{equation}
    \label{eq:sinusoidal_rms}
    \text{err}_{\text{RMS}} = \sqrt{\frac{1}{N}%
    \sum\limits_{i,j}\abs{z_{i,j}-g(x_{i},y_{j})}^{2}},
\end{equation}
where we sum over all of the $N$ computed points, with $g(x,y)$ given
by \cref{eq:sinusoidal_definition,eq:sinusoidal_surface_params}.
The RMS error is shown as a function of the mesh point density in
\cref{fig:sinusoidal_manifold_errors}. It appears to scale quadratically
with $\Delta_{\min}$, the smallest permitted separation between neighboring
mesh points --- which doubles as a measure of the overall mesh density.

\begin{figure}[htpb]
    \centering
    \resizebox{0.9\textwidth}{!}{
    \importpgf{figures/mpl-figs}{sinusoidal-errorscaling.pgf}}
    \caption[RMS error of geodesic level set approximations to an analytically
    known three-dimensional surface, as a function of mesh point density]
    {RMS error of geodesic level set approximations to an analytically known
    three-dimensional surface, as a function of mesh point density.
    Using seven different mesh point densities (here denoted by their
    $\Delta_{\min}$), we computed approximations of the sinusoidal surface
    defined by \cref{eq:sinusoidal_definition,eq:sinusoidal_surface_params} ---
    four of which are shown in \cref{fig:sinusoidal_manifolds} --- whose
    RMS error (as defined in \cref{eq:sinusoidal_rms}) are shown as solid
    dots. Note how the RMS error appears to increase quadratically
    as a function of the mesh density.
}
    \label{fig:sinusoidal_manifold_errors}
\end{figure}



Note that, due to the local (pseudo-)planar nature of the parametrization of
manifolds by means of points organized in level sets, the number of mesh points
in the parametrization of a manifold \emph{increases} quadratically as
$\Delta_{\min}$ decreases. Thus, the decrease in numerical error comes at the
cost of an increased consumption of computational resources. For instance,
the method's most computationally costly operations (namely, generating
trajectories to identify new mesh points and checking for self-intersections,
cf.\ sections~\ref{sec:revised_approach_to_computing_new_mesh_points} and
\ref{sub:continuous_self_intersection_checks}, respectively) need to be
performed far more frequently when the number of mesh points increases.
Perhaps more crucially, an increased number of mesh points leads to an
increase in the required memory. These considerations, together with the
ones to follow in the immediately forthcoming section, became the foundation
on which we based our choice of mesh point density for the computation of
LCSs (which will be elaborated upon in
\cref{sec:computed_lcss_in_the_abc_flow,sec:computed_lcss_in_the_forde_fjord}).

\subsection{Sample manifolds computed from the steady ABC flow}
\label{sub:sample_manifolds_computed_from_the_steady_abc_flow}

\Cref{fig:abc_manifold_convergence} shows a sample manifold obtained for the
steady ABC flow (cf.\
\cref{sec:flow_systems_defined_by_analytical_velocity_fields}), computed for a
few different mesh point densities, and otherwise similar parameters as were
used in order to approximate the sinusoidal surface (see
\cref{tab:sinusoidal_manifold_params,fig:sinusoidal_manifolds}). From
\cref{fig:abc_manifold_convergence}, it is apparent that increasing the mesh
point density generally results in increased resolution. However, there appears
to be some density threshold, above which unwanted numerical artefacts
start to emerge. This is particularly evident in
\cref{fig:abc_manifold_minsep=0.01}, where the sharp indentation appears to be
out of place. Such oddities could be due to accumulation of numerical
error when computing new mesh points from ficticious ancestor points
(as described in \cref{sub:maintaining_mesh_point_density}) --- which naturally
occurs more often with increased mesh point density. Accordingly, our
choice of mesh point density for (both variants of) the ABC flow (which will be
presented in \cref{tab:abc_manifold_params}) was guided by a desire to limit
the amount of such oddities, in addition to the memory required in order to
store each manifold (as mentioned in
\cref{sub:an_analytical_manifold_test_case}).

\begin{figure}[htpb]
    \centering
    \begin{subfigure}[b]{0.475\textwidth}
        \centering
        \importpgf{figures/mpl-figs}{convergence-minsep=0.16-small.pgf}
        \caption[]{{\small min\_sep = 0.16}}
        \label{fig:u0_dom_err_bs32}
    \end{subfigure}
    \begin{subfigure}[b]{0.475\textwidth}
        \centering
        \importpgf{figures/mpl-figs}{convergence-minsep=0.08-small.pgf}
        \caption[]{{\small min\_sep = 0.08}}
        \label{fig:u0_dom_err_bs54}
    \end{subfigure}

    \begin{subfigure}[b]{0.475\textwidth}
        \centering
        \importpgf{figures/mpl-figs}{convergence-minsep=0.04-small.pgf}
        \caption[]{{\small min\_sep = 0.04}}
        \label{fig:u0_dom_err_dp54}
    \end{subfigure}
    \begin{subfigure}[b]{0.475\textwidth}
        \centering
        \importpgf{figures/mpl-figs}{convergence-minsep=0.01-small.pgf}
        \caption[]{{\small min\_sep = 0.01}}
        \label{fig:u0_dom_err_dp87}
    \end{subfigure}
    \caption[Veni, vidi, Aviici]{Numerical convergence as min\_sep decreases}
    \label{fig:u0_dom_errs}
\end{figure}


\begin{figure}[htpb]
    \centering
    \hspace*{\fill}
    \begin{subfigure}[b]{0.45\textwidth}
        \centering
        \resizebox{0.9\linewidth}{!}{\importpgf{figures/mpl-figs}{verification-forced-small.pgf}}
        \caption[]{{\small Trajectories computed in the direction
        \\\phantom{(a)} field given by \cref{eq:revised_direction_field}.}}
        \label{fig:verification_forced_outwards}
    \end{subfigure}\hfill%
    \begin{subfigure}[b]{0.45\textwidth}
        \centering
        \resizebox{0.9\linewidth}{!}{\importpgf{figures/mpl-figs}{verification-notforced-small.pgf}}
        \caption[]{{\small Trajectories of the direction field given by
                \\\phantom{(b)} \cref{eq:dynamialsystem_initialposition},
        for various weights $(a,b)$.}}
        \label{fig:verification_pure_linear_combination}
    \end{subfigure}
    \hspace*{\fill}
    \caption[Trajectories orthogonal to the $\vct{\xi}_{3}$-direction field,
    superimposed onto a computed manifold surface in the steady ABC flow]
    {Trajectories orthogonal to the $\vct{\xi}_{3}$-direction field,
        superimposed onto a computed manifold surface in the steady ABC flow.
        The manifold was computed using the parameters given in
        \cref{tab:abc_manifold_params}. The trajectories were computed as
        solution curves of equations~\eqref{eq:revised_direction_field} and
        \eqref{eq:dynamialsystem_initialposition}, respectively, starting at (a
        small circle laying within the manifold, centered in) the manifold
        epicentre $\vct{x}_{0}$. Note how none of the trajectories ever leave
        the manifold. The trajectories shown in
        (\subref*{fig:verification_forced_outwards}) cover the entire manifold,
        which is as expected, as the manifold mesh points were computed as end
        points of trajectories in a similar direction field (cf.\
        \cref{sec:revised_approach_to_computing_new_mesh_points}). The
        trajectories shown in
        (\subref*{fig:verification_pure_linear_combination}) never leave
        the manifold, but face difficulties reaching certain regions of it,
        as can be seen from locally reduced density of trajectories. For
        trajectories passing through $\vct{x}_{0}$, such
        regions are likely only accessible for very particular weights $(a,b)$.
    }
    \label{fig:verification_of_manifold_invariance}
\end{figure}



Motivated by \cref{rmk:invariance_lcs}, we sought to verify that our computed
manifolds, by virtue of containing repelling LCSs, act as invariant manifolds
for arbitrary linear combinations of the $\vct{\xi}_{1}$- and
$\vct{\xi}_{2}$-direction fields.
\Cref{fig:verification_of_manifold_invariance} shows a sample manifold obtained
for the steady ABC flow (described in detail in
\cref{sec:flow_systems_defined_by_analytical_velocity_fields}), using the
parameter values provided in \cref{tab:abc_manifold_params} --- the very same
that we used in order to compute LCSs in said flow (more to follow in
\cref{sec:computed_lcss_in_the_abc_flow}). It also shows $200$ different
trajectories launched from (a small circle laying within the manifold, centered
in) the manifold epicentre $\vct{x}_{0}$ and computed using the Dormand-Prince
8(7) adaptive ODE solver (see \cref{tab:butcherdopri87,%
sub:the_implementation_of_dynamic_runge_kutta_step_size}). In
\cref{fig:verification_forced_outwards}, the trajectories are solution
curves of \cref{eq:revised_direction_field}, where $\vct{t}_{i,j}$ was kept
constant along each trajectory. Meanwhile,
\cref{fig:verification_pure_linear_combination} shows solution curves
of \cref{eq:dynamialsystem_initialposition}, for $200$ different pairs of
weights $(a,b)$ such that the initial trajectory directions were evenly
distributed in the plane defined by the coordinate $\vct{x}_{0}$ and
the unit normal $\vct{\xi}_{3}(\vct{x}_{0})$.

Unsurprisingly, the curves in \cref{fig:verification_forced_outwards} all stay
within the computed manifold, as all of its constituent mesh points were
computed as endpoints of trajectories in a similarly defined direction field
(see \cref{sec:revised_approach_to_computing_new_mesh_points}). The same
applies for the curves shown in
\cref{fig:verification_pure_linear_combination}; however, there seems to be
some regions of the manifold which are particularly challenging to hit ---
signified by a decreased trajectory density --- it appears that very specific
weights $(a,b)$ are required to reach them. Regardless, none of the computed
trajectories ever leave the computed surface, which confirms that the surface
\emph{is} in fact an invariant manifold of the $\vct{\xi}_{1}$- and
$\vct{\xi}_{2}$-direction fields, in compliance with LCS existence criterion~%
\eqref{eq:lcs_condition_c} (see also \cref{rmk:invariance_lcs}).

