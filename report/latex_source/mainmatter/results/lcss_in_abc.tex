\section{Computed LCSs in the ABC flow}
\label{sec:computed_lcss_in_the_abc_flow}

Having computed Cauchy-Green strain eigenvalues and -vectors for both
variants of the ABC flow (as presented in
\cref{sec:flow_systems_defined_by_analytical_velocity_fields}), we used
the $\mathcal{U}_{0}$ domain --- i.e., the grid points satisfying the LCS
existence criteria~ \eqref{eq:lcs_condition_a},~\eqref{eq:lcs_condition_b}
and~\eqref{eq:lcs_condition_b} (the implementation of which are
described in
\cref{sub:identifying_suitable_initial_conditions_for_developing_lcss}) --- as
a first approximation to where we may reasonably expect to find repelling LCSs.
Four different views of the $\mathcal{U}_{0}$ domains for the steady
and unsteady ABC flows are shown in \cref{fig:steady_abd,fig:unsteady_abd},
respectively.

\begin{figure}[htpb]
    \centering
    \begin{subfigure}[b]{0.475\textwidth}
        \centering
        \importpgf{figures/mpl-figs}{indep-abd-domain-view1-small.pgf}
        \caption[]{{\small View along the negative $z$-axis}}
        \label{fig:steady_abd_z}
    \end{subfigure}
    \begin{subfigure}[b]{0.475\textwidth}
        \centering
        \importpgf{figures/mpl-figs}{indep-abd-domain-view2-small.pgf}
        \caption[]{{\small View along the positive $y$-axis}}
        \label{fig:steady_abd_y}
    \end{subfigure}

    \begin{subfigure}[b]{0.475\textwidth}
        \centering
        \importpgf{figures/mpl-figs}{indep-abd-domain-view3-small.pgf}
        \caption[]{{\small View along the positive $x$-axis}}
        \label{fig:steady_abd_x}
    \end{subfigure}
    \begin{subfigure}[b]{0.475\textwidth}
        \centering
        \importpgf{figures/mpl-figs}{indep-abd-domain-view4-small.pgf}
        \caption[]{{\small Approximately isometric view}}
        \label{fig:steady_abd_isometric}
    \end{subfigure}
    \caption[Four views of the $\mathcal{U}_{0}$ domain obtained for transport
    in the steady ABC flow]
    {
        Four views of the $\mathcal{U}_{0}$ domain obtained for transport in the
        steady ABC flow, identified as the grid points which satisfy the LCS
        criteria \eqref{eq:lcs_condition_a},~\eqref{eq:lcs_condition_b} and~
        \eqref{eq:lcs_condition_d} (details on the implementation are available
        in
        \cref{sub:identifying_suitable_initial_conditions_for_developing_lcss}).
        In total, the domain consists of \numprint{340951} different points
        (cf.\ \cref{tab:initialconditionparams}).
}
    \label{fig:steady_abd}
\end{figure}



\begin{figure}[htpb]
    \centering
    \begin{subfigure}[b]{0.475\textwidth}
        \centering
        \importpgf{figures/mpl-figs}{dep-abd-domain-view1-small.pgf}
        \caption[]{{\small View along the negative $z$-axis}}
        \label{fig:unsteady_abd_z}
    \end{subfigure}
    \begin{subfigure}[b]{0.475\textwidth}
        \centering
        \importpgf{figures/mpl-figs}{dep-abd-domain-view2-small.pgf}
        \caption[]{{\small View along the positive $y$-axis}}
        \label{fig:unsteady_abd_y}
    \end{subfigure}

    \begin{subfigure}[b]{0.475\textwidth}
        \centering
        \importpgf{figures/mpl-figs}{dep-abd-domain-view3-small.pgf}
        \caption[]{{\small View along the positive $x$-axis}}
        \label{fig:unsteady_abd_x}
    \end{subfigure}
    \begin{subfigure}[b]{0.475\textwidth}
        \centering
        \importpgf{figures/mpl-figs}{dep-abd-domain-view4-small.pgf}
        \caption[]{{\small Approximately isometric view}}
        \label{fig:unsteady_abd_isometric}
    \end{subfigure}
    \caption[Four views of the $\mathcal{U}_{0}$ domain obtained for transport
    in the unsteady ABC flow]
    {
        Four views of the $\mathcal{U}_{0}$ domain obtained for transport in the
        unsteady ABC flow, identified as the grid points which satisfy the LCS
        criteria \eqref{eq:lcs_condition_a},~\eqref{eq:lcs_condition_b} and~
        \eqref{eq:lcs_condition_d} (details on the implementation are available
        in
        \cref{sub:identifying_suitable_initial_conditions_for_developing_lcss}).
        In total, the domain consists of \numprint{361461} different points
        (cf.\ \cref{tab:initialconditionparams}).
}
    \label{fig:unsteady_abd}
\end{figure}



Note that although the $\mathcal{U}_{0}$ domains for the two
flow variants do not contain the same number of points (cf.\
\cref{tab:initialconditionparams}), the macroscopic trends remain the same. The
difference lies in minute details, such as the topmost cavities in
\cref{fig:steady_abd_z,fig:unsteady_abd_z} being of slightly different sizes,
or the point densities along the north east `bands' in \cref{fig:steady_abd_y,%
fig:unsteady_abd_y} being somewhat dissimilar. This is as expected, seeing
as the two underlying transport systems are lightly perturbed versions of
one another.

Using the filtering parameters provided in \cref{tab:gridparams}, we identified
initial conditions for the development of manifolds as a subsets of the
$\mathcal{U}_{0}$ domains --- yielding a total of \numprint{618} and
\numprint{676} points for the steady and unsteady variants of the ABC flow,
respectively.  Then, we computed manifolds and extracted LCSs using the
parameters in \cref{tab:initialconditionparams,%
tab:abc_manifold_params} and the method outlined in \cref{sec:preliminaries%
_for_computing_repelling_lcss_in_3d_flow_by_means_of_geodesic_level_sets,%
    sec:revised_approach_to_computing_new_mesh_points,%
    sec:managing_mesh_accuracy,%
    sec:continuously_reconstructing_three_dimensional_manifold_surfaces_from_%
    point_meshes,%
    sec:macroscale_stopping_criteria_for_the_expansion_of_computed_manifolds,%
    sec:identifying_lcss_as_subsets_of_computed_manifolds}. This resulted
in a total of \numprint{22} LCS surfaces for the steady flow, and
\numprint{31} surfaces for the unsteady flow.

\begin{table}[htpb]
    \centering
    \caption[Parameters used to compute manifolds, and subsequently
    repelling LCSs, in the ABC flow (both variants)]
    {Parameters used to compute manifolds, and subsequently repelling LCSs,
        in both variants of the ABC flow (see
        \cref{sec:flow_systems_defined_by_analytical_velocity_fields}). Note
        that the interset distances $\Delta_{i}$ were dynamically altered, as
        described in
        \cref{sub:a_curvature_based_approach_to_determining_interset%
        _separations}. Accordingly, only the \emph{first} interset distance,
        $\Delta_{1}$, was set explicitly.
    }
    \label{tab:abc_manifold_params}
    \begin{tabular}{ccc}
        \toprule
        Parameter & Value & Description\\
        \midrule
        $\delta_{\text{init}}$ & $10^{-3}$ %
        & \makecell{Separation of innermost geodesic level set \\
        from the manifold epicentre, cf. \cref{fig:innermost_levelset}}%
        \\[9pt]
        %
        $\Delta_{\min}$, $\Delta_{\max}$
        & $0.04$, $0.16$ %
        & \makecell{Boundaries for interpoint \\separations (details
        found in \cref{sub:maintaining_mesh_point_density})}%
        \\[9pt]
        %
        $\Delta_{1}$ %
        & $2\Delta_{\min}$ %
        & \makecell{Interset distance used to compute the second \\ geodesic
        level set (see
        \cref{sec:revised_approach_to_computing_new_mesh_points,%
        sub:a_curvature_based_approach_to_determining_interset_separations})}%
        \\[9pt]
        %
        $\gamma_{\|}$ %
        & $10^{-4}$ %
        & \makecell{Tolerance for detecting regions in which
        $\vct{\xi}_{3}$\\ is (anti-)parallel to $\vct{t}_{i,j}$
    (see \cref{eq:revised_xi3_tan_parallel})}
        \\[9pt]
        %
        $\gamma_{\Delta}$ %
        & $5\cdot10^{-3}$ %
        & \makecell{Tolerance for the separation of a mesh point\\ from
        its ancestor (per \cref{eq:revised_dist_tol})}
        \\[9pt]
        $\gamma_{\text{arc}}$ %
        & 5 %
        & \makecell{Sets an upper limit to trajectory lengths as \\
        $\gamma_{\text{arc}}\Delta_{i}$ (briefly mentioned in
        \cref{sub:computing_pseudoradial_trajectories_directly})}
        \\[9pt]
        %
        $\gamma_{\circlearrowleft}$ %
        & $7\cdot10^{-1}$
        & \makecell{Sets an upper limit to the extent of loop-like\\
        segments of any level set (see
        \cref{sub:limiting_the_accumulation_of_numerical_noise})}
        \\[9pt]
        %
        \makecell[c]{$\alpha_{\uparrow}$\\ $\alpha_{\downarrow}$ \\[1.5pt]%
        ${(\Delta\alpha)}_{\uparrow}$, ${(\Delta\alpha)}_{\downarrow}$} &
        \makecell[c]{$8.7\cdot10^{-2}\;\si{\radian}$ %
            \phantom{2}$({5}\si{\degree})$\\ %
            ${4.4\cdot10^{-1}}\;\si{\radian}$ $({25}\si{\degree})$\\[1.5pt]%
        $2\Delta_{\min}\alpha_{\uparrow}$, %
        $2\Delta_{\min}\alpha_{\downarrow}$}%
        & \makecell[c]{Used in a curvature-based approach to adjust\\
        interset distances (outlined in
        \cref{sub:a_curvature_based_approach_to_determining_interset%
        _separations})}
        \\[18pt]
        %
        $\gamma_{\cap}$ &
        $5$ &
        \makecell[c]{Used for terminating the expansion of \\
        self-intersecting manifolds (cf.\
        \cref{sub:continuous_self_intersection_checks})}
        \\[9pt]
        %
        $\gamma_{\square}$ &
        $1.75$ &
        \makecell[c]{Relaxation parameter for extracting LCSs from\\ the
            computed manifolds (see
        \cref{sec:identifying_lcss_as_subsets_of_computed_manifolds})}
        \\[9pt]
        %
        $\mathcal{W}_{\text{min}}$ &
        $6.0$ &
        \makecell[c]{Filters away the smallest LCSs measured in\\
        (pseudo-)surface area (see
        \cref{sec:identifying_lcss_as_subsets_of_computed_manifolds})}
        \\
        \bottomrule
    \end{tabular}
\end{table}



The LCSs present in the steady flow turn out to form two distinct,
smooth and coherent structures, which are shown in \cref{fig:steady_lcss}. The
structures lie close together, yet appear not to be connected --- accordingly,
the two structures are highlighted by different colors for the purpose of
facilitating visual comparisons. Note in particular the correspondence between
the $\mathcal{U}_{0}$  domain, shown in \cref{fig:steady_abd}, and the computed
LCSs, shown in \cref{fig:steady_lcss} --- where the perspective of each
subfigure is the same as that of the corresponding subfigure in
\cref{fig:steady_abd}. Two prominent similarities are the tunnel-like structure
apparent in the middle right of \cref{fig:steady_abd_y,fig:steady_lcss_y}, and
the indent which manifests near the bottom right corner of
\cref{fig:steady_abd_x,fig:steady_lcss_x}.

\begin{figure}[htpb]
    \centering
    \hspace*{\fill}
    \begin{subfigure}[b]{0.42\textwidth}
        \centering
        \resizebox{0.9\linewidth}{!}{\importpgf{figures/mpl-figs}{indep-lcss-view1-small.pgf}}
        \caption[]{{\small View along the negative $z$-axis}}
        \label{fig:steady_lcss_z}
    \end{subfigure}\hfill%
    \begin{subfigure}[b]{0.42\textwidth}
        \centering
        \resizebox{0.9\linewidth}{!}{\importpgf{figures/mpl-figs}{indep-lcss-view2-small.pgf}}
        \caption[]{{\small View along the positive $y$-axis}}
        \label{fig:steady_lcss_y}
    \end{subfigure}%
    \hspace*{\fill}

    \hspace*{\fill}
    \begin{subfigure}[b]{0.42\textwidth}
        \centering
        \resizebox{0.9\linewidth}{!}{\importpgf{figures/mpl-figs}{indep-lcss-view3-small.pgf}}
        \caption[]{{\small View along the positive $x$-axis}}
        \label{fig:steady_lcss_x}
    \end{subfigure}\hfill%
    \begin{subfigure}[b]{0.42\textwidth}
        \centering
        \resizebox{0.9\linewidth}{!}{\importpgf{figures/mpl-figs}{indep-lcss-view4-small.pgf}}
        \caption[]{{\small Approximately isometric view}}
        \label{fig:steady_lcss_isometric}
    \end{subfigure}%
    \hspace*{\fill}
    \caption[Four views of the repelling LCSs obtained for transport in the
    steady ABC flow]
    {
        Four views of the repelling LCSs obtained for transport in the steady
        ABC flow, for the time interval $\mathcal{I}=[0,5]$ (see
        \cref{sub:steady_arnold_beltrami_childress_flow}). A total of
        \numprint{22} surfaces constitute two distinct, smooth and coherent
        structures, which are shown in different colors. We provide four
        different viewing angles (the same as the ones used in
        \cref{fig:steady_abd}, which shows the computed $\mathcal{U}_{0}$
        domain), chosen in order convey the three-dimensional structures in as
        great detail as possible.
}
    \label{fig:steady_lcss}
\end{figure}



The computed LCSs in the unsteady flow constitute three distinct, smooth and
coherent structures, which are shown in \cref{fig:unsteady_lcss}. Although
the structures lie adjacent to eachother, they do not seem to be connected.
Thus, the different structures are indicated using disparate colors, yet
again to facilitate visual comparisons. Just like for the LCSs in the steady
flow, the computed LCSs strongly resemble subsets of the $\mathcal{U}_{0}$,
shown in \cref{fig:unsteady_abd} --- where the viewing angle of each
$\mathcal{U}_{0}$ domain subfigure is the same as that of the corresponding
subfigure in \cref{fig:unsteady_lcss}. The two largest structures apparent in
\cref{fig:unsteady_lcss} harmonize with the two dominant structures found for
the steady flow (see \cref{fig:steady_lcss}). The smallest structure appears
reasonable when considered together with the $\mathcal{U}_{0}$ domain --- shown
in \cref{fig:unsteady_abd} --- in particular, it further enhances the
tubular structures manifesting in the middle right of
\cref{fig:unsteady_abd_y,fig:unsteady_lcss_y}, and the middle top of
\cref{fig:unsteady_abd_x,fig:unsteady_lcss_x}.

\begin{figure}[htpb]
    \centering
    \begin{subfigure}[b]{0.475\textwidth}
        \centering
        %% Creator: Matplotlib, PGF backend
%%
%% To include the figure in your LaTeX document, write
%%   \input{<filename>.pgf}
%%
%% Make sure the required packages are loaded in your preamble
%%   \usepackage{pgf}
%%
%% Figures using additional raster images can only be included by \input if
%% they are in the same directory as the main LaTeX file. For loading figures
%% from other directories you can use the `import` package
%%   \usepackage{import}
%% and then include the figures with
%%   \import{<path to file>}{<filename>.pgf}
%%
%% Matplotlib used the following preamble
%%   \usepackage{fontspec}
%%   \setmainfont{DejaVu Serif}
%%   \setsansfont{DejaVu Sans}
%%   \setmonofont{DejaVu Sans Mono}
%%
\begingroup%
\makeatletter%
\begin{pgfpicture}%
\pgfpathrectangle{\pgfpointorigin}{\pgfqpoint{2.660000in}{1.740000in}}%
\pgfusepath{use as bounding box, clip}%
\begin{pgfscope}%
\pgfsetbuttcap%
\pgfsetmiterjoin%
\definecolor{currentfill}{rgb}{1.000000,1.000000,1.000000}%
\pgfsetfillcolor{currentfill}%
\pgfsetlinewidth{0.000000pt}%
\definecolor{currentstroke}{rgb}{1.000000,1.000000,1.000000}%
\pgfsetstrokecolor{currentstroke}%
\pgfsetdash{}{0pt}%
\pgfpathmoveto{\pgfqpoint{0.000000in}{0.000000in}}%
\pgfpathlineto{\pgfqpoint{2.660000in}{0.000000in}}%
\pgfpathlineto{\pgfqpoint{2.660000in}{1.740000in}}%
\pgfpathlineto{\pgfqpoint{0.000000in}{1.740000in}}%
\pgfpathclose%
\pgfusepath{fill}%
\end{pgfscope}%
\begin{pgfscope}%
\pgfsetbuttcap%
\pgfsetmiterjoin%
\definecolor{currentfill}{rgb}{1.000000,1.000000,1.000000}%
\pgfsetfillcolor{currentfill}%
\pgfsetlinewidth{0.000000pt}%
\definecolor{currentstroke}{rgb}{0.000000,0.000000,0.000000}%
\pgfsetstrokecolor{currentstroke}%
\pgfsetstrokeopacity{0.000000}%
\pgfsetdash{}{0pt}%
\pgfpathmoveto{\pgfqpoint{-0.798000in}{-0.261000in}}%
\pgfpathlineto{\pgfqpoint{3.192000in}{-0.261000in}}%
\pgfpathlineto{\pgfqpoint{3.192000in}{2.088000in}}%
\pgfpathlineto{\pgfqpoint{-0.798000in}{2.088000in}}%
\pgfpathclose%
\pgfusepath{fill}%
\end{pgfscope}%
\begin{pgfscope}%
\pgfsetbuttcap%
\pgfsetmiterjoin%
\pgfsetlinewidth{0.000000pt}%
\definecolor{currentstroke}{rgb}{1.000000,1.000000,1.000000}%
\pgfsetstrokecolor{currentstroke}%
\pgfsetstrokeopacity{0.000000}%
\pgfsetdash{}{0pt}%
\pgfpathmoveto{\pgfqpoint{0.181337in}{0.293965in}}%
\pgfpathlineto{\pgfqpoint{0.185027in}{1.550475in}}%
\pgfpathlineto{\pgfqpoint{0.068195in}{1.665415in}}%
\pgfpathlineto{\pgfqpoint{0.063651in}{0.271092in}}%
\pgfusepath{}%
\end{pgfscope}%
\begin{pgfscope}%
\pgfsetbuttcap%
\pgfsetmiterjoin%
\pgfsetlinewidth{0.000000pt}%
\definecolor{currentstroke}{rgb}{1.000000,1.000000,1.000000}%
\pgfsetstrokecolor{currentstroke}%
\pgfsetstrokeopacity{0.000000}%
\pgfsetdash{}{0pt}%
\pgfpathmoveto{\pgfqpoint{0.185027in}{1.550475in}}%
\pgfpathlineto{\pgfqpoint{2.316811in}{1.550475in}}%
\pgfpathlineto{\pgfqpoint{2.433643in}{1.665415in}}%
\pgfpathlineto{\pgfqpoint{0.068195in}{1.665415in}}%
\pgfusepath{}%
\end{pgfscope}%
\begin{pgfscope}%
\pgfsetbuttcap%
\pgfsetmiterjoin%
\pgfsetlinewidth{0.000000pt}%
\definecolor{currentstroke}{rgb}{1.000000,1.000000,1.000000}%
\pgfsetstrokecolor{currentstroke}%
\pgfsetstrokeopacity{0.000000}%
\pgfsetdash{}{0pt}%
\pgfpathmoveto{\pgfqpoint{0.181337in}{0.293965in}}%
\pgfpathlineto{\pgfqpoint{2.320500in}{0.293965in}}%
\pgfpathlineto{\pgfqpoint{2.316811in}{1.550475in}}%
\pgfpathlineto{\pgfqpoint{0.185027in}{1.550475in}}%
\pgfusepath{}%
\end{pgfscope}%
\begin{pgfscope}%
\pgfsetrectcap%
\pgfsetroundjoin%
\pgfsetlinewidth{0.803000pt}%
\definecolor{currentstroke}{rgb}{0.000000,0.000000,0.000000}%
\pgfsetstrokecolor{currentstroke}%
\pgfsetdash{}{0pt}%
\pgfpathmoveto{\pgfqpoint{0.181337in}{0.293965in}}%
\pgfpathlineto{\pgfqpoint{2.320500in}{0.293965in}}%
\pgfusepath{stroke}%
\end{pgfscope}%
\begin{pgfscope}%
\pgftext[x=1.250919in,y=0.078474in,,]{\sffamily\fontsize{10.000000}{12.000000}\selectfont \(\displaystyle x\)}%
\end{pgfscope}%
\begin{pgfscope}%
\pgfsetbuttcap%
\pgfsetroundjoin%
\pgfsetlinewidth{0.803000pt}%
\definecolor{currentstroke}{rgb}{0.690196,0.690196,0.690196}%
\pgfsetstrokecolor{currentstroke}%
\pgfsetdash{}{0pt}%
\pgfpathmoveto{\pgfqpoint{0.338013in}{0.293965in}}%
\pgfpathlineto{\pgfqpoint{0.341162in}{1.550475in}}%
\pgfpathlineto{\pgfqpoint{0.241445in}{1.665415in}}%
\pgfusepath{stroke}%
\end{pgfscope}%
\begin{pgfscope}%
\pgfsetbuttcap%
\pgfsetroundjoin%
\pgfsetlinewidth{0.803000pt}%
\definecolor{currentstroke}{rgb}{0.690196,0.690196,0.690196}%
\pgfsetstrokecolor{currentstroke}%
\pgfsetdash{}{0pt}%
\pgfpathmoveto{\pgfqpoint{0.915721in}{0.293965in}}%
\pgfpathlineto{\pgfqpoint{0.916877in}{1.550475in}}%
\pgfpathlineto{\pgfqpoint{0.880263in}{1.665415in}}%
\pgfusepath{stroke}%
\end{pgfscope}%
\begin{pgfscope}%
\pgfsetbuttcap%
\pgfsetroundjoin%
\pgfsetlinewidth{0.803000pt}%
\definecolor{currentstroke}{rgb}{0.690196,0.690196,0.690196}%
\pgfsetstrokecolor{currentstroke}%
\pgfsetdash{}{0pt}%
\pgfpathmoveto{\pgfqpoint{1.493428in}{0.293965in}}%
\pgfpathlineto{\pgfqpoint{1.492592in}{1.550475in}}%
\pgfpathlineto{\pgfqpoint{1.519081in}{1.665415in}}%
\pgfusepath{stroke}%
\end{pgfscope}%
\begin{pgfscope}%
\pgfsetbuttcap%
\pgfsetroundjoin%
\pgfsetlinewidth{0.803000pt}%
\definecolor{currentstroke}{rgb}{0.690196,0.690196,0.690196}%
\pgfsetstrokecolor{currentstroke}%
\pgfsetdash{}{0pt}%
\pgfpathmoveto{\pgfqpoint{2.071136in}{0.293965in}}%
\pgfpathlineto{\pgfqpoint{2.068306in}{1.550475in}}%
\pgfpathlineto{\pgfqpoint{2.157899in}{1.665415in}}%
\pgfusepath{stroke}%
\end{pgfscope}%
\begin{pgfscope}%
\pgfsetrectcap%
\pgfsetroundjoin%
\pgfsetlinewidth{0.803000pt}%
\definecolor{currentstroke}{rgb}{0.000000,0.000000,0.000000}%
\pgfsetstrokecolor{currentstroke}%
\pgfsetdash{}{0pt}%
\pgfpathmoveto{\pgfqpoint{0.338038in}{0.304052in}}%
\pgfpathlineto{\pgfqpoint{0.337963in}{0.273790in}}%
\pgfusepath{stroke}%
\end{pgfscope}%
\begin{pgfscope}%
\pgftext[x=0.342401in,y=0.231657in,,top]{\sffamily\fontsize{10.000000}{12.000000}\selectfont \(\displaystyle 0\)}%
\end{pgfscope}%
\begin{pgfscope}%
\pgfsetrectcap%
\pgfsetroundjoin%
\pgfsetlinewidth{0.803000pt}%
\definecolor{currentstroke}{rgb}{0.000000,0.000000,0.000000}%
\pgfsetstrokecolor{currentstroke}%
\pgfsetdash{}{0pt}%
\pgfpathmoveto{\pgfqpoint{0.915730in}{0.304052in}}%
\pgfpathlineto{\pgfqpoint{0.915702in}{0.273790in}}%
\pgfusepath{stroke}%
\end{pgfscope}%
\begin{pgfscope}%
\pgftext[x=0.917332in,y=0.231657in,,top]{\sffamily\fontsize{10.000000}{12.000000}\selectfont \(\displaystyle 2\)}%
\end{pgfscope}%
\begin{pgfscope}%
\pgfsetrectcap%
\pgfsetroundjoin%
\pgfsetlinewidth{0.803000pt}%
\definecolor{currentstroke}{rgb}{0.000000,0.000000,0.000000}%
\pgfsetstrokecolor{currentstroke}%
\pgfsetdash{}{0pt}%
\pgfpathmoveto{\pgfqpoint{1.493421in}{0.304052in}}%
\pgfpathlineto{\pgfqpoint{1.493442in}{0.273790in}}%
\pgfusepath{stroke}%
\end{pgfscope}%
\begin{pgfscope}%
\pgftext[x=1.492263in,y=0.231657in,,top]{\sffamily\fontsize{10.000000}{12.000000}\selectfont \(\displaystyle 4\)}%
\end{pgfscope}%
\begin{pgfscope}%
\pgfsetrectcap%
\pgfsetroundjoin%
\pgfsetlinewidth{0.803000pt}%
\definecolor{currentstroke}{rgb}{0.000000,0.000000,0.000000}%
\pgfsetstrokecolor{currentstroke}%
\pgfsetdash{}{0pt}%
\pgfpathmoveto{\pgfqpoint{2.071113in}{0.304052in}}%
\pgfpathlineto{\pgfqpoint{2.071181in}{0.273790in}}%
\pgfusepath{stroke}%
\end{pgfscope}%
\begin{pgfscope}%
\pgftext[x=2.067194in,y=0.231657in,,top]{\sffamily\fontsize{10.000000}{12.000000}\selectfont \(\displaystyle 6\)}%
\end{pgfscope}%
\begin{pgfscope}%
\pgfsetrectcap%
\pgfsetroundjoin%
\pgfsetlinewidth{0.803000pt}%
\definecolor{currentstroke}{rgb}{0.000000,0.000000,0.000000}%
\pgfsetstrokecolor{currentstroke}%
\pgfsetdash{}{0pt}%
\pgfpathmoveto{\pgfqpoint{2.316811in}{1.550475in}}%
\pgfpathlineto{\pgfqpoint{2.320500in}{0.293965in}}%
\pgfusepath{stroke}%
\end{pgfscope}%
\begin{pgfscope}%
\pgftext[x=2.538288in,y=0.918793in,,]{\sffamily\fontsize{10.000000}{12.000000}\selectfont \(\displaystyle y\)}%
\end{pgfscope}%
\begin{pgfscope}%
\pgfsetbuttcap%
\pgfsetroundjoin%
\pgfsetlinewidth{0.803000pt}%
\definecolor{currentstroke}{rgb}{0.690196,0.690196,0.690196}%
\pgfsetstrokecolor{currentstroke}%
\pgfsetdash{}{0pt}%
\pgfpathmoveto{\pgfqpoint{0.064010in}{0.381351in}}%
\pgfpathlineto{\pgfqpoint{0.181629in}{0.393291in}}%
\pgfpathlineto{\pgfqpoint{2.320209in}{0.393291in}}%
\pgfusepath{stroke}%
\end{pgfscope}%
\begin{pgfscope}%
\pgfsetbuttcap%
\pgfsetroundjoin%
\pgfsetlinewidth{0.803000pt}%
\definecolor{currentstroke}{rgb}{0.690196,0.690196,0.690196}%
\pgfsetstrokecolor{currentstroke}%
\pgfsetdash{}{0pt}%
\pgfpathmoveto{\pgfqpoint{0.065240in}{0.758791in}}%
\pgfpathlineto{\pgfqpoint{0.182628in}{0.733352in}}%
\pgfpathlineto{\pgfqpoint{2.319210in}{0.733352in}}%
\pgfusepath{stroke}%
\end{pgfscope}%
\begin{pgfscope}%
\pgfsetbuttcap%
\pgfsetroundjoin%
\pgfsetlinewidth{0.803000pt}%
\definecolor{currentstroke}{rgb}{0.690196,0.690196,0.690196}%
\pgfsetstrokecolor{currentstroke}%
\pgfsetdash{}{0pt}%
\pgfpathmoveto{\pgfqpoint{0.066468in}{1.135449in}}%
\pgfpathlineto{\pgfqpoint{0.183624in}{1.072778in}}%
\pgfpathlineto{\pgfqpoint{2.318214in}{1.072778in}}%
\pgfusepath{stroke}%
\end{pgfscope}%
\begin{pgfscope}%
\pgfsetbuttcap%
\pgfsetroundjoin%
\pgfsetlinewidth{0.803000pt}%
\definecolor{currentstroke}{rgb}{0.690196,0.690196,0.690196}%
\pgfsetstrokecolor{currentstroke}%
\pgfsetdash{}{0pt}%
\pgfpathmoveto{\pgfqpoint{0.067693in}{1.511328in}}%
\pgfpathlineto{\pgfqpoint{0.184619in}{1.411571in}}%
\pgfpathlineto{\pgfqpoint{2.317219in}{1.411571in}}%
\pgfusepath{stroke}%
\end{pgfscope}%
\begin{pgfscope}%
\pgfsetrectcap%
\pgfsetroundjoin%
\pgfsetlinewidth{0.803000pt}%
\definecolor{currentstroke}{rgb}{0.000000,0.000000,0.000000}%
\pgfsetstrokecolor{currentstroke}%
\pgfsetdash{}{0pt}%
\pgfpathmoveto{\pgfqpoint{2.303100in}{0.393291in}}%
\pgfpathlineto{\pgfqpoint{2.354426in}{0.393291in}}%
\pgfusepath{stroke}%
\end{pgfscope}%
\begin{pgfscope}%
\pgftext[x=2.422307in,y=0.393832in,,top]{\sffamily\fontsize{10.000000}{12.000000}\selectfont \(\displaystyle 0\)}%
\end{pgfscope}%
\begin{pgfscope}%
\pgfsetrectcap%
\pgfsetroundjoin%
\pgfsetlinewidth{0.803000pt}%
\definecolor{currentstroke}{rgb}{0.000000,0.000000,0.000000}%
\pgfsetstrokecolor{currentstroke}%
\pgfsetdash{}{0pt}%
\pgfpathmoveto{\pgfqpoint{2.302118in}{0.733352in}}%
\pgfpathlineto{\pgfqpoint{2.353396in}{0.733352in}}%
\pgfusepath{stroke}%
\end{pgfscope}%
\begin{pgfscope}%
\pgftext[x=2.421218in,y=0.732201in,,top]{\sffamily\fontsize{10.000000}{12.000000}\selectfont \(\displaystyle 2\)}%
\end{pgfscope}%
\begin{pgfscope}%
\pgfsetrectcap%
\pgfsetroundjoin%
\pgfsetlinewidth{0.803000pt}%
\definecolor{currentstroke}{rgb}{0.000000,0.000000,0.000000}%
\pgfsetstrokecolor{currentstroke}%
\pgfsetdash{}{0pt}%
\pgfpathmoveto{\pgfqpoint{2.301137in}{1.072778in}}%
\pgfpathlineto{\pgfqpoint{2.352367in}{1.072778in}}%
\pgfusepath{stroke}%
\end{pgfscope}%
\begin{pgfscope}%
\pgftext[x=2.420132in,y=1.069941in,,top]{\sffamily\fontsize{10.000000}{12.000000}\selectfont \(\displaystyle 4\)}%
\end{pgfscope}%
\begin{pgfscope}%
\pgfsetrectcap%
\pgfsetroundjoin%
\pgfsetlinewidth{0.803000pt}%
\definecolor{currentstroke}{rgb}{0.000000,0.000000,0.000000}%
\pgfsetstrokecolor{currentstroke}%
\pgfsetdash{}{0pt}%
\pgfpathmoveto{\pgfqpoint{2.300158in}{1.411571in}}%
\pgfpathlineto{\pgfqpoint{2.351340in}{1.411571in}}%
\pgfusepath{stroke}%
\end{pgfscope}%
\begin{pgfscope}%
\pgftext[x=2.419048in,y=1.407056in,,top]{\sffamily\fontsize{10.000000}{12.000000}\selectfont \(\displaystyle 6\)}%
\end{pgfscope}%
\begin{pgfscope}%
\pgfsetrectcap%
\pgfsetroundjoin%
\pgfsetlinewidth{0.803000pt}%
\definecolor{currentstroke}{rgb}{0.000000,0.000000,0.000000}%
\pgfsetstrokecolor{currentstroke}%
\pgfsetdash{}{0pt}%
\pgfpathmoveto{\pgfqpoint{2.316811in}{1.550475in}}%
\pgfpathlineto{\pgfqpoint{2.433643in}{1.665415in}}%
\pgfusepath{stroke}%
\end{pgfscope}%
\begin{pgfscope}%
\pgftext[x=2.513410in,y=1.688047in,,]{\sffamily\fontsize{10.000000}{12.000000}\selectfont \(\displaystyle z\)}%
\end{pgfscope}%
\begin{pgfscope}%
\pgfsetbuttcap%
\pgfsetroundjoin%
\pgfsetlinewidth{0.803000pt}%
\definecolor{currentstroke}{rgb}{0.690196,0.690196,0.690196}%
\pgfsetstrokecolor{currentstroke}%
\pgfsetdash{}{0pt}%
\pgfpathmoveto{\pgfqpoint{2.324579in}{1.558117in}}%
\pgfpathlineto{\pgfqpoint{0.177259in}{1.558117in}}%
\pgfpathlineto{\pgfqpoint{0.173515in}{0.292445in}}%
\pgfusepath{stroke}%
\end{pgfscope}%
\begin{pgfscope}%
\pgfsetbuttcap%
\pgfsetroundjoin%
\pgfsetlinewidth{0.803000pt}%
\definecolor{currentstroke}{rgb}{0.690196,0.690196,0.690196}%
\pgfsetstrokecolor{currentstroke}%
\pgfsetdash{}{0pt}%
\pgfpathmoveto{\pgfqpoint{2.354227in}{1.587285in}}%
\pgfpathlineto{\pgfqpoint{0.147611in}{1.587285in}}%
\pgfpathlineto{\pgfqpoint{0.143658in}{0.286642in}}%
\pgfusepath{stroke}%
\end{pgfscope}%
\begin{pgfscope}%
\pgfsetbuttcap%
\pgfsetroundjoin%
\pgfsetlinewidth{0.803000pt}%
\definecolor{currentstroke}{rgb}{0.690196,0.690196,0.690196}%
\pgfsetstrokecolor{currentstroke}%
\pgfsetdash{}{0pt}%
\pgfpathmoveto{\pgfqpoint{2.385558in}{1.618109in}}%
\pgfpathlineto{\pgfqpoint{0.116280in}{1.618109in}}%
\pgfpathlineto{\pgfqpoint{0.112098in}{0.280508in}}%
\pgfusepath{stroke}%
\end{pgfscope}%
\begin{pgfscope}%
\pgfsetbuttcap%
\pgfsetroundjoin%
\pgfsetlinewidth{0.803000pt}%
\definecolor{currentstroke}{rgb}{0.690196,0.690196,0.690196}%
\pgfsetstrokecolor{currentstroke}%
\pgfsetdash{}{0pt}%
\pgfpathmoveto{\pgfqpoint{2.418721in}{1.650735in}}%
\pgfpathlineto{\pgfqpoint{0.083117in}{1.650735in}}%
\pgfpathlineto{\pgfqpoint{0.078686in}{0.274014in}}%
\pgfusepath{stroke}%
\end{pgfscope}%
\begin{pgfscope}%
\pgfsetrectcap%
\pgfsetroundjoin%
\pgfsetlinewidth{0.803000pt}%
\definecolor{currentstroke}{rgb}{0.000000,0.000000,0.000000}%
\pgfsetstrokecolor{currentstroke}%
\pgfsetdash{}{0pt}%
\pgfpathmoveto{\pgfqpoint{2.307400in}{1.558117in}}%
\pgfpathlineto{\pgfqpoint{2.358936in}{1.558117in}}%
\pgfusepath{stroke}%
\end{pgfscope}%
\begin{pgfscope}%
\pgfsetrectcap%
\pgfsetroundjoin%
\pgfsetlinewidth{0.803000pt}%
\definecolor{currentstroke}{rgb}{0.000000,0.000000,0.000000}%
\pgfsetstrokecolor{currentstroke}%
\pgfsetdash{}{0pt}%
\pgfpathmoveto{\pgfqpoint{2.336574in}{1.587285in}}%
\pgfpathlineto{\pgfqpoint{2.389532in}{1.587285in}}%
\pgfusepath{stroke}%
\end{pgfscope}%
\begin{pgfscope}%
\pgfsetrectcap%
\pgfsetroundjoin%
\pgfsetlinewidth{0.803000pt}%
\definecolor{currentstroke}{rgb}{0.000000,0.000000,0.000000}%
\pgfsetstrokecolor{currentstroke}%
\pgfsetdash{}{0pt}%
\pgfpathmoveto{\pgfqpoint{2.367404in}{1.618109in}}%
\pgfpathlineto{\pgfqpoint{2.421867in}{1.618109in}}%
\pgfusepath{stroke}%
\end{pgfscope}%
\begin{pgfscope}%
\pgfsetrectcap%
\pgfsetroundjoin%
\pgfsetlinewidth{0.803000pt}%
\definecolor{currentstroke}{rgb}{0.000000,0.000000,0.000000}%
\pgfsetstrokecolor{currentstroke}%
\pgfsetdash{}{0pt}%
\pgfpathmoveto{\pgfqpoint{2.400036in}{1.650735in}}%
\pgfpathlineto{\pgfqpoint{2.456091in}{1.650735in}}%
\pgfusepath{stroke}%
\end{pgfscope}%
\begin{pgfscope}%
\pgfsys@transformshift{0.284286in}{0.362857in}%
\pgftext[left,bottom]{\pgfimage[interpolate=true,width=1.978571in,height=1.170000in]{figures/mpl-figs/dep-lcss-view1-small-img0.png}}%
\end{pgfscope}%
\end{pgfpicture}%
\makeatother%
\endgroup%

        \caption[]{{\small View along the negative $z$-axis}}
        \label{fig:u0_dom_err_bs32}
    \end{subfigure}
    \begin{subfigure}[b]{0.475\textwidth}
        \centering
        %% Creator: Matplotlib, PGF backend
%%
%% To include the figure in your LaTeX document, write
%%   \input{<filename>.pgf}
%%
%% Make sure the required packages are loaded in your preamble
%%   \usepackage{pgf}
%%
%% Figures using additional raster images can only be included by \input if
%% they are in the same directory as the main LaTeX file. For loading figures
%% from other directories you can use the `import` package
%%   \usepackage{import}
%% and then include the figures with
%%   \import{<path to file>}{<filename>.pgf}
%%
%% Matplotlib used the following preamble
%%   \usepackage{fontspec}
%%   \setmainfont{DejaVu Serif}
%%   \setsansfont{DejaVu Sans}
%%   \setmonofont{DejaVu Sans Mono}
%%
\begingroup%
\makeatletter%
\begin{pgfpicture}%
\pgfpathrectangle{\pgfpointorigin}{\pgfqpoint{2.660000in}{1.740000in}}%
\pgfusepath{use as bounding box, clip}%
\begin{pgfscope}%
\pgfsetbuttcap%
\pgfsetmiterjoin%
\definecolor{currentfill}{rgb}{1.000000,1.000000,1.000000}%
\pgfsetfillcolor{currentfill}%
\pgfsetlinewidth{0.000000pt}%
\definecolor{currentstroke}{rgb}{1.000000,1.000000,1.000000}%
\pgfsetstrokecolor{currentstroke}%
\pgfsetdash{}{0pt}%
\pgfpathmoveto{\pgfqpoint{0.000000in}{0.000000in}}%
\pgfpathlineto{\pgfqpoint{2.660000in}{0.000000in}}%
\pgfpathlineto{\pgfqpoint{2.660000in}{1.740000in}}%
\pgfpathlineto{\pgfqpoint{0.000000in}{1.740000in}}%
\pgfpathclose%
\pgfusepath{fill}%
\end{pgfscope}%
\begin{pgfscope}%
\pgfsetbuttcap%
\pgfsetmiterjoin%
\definecolor{currentfill}{rgb}{1.000000,1.000000,1.000000}%
\pgfsetfillcolor{currentfill}%
\pgfsetlinewidth{0.000000pt}%
\definecolor{currentstroke}{rgb}{0.000000,0.000000,0.000000}%
\pgfsetstrokecolor{currentstroke}%
\pgfsetstrokeopacity{0.000000}%
\pgfsetdash{}{0pt}%
\pgfpathmoveto{\pgfqpoint{-0.798000in}{-0.174000in}}%
\pgfpathlineto{\pgfqpoint{3.192000in}{-0.174000in}}%
\pgfpathlineto{\pgfqpoint{3.192000in}{2.175000in}}%
\pgfpathlineto{\pgfqpoint{-0.798000in}{2.175000in}}%
\pgfpathclose%
\pgfusepath{fill}%
\end{pgfscope}%
\begin{pgfscope}%
\pgfsetbuttcap%
\pgfsetmiterjoin%
\pgfsetlinewidth{0.000000pt}%
\definecolor{currentstroke}{rgb}{1.000000,1.000000,1.000000}%
\pgfsetstrokecolor{currentstroke}%
\pgfsetstrokeopacity{0.000000}%
\pgfsetdash{}{0pt}%
\pgfpathmoveto{\pgfqpoint{0.065888in}{0.334590in}}%
\pgfpathlineto{\pgfqpoint{0.183218in}{0.403664in}}%
\pgfpathlineto{\pgfqpoint{0.183218in}{1.660822in}}%
\pgfpathlineto{\pgfqpoint{0.065888in}{1.729897in}}%
\pgfusepath{}%
\end{pgfscope}%
\begin{pgfscope}%
\pgfsetbuttcap%
\pgfsetmiterjoin%
\pgfsetlinewidth{0.000000pt}%
\definecolor{currentstroke}{rgb}{1.000000,1.000000,1.000000}%
\pgfsetstrokecolor{currentstroke}%
\pgfsetstrokeopacity{0.000000}%
\pgfsetdash{}{0pt}%
\pgfpathmoveto{\pgfqpoint{0.183218in}{0.403664in}}%
\pgfpathlineto{\pgfqpoint{2.318620in}{0.403664in}}%
\pgfpathlineto{\pgfqpoint{2.318620in}{1.660822in}}%
\pgfpathlineto{\pgfqpoint{0.183218in}{1.660822in}}%
\pgfusepath{}%
\end{pgfscope}%
\begin{pgfscope}%
\pgfsetbuttcap%
\pgfsetmiterjoin%
\pgfsetlinewidth{0.000000pt}%
\definecolor{currentstroke}{rgb}{1.000000,1.000000,1.000000}%
\pgfsetstrokecolor{currentstroke}%
\pgfsetstrokeopacity{0.000000}%
\pgfsetdash{}{0pt}%
\pgfpathmoveto{\pgfqpoint{0.065888in}{0.334590in}}%
\pgfpathlineto{\pgfqpoint{2.435950in}{0.334590in}}%
\pgfpathlineto{\pgfqpoint{2.318620in}{0.403664in}}%
\pgfpathlineto{\pgfqpoint{0.183218in}{0.403664in}}%
\pgfusepath{}%
\end{pgfscope}%
\begin{pgfscope}%
\pgfsetrectcap%
\pgfsetroundjoin%
\pgfsetlinewidth{0.803000pt}%
\definecolor{currentstroke}{rgb}{0.000000,0.000000,0.000000}%
\pgfsetstrokecolor{currentstroke}%
\pgfsetdash{}{0pt}%
\pgfpathmoveto{\pgfqpoint{0.065888in}{0.334590in}}%
\pgfpathlineto{\pgfqpoint{2.435950in}{0.334590in}}%
\pgfusepath{stroke}%
\end{pgfscope}%
\begin{pgfscope}%
\pgftext[x=1.250919in,y=0.095077in,,]{\sffamily\fontsize{10.000000}{12.000000}\selectfont \(\displaystyle x\)}%
\end{pgfscope}%
\begin{pgfscope}%
\pgfsetbuttcap%
\pgfsetroundjoin%
\pgfsetlinewidth{0.803000pt}%
\definecolor{currentstroke}{rgb}{0.690196,0.690196,0.690196}%
\pgfsetstrokecolor{currentstroke}%
\pgfsetdash{}{0pt}%
\pgfpathmoveto{\pgfqpoint{0.239475in}{0.334590in}}%
\pgfpathlineto{\pgfqpoint{0.339618in}{0.403664in}}%
\pgfpathlineto{\pgfqpoint{0.339618in}{1.660822in}}%
\pgfusepath{stroke}%
\end{pgfscope}%
\begin{pgfscope}%
\pgfsetbuttcap%
\pgfsetroundjoin%
\pgfsetlinewidth{0.803000pt}%
\definecolor{currentstroke}{rgb}{0.690196,0.690196,0.690196}%
\pgfsetstrokecolor{currentstroke}%
\pgfsetdash{}{0pt}%
\pgfpathmoveto{\pgfqpoint{0.879540in}{0.334590in}}%
\pgfpathlineto{\pgfqpoint{0.916310in}{0.403664in}}%
\pgfpathlineto{\pgfqpoint{0.916310in}{1.660822in}}%
\pgfusepath{stroke}%
\end{pgfscope}%
\begin{pgfscope}%
\pgfsetbuttcap%
\pgfsetroundjoin%
\pgfsetlinewidth{0.803000pt}%
\definecolor{currentstroke}{rgb}{0.690196,0.690196,0.690196}%
\pgfsetstrokecolor{currentstroke}%
\pgfsetdash{}{0pt}%
\pgfpathmoveto{\pgfqpoint{1.519604in}{0.334590in}}%
\pgfpathlineto{\pgfqpoint{1.493002in}{0.403664in}}%
\pgfpathlineto{\pgfqpoint{1.493002in}{1.660822in}}%
\pgfusepath{stroke}%
\end{pgfscope}%
\begin{pgfscope}%
\pgfsetbuttcap%
\pgfsetroundjoin%
\pgfsetlinewidth{0.803000pt}%
\definecolor{currentstroke}{rgb}{0.690196,0.690196,0.690196}%
\pgfsetstrokecolor{currentstroke}%
\pgfsetdash{}{0pt}%
\pgfpathmoveto{\pgfqpoint{2.159669in}{0.334590in}}%
\pgfpathlineto{\pgfqpoint{2.069694in}{0.403664in}}%
\pgfpathlineto{\pgfqpoint{2.069694in}{1.660822in}}%
\pgfusepath{stroke}%
\end{pgfscope}%
\begin{pgfscope}%
\pgfsetrectcap%
\pgfsetroundjoin%
\pgfsetlinewidth{0.803000pt}%
\definecolor{currentstroke}{rgb}{0.000000,0.000000,0.000000}%
\pgfsetstrokecolor{currentstroke}%
\pgfsetdash{}{0pt}%
\pgfpathmoveto{\pgfqpoint{0.240363in}{0.335202in}}%
\pgfpathlineto{\pgfqpoint{0.237694in}{0.333361in}}%
\pgfusepath{stroke}%
\end{pgfscope}%
\begin{pgfscope}%
\pgftext[x=0.235721in,y=0.284868in,,top]{\sffamily\fontsize{10.000000}{12.000000}\selectfont \(\displaystyle 0\)}%
\end{pgfscope}%
\begin{pgfscope}%
\pgfsetrectcap%
\pgfsetroundjoin%
\pgfsetlinewidth{0.803000pt}%
\definecolor{currentstroke}{rgb}{0.000000,0.000000,0.000000}%
\pgfsetstrokecolor{currentstroke}%
\pgfsetdash{}{0pt}%
\pgfpathmoveto{\pgfqpoint{0.879866in}{0.335202in}}%
\pgfpathlineto{\pgfqpoint{0.878886in}{0.333361in}}%
\pgfusepath{stroke}%
\end{pgfscope}%
\begin{pgfscope}%
\pgftext[x=0.878161in,y=0.284868in,,top]{\sffamily\fontsize{10.000000}{12.000000}\selectfont \(\displaystyle 2\)}%
\end{pgfscope}%
\begin{pgfscope}%
\pgfsetrectcap%
\pgfsetroundjoin%
\pgfsetlinewidth{0.803000pt}%
\definecolor{currentstroke}{rgb}{0.000000,0.000000,0.000000}%
\pgfsetstrokecolor{currentstroke}%
\pgfsetdash{}{0pt}%
\pgfpathmoveto{\pgfqpoint{1.519368in}{0.335202in}}%
\pgfpathlineto{\pgfqpoint{1.520078in}{0.333361in}}%
\pgfusepath{stroke}%
\end{pgfscope}%
\begin{pgfscope}%
\pgftext[x=1.520602in,y=0.284868in,,top]{\sffamily\fontsize{10.000000}{12.000000}\selectfont \(\displaystyle 4\)}%
\end{pgfscope}%
\begin{pgfscope}%
\pgfsetrectcap%
\pgfsetroundjoin%
\pgfsetlinewidth{0.803000pt}%
\definecolor{currentstroke}{rgb}{0.000000,0.000000,0.000000}%
\pgfsetstrokecolor{currentstroke}%
\pgfsetdash{}{0pt}%
\pgfpathmoveto{\pgfqpoint{2.158871in}{0.335202in}}%
\pgfpathlineto{\pgfqpoint{2.161270in}{0.333361in}}%
\pgfusepath{stroke}%
\end{pgfscope}%
\begin{pgfscope}%
\pgftext[x=2.163042in,y=0.284868in,,top]{\sffamily\fontsize{10.000000}{12.000000}\selectfont \(\displaystyle 6\)}%
\end{pgfscope}%
\begin{pgfscope}%
\pgfsetrectcap%
\pgfsetroundjoin%
\pgfsetlinewidth{0.803000pt}%
\definecolor{currentstroke}{rgb}{0.000000,0.000000,0.000000}%
\pgfsetstrokecolor{currentstroke}%
\pgfsetdash{}{0pt}%
\pgfpathmoveto{\pgfqpoint{2.318620in}{0.403664in}}%
\pgfpathlineto{\pgfqpoint{2.435950in}{0.334590in}}%
\pgfusepath{stroke}%
\end{pgfscope}%
\begin{pgfscope}%
\pgftext[x=2.544348in,y=0.270773in,,]{\sffamily\fontsize{10.000000}{12.000000}\selectfont \(\displaystyle y\)}%
\end{pgfscope}%
\begin{pgfscope}%
\pgfsetbuttcap%
\pgfsetroundjoin%
\pgfsetlinewidth{0.803000pt}%
\definecolor{currentstroke}{rgb}{0.690196,0.690196,0.690196}%
\pgfsetstrokecolor{currentstroke}%
\pgfsetdash{}{0pt}%
\pgfpathmoveto{\pgfqpoint{0.076061in}{1.723908in}}%
\pgfpathlineto{\pgfqpoint{0.076061in}{0.340579in}}%
\pgfpathlineto{\pgfqpoint{2.425777in}{0.340579in}}%
\pgfusepath{stroke}%
\end{pgfscope}%
\begin{pgfscope}%
\pgfsetbuttcap%
\pgfsetroundjoin%
\pgfsetlinewidth{0.803000pt}%
\definecolor{currentstroke}{rgb}{0.690196,0.690196,0.690196}%
\pgfsetstrokecolor{currentstroke}%
\pgfsetdash{}{0pt}%
\pgfpathmoveto{\pgfqpoint{0.093097in}{1.713878in}}%
\pgfpathlineto{\pgfqpoint{0.093097in}{0.350608in}}%
\pgfpathlineto{\pgfqpoint{2.408741in}{0.350608in}}%
\pgfusepath{stroke}%
\end{pgfscope}%
\begin{pgfscope}%
\pgfsetbuttcap%
\pgfsetroundjoin%
\pgfsetlinewidth{0.803000pt}%
\definecolor{currentstroke}{rgb}{0.690196,0.690196,0.690196}%
\pgfsetstrokecolor{currentstroke}%
\pgfsetdash{}{0pt}%
\pgfpathmoveto{\pgfqpoint{0.109645in}{1.704136in}}%
\pgfpathlineto{\pgfqpoint{0.109645in}{0.360351in}}%
\pgfpathlineto{\pgfqpoint{2.392193in}{0.360351in}}%
\pgfusepath{stroke}%
\end{pgfscope}%
\begin{pgfscope}%
\pgfsetbuttcap%
\pgfsetroundjoin%
\pgfsetlinewidth{0.803000pt}%
\definecolor{currentstroke}{rgb}{0.690196,0.690196,0.690196}%
\pgfsetstrokecolor{currentstroke}%
\pgfsetdash{}{0pt}%
\pgfpathmoveto{\pgfqpoint{0.125728in}{1.694668in}}%
\pgfpathlineto{\pgfqpoint{0.125728in}{0.369819in}}%
\pgfpathlineto{\pgfqpoint{2.376110in}{0.369819in}}%
\pgfusepath{stroke}%
\end{pgfscope}%
\begin{pgfscope}%
\pgfsetbuttcap%
\pgfsetroundjoin%
\pgfsetlinewidth{0.803000pt}%
\definecolor{currentstroke}{rgb}{0.690196,0.690196,0.690196}%
\pgfsetstrokecolor{currentstroke}%
\pgfsetdash{}{0pt}%
\pgfpathmoveto{\pgfqpoint{0.141363in}{1.685463in}}%
\pgfpathlineto{\pgfqpoint{0.141363in}{0.379023in}}%
\pgfpathlineto{\pgfqpoint{2.360475in}{0.379023in}}%
\pgfusepath{stroke}%
\end{pgfscope}%
\begin{pgfscope}%
\pgfsetbuttcap%
\pgfsetroundjoin%
\pgfsetlinewidth{0.803000pt}%
\definecolor{currentstroke}{rgb}{0.690196,0.690196,0.690196}%
\pgfsetstrokecolor{currentstroke}%
\pgfsetdash{}{0pt}%
\pgfpathmoveto{\pgfqpoint{0.156570in}{1.676511in}}%
\pgfpathlineto{\pgfqpoint{0.156570in}{0.387976in}}%
\pgfpathlineto{\pgfqpoint{2.345268in}{0.387976in}}%
\pgfusepath{stroke}%
\end{pgfscope}%
\begin{pgfscope}%
\pgfsetbuttcap%
\pgfsetroundjoin%
\pgfsetlinewidth{0.803000pt}%
\definecolor{currentstroke}{rgb}{0.690196,0.690196,0.690196}%
\pgfsetstrokecolor{currentstroke}%
\pgfsetdash{}{0pt}%
\pgfpathmoveto{\pgfqpoint{0.171365in}{1.667800in}}%
\pgfpathlineto{\pgfqpoint{0.171365in}{0.396686in}}%
\pgfpathlineto{\pgfqpoint{2.330473in}{0.396686in}}%
\pgfusepath{stroke}%
\end{pgfscope}%
\begin{pgfscope}%
\pgfsetrectcap%
\pgfsetroundjoin%
\pgfsetlinewidth{0.803000pt}%
\definecolor{currentstroke}{rgb}{0.000000,0.000000,0.000000}%
\pgfsetstrokecolor{currentstroke}%
\pgfsetdash{}{0pt}%
\pgfpathmoveto{\pgfqpoint{2.406979in}{0.340579in}}%
\pgfpathlineto{\pgfqpoint{2.463372in}{0.340579in}}%
\pgfusepath{stroke}%
\end{pgfscope}%
\begin{pgfscope}%
\pgfsetrectcap%
\pgfsetroundjoin%
\pgfsetlinewidth{0.803000pt}%
\definecolor{currentstroke}{rgb}{0.000000,0.000000,0.000000}%
\pgfsetstrokecolor{currentstroke}%
\pgfsetdash{}{0pt}%
\pgfpathmoveto{\pgfqpoint{2.390216in}{0.350608in}}%
\pgfpathlineto{\pgfqpoint{2.445792in}{0.350608in}}%
\pgfusepath{stroke}%
\end{pgfscope}%
\begin{pgfscope}%
\pgfsetrectcap%
\pgfsetroundjoin%
\pgfsetlinewidth{0.803000pt}%
\definecolor{currentstroke}{rgb}{0.000000,0.000000,0.000000}%
\pgfsetstrokecolor{currentstroke}%
\pgfsetdash{}{0pt}%
\pgfpathmoveto{\pgfqpoint{2.373932in}{0.360351in}}%
\pgfpathlineto{\pgfqpoint{2.428713in}{0.360351in}}%
\pgfusepath{stroke}%
\end{pgfscope}%
\begin{pgfscope}%
\pgfsetrectcap%
\pgfsetroundjoin%
\pgfsetlinewidth{0.803000pt}%
\definecolor{currentstroke}{rgb}{0.000000,0.000000,0.000000}%
\pgfsetstrokecolor{currentstroke}%
\pgfsetdash{}{0pt}%
\pgfpathmoveto{\pgfqpoint{2.358107in}{0.369819in}}%
\pgfpathlineto{\pgfqpoint{2.412116in}{0.369819in}}%
\pgfusepath{stroke}%
\end{pgfscope}%
\begin{pgfscope}%
\pgfsetrectcap%
\pgfsetroundjoin%
\pgfsetlinewidth{0.803000pt}%
\definecolor{currentstroke}{rgb}{0.000000,0.000000,0.000000}%
\pgfsetstrokecolor{currentstroke}%
\pgfsetdash{}{0pt}%
\pgfpathmoveto{\pgfqpoint{2.342722in}{0.379023in}}%
\pgfpathlineto{\pgfqpoint{2.395981in}{0.379023in}}%
\pgfusepath{stroke}%
\end{pgfscope}%
\begin{pgfscope}%
\pgfsetrectcap%
\pgfsetroundjoin%
\pgfsetlinewidth{0.803000pt}%
\definecolor{currentstroke}{rgb}{0.000000,0.000000,0.000000}%
\pgfsetstrokecolor{currentstroke}%
\pgfsetdash{}{0pt}%
\pgfpathmoveto{\pgfqpoint{2.327759in}{0.387976in}}%
\pgfpathlineto{\pgfqpoint{2.380287in}{0.387976in}}%
\pgfusepath{stroke}%
\end{pgfscope}%
\begin{pgfscope}%
\pgfsetrectcap%
\pgfsetroundjoin%
\pgfsetlinewidth{0.803000pt}%
\definecolor{currentstroke}{rgb}{0.000000,0.000000,0.000000}%
\pgfsetstrokecolor{currentstroke}%
\pgfsetdash{}{0pt}%
\pgfpathmoveto{\pgfqpoint{2.313200in}{0.396686in}}%
\pgfpathlineto{\pgfqpoint{2.365018in}{0.396686in}}%
\pgfusepath{stroke}%
\end{pgfscope}%
\begin{pgfscope}%
\pgfsetrectcap%
\pgfsetroundjoin%
\pgfsetlinewidth{0.803000pt}%
\definecolor{currentstroke}{rgb}{0.000000,0.000000,0.000000}%
\pgfsetstrokecolor{currentstroke}%
\pgfsetdash{}{0pt}%
\pgfpathmoveto{\pgfqpoint{2.318620in}{0.403664in}}%
\pgfpathlineto{\pgfqpoint{2.318620in}{1.660822in}}%
\pgfusepath{stroke}%
\end{pgfscope}%
\begin{pgfscope}%
\pgftext[x=2.588256in,y=1.032243in,,]{\sffamily\fontsize{10.000000}{12.000000}\selectfont \(\displaystyle z\)}%
\end{pgfscope}%
\begin{pgfscope}%
\pgfsetbuttcap%
\pgfsetroundjoin%
\pgfsetlinewidth{0.803000pt}%
\definecolor{currentstroke}{rgb}{0.690196,0.690196,0.690196}%
\pgfsetstrokecolor{currentstroke}%
\pgfsetdash{}{0pt}%
\pgfpathmoveto{\pgfqpoint{2.318620in}{0.495741in}}%
\pgfpathlineto{\pgfqpoint{0.183218in}{0.495741in}}%
\pgfpathlineto{\pgfqpoint{0.065888in}{0.436784in}}%
\pgfusepath{stroke}%
\end{pgfscope}%
\begin{pgfscope}%
\pgfsetbuttcap%
\pgfsetroundjoin%
\pgfsetlinewidth{0.803000pt}%
\definecolor{currentstroke}{rgb}{0.690196,0.690196,0.690196}%
\pgfsetstrokecolor{currentstroke}%
\pgfsetdash{}{0pt}%
\pgfpathmoveto{\pgfqpoint{2.318620in}{0.835252in}}%
\pgfpathlineto{\pgfqpoint{0.183218in}{0.835252in}}%
\pgfpathlineto{\pgfqpoint{0.065888in}{0.813604in}}%
\pgfusepath{stroke}%
\end{pgfscope}%
\begin{pgfscope}%
\pgfsetbuttcap%
\pgfsetroundjoin%
\pgfsetlinewidth{0.803000pt}%
\definecolor{currentstroke}{rgb}{0.690196,0.690196,0.690196}%
\pgfsetstrokecolor{currentstroke}%
\pgfsetdash{}{0pt}%
\pgfpathmoveto{\pgfqpoint{2.318620in}{1.174763in}}%
\pgfpathlineto{\pgfqpoint{0.183218in}{1.174763in}}%
\pgfpathlineto{\pgfqpoint{0.065888in}{1.190424in}}%
\pgfusepath{stroke}%
\end{pgfscope}%
\begin{pgfscope}%
\pgfsetbuttcap%
\pgfsetroundjoin%
\pgfsetlinewidth{0.803000pt}%
\definecolor{currentstroke}{rgb}{0.690196,0.690196,0.690196}%
\pgfsetstrokecolor{currentstroke}%
\pgfsetdash{}{0pt}%
\pgfpathmoveto{\pgfqpoint{2.318620in}{1.514274in}}%
\pgfpathlineto{\pgfqpoint{0.183218in}{1.514274in}}%
\pgfpathlineto{\pgfqpoint{0.065888in}{1.567244in}}%
\pgfusepath{stroke}%
\end{pgfscope}%
\begin{pgfscope}%
\pgfsetrectcap%
\pgfsetroundjoin%
\pgfsetlinewidth{0.803000pt}%
\definecolor{currentstroke}{rgb}{0.000000,0.000000,0.000000}%
\pgfsetstrokecolor{currentstroke}%
\pgfsetdash{}{0pt}%
\pgfpathmoveto{\pgfqpoint{2.301537in}{0.495741in}}%
\pgfpathlineto{\pgfqpoint{2.352787in}{0.495741in}}%
\pgfusepath{stroke}%
\end{pgfscope}%
\begin{pgfscope}%
\pgftext[x=2.454331in,y=0.499292in,,top]{\sffamily\fontsize{10.000000}{12.000000}\selectfont \(\displaystyle 0\)}%
\end{pgfscope}%
\begin{pgfscope}%
\pgfsetrectcap%
\pgfsetroundjoin%
\pgfsetlinewidth{0.803000pt}%
\definecolor{currentstroke}{rgb}{0.000000,0.000000,0.000000}%
\pgfsetstrokecolor{currentstroke}%
\pgfsetdash{}{0pt}%
\pgfpathmoveto{\pgfqpoint{2.301537in}{0.835252in}}%
\pgfpathlineto{\pgfqpoint{2.352787in}{0.835252in}}%
\pgfusepath{stroke}%
\end{pgfscope}%
\begin{pgfscope}%
\pgftext[x=2.454331in,y=0.836556in,,top]{\sffamily\fontsize{10.000000}{12.000000}\selectfont \(\displaystyle 2\)}%
\end{pgfscope}%
\begin{pgfscope}%
\pgfsetrectcap%
\pgfsetroundjoin%
\pgfsetlinewidth{0.803000pt}%
\definecolor{currentstroke}{rgb}{0.000000,0.000000,0.000000}%
\pgfsetstrokecolor{currentstroke}%
\pgfsetdash{}{0pt}%
\pgfpathmoveto{\pgfqpoint{2.301537in}{1.174763in}}%
\pgfpathlineto{\pgfqpoint{2.352787in}{1.174763in}}%
\pgfusepath{stroke}%
\end{pgfscope}%
\begin{pgfscope}%
\pgftext[x=2.454331in,y=1.173819in,,top]{\sffamily\fontsize{10.000000}{12.000000}\selectfont \(\displaystyle 4\)}%
\end{pgfscope}%
\begin{pgfscope}%
\pgfsetrectcap%
\pgfsetroundjoin%
\pgfsetlinewidth{0.803000pt}%
\definecolor{currentstroke}{rgb}{0.000000,0.000000,0.000000}%
\pgfsetstrokecolor{currentstroke}%
\pgfsetdash{}{0pt}%
\pgfpathmoveto{\pgfqpoint{2.301537in}{1.514274in}}%
\pgfpathlineto{\pgfqpoint{2.352787in}{1.514274in}}%
\pgfusepath{stroke}%
\end{pgfscope}%
\begin{pgfscope}%
\pgftext[x=2.454331in,y=1.511083in,,top]{\sffamily\fontsize{10.000000}{12.000000}\selectfont \(\displaystyle 6\)}%
\end{pgfscope}%
\begin{pgfscope}%
\pgfsys@transformshift{0.235714in}{0.451429in}%
\pgftext[left,bottom]{\pgfimage[interpolate=true,width=2.012857in,height=1.162857in]{dep-lcss-view2-small-img0.png}}%
\end{pgfscope}%
\end{pgfpicture}%
\makeatother%
\endgroup%

        \caption[]{{\small View along the positive $y$-axis}}
        \label{fig:u0_dom_err_bs54}
    \end{subfigure}

    \begin{subfigure}[b]{0.475\textwidth}
        \centering
        %% Creator: Matplotlib, PGF backend
%%
%% To include the figure in your LaTeX document, write
%%   \input{<filename>.pgf}
%%
%% Make sure the required packages are loaded in your preamble
%%   \usepackage{pgf}
%%
%% Figures using additional raster images can only be included by \input if
%% they are in the same directory as the main LaTeX file. For loading figures
%% from other directories you can use the `import` package
%%   \usepackage{import}
%% and then include the figures with
%%   \import{<path to file>}{<filename>.pgf}
%%
%% Matplotlib used the following preamble
%%   \usepackage{fontspec}
%%   \setmainfont{DejaVu Serif}
%%   \setsansfont{DejaVu Sans}
%%   \setmonofont{DejaVu Sans Mono}
%%
\begingroup%
\makeatletter%
\begin{pgfpicture}%
\pgfpathrectangle{\pgfpointorigin}{\pgfqpoint{2.660000in}{1.740000in}}%
\pgfusepath{use as bounding box, clip}%
\begin{pgfscope}%
\pgfsetbuttcap%
\pgfsetmiterjoin%
\definecolor{currentfill}{rgb}{1.000000,1.000000,1.000000}%
\pgfsetfillcolor{currentfill}%
\pgfsetlinewidth{0.000000pt}%
\definecolor{currentstroke}{rgb}{1.000000,1.000000,1.000000}%
\pgfsetstrokecolor{currentstroke}%
\pgfsetdash{}{0pt}%
\pgfpathmoveto{\pgfqpoint{0.000000in}{0.000000in}}%
\pgfpathlineto{\pgfqpoint{2.660000in}{0.000000in}}%
\pgfpathlineto{\pgfqpoint{2.660000in}{1.740000in}}%
\pgfpathlineto{\pgfqpoint{0.000000in}{1.740000in}}%
\pgfpathclose%
\pgfusepath{fill}%
\end{pgfscope}%
\begin{pgfscope}%
\pgfsetbuttcap%
\pgfsetmiterjoin%
\definecolor{currentfill}{rgb}{1.000000,1.000000,1.000000}%
\pgfsetfillcolor{currentfill}%
\pgfsetlinewidth{0.000000pt}%
\definecolor{currentstroke}{rgb}{0.000000,0.000000,0.000000}%
\pgfsetstrokecolor{currentstroke}%
\pgfsetstrokeopacity{0.000000}%
\pgfsetdash{}{0pt}%
\pgfpathmoveto{\pgfqpoint{-0.585200in}{-0.174000in}}%
\pgfpathlineto{\pgfqpoint{3.059000in}{-0.174000in}}%
\pgfpathlineto{\pgfqpoint{3.059000in}{2.175000in}}%
\pgfpathlineto{\pgfqpoint{-0.585200in}{2.175000in}}%
\pgfpathclose%
\pgfusepath{fill}%
\end{pgfscope}%
\begin{pgfscope}%
\pgfsetbuttcap%
\pgfsetmiterjoin%
\pgfsetlinewidth{0.000000pt}%
\definecolor{currentstroke}{rgb}{1.000000,1.000000,1.000000}%
\pgfsetstrokecolor{currentstroke}%
\pgfsetstrokeopacity{0.000000}%
\pgfsetdash{}{0pt}%
\pgfpathmoveto{\pgfqpoint{2.259661in}{0.427012in}}%
\pgfpathlineto{\pgfqpoint{0.312631in}{0.427012in}}%
\pgfpathlineto{\pgfqpoint{0.309261in}{1.683521in}}%
\pgfpathlineto{\pgfqpoint{2.263030in}{1.683521in}}%
\pgfusepath{}%
\end{pgfscope}%
\begin{pgfscope}%
\pgfsetbuttcap%
\pgfsetmiterjoin%
\pgfsetlinewidth{0.000000pt}%
\definecolor{currentstroke}{rgb}{1.000000,1.000000,1.000000}%
\pgfsetstrokecolor{currentstroke}%
\pgfsetstrokeopacity{0.000000}%
\pgfsetdash{}{0pt}%
\pgfpathmoveto{\pgfqpoint{2.366367in}{0.312072in}}%
\pgfpathlineto{\pgfqpoint{2.259661in}{0.427012in}}%
\pgfpathlineto{\pgfqpoint{2.263030in}{1.683521in}}%
\pgfpathlineto{\pgfqpoint{2.370517in}{1.706394in}}%
\pgfusepath{}%
\end{pgfscope}%
\begin{pgfscope}%
\pgfsetbuttcap%
\pgfsetmiterjoin%
\pgfsetlinewidth{0.000000pt}%
\definecolor{currentstroke}{rgb}{1.000000,1.000000,1.000000}%
\pgfsetstrokecolor{currentstroke}%
\pgfsetstrokeopacity{0.000000}%
\pgfsetdash{}{0pt}%
\pgfpathmoveto{\pgfqpoint{2.366367in}{0.312072in}}%
\pgfpathlineto{\pgfqpoint{2.259661in}{0.427012in}}%
\pgfpathlineto{\pgfqpoint{0.312631in}{0.427012in}}%
\pgfpathlineto{\pgfqpoint{0.205925in}{0.312072in}}%
\pgfusepath{}%
\end{pgfscope}%
\begin{pgfscope}%
\pgfsetrectcap%
\pgfsetroundjoin%
\pgfsetlinewidth{0.803000pt}%
\definecolor{currentstroke}{rgb}{0.000000,0.000000,0.000000}%
\pgfsetstrokecolor{currentstroke}%
\pgfsetdash{}{0pt}%
\pgfpathmoveto{\pgfqpoint{0.312631in}{0.427012in}}%
\pgfpathlineto{\pgfqpoint{0.205925in}{0.312072in}}%
\pgfusepath{stroke}%
\end{pgfscope}%
\begin{pgfscope}%
\pgftext[x=0.098354in,y=0.267076in,,]{\sffamily\fontsize{10.000000}{12.000000}\selectfont \(\displaystyle x\)}%
\end{pgfscope}%
\begin{pgfscope}%
\pgfsetbuttcap%
\pgfsetroundjoin%
\pgfsetlinewidth{0.803000pt}%
\definecolor{currentstroke}{rgb}{0.690196,0.690196,0.690196}%
\pgfsetstrokecolor{currentstroke}%
\pgfsetdash{}{0pt}%
\pgfpathmoveto{\pgfqpoint{0.214528in}{0.321339in}}%
\pgfpathlineto{\pgfqpoint{2.357764in}{0.321339in}}%
\pgfpathlineto{\pgfqpoint{2.361849in}{1.704550in}}%
\pgfusepath{stroke}%
\end{pgfscope}%
\begin{pgfscope}%
\pgfsetbuttcap%
\pgfsetroundjoin%
\pgfsetlinewidth{0.803000pt}%
\definecolor{currentstroke}{rgb}{0.690196,0.690196,0.690196}%
\pgfsetstrokecolor{currentstroke}%
\pgfsetdash{}{0pt}%
\pgfpathmoveto{\pgfqpoint{0.230035in}{0.338042in}}%
\pgfpathlineto{\pgfqpoint{2.342257in}{0.338042in}}%
\pgfpathlineto{\pgfqpoint{2.346224in}{1.701225in}}%
\pgfusepath{stroke}%
\end{pgfscope}%
\begin{pgfscope}%
\pgfsetbuttcap%
\pgfsetroundjoin%
\pgfsetlinewidth{0.803000pt}%
\definecolor{currentstroke}{rgb}{0.690196,0.690196,0.690196}%
\pgfsetstrokecolor{currentstroke}%
\pgfsetdash{}{0pt}%
\pgfpathmoveto{\pgfqpoint{0.245099in}{0.354268in}}%
\pgfpathlineto{\pgfqpoint{2.327193in}{0.354268in}}%
\pgfpathlineto{\pgfqpoint{2.331047in}{1.697995in}}%
\pgfusepath{stroke}%
\end{pgfscope}%
\begin{pgfscope}%
\pgfsetbuttcap%
\pgfsetroundjoin%
\pgfsetlinewidth{0.803000pt}%
\definecolor{currentstroke}{rgb}{0.690196,0.690196,0.690196}%
\pgfsetstrokecolor{currentstroke}%
\pgfsetdash{}{0pt}%
\pgfpathmoveto{\pgfqpoint{0.259739in}{0.370039in}}%
\pgfpathlineto{\pgfqpoint{2.312552in}{0.370039in}}%
\pgfpathlineto{\pgfqpoint{2.316299in}{1.694857in}}%
\pgfusepath{stroke}%
\end{pgfscope}%
\begin{pgfscope}%
\pgfsetbuttcap%
\pgfsetroundjoin%
\pgfsetlinewidth{0.803000pt}%
\definecolor{currentstroke}{rgb}{0.690196,0.690196,0.690196}%
\pgfsetstrokecolor{currentstroke}%
\pgfsetdash{}{0pt}%
\pgfpathmoveto{\pgfqpoint{0.273974in}{0.385372in}}%
\pgfpathlineto{\pgfqpoint{2.298318in}{0.385372in}}%
\pgfpathlineto{\pgfqpoint{2.301961in}{1.691806in}}%
\pgfusepath{stroke}%
\end{pgfscope}%
\begin{pgfscope}%
\pgfsetbuttcap%
\pgfsetroundjoin%
\pgfsetlinewidth{0.803000pt}%
\definecolor{currentstroke}{rgb}{0.690196,0.690196,0.690196}%
\pgfsetstrokecolor{currentstroke}%
\pgfsetdash{}{0pt}%
\pgfpathmoveto{\pgfqpoint{0.287819in}{0.400285in}}%
\pgfpathlineto{\pgfqpoint{2.284473in}{0.400285in}}%
\pgfpathlineto{\pgfqpoint{2.288017in}{1.688838in}}%
\pgfusepath{stroke}%
\end{pgfscope}%
\begin{pgfscope}%
\pgfsetbuttcap%
\pgfsetroundjoin%
\pgfsetlinewidth{0.803000pt}%
\definecolor{currentstroke}{rgb}{0.690196,0.690196,0.690196}%
\pgfsetstrokecolor{currentstroke}%
\pgfsetdash{}{0pt}%
\pgfpathmoveto{\pgfqpoint{0.301291in}{0.414796in}}%
\pgfpathlineto{\pgfqpoint{2.271001in}{0.414796in}}%
\pgfpathlineto{\pgfqpoint{2.274450in}{1.685951in}}%
\pgfusepath{stroke}%
\end{pgfscope}%
\begin{pgfscope}%
\pgfsetrectcap%
\pgfsetroundjoin%
\pgfsetlinewidth{0.803000pt}%
\definecolor{currentstroke}{rgb}{0.000000,0.000000,0.000000}%
\pgfsetstrokecolor{currentstroke}%
\pgfsetdash{}{0pt}%
\pgfpathmoveto{\pgfqpoint{0.231674in}{0.321339in}}%
\pgfpathlineto{\pgfqpoint{0.180236in}{0.321339in}}%
\pgfusepath{stroke}%
\end{pgfscope}%
\begin{pgfscope}%
\pgfsetrectcap%
\pgfsetroundjoin%
\pgfsetlinewidth{0.803000pt}%
\definecolor{currentstroke}{rgb}{0.000000,0.000000,0.000000}%
\pgfsetstrokecolor{currentstroke}%
\pgfsetdash{}{0pt}%
\pgfpathmoveto{\pgfqpoint{0.246932in}{0.338042in}}%
\pgfpathlineto{\pgfqpoint{0.196239in}{0.338042in}}%
\pgfusepath{stroke}%
\end{pgfscope}%
\begin{pgfscope}%
\pgfsetrectcap%
\pgfsetroundjoin%
\pgfsetlinewidth{0.803000pt}%
\definecolor{currentstroke}{rgb}{0.000000,0.000000,0.000000}%
\pgfsetstrokecolor{currentstroke}%
\pgfsetdash{}{0pt}%
\pgfpathmoveto{\pgfqpoint{0.261756in}{0.354268in}}%
\pgfpathlineto{\pgfqpoint{0.211785in}{0.354268in}}%
\pgfusepath{stroke}%
\end{pgfscope}%
\begin{pgfscope}%
\pgfsetrectcap%
\pgfsetroundjoin%
\pgfsetlinewidth{0.803000pt}%
\definecolor{currentstroke}{rgb}{0.000000,0.000000,0.000000}%
\pgfsetstrokecolor{currentstroke}%
\pgfsetdash{}{0pt}%
\pgfpathmoveto{\pgfqpoint{0.276162in}{0.370039in}}%
\pgfpathlineto{\pgfqpoint{0.226894in}{0.370039in}}%
\pgfusepath{stroke}%
\end{pgfscope}%
\begin{pgfscope}%
\pgfsetrectcap%
\pgfsetroundjoin%
\pgfsetlinewidth{0.803000pt}%
\definecolor{currentstroke}{rgb}{0.000000,0.000000,0.000000}%
\pgfsetstrokecolor{currentstroke}%
\pgfsetdash{}{0pt}%
\pgfpathmoveto{\pgfqpoint{0.290169in}{0.385372in}}%
\pgfpathlineto{\pgfqpoint{0.241584in}{0.385372in}}%
\pgfusepath{stroke}%
\end{pgfscope}%
\begin{pgfscope}%
\pgfsetrectcap%
\pgfsetroundjoin%
\pgfsetlinewidth{0.803000pt}%
\definecolor{currentstroke}{rgb}{0.000000,0.000000,0.000000}%
\pgfsetstrokecolor{currentstroke}%
\pgfsetdash{}{0pt}%
\pgfpathmoveto{\pgfqpoint{0.303792in}{0.400285in}}%
\pgfpathlineto{\pgfqpoint{0.255873in}{0.400285in}}%
\pgfusepath{stroke}%
\end{pgfscope}%
\begin{pgfscope}%
\pgfsetrectcap%
\pgfsetroundjoin%
\pgfsetlinewidth{0.803000pt}%
\definecolor{currentstroke}{rgb}{0.000000,0.000000,0.000000}%
\pgfsetstrokecolor{currentstroke}%
\pgfsetdash{}{0pt}%
\pgfpathmoveto{\pgfqpoint{0.317048in}{0.414796in}}%
\pgfpathlineto{\pgfqpoint{0.269775in}{0.414796in}}%
\pgfusepath{stroke}%
\end{pgfscope}%
\begin{pgfscope}%
\pgfsetrectcap%
\pgfsetroundjoin%
\pgfsetlinewidth{0.803000pt}%
\definecolor{currentstroke}{rgb}{0.000000,0.000000,0.000000}%
\pgfsetstrokecolor{currentstroke}%
\pgfsetdash{}{0pt}%
\pgfpathmoveto{\pgfqpoint{2.366367in}{0.312072in}}%
\pgfpathlineto{\pgfqpoint{0.205925in}{0.312072in}}%
\pgfusepath{stroke}%
\end{pgfscope}%
\begin{pgfscope}%
\pgftext[x=1.286146in,y=0.078917in,,]{\sffamily\fontsize{10.000000}{12.000000}\selectfont \(\displaystyle y\)}%
\end{pgfscope}%
\begin{pgfscope}%
\pgfsetbuttcap%
\pgfsetroundjoin%
\pgfsetlinewidth{0.803000pt}%
\definecolor{currentstroke}{rgb}{0.690196,0.690196,0.690196}%
\pgfsetstrokecolor{currentstroke}%
\pgfsetdash{}{0pt}%
\pgfpathmoveto{\pgfqpoint{2.109077in}{1.683521in}}%
\pgfpathlineto{\pgfqpoint{2.106239in}{0.427012in}}%
\pgfpathlineto{\pgfqpoint{2.196128in}{0.312072in}}%
\pgfusepath{stroke}%
\end{pgfscope}%
\begin{pgfscope}%
\pgfsetbuttcap%
\pgfsetroundjoin%
\pgfsetlinewidth{0.803000pt}%
\definecolor{currentstroke}{rgb}{0.690196,0.690196,0.690196}%
\pgfsetstrokecolor{currentstroke}%
\pgfsetdash{}{0pt}%
\pgfpathmoveto{\pgfqpoint{1.581355in}{1.683521in}}%
\pgfpathlineto{\pgfqpoint{1.580336in}{0.427012in}}%
\pgfpathlineto{\pgfqpoint{1.612582in}{0.312072in}}%
\pgfusepath{stroke}%
\end{pgfscope}%
\begin{pgfscope}%
\pgfsetbuttcap%
\pgfsetroundjoin%
\pgfsetlinewidth{0.803000pt}%
\definecolor{currentstroke}{rgb}{0.690196,0.690196,0.690196}%
\pgfsetstrokecolor{currentstroke}%
\pgfsetdash{}{0pt}%
\pgfpathmoveto{\pgfqpoint{1.053632in}{1.683521in}}%
\pgfpathlineto{\pgfqpoint{1.054434in}{0.427012in}}%
\pgfpathlineto{\pgfqpoint{1.029037in}{0.312072in}}%
\pgfusepath{stroke}%
\end{pgfscope}%
\begin{pgfscope}%
\pgfsetbuttcap%
\pgfsetroundjoin%
\pgfsetlinewidth{0.803000pt}%
\definecolor{currentstroke}{rgb}{0.690196,0.690196,0.690196}%
\pgfsetstrokecolor{currentstroke}%
\pgfsetdash{}{0pt}%
\pgfpathmoveto{\pgfqpoint{0.525910in}{1.683521in}}%
\pgfpathlineto{\pgfqpoint{0.528532in}{0.427012in}}%
\pgfpathlineto{\pgfqpoint{0.445491in}{0.312072in}}%
\pgfusepath{stroke}%
\end{pgfscope}%
\begin{pgfscope}%
\pgfsetrectcap%
\pgfsetroundjoin%
\pgfsetlinewidth{0.803000pt}%
\definecolor{currentstroke}{rgb}{0.000000,0.000000,0.000000}%
\pgfsetstrokecolor{currentstroke}%
\pgfsetdash{}{0pt}%
\pgfpathmoveto{\pgfqpoint{2.195331in}{0.313091in}}%
\pgfpathlineto{\pgfqpoint{2.197727in}{0.310028in}}%
\pgfusepath{stroke}%
\end{pgfscope}%
\begin{pgfscope}%
\pgftext[x=2.199568in,y=0.257886in,,top]{\sffamily\fontsize{10.000000}{12.000000}\selectfont \(\displaystyle 0\)}%
\end{pgfscope}%
\begin{pgfscope}%
\pgfsetrectcap%
\pgfsetroundjoin%
\pgfsetlinewidth{0.803000pt}%
\definecolor{currentstroke}{rgb}{0.000000,0.000000,0.000000}%
\pgfsetstrokecolor{currentstroke}%
\pgfsetdash{}{0pt}%
\pgfpathmoveto{\pgfqpoint{1.612296in}{0.313091in}}%
\pgfpathlineto{\pgfqpoint{1.613156in}{0.310028in}}%
\pgfusepath{stroke}%
\end{pgfscope}%
\begin{pgfscope}%
\pgftext[x=1.613816in,y=0.257886in,,top]{\sffamily\fontsize{10.000000}{12.000000}\selectfont \(\displaystyle 2\)}%
\end{pgfscope}%
\begin{pgfscope}%
\pgfsetrectcap%
\pgfsetroundjoin%
\pgfsetlinewidth{0.803000pt}%
\definecolor{currentstroke}{rgb}{0.000000,0.000000,0.000000}%
\pgfsetstrokecolor{currentstroke}%
\pgfsetdash{}{0pt}%
\pgfpathmoveto{\pgfqpoint{1.029262in}{0.313091in}}%
\pgfpathlineto{\pgfqpoint{1.028585in}{0.310028in}}%
\pgfusepath{stroke}%
\end{pgfscope}%
\begin{pgfscope}%
\pgftext[x=1.028065in,y=0.257886in,,top]{\sffamily\fontsize{10.000000}{12.000000}\selectfont \(\displaystyle 4\)}%
\end{pgfscope}%
\begin{pgfscope}%
\pgfsetrectcap%
\pgfsetroundjoin%
\pgfsetlinewidth{0.803000pt}%
\definecolor{currentstroke}{rgb}{0.000000,0.000000,0.000000}%
\pgfsetstrokecolor{currentstroke}%
\pgfsetdash{}{0pt}%
\pgfpathmoveto{\pgfqpoint{0.446227in}{0.313091in}}%
\pgfpathlineto{\pgfqpoint{0.444014in}{0.310028in}}%
\pgfusepath{stroke}%
\end{pgfscope}%
\begin{pgfscope}%
\pgftext[x=0.442313in,y=0.257886in,,top]{\sffamily\fontsize{10.000000}{12.000000}\selectfont \(\displaystyle 6\)}%
\end{pgfscope}%
\begin{pgfscope}%
\pgfsetrectcap%
\pgfsetroundjoin%
\pgfsetlinewidth{0.803000pt}%
\definecolor{currentstroke}{rgb}{0.000000,0.000000,0.000000}%
\pgfsetstrokecolor{currentstroke}%
\pgfsetdash{}{0pt}%
\pgfpathmoveto{\pgfqpoint{2.366367in}{0.312072in}}%
\pgfpathlineto{\pgfqpoint{2.370517in}{1.706394in}}%
\pgfusepath{stroke}%
\end{pgfscope}%
\begin{pgfscope}%
\pgftext[x=2.552944in,y=1.003746in,,]{\sffamily\fontsize{10.000000}{12.000000}\selectfont \(\displaystyle z\)}%
\end{pgfscope}%
\begin{pgfscope}%
\pgfsetbuttcap%
\pgfsetroundjoin%
\pgfsetlinewidth{0.803000pt}%
\definecolor{currentstroke}{rgb}{0.690196,0.690196,0.690196}%
\pgfsetstrokecolor{currentstroke}%
\pgfsetdash{}{0pt}%
\pgfpathmoveto{\pgfqpoint{2.366670in}{0.413832in}}%
\pgfpathlineto{\pgfqpoint{2.259907in}{0.518746in}}%
\pgfpathlineto{\pgfqpoint{0.312385in}{0.518746in}}%
\pgfusepath{stroke}%
\end{pgfscope}%
\begin{pgfscope}%
\pgfsetbuttcap%
\pgfsetroundjoin%
\pgfsetlinewidth{0.803000pt}%
\definecolor{currentstroke}{rgb}{0.690196,0.690196,0.690196}%
\pgfsetstrokecolor{currentstroke}%
\pgfsetdash{}{0pt}%
\pgfpathmoveto{\pgfqpoint{2.367788in}{0.789544in}}%
\pgfpathlineto{\pgfqpoint{2.260815in}{0.857398in}}%
\pgfpathlineto{\pgfqpoint{0.311477in}{0.857398in}}%
\pgfusepath{stroke}%
\end{pgfscope}%
\begin{pgfscope}%
\pgfsetbuttcap%
\pgfsetroundjoin%
\pgfsetlinewidth{0.803000pt}%
\definecolor{currentstroke}{rgb}{0.690196,0.690196,0.690196}%
\pgfsetstrokecolor{currentstroke}%
\pgfsetdash{}{0pt}%
\pgfpathmoveto{\pgfqpoint{2.368909in}{1.166034in}}%
\pgfpathlineto{\pgfqpoint{2.261725in}{1.196682in}}%
\pgfpathlineto{\pgfqpoint{0.310567in}{1.196682in}}%
\pgfusepath{stroke}%
\end{pgfscope}%
\begin{pgfscope}%
\pgfsetbuttcap%
\pgfsetroundjoin%
\pgfsetlinewidth{0.803000pt}%
\definecolor{currentstroke}{rgb}{0.690196,0.690196,0.690196}%
\pgfsetstrokecolor{currentstroke}%
\pgfsetdash{}{0pt}%
\pgfpathmoveto{\pgfqpoint{2.370032in}{1.543305in}}%
\pgfpathlineto{\pgfqpoint{2.262636in}{1.536601in}}%
\pgfpathlineto{\pgfqpoint{0.309656in}{1.536601in}}%
\pgfusepath{stroke}%
\end{pgfscope}%
\begin{pgfscope}%
\pgfsetrectcap%
\pgfsetroundjoin%
\pgfsetlinewidth{0.803000pt}%
\definecolor{currentstroke}{rgb}{0.000000,0.000000,0.000000}%
\pgfsetstrokecolor{currentstroke}%
\pgfsetdash{}{0pt}%
\pgfpathmoveto{\pgfqpoint{2.365723in}{0.414762in}}%
\pgfpathlineto{\pgfqpoint{2.368569in}{0.411966in}}%
\pgfusepath{stroke}%
\end{pgfscope}%
\begin{pgfscope}%
\pgftext[x=2.448129in,y=0.409672in,,top]{\sffamily\fontsize{10.000000}{12.000000}\selectfont \(\displaystyle 0\)}%
\end{pgfscope}%
\begin{pgfscope}%
\pgfsetrectcap%
\pgfsetroundjoin%
\pgfsetlinewidth{0.803000pt}%
\definecolor{currentstroke}{rgb}{0.000000,0.000000,0.000000}%
\pgfsetstrokecolor{currentstroke}%
\pgfsetdash{}{0pt}%
\pgfpathmoveto{\pgfqpoint{2.366839in}{0.790145in}}%
\pgfpathlineto{\pgfqpoint{2.369691in}{0.788337in}}%
\pgfusepath{stroke}%
\end{pgfscope}%
\begin{pgfscope}%
\pgftext[x=2.449337in,y=0.786853in,,top]{\sffamily\fontsize{10.000000}{12.000000}\selectfont \(\displaystyle 2\)}%
\end{pgfscope}%
\begin{pgfscope}%
\pgfsetrectcap%
\pgfsetroundjoin%
\pgfsetlinewidth{0.803000pt}%
\definecolor{currentstroke}{rgb}{0.000000,0.000000,0.000000}%
\pgfsetstrokecolor{currentstroke}%
\pgfsetdash{}{0pt}%
\pgfpathmoveto{\pgfqpoint{2.367958in}{1.166306in}}%
\pgfpathlineto{\pgfqpoint{2.370816in}{1.165489in}}%
\pgfusepath{stroke}%
\end{pgfscope}%
\begin{pgfscope}%
\pgftext[x=2.450547in,y=1.164818in,,top]{\sffamily\fontsize{10.000000}{12.000000}\selectfont \(\displaystyle 4\)}%
\end{pgfscope}%
\begin{pgfscope}%
\pgfsetrectcap%
\pgfsetroundjoin%
\pgfsetlinewidth{0.803000pt}%
\definecolor{currentstroke}{rgb}{0.000000,0.000000,0.000000}%
\pgfsetstrokecolor{currentstroke}%
\pgfsetdash{}{0pt}%
\pgfpathmoveto{\pgfqpoint{2.369079in}{1.543245in}}%
\pgfpathlineto{\pgfqpoint{2.371943in}{1.543424in}}%
\pgfusepath{stroke}%
\end{pgfscope}%
\begin{pgfscope}%
\pgftext[x=2.451759in,y=1.543571in,,top]{\sffamily\fontsize{10.000000}{12.000000}\selectfont \(\displaystyle 6\)}%
\end{pgfscope}%
\begin{pgfscope}%
\pgfsys@transformshift{0.351429in}{0.418571in}%
\pgftext[left,bottom]{\pgfimage[interpolate=true,width=1.864286in,height=1.191429in]{dep-lcss-view3-small-img0.png}}%
\end{pgfscope}%
\end{pgfpicture}%
\makeatother%
\endgroup%

        \caption[]{{\small View along the positive $x$-axis}}
        \label{fig:u0_dom_err_dp54}
    \end{subfigure}
    \begin{subfigure}[b]{0.475\textwidth}
        \centering
        %% Creator: Matplotlib, PGF backend
%%
%% To include the figure in your LaTeX document, write
%%   \input{<filename>.pgf}
%%
%% Make sure the required packages are loaded in your preamble
%%   \usepackage{pgf}
%%
%% Figures using additional raster images can only be included by \input if
%% they are in the same directory as the main LaTeX file. For loading figures
%% from other directories you can use the `import` package
%%   \usepackage{import}
%% and then include the figures with
%%   \import{<path to file>}{<filename>.pgf}
%%
%% Matplotlib used the following preamble
%%   \usepackage{fontspec}
%%   \setmainfont{DejaVu Serif}
%%   \setsansfont{DejaVu Sans}
%%   \setmonofont{DejaVu Sans Mono}
%%
\begingroup%
\makeatletter%
\begin{pgfpicture}%
\pgfpathrectangle{\pgfpointorigin}{\pgfqpoint{2.660000in}{1.740000in}}%
\pgfusepath{use as bounding box, clip}%
\begin{pgfscope}%
\pgfsetbuttcap%
\pgfsetmiterjoin%
\definecolor{currentfill}{rgb}{1.000000,1.000000,1.000000}%
\pgfsetfillcolor{currentfill}%
\pgfsetlinewidth{0.000000pt}%
\definecolor{currentstroke}{rgb}{1.000000,1.000000,1.000000}%
\pgfsetstrokecolor{currentstroke}%
\pgfsetdash{}{0pt}%
\pgfpathmoveto{\pgfqpoint{0.000000in}{0.000000in}}%
\pgfpathlineto{\pgfqpoint{2.660000in}{0.000000in}}%
\pgfpathlineto{\pgfqpoint{2.660000in}{1.740000in}}%
\pgfpathlineto{\pgfqpoint{0.000000in}{1.740000in}}%
\pgfpathclose%
\pgfusepath{fill}%
\end{pgfscope}%
\begin{pgfscope}%
\pgfsetbuttcap%
\pgfsetmiterjoin%
\definecolor{currentfill}{rgb}{1.000000,1.000000,1.000000}%
\pgfsetfillcolor{currentfill}%
\pgfsetlinewidth{0.000000pt}%
\definecolor{currentstroke}{rgb}{0.000000,0.000000,0.000000}%
\pgfsetstrokecolor{currentstroke}%
\pgfsetstrokeopacity{0.000000}%
\pgfsetdash{}{0pt}%
\pgfpathmoveto{\pgfqpoint{0.000000in}{0.087000in}}%
\pgfpathlineto{\pgfqpoint{2.793000in}{0.087000in}}%
\pgfpathlineto{\pgfqpoint{2.793000in}{1.740000in}}%
\pgfpathlineto{\pgfqpoint{0.000000in}{1.740000in}}%
\pgfpathclose%
\pgfusepath{fill}%
\end{pgfscope}%
\begin{pgfscope}%
\pgfsetbuttcap%
\pgfsetmiterjoin%
\pgfsetlinewidth{0.000000pt}%
\definecolor{currentstroke}{rgb}{1.000000,1.000000,1.000000}%
\pgfsetstrokecolor{currentstroke}%
\pgfsetstrokeopacity{0.000000}%
\pgfsetdash{}{0pt}%
\pgfpathmoveto{\pgfqpoint{1.434243in}{1.061013in}}%
\pgfpathlineto{\pgfqpoint{2.506765in}{0.618459in}}%
\pgfpathlineto{\pgfqpoint{2.588784in}{1.277488in}}%
\pgfpathlineto{\pgfqpoint{1.434243in}{1.718340in}}%
\pgfusepath{}%
\end{pgfscope}%
\begin{pgfscope}%
\pgfsetbuttcap%
\pgfsetmiterjoin%
\pgfsetlinewidth{0.000000pt}%
\definecolor{currentstroke}{rgb}{1.000000,1.000000,1.000000}%
\pgfsetstrokecolor{currentstroke}%
\pgfsetstrokeopacity{0.000000}%
\pgfsetdash{}{0pt}%
\pgfpathmoveto{\pgfqpoint{1.434243in}{1.061013in}}%
\pgfpathlineto{\pgfqpoint{0.361722in}{0.618459in}}%
\pgfpathlineto{\pgfqpoint{0.279702in}{1.277488in}}%
\pgfpathlineto{\pgfqpoint{1.434243in}{1.718340in}}%
\pgfusepath{}%
\end{pgfscope}%
\begin{pgfscope}%
\pgfsetbuttcap%
\pgfsetmiterjoin%
\pgfsetlinewidth{0.000000pt}%
\definecolor{currentstroke}{rgb}{1.000000,1.000000,1.000000}%
\pgfsetstrokecolor{currentstroke}%
\pgfsetstrokeopacity{0.000000}%
\pgfsetdash{}{0pt}%
\pgfpathmoveto{\pgfqpoint{1.434243in}{1.061013in}}%
\pgfpathlineto{\pgfqpoint{0.361722in}{0.618459in}}%
\pgfpathlineto{\pgfqpoint{1.434243in}{0.129092in}}%
\pgfpathlineto{\pgfqpoint{2.506765in}{0.618459in}}%
\pgfusepath{}%
\end{pgfscope}%
\begin{pgfscope}%
\pgfsetrectcap%
\pgfsetroundjoin%
\pgfsetlinewidth{0.803000pt}%
\definecolor{currentstroke}{rgb}{0.000000,0.000000,0.000000}%
\pgfsetstrokecolor{currentstroke}%
\pgfsetdash{}{0pt}%
\pgfpathmoveto{\pgfqpoint{2.506765in}{0.618459in}}%
\pgfpathlineto{\pgfqpoint{1.434243in}{0.129092in}}%
\pgfusepath{stroke}%
\end{pgfscope}%
\begin{pgfscope}%
\pgftext[x=2.205317in,y=0.156800in,,]{\sffamily\fontsize{10.000000}{12.000000}\selectfont \(\displaystyle x\)}%
\end{pgfscope}%
\begin{pgfscope}%
\pgfsetbuttcap%
\pgfsetroundjoin%
\pgfsetlinewidth{0.803000pt}%
\definecolor{currentstroke}{rgb}{0.690196,0.690196,0.690196}%
\pgfsetstrokecolor{currentstroke}%
\pgfsetdash{}{0pt}%
\pgfpathmoveto{\pgfqpoint{2.431882in}{0.584292in}}%
\pgfpathlineto{\pgfqpoint{1.359184in}{1.030041in}}%
\pgfpathlineto{\pgfqpoint{1.353718in}{1.687592in}}%
\pgfusepath{stroke}%
\end{pgfscope}%
\begin{pgfscope}%
\pgfsetbuttcap%
\pgfsetroundjoin%
\pgfsetlinewidth{0.803000pt}%
\definecolor{currentstroke}{rgb}{0.690196,0.690196,0.690196}%
\pgfsetstrokecolor{currentstroke}%
\pgfsetdash{}{0pt}%
\pgfpathmoveto{\pgfqpoint{2.150923in}{0.456097in}}%
\pgfpathlineto{\pgfqpoint{1.077801in}{0.913934in}}%
\pgfpathlineto{\pgfqpoint{1.051477in}{1.572184in}}%
\pgfusepath{stroke}%
\end{pgfscope}%
\begin{pgfscope}%
\pgfsetbuttcap%
\pgfsetroundjoin%
\pgfsetlinewidth{0.803000pt}%
\definecolor{currentstroke}{rgb}{0.690196,0.690196,0.690196}%
\pgfsetstrokecolor{currentstroke}%
\pgfsetdash{}{0pt}%
\pgfpathmoveto{\pgfqpoint{1.862099in}{0.324313in}}%
\pgfpathlineto{\pgfqpoint{0.788928in}{0.794737in}}%
\pgfpathlineto{\pgfqpoint{0.740589in}{1.453474in}}%
\pgfusepath{stroke}%
\end{pgfscope}%
\begin{pgfscope}%
\pgfsetbuttcap%
\pgfsetroundjoin%
\pgfsetlinewidth{0.803000pt}%
\definecolor{currentstroke}{rgb}{0.690196,0.690196,0.690196}%
\pgfsetstrokecolor{currentstroke}%
\pgfsetdash{}{0pt}%
\pgfpathmoveto{\pgfqpoint{1.565074in}{0.188787in}}%
\pgfpathlineto{\pgfqpoint{0.492263in}{0.672324in}}%
\pgfpathlineto{\pgfqpoint{0.420678in}{1.331318in}}%
\pgfusepath{stroke}%
\end{pgfscope}%
\begin{pgfscope}%
\pgfsetrectcap%
\pgfsetroundjoin%
\pgfsetlinewidth{0.803000pt}%
\definecolor{currentstroke}{rgb}{0.000000,0.000000,0.000000}%
\pgfsetstrokecolor{currentstroke}%
\pgfsetdash{}{0pt}%
\pgfpathmoveto{\pgfqpoint{2.422871in}{0.588036in}}%
\pgfpathlineto{\pgfqpoint{2.449925in}{0.576794in}}%
\pgfusepath{stroke}%
\end{pgfscope}%
\begin{pgfscope}%
\pgftext[x=2.482485in,y=0.532509in,,top]{\sffamily\fontsize{10.000000}{12.000000}\selectfont \(\displaystyle 0\)}%
\end{pgfscope}%
\begin{pgfscope}%
\pgfsetrectcap%
\pgfsetroundjoin%
\pgfsetlinewidth{0.803000pt}%
\definecolor{currentstroke}{rgb}{0.000000,0.000000,0.000000}%
\pgfsetstrokecolor{currentstroke}%
\pgfsetdash{}{0pt}%
\pgfpathmoveto{\pgfqpoint{2.141903in}{0.459945in}}%
\pgfpathlineto{\pgfqpoint{2.168986in}{0.448390in}}%
\pgfusepath{stroke}%
\end{pgfscope}%
\begin{pgfscope}%
\pgftext[x=2.202509in,y=0.403730in,,top]{\sffamily\fontsize{10.000000}{12.000000}\selectfont \(\displaystyle 2\)}%
\end{pgfscope}%
\begin{pgfscope}%
\pgfsetrectcap%
\pgfsetroundjoin%
\pgfsetlinewidth{0.803000pt}%
\definecolor{currentstroke}{rgb}{0.000000,0.000000,0.000000}%
\pgfsetstrokecolor{currentstroke}%
\pgfsetdash{}{0pt}%
\pgfpathmoveto{\pgfqpoint{1.853072in}{0.328270in}}%
\pgfpathlineto{\pgfqpoint{1.880175in}{0.316389in}}%
\pgfusepath{stroke}%
\end{pgfscope}%
\begin{pgfscope}%
\pgftext[x=1.914703in,y=0.271350in,,top]{\sffamily\fontsize{10.000000}{12.000000}\selectfont \(\displaystyle 4\)}%
\end{pgfscope}%
\begin{pgfscope}%
\pgfsetrectcap%
\pgfsetroundjoin%
\pgfsetlinewidth{0.803000pt}%
\definecolor{currentstroke}{rgb}{0.000000,0.000000,0.000000}%
\pgfsetstrokecolor{currentstroke}%
\pgfsetdash{}{0pt}%
\pgfpathmoveto{\pgfqpoint{1.556045in}{0.192857in}}%
\pgfpathlineto{\pgfqpoint{1.583157in}{0.180637in}}%
\pgfusepath{stroke}%
\end{pgfscope}%
\begin{pgfscope}%
\pgftext[x=1.618735in,y=0.135215in,,top]{\sffamily\fontsize{10.000000}{12.000000}\selectfont \(\displaystyle 6\)}%
\end{pgfscope}%
\begin{pgfscope}%
\pgfsetrectcap%
\pgfsetroundjoin%
\pgfsetlinewidth{0.803000pt}%
\definecolor{currentstroke}{rgb}{0.000000,0.000000,0.000000}%
\pgfsetstrokecolor{currentstroke}%
\pgfsetdash{}{0pt}%
\pgfpathmoveto{\pgfqpoint{0.361722in}{0.618459in}}%
\pgfpathlineto{\pgfqpoint{1.434243in}{0.129092in}}%
\pgfusepath{stroke}%
\end{pgfscope}%
\begin{pgfscope}%
\pgftext[x=0.663170in,y=0.156800in,,]{\sffamily\fontsize{10.000000}{12.000000}\selectfont \(\displaystyle y\)}%
\end{pgfscope}%
\begin{pgfscope}%
\pgfsetbuttcap%
\pgfsetroundjoin%
\pgfsetlinewidth{0.803000pt}%
\definecolor{currentstroke}{rgb}{0.690196,0.690196,0.690196}%
\pgfsetstrokecolor{currentstroke}%
\pgfsetdash{}{0pt}%
\pgfpathmoveto{\pgfqpoint{1.520902in}{1.685250in}}%
\pgfpathlineto{\pgfqpoint{1.515018in}{1.027683in}}%
\pgfpathlineto{\pgfqpoint{0.442308in}{0.581689in}}%
\pgfusepath{stroke}%
\end{pgfscope}%
\begin{pgfscope}%
\pgfsetbuttcap%
\pgfsetroundjoin%
\pgfsetlinewidth{0.803000pt}%
\definecolor{currentstroke}{rgb}{0.690196,0.690196,0.690196}%
\pgfsetstrokecolor{currentstroke}%
\pgfsetdash{}{0pt}%
\pgfpathmoveto{\pgfqpoint{1.823366in}{1.569757in}}%
\pgfpathlineto{\pgfqpoint{1.796598in}{0.911495in}}%
\pgfpathlineto{\pgfqpoint{0.723470in}{0.453401in}}%
\pgfusepath{stroke}%
\end{pgfscope}%
\begin{pgfscope}%
\pgfsetbuttcap%
\pgfsetroundjoin%
\pgfsetlinewidth{0.803000pt}%
\definecolor{currentstroke}{rgb}{0.690196,0.690196,0.690196}%
\pgfsetstrokecolor{currentstroke}%
\pgfsetdash{}{0pt}%
\pgfpathmoveto{\pgfqpoint{2.134487in}{1.450958in}}%
\pgfpathlineto{\pgfqpoint{2.085675in}{0.792213in}}%
\pgfpathlineto{\pgfqpoint{1.012507in}{0.321521in}}%
\pgfusepath{stroke}%
\end{pgfscope}%
\begin{pgfscope}%
\pgfsetbuttcap%
\pgfsetroundjoin%
\pgfsetlinewidth{0.803000pt}%
\definecolor{currentstroke}{rgb}{0.690196,0.690196,0.690196}%
\pgfsetstrokecolor{currentstroke}%
\pgfsetdash{}{0pt}%
\pgfpathmoveto{\pgfqpoint{2.454642in}{1.328709in}}%
\pgfpathlineto{\pgfqpoint{2.382553in}{0.669712in}}%
\pgfpathlineto{\pgfqpoint{1.309754in}{0.185894in}}%
\pgfusepath{stroke}%
\end{pgfscope}%
\begin{pgfscope}%
\pgfsetrectcap%
\pgfsetroundjoin%
\pgfsetlinewidth{0.803000pt}%
\definecolor{currentstroke}{rgb}{0.000000,0.000000,0.000000}%
\pgfsetstrokecolor{currentstroke}%
\pgfsetdash{}{0pt}%
\pgfpathmoveto{\pgfqpoint{0.451319in}{0.585436in}}%
\pgfpathlineto{\pgfqpoint{0.424264in}{0.574188in}}%
\pgfusepath{stroke}%
\end{pgfscope}%
\begin{pgfscope}%
\pgftext[x=0.391685in,y=0.529895in,,top]{\sffamily\fontsize{10.000000}{12.000000}\selectfont \(\displaystyle 0\)}%
\end{pgfscope}%
\begin{pgfscope}%
\pgfsetrectcap%
\pgfsetroundjoin%
\pgfsetlinewidth{0.803000pt}%
\definecolor{currentstroke}{rgb}{0.000000,0.000000,0.000000}%
\pgfsetstrokecolor{currentstroke}%
\pgfsetdash{}{0pt}%
\pgfpathmoveto{\pgfqpoint{0.732490in}{0.457252in}}%
\pgfpathlineto{\pgfqpoint{0.705407in}{0.445691in}}%
\pgfusepath{stroke}%
\end{pgfscope}%
\begin{pgfscope}%
\pgftext[x=0.671864in,y=0.401023in,,top]{\sffamily\fontsize{10.000000}{12.000000}\selectfont \(\displaystyle 2\)}%
\end{pgfscope}%
\begin{pgfscope}%
\pgfsetrectcap%
\pgfsetroundjoin%
\pgfsetlinewidth{0.803000pt}%
\definecolor{currentstroke}{rgb}{0.000000,0.000000,0.000000}%
\pgfsetstrokecolor{currentstroke}%
\pgfsetdash{}{0pt}%
\pgfpathmoveto{\pgfqpoint{1.021534in}{0.325480in}}%
\pgfpathlineto{\pgfqpoint{0.994431in}{0.313592in}}%
\pgfusepath{stroke}%
\end{pgfscope}%
\begin{pgfscope}%
\pgftext[x=0.959881in,y=0.268545in,,top]{\sffamily\fontsize{10.000000}{12.000000}\selectfont \(\displaystyle 4\)}%
\end{pgfscope}%
\begin{pgfscope}%
\pgfsetrectcap%
\pgfsetroundjoin%
\pgfsetlinewidth{0.803000pt}%
\definecolor{currentstroke}{rgb}{0.000000,0.000000,0.000000}%
\pgfsetstrokecolor{currentstroke}%
\pgfsetdash{}{0pt}%
\pgfpathmoveto{\pgfqpoint{1.318784in}{0.189966in}}%
\pgfpathlineto{\pgfqpoint{1.291672in}{0.177739in}}%
\pgfusepath{stroke}%
\end{pgfscope}%
\begin{pgfscope}%
\pgftext[x=1.256071in,y=0.132308in,,top]{\sffamily\fontsize{10.000000}{12.000000}\selectfont \(\displaystyle 6\)}%
\end{pgfscope}%
\begin{pgfscope}%
\pgfsetrectcap%
\pgfsetroundjoin%
\pgfsetlinewidth{0.803000pt}%
\definecolor{currentstroke}{rgb}{0.000000,0.000000,0.000000}%
\pgfsetstrokecolor{currentstroke}%
\pgfsetdash{}{0pt}%
\pgfpathmoveto{\pgfqpoint{0.361722in}{0.618459in}}%
\pgfpathlineto{\pgfqpoint{0.279702in}{1.277488in}}%
\pgfusepath{stroke}%
\end{pgfscope}%
\begin{pgfscope}%
\pgftext[x=0.082110in,y=0.935838in,,]{\sffamily\fontsize{10.000000}{12.000000}\selectfont \(\displaystyle z\)}%
\end{pgfscope}%
\begin{pgfscope}%
\pgfsetbuttcap%
\pgfsetroundjoin%
\pgfsetlinewidth{0.803000pt}%
\definecolor{currentstroke}{rgb}{0.690196,0.690196,0.690196}%
\pgfsetstrokecolor{currentstroke}%
\pgfsetdash{}{0pt}%
\pgfpathmoveto{\pgfqpoint{0.356112in}{0.663533in}}%
\pgfpathlineto{\pgfqpoint{1.434243in}{1.106123in}}%
\pgfpathlineto{\pgfqpoint{2.512375in}{0.663533in}}%
\pgfusepath{stroke}%
\end{pgfscope}%
\begin{pgfscope}%
\pgfsetbuttcap%
\pgfsetroundjoin%
\pgfsetlinewidth{0.803000pt}%
\definecolor{currentstroke}{rgb}{0.690196,0.690196,0.690196}%
\pgfsetstrokecolor{currentstroke}%
\pgfsetdash{}{0pt}%
\pgfpathmoveto{\pgfqpoint{0.334910in}{0.833887in}}%
\pgfpathlineto{\pgfqpoint{1.434243in}{1.276413in}}%
\pgfpathlineto{\pgfqpoint{2.533576in}{0.833887in}}%
\pgfusepath{stroke}%
\end{pgfscope}%
\begin{pgfscope}%
\pgfsetbuttcap%
\pgfsetroundjoin%
\pgfsetlinewidth{0.803000pt}%
\definecolor{currentstroke}{rgb}{0.690196,0.690196,0.690196}%
\pgfsetstrokecolor{currentstroke}%
\pgfsetdash{}{0pt}%
\pgfpathmoveto{\pgfqpoint{0.312858in}{1.011076in}}%
\pgfpathlineto{\pgfqpoint{1.434243in}{1.453194in}}%
\pgfpathlineto{\pgfqpoint{2.555628in}{1.011076in}}%
\pgfusepath{stroke}%
\end{pgfscope}%
\begin{pgfscope}%
\pgfsetbuttcap%
\pgfsetroundjoin%
\pgfsetlinewidth{0.803000pt}%
\definecolor{currentstroke}{rgb}{0.690196,0.690196,0.690196}%
\pgfsetstrokecolor{currentstroke}%
\pgfsetdash{}{0pt}%
\pgfpathmoveto{\pgfqpoint{0.289904in}{1.195520in}}%
\pgfpathlineto{\pgfqpoint{1.434243in}{1.636844in}}%
\pgfpathlineto{\pgfqpoint{2.578583in}{1.195520in}}%
\pgfusepath{stroke}%
\end{pgfscope}%
\begin{pgfscope}%
\pgfsetrectcap%
\pgfsetroundjoin%
\pgfsetlinewidth{0.803000pt}%
\definecolor{currentstroke}{rgb}{0.000000,0.000000,0.000000}%
\pgfsetstrokecolor{currentstroke}%
\pgfsetdash{}{0pt}%
\pgfpathmoveto{\pgfqpoint{0.365169in}{0.667251in}}%
\pgfpathlineto{\pgfqpoint{0.337976in}{0.656088in}}%
\pgfusepath{stroke}%
\end{pgfscope}%
\begin{pgfscope}%
\pgftext[x=0.252648in,y=0.663533in,,top]{\sffamily\fontsize{10.000000}{12.000000}\selectfont \(\displaystyle 0\)}%
\end{pgfscope}%
\begin{pgfscope}%
\pgfsetrectcap%
\pgfsetroundjoin%
\pgfsetlinewidth{0.803000pt}%
\definecolor{currentstroke}{rgb}{0.000000,0.000000,0.000000}%
\pgfsetstrokecolor{currentstroke}%
\pgfsetdash{}{0pt}%
\pgfpathmoveto{\pgfqpoint{0.344154in}{0.837608in}}%
\pgfpathlineto{\pgfqpoint{0.316400in}{0.826436in}}%
\pgfusepath{stroke}%
\end{pgfscope}%
\begin{pgfscope}%
\pgftext[x=0.229411in,y=0.833887in,,top]{\sffamily\fontsize{10.000000}{12.000000}\selectfont \(\displaystyle 2\)}%
\end{pgfscope}%
\begin{pgfscope}%
\pgfsetrectcap%
\pgfsetroundjoin%
\pgfsetlinewidth{0.803000pt}%
\definecolor{currentstroke}{rgb}{0.000000,0.000000,0.000000}%
\pgfsetstrokecolor{currentstroke}%
\pgfsetdash{}{0pt}%
\pgfpathmoveto{\pgfqpoint{0.322297in}{1.014798in}}%
\pgfpathlineto{\pgfqpoint{0.293958in}{1.003625in}}%
\pgfusepath{stroke}%
\end{pgfscope}%
\begin{pgfscope}%
\pgftext[x=0.205243in,y=1.011076in,,top]{\sffamily\fontsize{10.000000}{12.000000}\selectfont \(\displaystyle 4\)}%
\end{pgfscope}%
\begin{pgfscope}%
\pgfsetrectcap%
\pgfsetroundjoin%
\pgfsetlinewidth{0.803000pt}%
\definecolor{currentstroke}{rgb}{0.000000,0.000000,0.000000}%
\pgfsetstrokecolor{currentstroke}%
\pgfsetdash{}{0pt}%
\pgfpathmoveto{\pgfqpoint{0.299545in}{1.199238in}}%
\pgfpathlineto{\pgfqpoint{0.270596in}{1.188074in}}%
\pgfusepath{stroke}%
\end{pgfscope}%
\begin{pgfscope}%
\pgftext[x=0.180085in,y=1.195520in,,top]{\sffamily\fontsize{10.000000}{12.000000}\selectfont \(\displaystyle 6\)}%
\end{pgfscope}%
\begin{pgfscope}%
\pgfsys@transformshift{0.564286in}{0.398571in}%
\pgftext[left,bottom]{\pgfimage[interpolate=true,width=1.808571in,height=0.934286in]{dep-lcss-view4-small-img0.png}}%
\end{pgfscope}%
\end{pgfpicture}%
\makeatother%
\endgroup%

        \caption[]{{\small Approximately isometric view}}
        \label{fig:u0_dom_err_dp87}
    \end{subfigure}
    \caption[Aviici is love, Aviici is life]{LCSs obtained for dynamic ABC flow}
    \label{fig:u0_dom_errs}
\end{figure}



