\section{Computed LCSs in the ABC flow}
\label{sec:computed_lcss_in_the_abc_flow}

Having computed Cauchy-Green strain eigenvalues and -vectors for both
variants of the ABC flow (as presented in
\cref{sec:flow_systems_defined_by_analytical_velocity_fields}), we used
the $\mathcal{U}_{0}$ domain --- i.e., the grid points satisfying the LCS
existence criteria~ \eqref{eq:lcs_condition_a},~\eqref{eq:lcs_condition_b}
and~\eqref{eq:lcs_condition_b} (the implementation of which are
described in
\cref{sub:identifying_suitable_initial_conditions_for_developing_lcss}) --- as a
first approximation to where we may reasonably expect to find repelling LCSs.
Four different views of the $\mathcal{U}_{0}$ domains for the steady
and unsteady ABC flows are shown in \cref{fig:steady_abd,fig:unsteady_abd},
respectively.

\begin{figure}[htpb]
    \centering
    \begin{subfigure}[b]{0.475\textwidth}
        \centering
        \importpgf{figures/mpl-figs}{indep-abd-domain-view1-small.pgf}
        \caption[]{{\small View along the negative $z$-axis}}
        \label{fig:steady_abd_z}
    \end{subfigure}
    \begin{subfigure}[b]{0.475\textwidth}
        \centering
        \importpgf{figures/mpl-figs}{indep-abd-domain-view2-small.pgf}
        \caption[]{{\small View along the positive $y$-axis}}
        \label{fig:steady_abd_y}
    \end{subfigure}

    \begin{subfigure}[b]{0.475\textwidth}
        \centering
        \importpgf{figures/mpl-figs}{indep-abd-domain-view3-small.pgf}
        \caption[]{{\small View along the positive $x$-axis}}
        \label{fig:steady_abd_x}
    \end{subfigure}
    \begin{subfigure}[b]{0.475\textwidth}
        \centering
        \importpgf{figures/mpl-figs}{indep-abd-domain-view4-small.pgf}
        \caption[]{{\small Approximately isometric view}}
        \label{fig:steady_abd_isometric}
    \end{subfigure}
    \caption[Four views of the $\mathcal{U}_{0}$ domain obtained for transport
    in the steady ABC flow]
    {
        Four views of the $\mathcal{U}_{0}$ domain obtained for transport in the
        steady ABC flow, identified as the grid points which satisfy the LCS
        criteria \eqref{eq:lcs_condition_a},~\eqref{eq:lcs_condition_b} and~
        \eqref{eq:lcs_condition_d} (details on the implementation are available
        in
        \cref{sub:identifying_suitable_initial_conditions_for_developing_lcss}).
        In total, the domain consists of \numprint{340951} different points
        (cf.\ \cref{tab:initialconditionparams}).
}
    \label{fig:steady_abd}
\end{figure}



\begin{figure}[htpb]
    \centering
    \begin{subfigure}[b]{0.475\textwidth}
        \centering
        \importpgf{figures/mpl-figs}{dep-abd-domain-view1-small.pgf}
        \caption[]{{\small View along the negative $z$-axis}}
        \label{fig:unsteady_abd_z}
    \end{subfigure}
    \begin{subfigure}[b]{0.475\textwidth}
        \centering
        \importpgf{figures/mpl-figs}{dep-abd-domain-view2-small.pgf}
        \caption[]{{\small View along the positive $y$-axis}}
        \label{fig:unsteady_abd_y}
    \end{subfigure}

    \begin{subfigure}[b]{0.475\textwidth}
        \centering
        \importpgf{figures/mpl-figs}{dep-abd-domain-view3-small.pgf}
        \caption[]{{\small View along the positive $x$-axis}}
        \label{fig:unsteady_abd_x}
    \end{subfigure}
    \begin{subfigure}[b]{0.475\textwidth}
        \centering
        \importpgf{figures/mpl-figs}{dep-abd-domain-view4-small.pgf}
        \caption[]{{\small Approximately isometric view}}
        \label{fig:unsteady_abd_isometric}
    \end{subfigure}
    \caption[Four views of the $\mathcal{U}_{0}$ domain obtained for transport
    in the unsteady ABC flow]
    {
        Four views of the $\mathcal{U}_{0}$ domain obtained for transport in the
        unsteady ABC flow, for the time interval $\mathcal{I}=[0,5]$,
        identified as the grid points which satisfy the LCS criteria
        \eqref{eq:lcs_condition_a},~\eqref{eq:lcs_condition_b} and~
        \eqref{eq:lcs_condition_d} (details on the implementation are available
        in
        \cref{sub:identifying_suitable_initial_conditions_for_developing_lcss}).
        In total, the domain consists of \numprint{361461} different points
        (cf.\ \cref{tab:initialconditionparams}).
}
    \label{fig:unsteady_abd}
\end{figure}



Note that, although the $\mathcal{U}_{0}$ domains for the two
flow variants do not contain the same number of points (cf.\
\cref{tab:initialconditionparams}), the macroscopic trends remain the same. The
difference lies in minute details, such as the topmost cavities in
\cref{fig:steady_abd_z,fig:unsteady_abd_z} being of slightly different sizes,
or the point densities along the north east `bands' in \cref{fig:steady_abd_y,%
fig:unsteady_abd_y} being somewhat dissimilar. This is as expected, seeing
as the two underlying transport systems are lightly perturbed versions of
one another.

Using an approximately evenly sampled subset of the $\mathcal{U}_{0}$ domain
as initial conditions, \numprint{618} and \numprint{676} for the steady
and unsteady variants of the ABC flow, respectively, we then computed manifolds
and extracted LCSs using the parameters in \cref{tab:initialconditionparams,%
tab:abc_manifold_params} and the method
outlined in \cref{sec:preliminaries_for_computing_repelling_lcss_in_3d_flow_by_means_of_geodesic_level_sets,%
    sec:revised_approach_to_computing_new_mesh_points,%
    sec:managing_mesh_accuracy,%
    sec:continuously_reconstructing_three_dimensional_manifold_surfaces_from_%
    point_meshes,%
    sec:macroscale_stopping_criteria_for_the_expansion_of_computed_manifolds,%
    sec:identifying_lcss_as_subsets_of_computed_manifolds}. This resulted
in a total of \numprint{22} LCS surfaces for the steady flow, and
\numprint{31} surfaces for the unsteady flow.

\begin{table}[htpb]
    \centering
    \caption[Parameters used to compute manifolds, and subsequently
    repelling LCSs, in the ABC flow (both variants)]
    {Parameters used to compute manifolds, and subsequently repelling LCSs,
        in both variants of the ABC flow (see
        \cref{sec:flow_systems_defined_by_analytical_velocity_fields}). Note
        that the interset distances $\Delta_{i}$ were dynamically altered, as
        described in
        \cref{sub:a_curvature_based_approach_to_determining_interset%
        _separations}. Accordingly, only the \emph{first} interset distance,
        $\Delta_{1}$, was set explicitly.
    }
    \label{tab:abc_manifold_params}
    \begin{tabular}{ccc}
        \toprule
        Parameter & Value & Description\\
        \midrule
        $\delta_{\text{init}}$ & $10^{-3}$ %
        & \makecell{Separation of innermost geodesic level set \\
        from the manifold epicentre, cf. \cref{fig:innermost_levelset}}%
        \\[9pt]
        %
        $\Delta_{\min}$, $\Delta_{\max}$
        & $0.04$, $0.16$ %
        & \makecell{Boundaries for interpoint separations \\(details
        found in \cref{sub:maintaining_mesh_point_density})}%
        \\[9pt]
        %
        $\Delta_{1}$ %
        & $2\Delta_{\min}$ %
        & \makecell{Interset distance used to compute the second \\ geodesic
        level set (see
        \cref{sec:revised_approach_to_computing_new_mesh_points,%
        sub:a_curvature_based_approach_to_determining_interset_separations})}%
        \\[9pt]
        %
        $\gamma_{\|}$ %
        & $10^{-4}$ %
        & \makecell{Tolerance for detecting regions in which
        $\vct{\xi}_{3}$\\ is (anti-)parallel to $\vct{t}_{i,j}$
    (see \cref{eq:revised_xi3_tan_parallel})}
        \\[9pt]
        %
        $\gamma_{\Delta}$ %
        & $5\cdot10^{-3}$ %
        & \makecell{Tolerance for the separation of a mesh point\\ from
        its ancestor (per \cref{eq:revised_dist_tol})}
        \\[9pt]
        $\gamma_{\text{arc}}$ %
        & 5 %
        & \makecell{Sets an upper limit to trajectory lengths as \\
        $\gamma_{\text{arc}}\Delta_{i}$ (briefly mentioned in
        \cref{sub:computing_pseudoradial_trajectories_directly})}
        \\[9pt]
        %
        $\gamma_{\circlearrowleft}$ %
        & $7\cdot10^{-1}$
        & \makecell{Sets an upper limit to the extent of loop-like\\
        segments of any level set (see
        \cref{sub:limiting_the_accumulation_of_numerical_noise})}
        \\[9pt]
        %
        \makecell[c]{$\alpha_{\uparrow}$\\ $\alpha_{\downarrow}$ \\[1.5pt]%
        ${(\Delta\alpha)}_{\uparrow}$, ${(\Delta\alpha)}_{\downarrow}$} &
        \makecell[c]{$8.7\cdot10^{-2}\;\si{\radian}$ %
            \phantom{2}$({5}\si{\degree})$\\ %
            ${4.4\cdot10^{-1}}\;\si{\radian}$ $({25}\si{\degree})$\\[1.5pt]%
        $2\Delta_{\min}\alpha_{\uparrow}$, %
        $2\Delta_{\min}\alpha_{\downarrow}$}%
        & \makecell[c]{Used in a curvature-based approach to adjust\\
        interset distances (outlined in
        \cref{sub:a_curvature_based_approach_to_determining_interset%
        _separations})}
        \\[18pt]
        %
        $\gamma_{\cap}$ &
        $5$ &
        \makecell[c]{Used for terminating the expansion of \\
        self-intersecting manifolds (cf.\
        \cref{sub:continuous_self_intersection_checks})}
        \\[9pt]
        %
        $\gamma_{\square}$ &
        $1.75$ &
        \makecell[c]{Relaxation parameter for extracting LCSs from\\ the
            computed manifolds (see
        \cref{sec:identifying_lcss_as_subsets_of_computed_manifolds})}
        \\[9pt]
        %
        $\mathcal{W}_{\text{min}}$ &
        $6.0$ &
        \makecell[c]{Filters away the smallest LCSs measured in\\
        (pseudo-)surface area (see
        \cref{sec:identifying_lcss_as_subsets_of_computed_manifolds})}
        \\
        \bottomrule
    \end{tabular}
\end{table}



The LCSs present in the steady flow turn out to form two distinct,
smooth and coherent structures, which are shown in \cref{fig:steady_lcss}. The
structures lie close together, yet appear not to be connected --- accordingly,
the two structures are highlighted by different colors for the purpose of
facilitating visual comparisons. Note in particular the correspondence between
the $\mathcal{U}_{0}$  domain, shown in \cref{fig:steady_abd}, and the computed
LCSs, shown in \cref{fig:steady_lcss} --- where the perspective of each
subfigure is the same as that of the corresponding subfigure in
\cref{fig:steady_abd}. Two prominent similarities are the tunnel-like structure
apparent in the middle right of \cref{fig:steady_abd_y,fig:steady_lcss_y}, and
the indent which manifests near the bottom right corner of
\cref{fig:steady_abd_x,fig:steady_lcss_x}.

\begin{figure}[htpb]
    \centering
    \hspace*{\fill}
    \begin{subfigure}[b]{0.42\textwidth}
        \centering
        \resizebox{0.9\linewidth}{!}{\importpgf{figures/mpl-figs}{indep-lcss-view1-small.pgf}}
        \caption[]{{\small View along the negative $z$-axis}}
        \label{fig:steady_lcss_z}
    \end{subfigure}\hfill%
    \begin{subfigure}[b]{0.42\textwidth}
        \centering
        \resizebox{0.9\linewidth}{!}{\importpgf{figures/mpl-figs}{indep-lcss-view2-small.pgf}}
        \caption[]{{\small View along the positive $y$-axis}}
        \label{fig:steady_lcss_y}
    \end{subfigure}%
    \hspace*{\fill}

    \hspace*{\fill}
    \begin{subfigure}[b]{0.42\textwidth}
        \centering
        \resizebox{0.9\linewidth}{!}{\importpgf{figures/mpl-figs}{indep-lcss-view3-small.pgf}}
        \caption[]{{\small View along the positive $x$-axis}}
        \label{fig:steady_lcss_x}
    \end{subfigure}\hfill%
    \begin{subfigure}[b]{0.42\textwidth}
        \centering
        \resizebox{0.9\linewidth}{!}{\importpgf{figures/mpl-figs}{indep-lcss-view4-small.pgf}}
        \caption[]{{\small Approximately isometric view}}
        \label{fig:steady_lcss_isometric}
    \end{subfigure}%
    \hspace*{\fill}
    \caption[Four views of the repelling LCSs obtained for transport in the
    steady ABC flow]
    {
        Four views of the repelling LCSs obtained for transport in the steady
        ABC flow, for the time interval $\mathcal{I}=[0,5]$ (see
        \cref{sub:steady_arnold_beltrami_childress_flow}). A total of
        \numprint{22} surfaces constitute two distinct, smooth and coherent
        structures, which are shown in different colors. We provide four
        different viewing angles (the same as the ones used in
        \cref{fig:steady_abd}, which shows the computed $\mathcal{U}_{0}$
        domain), chosen in order convey the three-dimensional structures in as
        great detail as possible.
}
    \label{fig:steady_lcss}
\end{figure}



The computed LCSs in the unsteady flow constitute three distinct, smooth and
coherent structures, which are shown in \cref{fig:unsteady_lcss}. Although
the structures lie adjacent to eachother, they do not seem to be connected.
Thus, the different structures are indicated using disparate colors, yet
again to facilitate visual comparisons. Just like for the LCSs in the steady
flow, the computed LCSs strongly resemble subsets of the $\mathcal{U}_{0}$,
shown in \cref{fig:unsteady_abd} --- where the viewing angle of each
$\mathcal{U}_{0}$ domain subfigure is the same as that of the corresponding
subfigure in \cref{fig:unsteady_lcss}. The two largest structures apparent in
\cref{fig:unsteady_lcss} harmonize with the two dominant structures found for
the steady flow (see \cref{fig:steady_lcss}). The smallest structure appears
reasonable when considered together with the $\mathcal{U}_{0}$ domain --- shown
in \cref{fig:unsteady_abd} --- in particular, it further enhances the
tubular structures manifesting in the middle right of \cref{fig:unsteady_abd_y,%
    fig:unsteady_lcss_y}, and the middle top of \cref{fig:unsteady_abd_x,%
fig:unsteady_lcss_x}.

\begin{figure}[htpb]
    \centering
    \hspace*{\fill}
    \begin{subfigure}[b]{0.42\textwidth}
        \centering
        \resizebox{0.9\linewidth}{!}{\importpgf{figures/mpl-figs}{dep-lcss-view1-small.pgf}}
        \caption[]{{\small View along the negative $z$-axis}}
        \label{fig:unsteady_lcss_z}
    \end{subfigure}\hfill%
    \begin{subfigure}[b]{0.42\textwidth}
        \centering
        \resizebox{0.9\linewidth}{!}{\importpgf{figures/mpl-figs}{dep-lcss-view2-small.pgf}}
        \caption[]{{\small View along the positive $y$-axis}}
        \label{fig:unsteady_lcss_y}
    \end{subfigure}%
    \hspace*{\fill}

    \hspace*{\fill}
    \begin{subfigure}[b]{0.42\textwidth}
        \centering
        \resizebox{0.9\linewidth}{!}{\importpgf{figures/mpl-figs}{dep-lcss-view3-small.pgf}}
        \caption[]{{\small View along the negative $x$-axis}}
        \label{fig:unsteady_lcss_x}
    \end{subfigure}\hfill%
    \begin{subfigure}[b]{0.42\textwidth}
        \centering
        \resizebox{0.9\linewidth}{!}{\importpgf{figures/mpl-figs}{dep-lcss-view4-small.pgf}}
        \caption[]{{\small Approximately isometric view}}
        \label{fig:unsteady_lcss_isometric}
    \end{subfigure}%
    \hspace*{\fill}
    \caption[Four views of the repelling LCSs obtained for transport in the
    unsteady ABC \newline{}flow]
    {
        Four views of the repelling LCSs obtained for transport in the unsteady
        ABC flow, for the time interval $\mathcal{I}=[0,5]$ (see
        \cref{sub:unsteady_arnold_beltrami_childress_flow}). A total of
        \numprint{31} surfaces constitute three distinct, reasonably smooth and
        coherent structures, which are shown in different colors. We provide
        four different viewing angles (the same as the ones used in
        \cref{fig:unsteady_abd}, which shows the computed $\mathcal{U}_{0}$
        domain), chosen in order convey the three-dimensional structures in as
        great detail as possible.
}
    \label{fig:unsteady_lcss}
\end{figure}



