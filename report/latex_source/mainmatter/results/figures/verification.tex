\begin{figure}[htpb]
    \centering
    \hspace*{\fill}
    \begin{subfigure}[b]{0.45\textwidth}
        \centering
        \resizebox{0.9\linewidth}{!}{\importpgf{figures/mpl-figs}{verification-forced-small.pgf}}
        \caption[]{{\small Trajectories computed in the direction \\\phantom{(a)} field given by \cref{eq:revised_direction_field}}}
        \label{fig:verification_forced_outwards}
    \end{subfigure}\hfill%
    \begin{subfigure}[b]{0.45\textwidth}
        \centering
        \resizebox{0.9\linewidth}{!}{\importpgf{figures/mpl-figs}{verification-notforced-small.pgf}}
        \caption[]{{\small Trajectories of the direction field given by
                \\\phantom{(b)} \cref{eq:dynamialsystem_initialposition},
        for various weights $(a,b)$}}
        \label{fig:verification_pure_linear_combination}
    \end{subfigure}
    \hspace*{\fill}
    \caption[Trajectories orthogonal to the $\vct{\xi}_{3}$-direction field,
    superimposed onto a computed manifold surface in the steady ABC flow]
    {Trajectories orthogonal to the $\vct{\xi}_{3}$-direction field,
        superimposed onto a computed manifold surface in the steady ABC flow.
        The manifold was computed using the parameters given in
        \cref{tab:abc_manifold_params}. The trajectories were computed as
        solution curves of
        \cref{eq:revised_direction_field,eq:dynamialsystem_initialposition},
        respectively, starting at (a small circle laying within the manifold,
        centered at) the manifold epicentre $\vct{x}_{0}$. Note how none of
        the trajectories ever leave the manifold. The trajectories shown in
        (\subref*{fig:verification_forced_outwards}) cover the entire manifold,
        which is as expected, as the manifold mesh points were computed as end
        points of trajectories in a similar direction field (cf.\
        \cref{sec:revised_approach_to_computing_new_mesh_points}). The
        trajectories shown in
        (\subref*{fig:verification_pure_linear_combination}) never leave
        the manifold, but face difficulties reaching certain regions of it,
        as can be seen from locally reduced density of trajectories. For
        trajectories passing through $\vct{x}_{0}$, such
        regions are likely only accessible for very particular weights $(a,b)$.
    }
    \label{fig:verification_of_manifold_invariance}
\end{figure}

