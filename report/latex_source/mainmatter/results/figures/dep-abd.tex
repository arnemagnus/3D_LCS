\begin{figure}[htpb]
    \centering
    \begin{subfigure}[b]{0.475\textwidth}
        \centering
        \importpgf{figures/mpl-figs}{dep-abd-domain-view1-small.pgf}
        \caption[]{{\small View along the negative $z$-axis}}
        \label{fig:unsteady_abd_z}
    \end{subfigure}
    \begin{subfigure}[b]{0.475\textwidth}
        \centering
        \importpgf{figures/mpl-figs}{dep-abd-domain-view2-small.pgf}
        \caption[]{{\small View along the positive $y$-axis}}
        \label{fig:unsteady_abd_y}
    \end{subfigure}

    \begin{subfigure}[b]{0.475\textwidth}
        \centering
        \importpgf{figures/mpl-figs}{dep-abd-domain-view3-small.pgf}
        \caption[]{{\small View along the positive $x$-axis}}
        \label{fig:unsteady_abd_x}
    \end{subfigure}
    \begin{subfigure}[b]{0.475\textwidth}
        \centering
        \importpgf{figures/mpl-figs}{dep-abd-domain-view4-small.pgf}
        \caption[]{{\small Approximately isometric view}}
        \label{fig:unsteady_abd_isometric}
    \end{subfigure}
    \caption[Four views of the $\mathcal{U}_{0}$ domain obtained for transport
    in the unsteady ABC flow]
    {
        Four views of the $\mathcal{U}_{0}$ domain obtained for transport in the
        unsteady ABC flow, identified as the grid points which satisfy the LCS
        criteria \eqref{eq:lcs_condition_a},~\eqref{eq:lcs_condition_b} and~
        \eqref{eq:lcs_condition_d} (details on the implementation are available
        in
        \cref{sub:identifying_suitable_initial_conditions_for_developing_lcss}).
        In total, the domain consists of \numprint{361461} different points
        (cf.\ \cref{tab:initialconditionparams}).
}
    \label{fig:unsteady_abd}
\end{figure}

