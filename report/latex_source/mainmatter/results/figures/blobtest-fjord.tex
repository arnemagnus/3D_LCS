\begin{figure}[htpb]
    \centering
    \begin{subfigure}[b]{0.475\textwidth}
        \centering
        \importpgf{figures/mpl-figs}{blobtest-fjord-pre-small.pgf}
        \caption[]{{\small Initial state, June 1 2013, 00:00}}
        \label{fig:blobtest-fjord-pre}
    \end{subfigure}
    \begin{subfigure}[b]{0.475\textwidth}
        \centering
        \importpgf{figures/mpl-figs}{blobtest-fjord-post-small.pgf}
        \caption[]{{\small Final state, June 1 2013, 12:00}}
        \label{fig:blobtest-fjord-post}
    \end{subfigure}
    \caption[Advection of tracers in order to verify the repelling nature of
    the computed LCSs in the Førde fjord]
    {%
        Advection of tracers in order to verify the repelling nature of the
        computed LCSs in the Førde fjord. At midnight June 1, 2013, two blobs
        of initial conditions are placed at opposite sides of the most strongly
        repulsive LCS identified for flow in the Førde fjord (see
        \cref{fig:fjord_lcss}) as shown in (\subref*{fig:blobtest-pre}). The
        two blobs, and the mesh points in the parametrization of the LCS, are
        then advected in the model data for the oceanic currents briefly
        described in \cref{sub:oceanic_currents_in_the_forde_fjord}, using the
        Dormand-Prince 8(7) ODE solver, for the 12 hour time interval of
        interest, where the final state is shown in
        (\subref*{fig:blobtest-post}). Note that, although the LCS
        triangulation breaks down under the advection, the two blobs of
        particles remain on fairly compact, and remain on the opposite sides of
        the LCS, never crossing the LCS surface. This indicates that the local
        centre of strongest repulsion remains located \emph{inbetween} the
        two blobs of particles throughout --- which
        is the exact behaviour we expect for a repelling LCS.
    }
    \label{fig:blobtest-fjord}
\end{figure}

