\begin{figure}[htpb]
    \centering
    \begin{subfigure}[b]{0.475\textwidth}
        \centering
        \importpgf{figures/mpl-figs}{indep-lcss-view1-small.pgf}
        \caption[]{{\small View along the negative $z$-axis}}
        \label{fig:steady_lcss_z}
    \end{subfigure}
    \begin{subfigure}[b]{0.475\textwidth}
        \centering
        \importpgf{figures/mpl-figs}{indep-lcss-view2-small.pgf}
        \caption[]{{\small View along the positive $y$-axis}}
        \label{fig:steady_lcss_y}
    \end{subfigure}

    \begin{subfigure}[b]{0.475\textwidth}
        \centering
        \importpgf{figures/mpl-figs}{indep-lcss-view3-small.pgf}
        \caption[]{{\small View along the positive $x$-axis}}
        \label{fig:steady_lcss_x}
    \end{subfigure}
    \begin{subfigure}[b]{0.475\textwidth}
        \centering
        \importpgf{figures/mpl-figs}{indep-lcss-view4-small.pgf}
        \caption[]{{\small Approximately isometric view}}
        \label{fig:steady_lcss_isometric}
    \end{subfigure}
    \caption[Four views of the repelling LCSs obtained for transport in the
    steady ABC flow]
    {
        Four views of the repelling LCSs obtained for transport in the steady
        ABC flow, for the time interval $\mathcal{I}=[0,5]$ (see
        \cref{sub:steady_arnold_beltrami_childress_flow}). A total of
        \numprint{22} surfaces constitute two distinct, smooth and coherent
        structures, which are shown in different colors. We provide four
        different viewing angles (the same as the ones used in
        \cref{fig:steady_abd}, which shows the computed $\mathcal{U}_{0}$
        domain), chosen in order convey the three-dimensional structures in as
        great detail as possible.
}
    \label{fig:steady_lcss}
\end{figure}

