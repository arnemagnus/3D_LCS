\begin{figure}[htpb]
    \centering
    \begin{subfigure}[b]{0.475\textwidth}
        \centering
        \importpgf{figures/mpl-figs}{sinusoidal-minsep=0.2.pgf}
        \caption[]{{\small $\Delta_{\min}= 2.0\cdot10^{-1}$}}
        \label{fig:sinusoidal_minsep=0.2}
    \end{subfigure}
    \begin{subfigure}[b]{0.475\textwidth}
        \centering
        \importpgf{figures/mpl-figs}{sinusoidal-minsep=0.12.pgf}
        \caption[]{{\small $\Delta_{\min}=1.2\cdot10^{-1}$}}
        \label{fig:sinusoidal_minsep=0.12}
    \end{subfigure}

    \begin{subfigure}[b]{0.475\textwidth}
        \centering
        \importpgf{figures/mpl-figs}{sinusoidal-minsep=0.04.pgf}
        \caption[]{{\small $\Delta_{\min} = 4.0\cdot10^{-2}$}}
        \label{fig:sinusoidal_minsep=0.04}
    \end{subfigure}
    \begin{subfigure}[b]{0.475\textwidth}
        \centering
        \importpgf{figures/mpl-figs}{sinusoidal-minsep=0.01.pgf}
        \caption[]{{\small $\Delta_{\min} = 1.0\cdot10^{-2}$}}
        \label{fig:sinusoidal_minsep=0.01}
    \end{subfigure}
    \caption[Geodesic level set approximation to analytically known three-dimensional surface]
    {Geodesic level set approximation to analytically known three-dimensional surface.
        The underlying sinusoidal surface is given by
        \cref{eq:sinusoidal_definition,eq:sinusoidal_surface_params}. The
        manifolds were all computed using the parameters given in
        \cref{tab:sinusoidal_manifold_params}, with varying mesh point
        densities (given by the value of $\Delta_{\min}$). Each of the
        approximations were grown from the initial position
        $\vct{x}_{0}=(\pi,\pi,\pi)$. Note how all of the manifolds are able
        to capture the macroscale behaviour of the underlying surface,
        although the microscale behaviour is increasingly well-resolved
        when the mesh point density increases.
    }
    \label{fig:sinusoidal_manifolds}
\end{figure}

