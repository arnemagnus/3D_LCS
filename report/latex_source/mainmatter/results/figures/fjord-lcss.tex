\begin{figure}[htpb]
    \centering
    \resizebox{0.5\textwidth}{!}{\importpgf{figures/mpl-figs}{fjord-lcs-colorbar.pgf}}
    \vspace{10.0pt}

    \begin{subfigure}[b]{0.475\textwidth}
        \centering
        \importpgf{figures/mpl-figs}{fjord-lcss-view1-small.pgf}
        \caption[]{{\small Top-down view along the $z$-axis}}
        \label{fig:fjord_lcss_z}
    \end{subfigure}
    \begin{subfigure}[b]{0.475\textwidth}
        \centering
        \importpgf{figures/mpl-figs}{fjord-lcss-view2-small.pgf}
        \caption[]{{\small View along the positive $y$-axis}}
        \label{fig:fjord_lcss_y}
    \end{subfigure}

    \begin{subfigure}[b]{0.475\textwidth}
        \centering
        \importpgf{figures/mpl-figs}{fjord-lcss-view3-small.pgf}
        \caption[]{{\small View along the positive $x$-axis}}
        \label{fig:fjord_lcss_x}
    \end{subfigure}
    \begin{subfigure}[b]{0.475\textwidth}
        \centering
        \importpgf{figures/mpl-figs}{fjord-lcss-view4-small.pgf}
        \caption[]{{\small Approximately isometric view}}
        \label{fig:fjord_lcss_isometric}
    \end{subfigure}
    \caption[Four views of the repelling LCSs obtained for transport in the
    Førde fjord]
    {
        Four views of the repelling LCSs obtained for transport in the Førde
        fjord, over a time interval of \numprint{12} hours (see
        \cref{sec:flow_systems_defined_by_gridded_velocity_data}). A total
        of \numprint{110} distinct surfaces are shown, which are colored
        according to their relative repulsion averages (per
        \cref{eq:fjord_lcs_colorscaling}) using the perceptually uniform
        colormap shown at the top. We provide four different viewing angles
        (the same as the ones used in \cref{fig:fjord_abd}, which shows
        the computed $\mathcal{U}_{0}$ domain), chosen in order to convey
        the three-dimensional structures in as great detail as possible.
}
    \label{fig:fjord_lcss}
\end{figure}

