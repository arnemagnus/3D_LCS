\begin{figure}[htpb]
    \centering
    \begin{subfigure}[b]{0.475\textwidth}
        \centering
        \importpgf{figures/mpl-figs}{blobtest-pre-small.pgf}
        \caption[]{{\small Initial state, at $t=0$}}
        \label{fig:blobtest-pre}
    \end{subfigure}
    \begin{subfigure}[b]{0.475\textwidth}
        \centering
        \importpgf{figures/mpl-figs}{blobtest-post-small.pgf}
        \caption[]{{\small Final state, at $t=5$}}
        \label{fig:blobtest-post}
    \end{subfigure}
    \caption[Advection of tracers in order to verify the repelling nature of
    the computed LCSs]
    {%
        Advection of tracers in order to verify the repelling nature of the
        computed LCSs. At $t=0$, two blobs of initial conditions are placed
        at opposite sides of a repelling LCS identified in the
        steady ABC flow --- more on which to follow in
        \cref{sec:computed_lcss_in_the_abc_flow} --- as shown in
        (\subref*{fig:blobtest-pre}). The two blobs, and the mesh points in
        the parametrization of the LCS, are then advected by the velocity
        field given by \cref{eq:abc_flow,eq:abc_params_stationary} using the
        Dormand-Prince 8(7) ODE solver until $t=5$, for which the corresponding
        state is shown in (\subref*{fig:blobtest-post}). Note that, although
        the LCS triangulation breaks down under the advection, the two blobs
        of particles remain on fairly compact, and remain on the opposite
        sides of the LCS, never crossing the LCS surface. This indicates
        that the local centre of strongest repulsion remains located
        \emph{inbetween} the two blobs of particles throughout --- which
        is the exact behaviour we expect for a repelling LCS.
    }
    \label{fig:blobtest}
\end{figure}

