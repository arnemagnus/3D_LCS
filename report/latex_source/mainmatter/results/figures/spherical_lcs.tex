\begin{figure}[htpb]
    \centering
    \resizebox{0.9\textwidth}{!}{
        \importpgf{figures/mpl-figs}{spherical-lcs.pgf}
    }
    \caption[The single repelling LCS present in the purely radial velocity
    field]
    {The single repelling LCS present in the purely radial velocity field given
        by \cref{eq:spherical_velocity_field} with $r=1$. In computing
        the underlying manifolds, we used the filtering parameter $\nu=20$
        (cf.\
        \cref{sub:identifying_suitable_initial_conditions_for_developing_lcss,%
        tab:initialconditionparams})
        and otherwise the same parameter values as given in
        \cref{tab:abc_manifold_params}. To extract the LCSs, we used
        the tolerance parameter $\gamma_{\square}=1.2$, keeping only
        the LCSs whose (pseudo-)surface area was greater than or equal to
        $\mathcal{W}_{\min}=1$ (see
        \cref{sec:identifying_lcss_as_subsets_of_computed_manifolds}).
        This resulted in 3 identical (to numerical precision) copies of a
        single, spherical, repelling LCS of unit radius --- which is exactly as
        expected, given the sharp repulsion maximum at $\norm{\vct{x}}=1$, as
        shown in \cref{fig:spherical_lm3}.
    }
    \label{fig:spherical_lcs}
\end{figure}

