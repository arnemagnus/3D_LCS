\begin{table}[htpb]
    \centering
    \caption[Parameters used to compute manifolds, and subsequently
    repelling LCSs, in the Førde fjord]
    {
        Parameters used to compute manifolds, and subsequently repelling
        LCSs, in the Førde fjord (see
        \cref{sec:flow_systems_defined_by_gridded_velocity_data}). Note that
        the interset distances $\Delta_{i}$ were dynamically altered, as
        described in
        \cref{sub:a_curvature_based_approach_to_determining_interset%
        _separations}. Accordingly, only the \emph{first} interset distance,
        $\Delta_{1}$, was set explicitly.  Moreover, as the domain of interest
        is scaled in units of metre, so too are $\delta_{\text{init}}$,
        $\Delta_{\min}$, $\Delta_{\max}$ and $\Delta_{1}$; whereas
        $\mathcal{W}_{\min}$ is given in square metre.
}
    \label{tab:fjord_manifold_params}
    \begin{tabular}{ccc}
        \toprule
        Parameter & Value & Description\\
        \midrule
        $\delta_{\text{init}}$ & $10^{-1}$ %
        & \makecell{Separation of innermost geodesic level set \\
        from the manifold epicentre, cf. \cref{fig:innermost_levelset}}%
        \\[9pt]
        %
        $\Delta_{\min}$, $\Delta_{\max}$
        & $2$, $8$ %
        & \makecell{Boundaries for interpoint \\separations (details
        found in \cref{sub:maintaining_mesh_point_density})}%
        \\[9pt]
        %
        $\Delta_{1}$ %
        & $2\Delta_{\min}$ %
        & \makecell{Interset distance used to compute the second \\ geodesic
        level set (see
        \cref{sec:revised_approach_to_computing_new_mesh_points,%
        sub:a_curvature_based_approach_to_determining_interset_separations})}%
        \\[9pt]
        %
        $\gamma_{\|}$ %
        & $10^{-4}$ %
        & \makecell{Tolerance for detecting regions in which
        $\vct{\xi}_{3}$\\ is (anti-)parallel to $\vct{t}_{i,j}$
    (see \cref{eq:revised_xi3_tan_parallel})}
        \\[9pt]
        %
        $\gamma_{\Delta}$ %
        & $5\cdot10^{-3}$ %
        & \makecell{Tolerance for the separation of a mesh point\\ from
        its ancestor (per \cref{eq:revised_dist_tol})}
        \\[9pt]
        $\gamma_{\text{arc}}$ %
        & 5 %
        & \makecell{Sets an upper limit to trajectory lengths as \\
        $\gamma_{\text{arc}}\Delta_{i}$ (briefly mentioned in
        \cref{sub:computing_pseudoradial_trajectories_directly})}
        \\[9pt]
        %
        $\gamma_{\circlearrowleft}$ %
        & $7\cdot10^{-1}$
        & \makecell{Sets an upper limit to the extent of loop-like\\
        segments of any level set (see
        \cref{sub:limiting_the_accumulation_of_numerical_noise})}
        \\[9pt]
        %
        \makecell[c]{$\alpha_{\uparrow}$\\ $\alpha_{\downarrow}$ \\[1.5pt]%
        ${(\Delta\alpha)}_{\uparrow}$, ${(\Delta\alpha)}_{\downarrow}$} &
        \makecell[c]{$8.7\cdot10^{-2}\;\si{\radian}$%
            \phantom{2}$({5}\si{\degree})$\\ %
            ${4.4\cdot10^{-1}}\;\si{\radian}$ $({25}\si{\degree})$\\[1.5pt]%
        $2\Delta_{\min}\alpha_{\uparrow}$, %
        $2\Delta_{\min}\alpha_{\downarrow}$}%
        & \makecell[c]{Used in a curvature-based approach to adjust\\
        interset distances (outlined in
        \cref{sub:a_curvature_based_approach_to_determining_interset%
        _separations})}
        \\[18pt]
        %
        $\gamma_{\cap}$ &
        $5$ &
        \makecell[c]{Used for terminating the expansion of \\
        self-intersecting manifolds (cf.\
        \cref{sub:continuous_self_intersection_checks})}
        \\[9pt]
        %
        $\gamma_{\square}$ &
        $1.2$ &
        \makecell[c]{Relaxation parameter for extracting LCSs from\\ the
            computed manifolds (see
        \cref{sec:identifying_lcss_as_subsets_of_computed_manifolds})}
        \\[9pt]
        %
        $\mathcal{W}_{\text{min}}$ &
        \numprint{20000} &
        \makecell[c]{Filters away the smallest LCSs measured in\\
        (pseudo-)surface area (see
        \cref{sec:identifying_lcss_as_subsets_of_computed_manifolds})}
        \\
        \bottomrule
    \end{tabular}
\end{table}

