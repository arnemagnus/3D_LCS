\begin{table}[htpb]
    \centering
    \caption[Parameter values used to approximate an analytically known
    three-dimensional surface by the method of geodesic level sets]
    {Parameter values used to approximate an analytically known
        three-dimensional surface (see
        \cref{eq:sinusoidal_definition,eq:sinusoidal_surface_params}) by
        the method of geodesic level sets. Note that the interset distances
        $\Delta_{i}$ were dynamically altered, as described in
        \cref{sub:a_curvature_based_approach_to_determining_interset%
        _separations}. Accordingly, only the \emph{first} interset distance,
        $\Delta_{1}$, was set explicitly. Moreover, while we varied the
        overall mesh point density by altering $\Delta_{\min}$ and
        $\Delta_{\max}$, we kept the ratio between them constant.
    }
    \label{tab:sinusoidal_manifold_params}
    \begin{tabular}{ccc}
        \toprule
        Parameter & Value & Description\\
        \midrule
        $\delta_{\text{init}}$ & $10^{-3}$ %
        & \makecell{Separation of innermost geodesic level set \\
        from the manifold epicentre, cf. \cref{fig:innermost_levelset}}%
        \\[9pt]
        %
        $\Delta_{\min}$, $\Delta_{\max}$
        & $\dfrac{\Delta_{\max}}{\Delta_{\min}} = 4$ %
        & \makecell{(Variable) boundaries for interpoint \\separations (details
        found in \cref{sub:maintaining_mesh_point_density})}%
        \\[9pt]
        %
        $\Delta_{1}$ %
        & $2\Delta_{\min}$ %
        & \makecell{Interset distance used to compute the second \\ geodesic
        level set (see \cref{sec:revised_approach_to_computing_new_mesh_points,%
        sub:a_curvature_based_approach_to_determining_interset_separations})}%
        \\[9pt]
        %
        $\gamma_{\|}$ %
        & $10^{-4}$ %
        & \makecell{Tolerance for detecting regions in which
        $\vct{\xi}_{3}$\\ is (anti-)parallel to $\vct{t}_{i,j}$
    (see \cref{eq:revised_xi3_tan_parallel})}
        \\[9pt]
        %
        $\gamma_{\Delta}$ %
        & $5\cdot10^{-3}$ %
        & \makecell{Tolerance for the separation of a mesh point\\ from
        its ancestor (per \cref{eq:revised_dist_tol})}
        \\[9pt]
        $\gamma_{\text{arc}}$ %
        & 5 %
        & \makecell{Sets an upper limit to trajectory lengths as \\
        $\gamma_{\text{arc}}\Delta_{i}$ (briefly mentioned in
        \cref{sub:computing_pseudoradial_trajectories_directly})}
        \\[9pt]
        %
        $\gamma_{\circlearrowleft}$ %
        & $7\cdot10^{-1}$
        & \makecell{Sets an upper limit to the extent of loop-like\\
        segments of any level set (see
        \cref{sub:limiting_the_accumulation_of_numerical_noise})}
        \\[9pt]
        %
        \makecell[c]{$\alpha_{\uparrow}$\\ $\alpha_{\downarrow}$ \\[1.5pt]%
        ${(\delta\alpha)}_{\uparrow}$, ${(\delta\alpha)}_{\downarrow}$} &
        \makecell[c]{$8.7\cdot10^{-2}\;\si{\radian}$ \phantom{2}$(5\si{\degree})$\\ %
            $4.4\cdot10^{-1}\;\si{\radian}$ $(25\si{\degree})$\\[1.5pt]%
        $2\delta_{\max}\alpha_{\uparrow}$, $2\delta_{\min}\alpha_{\downarrow}$}%
        & \makecell[c]{Used in a curvature-based approach to adjust\\
        interset distances (outlined in
        \cref{sub:a_curvature_based_approach_to_determining_interset_separations})}
        \\[18pt]
        \bottomrule
    \end{tabular}
\end{table}

