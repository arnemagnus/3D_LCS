When analyzing complex dynamical systems, such as the nonlinear many-body
problems arising in transport phenomena by virtue of oceanic currents or
atmospheric winds, the conventional approach to predicting future states
by means of simulating the trajectories of phase points is frequently
insufficient. This is due to the resulting predictions being very sensitive to
small changes in time and initial positions. One way of addressing the delicate
dependence on initial conditions is to run ensembles of different models for
the same underlying physical systems, with increasing spatial and temporal
resolution. Another is to run ensembles of simulations for the same physical
model, using perturbed initial conditions. These sorts of approaches are made
possible because the fundamental dynamics are known --- yet, for complex
transport systems, the computational cost quickly grows beyond the available
resources, in terms of computation time or memory. For many practical purposes,
however, microscopic details matter little in comparison to the overarching
trends in the system, which means that a less ambitious approach, merely aiming
to understand the macroscopics of the transport phenomenon, is often
justifiable.

At the turn of the millennium, the concept of Lagrangian coherent structures
saw the light of day, emerging from the intersection between nonlinear
dynamics, that is, the underlying mathematical principles of chaos theory, and
fluid dynamics \parencite{haller2000lagrangian}. These provide a new framework
for understanding transport phenomena in conceptual fluid flow systems.
Lagrangian coherent structures can be described as time-evolving ``landscapes''
in a multidimensional space, which dictate macroscopic flow patterns in
dynamical systems. In particular, such structures define the interfaces of
dynamically distinct, invariant regions. An invariant region in fluid dynamics
is characterized as a domain where all particle trajectories that originate
within the region, remain in it, although the region itself can move and deform
with time. So, simply put; Lagrangian coherent structures enable us to make
predictions regarding the future states of flow systems.

There are two possible perspectives regarding the description of fluid flow.
The Eulerian approach is to consider the properties of a flow field at a set of
fixed points in time and space. An example is the concept of velocity fields,
which describe the local and instantaneous velocities at all points within
their domains. The Lagrangian point of view, on the other hand, concerns the
developing velocity of each fluid element along their paths, as they are
transported by the flow. Unlike the Eulerian perspective, the Lagrangian
mindset is objective, as in frame-invariant. That is, properties of Lagrangian
fields are unchanged by time-dependent translations and rotations of the
reference frame. For unsteady flow systems, which are more common than steady
flow systems in nature, there exists no self-evident preferred frame of
reference. Thus, any transport-dictating dynamical structures should
hold for \emph{any} choice of reference frame. This is the main
rationale for which \emph{Lagrangian}, rather than \emph{Eulerian}, coherent
structures have been pursued.

A generic flow system can be described as a structure whose state depends on
flowing streams of energy, material or information. Conventional examples of
flow systems include the transport of physical properties such as pressure,
temperature or matter in fluids, and the transport of charge in electrical
currents. A large variety of phenomena can reasonably be modelled as flow
systems, such as the classical harmonic oscillator, or the interaction between
predator and prey in closed systems
\parencite[parts I--II]{strogatz2014nonlinear}. In doing so, valuable pieces
of insight can be obtained from well-understood properties of generic flows.
In recent years, analyses based upon Lagrangian coherent structures have been
conducted for a variety of naturally occuring phenomena which are not
commonly considered as flow systems. Two prominent examples are how
\textcite{olascoaga2008tracing} used Lagrangian coherent structures in order
to forecast the development of toxic algae in the ocean, and
\textcite{ali2007lagrangian} used Lagrangian coherent structures to predict
the formation and stability of human crowd patterns. As these examples
emphasize, the theory of Lagrangian coherent structures is applicable to a
wide range of systems.

The focus of this project has been the computation of Lagrangian coherent
structures in three-dimensional flow systems. Although the framework for
detecting Lagrangian coherent structures is mathematically valid for any number
of dimensions, little work has previously been dedicated to three-dimensional
systems. The two-dimensional case has been treated extensively (consider, for
instance, the works of \textcite{haller2000lagrangian},
\textcite{farazmand2012computing} and \textcite{onu2015lcstool}), as many
real-world transport systems --- such as the transport of garbage patches, or
remnants of oil spills, by oceanic surface currents --- can reasonably be
treated as two-dimensional. Two-dimensional analysis is, however, not always
suffucient; the scattering of e.g.\ volcanic ashes by the Earth's wind systems
is an example of an irrefutably three-dimensional transport phenomenon.
As concrete areas of application, identifying the Lagrangian coherent
structures present in the underlying circulation allows for predicting the
spread of oil spills on the ocean surface in order to isolate them and
accelerate the cleanup process before the oil is able to reach the
coastline, or providing airline companies an opportunity to divert flights
which would otherwise be exposed to volcanic ash clouds. Put simply, Lagrangian
flow analysis could potentially mitigate humanitarian and natural calamities
alike.

The limited extent to which analysis of three-dimensional flow systems with the
intention of identifying Lagrangian coherent structures has previously
conducted, is emphasized by the common practice being to compute a series of
two-dimensional projections followed by joining them together using
some kind of interpolation method (see the works of
\textcite{blazevski2014hyperbolic} and \textcite{oettinger2016autonomous}).
Although yielding three-dimensional surfaces, this approach potentially
neglects intricacies pertaining to fully three-dimensional transport phenomena.
A plausible justification for using this approach is the increase in complexity
when including the depth dimension, further increasing the amount of
computational resources required for full-scale analysis. Moreover --- as
previously mentioned --- many significant transport systems can reasonably be
treated as two-dimensional. Hence, the extent to which it is appropriate to
substitute analysis of two-dimensional Lagrangian coherent structures with that
of their three-dimensional counterparts remains an open question.

For this project, we aimed to adapt the method of geodesic level sets for
computing manifolds of three-dimensional vector fields, as outlined by
\textcite{krauskopf2005survey} (first introduced by
\textcite{krauskopf2003computing}), to identify Lagrangian coherent structures
as barriers to transport in three-dimensional flow systems. This involved
extensive conceptual derivations, in addition to quite a bit of programming
ingenuity. Accordingly, the project is best classified as one of
\emph{computational physics}, which might be viewed as complementary to the
more traditional branches of \emph{theoretical} and \emph{experimental}
physics.

This thesis is structured based on the idea that readers possessing
at least an undergraduate level of knowledge of physics and mathematics, in
addition to a rudimentary understanding of programming, should be able to
understand and repeat the numerical investigations which have been conducted.
To this end, the immediately forthcoming chapter contains a description of the
numerical integration and interpolation schemes which we used to simulate
general transport phenomena, in addition to a generic yet brief mathematical
description of the kind of flow systems considered --- and the Lagrangian
coherent structures situated therein, based on their variational theory. In the
ensuing chapter, we present the different transport models we considered,
followed by a rigorous description of how we implemented the variational
principles of Lagrangian coherent structures in a variation of the method of
geodesic level sets, in addition to how we made sure to utilize the available
computational resources in an efficient manner throughout. Then, we present a
few tests we used in order to verify that the computed three-dimensional
structures behave as expected for Lagrangian coherent structures, closely
followed by a presentation of the Lagrangian coherent structures we identified
in the various transport models. Lastly, we discuss the strengths and
weaknesses of using our variation of the method of geodesic level sets to
compute three-dimensional Lagrangian coherent structures, before drawing
conclusions on the project as a whole.
