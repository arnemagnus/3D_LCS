A generic flow system can be described as a structure whose state depends on
flowing streams of energy, material, or information. Traditional examples of
such systems are the transport of properties such as pressure, temperature or
matter in fluids, and the transport of charge in electrical currents. However,
valuable insight into various phenomena, including, but not limited to, algal
blooms in the ocean and crowd patterns formed by humans, can be obtained by
regarding them as flow systems.

Lagrangian coherent structures (henceforth abbreviated to LCSs) can be
described as `landscapes' within a multidimensional phase space, which exert a
major influence upon the flow patterns in dynamical systems. Compared to the
conventional approach of increasing the spatial and temporal resolution of the
model(s) involved in numerical simulations, LCSs provide a simplified means of
predicting the future states of complex systems. In this context, a complex
system is a system which exhibits sensitive dependence on initial conditions.
From their variational theory, hyperbolic LCSs are identified as locally most
repelling or attracting material surfaces. Somewhat simplified, such surfaces
can be considered as generalized trajectories, created by the underlying
transport phenomenon. When investigating transport systems, hyperbolic LCSs are
of particular interest, as they form the skeleton of observable flow patterns.

This project is centered around the computation of hyperbolic LCSs in
three-dimensional flows, using an adapted version of the method of geodesic
level sets for computing manifolds of three-dimensional vector fields.
