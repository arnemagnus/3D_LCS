A generic flow system can be described as a structure whose state depends on
flowing streams of energy, material, or information. Traditional examples of
such systems are the transport of properties such as pressure, temperature and
matter in fluids, and the transport of charge in electrical currents. However,
valuable insight into various phenomena, including, but not limited to, algal
blooms in the ocean and crowd patterns formed by humans, can be obtained by
regarding them as flow systems.

Lagrangian coherent structures (henceforth abbreviated to LCSs) can be
described as `landscapes' within a multidimensional phase space, which exert a
major influence upon the flow patterns in dynamical systems. Compared to the
conventional approach of increasing the resolution of the model(s) involved in
numerical simulations, LCSs provide a simpler means of predicting future states
of complex systems. In this context, a complex system is a system which
exhibits sensitive dependence on initial conditions. From their variational
theory, hyperbolic LCSs are identified as locally most repelling or attracting
material surfaces. Somewhat simplified, such surfaces can be considered as
generalized trajectories, created by the underlying transport phenomenon. When
investigating transport systems, hyperbolic LCSs are of particular interest, as
they form the skeleton of observable flow patterns.

Several important, naturally occuring transport phenomena --- such as the
spread of contaminations by oceanic surface currents --- can reasonably be
approximated as planar. This is one possible reason why previous
LCS research has mostly been conducted for two-dimensional systems. To the
extent that LCS analysis for three-dimensional flows has hitherto been
conducted, a common approach has been to combine a set of two-dimensional
projections, resulting in quasi-three-dimensional surfaces. This project is
centered around combining the underlying variational principles of hyperbolic
LCSs with an acknowledged technique for computing invariant manifolds of vector
fields, with a view to developing a robust framework for computing hyperbolic
LCSs in three-dimensional flow systems.

Our approach is described in detail, from its theoretical fundament, through
how we incorporated the underlying LCS theory into the computation of manifolds
and subsequent LCS surfaces, to some clever optimizations regarding
computational resource management. Using a few reference cases, we demonstrate
the efficacy of our method in reproducing simple three-dimensional manifolds
and LCSs. Moreover, the LCSs obtained for a simple, yet chaotic flow
system are shown to correspond well with those obtained for a time perturbed
version of the same system. This suggests that the computed LCS surfaces are
quite robust, and that the innate inaccuracies of model data for simulating
transport phenomena need not be a hindrance for applying LCS analysis to
real-world systems.

It remains to be seen whether or not computing truly three-dimensional LCS
surfaces yields a sufficient increase in descriptive power, compared to the
aforementioned quasi-three-dimensional approach, to warrant the increase in
conceptual complexity and computational resource consumption. This balancing
act might be context sensitive.
