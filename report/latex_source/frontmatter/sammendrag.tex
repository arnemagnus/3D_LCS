\begingroup
\vspace{3mm}
\abnormalparskip{0.5\baselineskip}
Et generisk strømningssystem kan beskrives som et system hvis tilstand
avhenger av flyt av energi-, material-, eller informasjonsstrømmer.
Tradisjonelle eksempler på slike systemer er transport av trykk, temperatur og
materie i fluidstrømninger, og ladningstransport forårsaket av elektriske
strømmer. Interessante sider ved andre fenomen, som algeoppblomstring, og
mønsterdannelser i folkemengder, kan avsløres ved å betrakte dem som
strømningssystemer.

Lagrange-koherente strukturer (heretter forkortet til LKSer) kan beskrives som
<<landskap>> i et flerdimensjonalt rom, som utøver stor innflytelse på
strømningsmønstre i dynamiske system. Hva gjelder prediksjoner for fremtidige
tilstander i komplekse systemer, kan bruk av LKSer utgjøre et enklere
analyseverktøy enn den tradisjonelle tilnærmingen som ofte går ut på iterativt
øke oppløsningen i modellene som ligger til grunn for numeriske simuleringer.
I denne sammenheng betegner et komplekst system
et system som er svært sensitivt til dets initialbetingelser. Fra deres
underliggende variasjonsteori kjennetegnes hyperbolske LKSer som de lokalt
sterkest frastøtende eller tiltrekkende materialoverflatene i systemet. Noe
forenklet kan denne typen overflater betraktes som en generalisering av banene
det underliggende transportfenomenet forårsaker. Ved analyse av
transportsystemer er hyperbolske LKSer spesielt interessante, ettersom disse
utgjør skjelettet av observable strømningsmønstre.

Flere viktige, naturlige transportfenomen -- deriblant spredning av
forurensning ved havoverflatestrømninger -- kan betraktes som
tilnærmet plane. Dette er en mulig årsak til at tidligere utført forskning på
LKSer i stor grad har omhandlet todimensjonale system; en annen kan være at å
introdusere en tredje dimensjon øker den underliggende matematiske
kompleksiteten betraktelig. I den grad LKSer i tredimensjonale strømninger har
blitt undersøkt så langt, har en vanlig tilnærming vært å sette sammen et sett
av planprojeksjoner, hvilket resulterer i kvasi-tredimensjonale overflater.
Dette prosjektet går ut på å kombinere de underliggende variasjonsprinsippene
for hyperbolske LKSer med en anerkjent teknikk for å beregne invariante
mangfoldigheter for vektorfelt, med den hensikt å utvikle et robust rammeverk
for å beregne hyperbolske LKSer i tredimensjonale strømningssystem.

Fremgangsmåten vår beskrives i detalj, fra dens teoretiske fundament, via
hvordan vi inkorporerte den underliggende LKS-teorien i beregninger av
mangfoldigheter og videre LKS-overflater, til noen fikse løsninger for å
begrense ressursforbruket. Ved analyse av noen referansesystem demonstrerer vi
metodens velegnethet for beregning av enkle tredimensjonale mangfoldigheter og
LKSer. Videre stemmer LKS-overflatene vi finner i et enkelt, men kaotisk
strømningssystem godt overens med overflatene som avdekkes i en tidsperturbert
versjon av det samme systemet. Dette tyder på at de beregnede LKS-overflatene
er robuste, og at den uunngåelige unøyaktigheten i modelldata for å simulere
reelle transportfenomen ikke trenger å stå til hinder for å anvende LKS-analyse
på virkelige strømninger.

Det gjenstår å se hvorvidt beregning av <<ekte>> tredimensjonale LKS-overflater
resulterer i tilstrekkelig økt innsikt, sammenlignet med den ovennevnte
kvasi-tredimensjonale tilnærmingen, til å forsvare den økte kompleksiteten og
det økte ressursforbruket. Denne avveiningen vil muligens
avhenge av kontekst.
\endgroup
