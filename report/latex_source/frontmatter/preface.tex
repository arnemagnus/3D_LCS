The submission of this thesis signifies my completion of all formal
requirements for acquiring the degree of MSc within the field of ``Physics and
Mathematics'', with a specialization in Applied Physics, at the Norwegian
University of Science and Technology. This thesis accounts for a total of 30
ECTS. All of the underlying work was performed in Trondheim, during the spring
semester of 2018 --- that is, my \nth{10} and (for now) final semester as a
university student --- under the supervision of Assoc. Prof. Tor Nordam. As a
consequence of our close collaboration for the work on our master's projects,
fellow student Simon Nordgreen and I present the same results in our respective
master's theses.

This thesis builds upon the work I did for my specialization project, which was
carried out during the fall semester of 2017 \parencite{loken2017sensitivity}.
The target audience possesses an understanding of physics and mathematics
corresponding to what is expected prior to the enrollment in a master's level
physics program. Intermediate proficiency with regards to numerical programming
pertaining to simulations would be advantageous. The program code which was
developed for this project --- a combination of modern Fortran, C++ and
Python --- is hosted at
\texttt{github}\footnote{\url{https://github.com/arnemagnus/3d_lcs}}; guidance
can be provided upon request. Familiarity with variational principles --- in
particular regarding stretch and strain in material flows --- is not a
necessary prerequisite, as all advanced concepts are explained in detail prior
to application.


\begin{minipage}[t]{\textwidth}
    \begin{flushright}
    Trondheim, June 2018\\
    Arne Magnus Tveita Løken
    \end{flushright}
\end{minipage}
