\documentclass{standalone}
\usepackage{tikz}
\usepackage{tikz-3dplot}
\usepackage{libertine}
\usepackage[libertine]{newtxmath}
\usetikzlibrary{calc,decorations,hobby}

% A simple empty decoration, that is used to ignore the last bit of the path
\pgfdeclaredecoration{ignore}{final}
{
\state{final}{}
}

% Declare the actual decoration.
\pgfdeclaremetadecoration{middle}{initial}{
    \state{initial}[
        width={0pt},
        next state=middle
    ]
    {\decoration{moveto}}

    \state{middle}[
        width={\pgfdecorationsegmentlength*\pgfmetadecoratedpathlength},
        next state=final
    ]
    {\decoration{curveto}}

    \state{final}
    {\decoration{ignore}}
}

% Create a key for easy access to the decoration
\tikzset{middle segment/.style={decoration={middle},decorate, segment length=#1}}

\def\getangle(#1)(#2)#3{%
  \begingroup%
    \pgftransformreset%
    \pgfmathanglebetweenpoints{\pgfpointanchor{#1}{center}}{\pgfpointanchor{#2}{center}}%
    \expandafter\xdef\csname angle#3\endcsname{\pgfmathresult}%
  \endgroup%
}

\begin{document}
\tdplotsetmaincoords{60}{120}

\begin{tikzpicture}[tdplot_main_coords]
	\pgfmathsetmacro{\innerscl}{3}
	\pgfmathsetmacro{\outerscl}{5}
    \pgfmathsetmacro{\num}{18} % = Degree span / separation, outer for loop

    % Inner geodesic level set
    \foreach [count = \i] \a in {10,30,...,350}%
    {%
        \coordinate (i\i) at ( {\innerscl*cos(\a)} , {\innerscl*sin(\a)} , 0) ;
    }%
    \draw[stroke=black!80,thin,dashed] (i1) to [ curve through ={(i2) .. (i3) .. (i4) .. (i5) .. (i6) .. (i7) .. (i8) .. (i9) .. (i10) .. (i11) .. (i12) .. (i13) .. (i14) .. (i15) .. (i16) .. (i17) .. (i18)}] (i1);
    \foreach \i in {1,...,\num}%
    {%
        \draw[stroke=black!80, fill=gray!60] (i\i) circle (3pt);
    }%

    % Outer geodesic level set
    \foreach [count = \i] \a in {10,30,...,350}%
    {%
        \coordinate (o\i) at ( {\outerscl*cos(\a)} , {\outerscl*sin(\a)} , 0.2) ;
    }%
    \draw[stroke=black!80,thin,dashed] (o1) to [ curve through ={(o2) .. (o3) .. (o4) .. (o5) .. (o6) .. (o7) .. (o8) .. (o9) .. (o10) .. (o11) .. (o12) .. (o13) .. (o14) .. (o15) .. (o16) .. (o17) .. (o18)}] (o1);
    \foreach \i in {1,...,\num}%
    {%
        \draw[stroke=black!80, fill=gray!60] (o\i) circle (3pt);
    }%



\end{tikzpicture}

\end{document}

