\documentclass{standalone}
\usepackage{tikz}
\usepackage{tikz-3dplot}
\usepackage{libertine}
\usepackage[libertine]{newtxmath}
\usetikzlibrary{calc,decorations}

% A simple empty decoration, that is used to ignore the last bit of the path
\pgfdeclaredecoration{ignore}{final}
{
\state{final}{}
}

% Declare the actual decoration.
\pgfdeclaremetadecoration{middle}{initial}{
    \state{initial}[
        width={0pt},
        next state=middle
    ]
    {\decoration{moveto}}

    \state{middle}[
        width={\pgfdecorationsegmentlength*\pgfmetadecoratedpathlength},
        next state=final
    ]
    {\decoration{curveto}}

    \state{final}
    {\decoration{ignore}}
}

% Create a key for easy access to the decoration
\tikzset{middle segment/.style={decoration={middle},decorate, segment length=#1}}

\def\getangle(#1)(#2)#3{%
  \begingroup%
    \pgftransformreset%
    \pgfmathanglebetweenpoints{\pgfpointanchor{#1}{center}}{\pgfpointanchor{#2}{center}}%
    \expandafter\xdef\csname angle#3\endcsname{\pgfmathresult}%
  \endgroup%
}

\begin{document}
\tdplotsetmaincoords{60}{20}

\begin{tikzpicture}[tdplot_main_coords]
    % Macro for the unit vector scale (i.e., the length of the unit vectors)
    \pgfmathsetmacro{\usclx}{1.35};
    \pgfmathsetmacro{\uscly}{1.35};
    \pgfmathsetmacro{\usclz}{1.35};

    % Macro for the axes parallel to the unit vectors
    \pgfmathsetmacro{\asclx}{5}
    \pgfmathsetmacro{\ascly}{5}
    \pgfmathsetmacro{\asclz}{3}

    % Macro for the vector separating origin point and aim point
    \pgfmathsetmacro{\x}{0.78*\asclx}
    \pgfmathsetmacro{\y}{0.64*\ascly}
    \pgfmathsetmacro{\z}{0.78*\asclz}

    % Set coordinates for origin point
    \coordinate (r) at (0,0,0);

    % Set coordinates for aiming point
    \coordinate (ra) at ($(r) + (\x,\y,\z)$);

    % Set coordinates for end points of unit vectors
    \coordinate (xi1) at ($(r) + (\usclx,0,0)$);
    \coordinate (xi2) at ($(r) + (0,\uscly,0)$);
    \coordinate (xi3) at ($(r) + (0,0,\usclz)$);

    % Set coordinates of orthogonal projection of aiming vector
    \coordinate (rort) at ($(r) + (0,0,\z)$);

    % Set coordinates of parallel projection of aiming vector
    \coordinate (rpar) at ($(r) + (\x,\y,0)$);

    % Shade plane spanned by xi1 and xi2
    \draw[fill=gray!15,draw opacity = 0] (r) -- ($(r)+(\asclx,0,0)$) -- ($(r)+(\asclx,\ascly,0)$) -- ($(r)+(0,\ascly,0)$) -- cycle;

    % Draw axes parallel to unit vectors
    \draw[thin,color=gray!60] (r) -- ($(r) + (\asclx,0,0)$); % xi1
    \draw[thin,color=gray!60] (r) -- ($(r) + (0,\ascly,0)$); % xi2
    \draw[thin,color=gray!60] (r) -- ($(r) + (0,0,\asclz)$); % xi3

    % Draw unit vectors
	\getangle(r)(xi1)b;
    \draw[->,very thick,color=black!90] (r) -- (xi1) node[midway,below,rotate=\angleb]{$\vec{\xi}_{1}(\vec{r})$};
	\getangle(r)(xi2)b;
    \draw[->,very thick,color=black!90] (r) -- (xi2) node[above,rotate=\angleb]{$\vec{\xi}_{2}(\vec{r})$};
    \draw[->,very thick,color=black!90] (r) -- (xi3) node[midway,left]{$\vec{\xi}_{3}(\vec{r})$};

    % Draw guide lines to orthogonal component of aim vector
    \draw[thin, dashed,color=black!65] (ra) -- (rort);

    % Draw guide lines to parallel component of aim vector
    \draw[thin, dashed] (ra) -- (rpar);

    % Draw nonmodified aim vector
    \draw[->,thin,middle segment = 0.99] (r) -- (ra);

    % Draw parallel component of aim vector
	\getangle(r)(rpar)b;
    \draw[->,thick] (r) -- (rpar) node[midway,below,rotate=\angleb]{$\vec{f}(\vec{r},\vec{r}_{\mathrm{aim}})$};

    % Draw origin point
    \draw[fill=white,stroke=black!90,fill opacity=1] (r) circle(2pt) node[below] {$\vec{r}$};

    % Draw aim point
    \draw[fill=gray!20,stroke=black!90] (ra) circle(2pt) node[right]{$\vec{r}_{\mathrm{aim}}$};

\end{tikzpicture}

\end{document}

