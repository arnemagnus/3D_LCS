\documentclass[crop]{standalone}
\usepackage{tikz}
\usepackage[]{tikz-3dplot}
\usepackage{pgfplots}
\pgfplotsset{compat=1.15}
\usepackage[]{amsmath}
\usepackage[]{libertine}
\usepackage[libertine]{newtxmath}
\usepackage[]{bm}
\usepackage[]{physics}
% Macros for greek letters in roman style, in math mode
\DeclareRobustCommand{\mathup}[1]{%
\begingroup\ensuremath\changegreek\mathrm{#1}\endgroup}
\DeclareRobustCommand{\mathbfup}[1]{%
\begingroup\ensuremath\changegreek\bm{\mathrm{#1}}\endgroup}


\makeatletter
\def\changegreek{\@for\next:={%
        alpha,beta,gamma,delta,epsilon,zeta,eta,theta,iota,kappa,lambda,mu,nu,%
        xi,pi,rho,sigma,tau,upsilon,phi,chi,psi,omega,varepsilon,varpi,%
    varrho,varsigma,varphi}%
\do{\expandafter\let\csname\next\expandafter\endcsname\csname\next up\endcsname}}
\makeatother

% Define vectors in bold, roman, lowercase font
\newcommand{\vct}[1]{\ensuremath{\mathbfup{\MakeLowercase{#1}}}}

% Define unit vectors in bold, roman, lowercase font, with hats
\newcommand{\uvct}[1]{\ensuremath{\mathbfup{\hat{\MakeLowercase{#1}}}}}

% Define matrices in bold, roman, uppercase font
\newcommand{\mtrx}[1]{\ensuremath{\mathbfup{\MakeUppercase{#1}}}}
\usetikzlibrary{%
    angles,%
    arrows.meta,%
    backgrounds,%
    calc,%
    decorations,%
    fit,%
    hobby,%
    patterns,%
    positioning,%
    quotes
}


\begin{document}

\tdplotsetmaincoords{60}{30}

\begin{tikzpicture}[tdplot_main_coords]
    \pgfmathsetmacro{\innerradius}{5}
    \pgfmathsetmacro{\outerradius}{7.5}
    \pgfmathsetmacro{\size}{2.5}
    \pgfmathsetmacro{\innerarclowerangle}{35}
    \pgfmathsetmacro{\innerarcupperangle}{100}
    \pgfmathsetmacro{\outerarclowerangle}{35}
    \pgfmathsetmacro{\outerarcupperangle}{95}

    % Place the set of initial points
    % Innermost level set
    \foreach [count=\i] \ang in {48,65,87}%
    {%
        \draw[draw=black!80,fill=gray!20] ( {\innerradius*cos(\ang)}, {\innerradius*sin(\ang)}, 0 ) circle (\size pt) coordinate (i\i);
    }
    % Outermost level set prior to insertion of new point
    \foreach [count=\i] \ang in {47,64,92}%
    {%
        \draw[draw=black!80,fill=gray!20] ( {\outerradius*cos(\ang)}, {\outerradius*sin(\ang)}, 0 ) circle (\size pt) coordinate (o\i);
    }
    % Define point for ghost trajectory
    \draw[draw=black!80,fill=gray!15] ( {\outerradius*cos(78)}, {\outerradius*sin(78)}, 0) circle (\size pt) coordinate (new);



    % Place nodes of inner level set
    \node[below left = -2.5pt and -5pt of i1] {\footnotesize $\mathcal{M}_{i,j-1}$};
    \node[below left = -2.5pt and -5pt of i2] {\footnotesize $\mathcal{M}_{i,j}$};
    \node[below left = -2.5pt and -5pt of i3] {\footnotesize $\mathcal{M}_{i,j+1}$};

    % Place nodes of outer level set
    \node[above right = 0pt and -5pt of o1] {\footnotesize $\mathcal{M}_{i+1,j-1}$};
    \node[above right = 0pt and -5pt of o2] {\footnotesize $\mathcal{M}_{i+1,j}$};
    \node[above right = -2.5pt and -5pt of new] {\footnotesize $\mathcal{M}_{i+1,j+\frac{1}{2}}$};
    \node[above right = -2.5pt and 0pt of o3] {\footnotesize $\mathcal{M}_{i+1,j+1}$};

    % Draw interpolation triangles
    \begin{scope}[on background layer]
        \draw[draw=none,pattern=north east lines, pattern color = gray!20] (i3) -- (new) -- (o3) -- cycle;
        \draw[draw=black!65,dashed] (i3) -- (new) -- (o3) -- cycle;

        \draw[draw=none,pattern=north east lines, pattern color = gray!20] (i1) -- (o1) -- (o2) -- cycle;
        \draw[draw=black!55,dashed] (i1) -- (o1) -- (o2) -- cycle;

        \draw[draw=none,pattern=north west lines, pattern color = gray!20] (i1)--(o2)--(i2)--cycle;
        \draw[draw=black!65,dashed] (i1) -- (o2) -- (i2) -- cycle;

        % Triangles relevant for this special case
        \draw[draw=black!65,dashed,fill=gray!30] (i2) -- (new) -- (i3) -- cycle;
        \draw[draw=black!65,dashed,fill=gray!15] (i2) -- (o2) -- (new) -- cycle;

    \end{scope}

\end{tikzpicture}
\end{document}
